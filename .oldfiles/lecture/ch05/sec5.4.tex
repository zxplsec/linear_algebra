\section{实对称矩阵的对角化}



\begin{dingyi}
  元素为复数的矩阵和向量,称为复矩阵和复向量。
\end{dingyi}

\begin{dingyi}
  设$a_{ij}$为复数,$\A=(a_{ij})_{m\times n},~\bar\A=(\bar a_{ij})_{m\times n}$,$\bar a_{ij}$是$a_{ij}$的共轭复数,则称$\bar \A$是$\A$的共轭矩阵。
\end{dingyi}


\begin{itemize}
\item $\overline {\overline \A} = \A$
\item ${\overline \A}^T = \overline{\A^T}$
\item 当$\A$为实对称矩阵时,${\overline \A}^T = \overline{\A^T}$
\end{itemize}





\begin{itemize}
\item $\overline{k\A}=\overline k ~\overline \A$
\item $\overline{\A+\B}=\overline \A+\overline \B$
\item $\overline{\A\B}=\overline \A\overline \B$
\item $\overline{(\A\B)}^T=\overline \B^T \overline \A^T$
\item $\overline{\A^{-1}}=(\overline \A)^{-1}$
\item $\det \bar\A = \overline{\det \A}$
\end{itemize}



\subsection{实对称矩阵的特征值和特征向量}



\begin{dingli}
  实对称矩阵$\A$的任一特征值都是实数。
\end{dingli}
\begin{proof}
$$
\begin{array}{rl}
  \A\xx = \lambda \xx
  &~~\Longrightarrow~~
    \overline{(\A\xx)}^T = \overline{\lambda \xx}^T\\[0.1in]
  &~~\Longrightarrow~~
    \overline{\xx}^T~\overline{\A}^T~\xx = \overline{\lambda} ~\overline{\xx}^T~\xx\\[0.1in]
  &~~\Longrightarrow~~
    \overline{\xx}^T~\A^T~\xx = \overline{\lambda} ~\overline{\xx}^T~\xx\\[0.1in]
  &~~\Longrightarrow~~
    \lambda \overline{\xx}^T~\xx = \overline{\lambda} ~\overline{\xx}^T~\xx\\[0.1in]
  &~~\Longrightarrow~~
    \lambda = \overline \lambda
\end{array}
$$
\end{proof}






\begin{dingli}
  实对称矩阵$\A$对应于不同特征值的特征向量是正交的。
\end{dingli}
\begin{proof}
设$\A\xx_1=\lambda_1\xx_1,~~\A\xx_1=\lambda_1\xx_1~ (\lambda_1\ne\lambda_2), ~~\A^T=\A$,则
$$
\lambda_1\xx_2^T\xx_1=\xx_2^T\A\xx_1=\xx_2^T\A^T\xx_1=(\A\xx_2)^T\xx_1=(\lambda_2\xx_2)^T\xx_1=\lambda_2\xx_2^T\xx_1
$$
由于$\lambda_1\ne\lambda_2$,所以
$$
\xx_2^T\xx_1=0.
$$
\end{proof}



