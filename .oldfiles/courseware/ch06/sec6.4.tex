\section{正定二次型和正定矩阵}

\begin{frame}
  \begin{footnotesize}
    \begin{block}{定义}
      如果对于任意的非零向量$\xx=(x_1,x_2,\cd,x_n)^T$,恒有
      $$
      \xx^T\A\xx=\sum_{i=1}^n\sum_{j=1}^na_{ij}x_ix_j>0,
      $$
      就称$\xx^T\A\xx$为正定二次型,称$\A$为正定矩阵。
    \end{block}
    \pause\vspace{0.1in}

    
    注:正定矩阵是针对对称矩阵而言的。
    
  \end{footnotesize}
\end{frame}

\begin{frame}
  \begin{footnotesize}
    \begin{block}{结论1}
      二次型$f(y_1,y_2,\cd,y_n)=d_1y_1^2+d_2y_2^2+\cd+d_ny_n^2$正定
      $~~~\Longleftrightarrow~~~d_i>0~~(i=1,2,\cd,n)$
    \end{block}\pause\proofname
    \begin{itemize}
    \item[$\Leftarrow$] 显然 \pause
    \item[$\Rightarrow$] 设$d_i\le 0$,取$y_i=1, y_j=0(j\ne i)$,代入二次型,得
      $$
      f(0,\cd,0,1,0,\cd,0)=d_i\le 0
      $$
      这与二次型$f(y_1,y_2,\cd,y_n)$正定矛盾。
    \end{itemize}
  \end{footnotesize}
\end{frame}


\begin{frame}
  \begin{footnotesize}
    \begin{block}{结论2}
      一个二次型$\xx^T\A\xx$,经过非退化线性变换$\xx=\C\yy$,化为$\yy^T(\C^T\A\C)\yy$,其正定性保持不变。即当
      $$\xx^T\A\xx~~~\xLongleftrightarrow[]{\ds \xx=\C\yy}~~~\yy^T(\C^T\A\C)\yy\quad (\C\mbox{可逆})$$
      时,等式两端的二次型有相同的正定性。
    \end{block}\pause\proofname
    $\forall \yy=(y_1,y_2,\cd,y_n)\ne\zero$,由于$\xx=\C\yy(\C\mbox{可逆})$,则$\xx\ne \zero$。若$\xx^T\A\xx$正定,则$\xx^T\A\xx>0$。
    从而有:$\forall \yy\ne\zero$,
    $$
    \yy^T(\C^T\A\C)\yy=\xx^T\A\xx>0
    $$
    故$\yy^T(\C^T\A\C)\yy$是正定二次型。\pause 反之亦然。
  \end{footnotesize}
\end{frame}


\begin{frame}
  \begin{footnotesize}
    \begin{block}{定理}
      若$\A$是$n$阶实对称矩阵,则以下命题等价:
      \begin{itemize}
      \item[(1)]$\xx^T\A\xx$是正定二次型($\A$是正定矩阵);
      \item[(2)]$\A$的正惯性指数为$n$,即$\A\simeq\II$;
      \item[(3)]存在可逆矩阵$\PP$使得$\A=\PP^T\PP$;
      \item[(4)]$\A$的$n$个特征值$\lambda_1,\lambda_2,\cd,\lambda_n$全大于零。
      \end{itemize}
    \end{block}
  \end{footnotesize}
\end{frame}

\begin{frame}
  \begin{footnotesize}
    \begin{exampleblock}{例}
      $\A\mbox{正定} ~~\Longrightarrow~~ \A^{-1}\mbox{正定}$
    \end{exampleblock}
  \end{footnotesize}
\end{frame}

\begin{frame}
  \begin{footnotesize}
    \begin{exampleblock}{例2}
      判断二次型
      $$
      f(x_1,x_2,x_3)=x_1^2+2x_2^2+3x_3^2+2x_1x_2-2x_2x_3
      $$
      是否为正定二次型。
    \end{exampleblock}
  \end{footnotesize}
\end{frame}

\begin{frame}
  \begin{footnotesize}
    \begin{exampleblock}{例3}
      判断二次型
      $$
      f(x_1,x_2,x_3)=3x_1^2+x_2^2+3x_3^2-4x_1x_2-4x_1x_3+4x_2x_3
      $$
      是否为正定二次型。
    \end{exampleblock}
  \end{footnotesize}
\end{frame}

\begin{frame}
  \begin{footnotesize}
    \begin{block}{定理}
      $$
      \A\mbox{正定}~~\Longrightarrow~~
      a_{ii}>0(i=1,2,\cd,n) \mbox{~~且~~}
      |\A|>0
      $$
    \end{block}
  \end{footnotesize}
\end{frame}

\begin{frame}
  \begin{footnotesize}
    \begin{block}{定理}
      $$\A\mbox{正定} ~~\Longleftrightarrow~~ \A\mbox{的$n$个顺序主子式全大于零。}$$
    \end{block}
  \end{footnotesize}
\end{frame}

\begin{frame}
  \begin{footnotesize}
    
  \end{footnotesize}
\end{frame}

\begin{frame}
  \begin{footnotesize}
    
  \end{footnotesize}
\end{frame}

\begin{frame}
  \begin{footnotesize}
    
  \end{footnotesize}
\end{frame}

\begin{frame}
  \begin{footnotesize}
    
  \end{footnotesize}
\end{frame}

\begin{frame}
  \begin{footnotesize}
    
  \end{footnotesize}
\end{frame}

\begin{frame}
  \begin{footnotesize}
    
  \end{footnotesize}
\end{frame}

\begin{frame}
  \begin{footnotesize}
    
  \end{footnotesize}
\end{frame}

\begin{frame}
  \begin{footnotesize}
    
  \end{footnotesize}
\end{frame}
