\subsection{正交矩阵及其性质}
\begin{frame}
  \begin{footnotesize}
    \begin{block}{定义}
      设$\A\in\RRR^{n\times n}$,如果
      $$
      \A^T\A=\II
      $$
      则称$\A$为正交矩阵。
    \end{block}
  \end{footnotesize}
\end{frame}


\begin{frame}
  \begin{footnotesize}
    \begin{block}{定理}
      $$
      \A\mbox{为}\mbox{正交矩阵}
      ~~\Longleftrightarrow~~
      \A\mbox{的列向量组为一组标准正交基。}
      $$
    \end{block}
    \pause
    \proofname
    将$\A$按列分块为$(\alphabd_1,\alphabd_2,\cd,\alphabd_n)$,则
    $$
    \A^T\A = \left(
    \begin{array}{c}
      \alphabd_1^T\\
      \alphabd_2^T\\
      \vd\\
      \alphabd_n^T
    \end{array}
    \right) (\alphabd_1,\alphabd_2,\cd,\alphabd_n) = \left(
    \begin{array}{cccc}
      \alphabd_1^T\alphabd_1&\alphabd_1^T\alphabd_2&\cd&\alphabd_1^T\alphabd_n\\
      \alphabd_2^T\alphabd_1&\alphabd_2^T\alphabd_2&\cd&\alphabd_2^T\alphabd_n\\
      \vd&\vd&&\vd\\
      \alphabd_n^T\alphabd_1&\alphabd_n^T\alphabd_2&\cd&\alphabd_n^T\alphabd_n
    \end{array}
    \right)
    $$
    \pause 
    因此
    $$
    \begin{array}{rl}
      \A^T\A=\II &~~\Longleftrightarrow~~
      \left\{
      \begin{array}{ll}
        \alphabd_i^T\alphabd_i=1,  &i=1,2,\cd,n\\[0.2cm]
        \alphabd_i^T\alphabd_j=0,  &j\ne i, ~~i,j=1,2,\cd,n
      \end{array}
      \right.\\[0.3in]
      &~~\Longleftrightarrow~~
      \A\mbox{的列向量组为一组标准正交基。}
    \end{array}
    $$
  \end{footnotesize}
\end{frame}


\begin{frame}
  \begin{footnotesize}
    \begin{block}{定理}
      设$\A,\B$皆为$n$阶正交矩阵,则
      \begin{itemize}
      \item[(1)] $|\A|=1\mbox{~或~} -1$
      \item[(2)] $\A^{-1}=\A^T$
      \item[(3)] $\A^T$也是正交矩阵
      \item[(4)] $\A\B$也是正交矩阵
      \end{itemize}
    \end{block}
  \end{footnotesize}
\end{frame}

\begin{frame}
  \begin{footnotesize}
    \begin{block}{定理}
      若列向量$\xx,\yy\in\RRR^n$在$n$阶正交矩阵$\A$的作用下变换为$\A\xx,\A\yy\in\RRR^n$,则向量的内积、长度与向量间的夹角都保持不变,即
      $$
      \begin{array}{c}
        (\A\xx,\A\yy)=(\xx,\yy),\\[0.1in]
        \|\A\xx\|=\|\xx\|, ~~\|\A\yy\|=\|\yy\|, \\[0.1in]
        <\A\xx,\A\yy>=<\xx,\yy>.
      \end{array}
      $$
    \end{block}
  \end{footnotesize}
\end{frame}
