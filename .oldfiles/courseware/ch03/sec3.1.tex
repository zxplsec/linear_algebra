%%%%%
\section{$n$维向量及其线性相关性}

\begin{frame}
\begin{footnotesize}
  考察三元齐次线性方程组
  \begin{equation}\label{ls}
    \left\{
    \begin{array}{l}
      a_{11}x_1+a_{12}x_2+a_{13}x_3=0,\\[0.1in]
      a_{21}x_1+a_{22}x_2+a_{23}x_3=0,\\[0.1in]
      a_{31}x_1+a_{32}x_2+a_{33}x_3=0
    \end{array}
    \right.
  \end{equation}
  我们用向量工具给出其几何解释。
  \pause
  记
  $$
  \ii = \left(
  \begin{array}{ccc}
    1&
    0&
    0
  \end{array}
  \right), ~~
  \jj = \left(
  \begin{array}{ccc}
    0&
    1&
    0
  \end{array}
  \right), ~~
  \kk = \left(
  \begin{array}{ccc}
    0&
    0&
    1
  \end{array}
  \right)
  $$
  以及
  $$
  \alphabd_i = \left(
  \begin{array}{ccc}
    a_{i1}&
    a_{i2}&
    a_{i3}
  \end{array}
  \right) = a_{i1} \ii + a_{i2} \jj + a_{i3} \kk
  $$
  该线性方程组的解可记为
  $$
  \xx = \left(
  \begin{array}{ccc}
    x_1&
    x_2&
    x_3
  \end{array}
  \right) = x_1\ii + x_2\jj + x_3\kk
  $$
\end{footnotesize}
\end{frame}


\begin{frame}
  \begin{footnotesize}
    \begin{block}{向量的垂直}
      两个向量$\uu=(u_1,u_2,u_3), ~~\vv=(v_1,v_2,v_3)$垂直的充分必要条件是
      $$
      u_1v_1+u_2v_2+u_3v_3=0.
      $$
    \end{block}
    \pause

    由以上方程组可看出,解向量$\xx$与$\alphabd_1,\alphabd_2,\alphabd_3$都垂直。\pause 故
    \begin{itemize}
    \item[(1)] 若$\alphabd_1,\alphabd_2,\alphabd_3$不共面,只有零向量与三者都垂直,即线性方程组(\ref{ls})只有零解;\\[0.1in] \pause
    \item[(2)] 若$\alphabd_1,\alphabd_2,\alphabd_3$共面但不共线,则与该平面垂直的向量都是线性方程组(\ref{ls})的解,
      故(\ref{ls})有无穷多个彼此平行的解向量;\\[0.1in] \pause
    \item[(3)] 若$\alphabd_1,\alphabd_2,\alphabd_3$共线,则过原点且与该直线垂直的平面上的全体向量都是(\ref{ls})的解向量,
      此时任一解向量均可表示为
      $$
      \xx = k_1 \xx^{(1)} + k_2 \xx^{(2)},
      $$
      其中$\xx^{(1)}, \xx^{(1)}$为(\ref{ls})的某两个不共线的非零解向量,$k_1,k_2$为任意常数。
    \end{itemize}
  \end{footnotesize}
\end{frame}

\begin{frame}
  \begin{footnotesize}
    \begin{block}{$n$维向量}
      数域$F$上的$n$个数$a_1,a_2,\cd,a_n$构成的有序数组称为数域$F$上的一个$n$维向量,记为
      \begin{equation}\label{vec}
        \alphabd = (a_1, a_2, \cd, a_n)
      \end{equation}
      其中$a_i$称为$\alphabd$的第$i$个分量。
    \end{block}
    \pause
    
    \begin{itemize}
    \item 形如(\ref{vec})的向量称为\red{行向量};\\[0.1in]
    \item 形如
      $$
      \alphabd = (a_1, a_2, \cd, a_n)^T = \left(
      \begin{array}{c}
        a_1\\
        a_2\\
        \vd\\
        a_n
      \end{array}
      \right)
      $$
      的向量称为\red{列向量}。
    \end{itemize}

    \pause
    数域$F$上全体$n$维向量组成的集合,记作$F^n$。\pause 设$\alphabd\in F^n$,则
    \begin{itemize}
    \item 当$F$取为$\mathbb R$时,$\alphabd$为实向量;
    \item 当$F$取为$\mathbb C$时,$\alphabd$为复向量。
    \end{itemize}
  \end{footnotesize}
\end{frame}


\begin{frame}
  \begin{footnotesize}
    \begin{block}{向量运算}
      设$\alphabd=(a_1,a_2,\cd,a_n),~~\betabd=(b_1,b_2,\cd,b_n)\in F^n$,$k\in F$,定义
      \begin{itemize}
      \item[(i)]
        $\alphabd=\betabd$当且仅当$a_i=b_i(i=1,2,\cd,n)$;\\[0.1in]
      \item[(ii)]向量加法
        $$
        \alphabd+\betabd=(a_1+b_1,a_2+b_2,\cd,a_n+b_n)
        $$\\[0.1in]
      \item[(iii)]向量数乘
        $$
        k\alphabd=(ka_1,ka_2,\cd,ka_n)
        $$
      \end{itemize}
    \end{block}
    \pause
    
    \begin{itemize}
    \item 在(iii)中取$k=-1$,得
      $$
      (-1)\alphabd = (-a_1,-a_2,\cd,-a_n)
      $$
      右端的向量称为$\alphabd$的负向量,记为$-\alphabd$. \\[0.1in]
    \item 向量的减法定义为
      $$
      \betabd-\alphabd = \betabd + (-\alphabd)
      $$
    \end{itemize}
    
  \end{footnotesize}
\end{frame}


\begin{frame}
  \begin{footnotesize}
    \begin{block}{向量的8条运算规则}
      设$\alphabd,\betabd,\gammabd\in F^n, 1,k,l\in F$,则
      \begin{itemize}
      \item[(1)] $\alphabd+\betabd=\betabd+\alphabd$\\[0.1in]
      \item[(2)] $(\alphabd+\betabd)+\gammabd=\alphabd+(\betabd+\gammabd)$\\[0.1in]
      \item[(3)] 对任一向量$\alphabd$,有$\alphabd+\zero=\alphabd$\\[0.1in]
      \item[(4)] 对任一向量$\alphabd$,存在负向量$-\alphabd$,使得$\alphabd+(-\alphabd)=\zero$\\[0.1in]
      \item[(5)] $1\alphabd=\alphabd$
      \item[(6)] $k(l\alphabd)=(kl)\alphabd$
      \item[(7)] $k(\alphabd+\betabd)=k\alphabd+k\betabd$
      \item[(8)] $(k+l)\alphabd=k\alphabd+l\alphabd$
      \end{itemize}
    \end{block}
    \pause
    \begin{block}{向量空间}
      数域$F$上的$n$维向量,在其中定义了上述加法与数乘运算,就称之为$F$上的$n$维向量空间,仍记为$F^n$。
      当$F=\mathbb R$时,叫做$n$维实向量空间,记作$\mathbb R^n$。
    \end{block}

  \end{footnotesize}
\end{frame}

\begin{frame}
  \begin{footnotesize}
    \begin{block}{定义(线性表示)}
      设$\alphabd_i\in F^n, k_i \in F (i=1,2,\cd,m)$,则向量
      $$
      \sum_{i=1}^m k_i\alphabd_i = k_1\alphabd_1 + k_2\alphabd_2+\cd + k_m\alphabd_m
      $$
      称为向量组$\alphabd_1,\alphabd_2,\cd,\alphabd_m$在数域$F$上的一个\red{线性组合}。
      \vspace{0.1in}

      如果记
      $$\betabd=\sum_{i=1}^m k_i\alphabd_i,$$
      则称$\betabd$可由$\alphabd_1,\alphabd_2,\cd,\alphabd_m$\red{线性表示}(或\red{线性表出})。
    \end{block}
  \end{footnotesize}
\end{frame}

\begin{frame}
  \begin{footnotesize}
    设有线性方程组$\A\xx=\bb$,其中$\A$为$m\times n$矩阵。记
    $$\A=(\alphabd_1,\alphabd_2,\cd,\alphabd_n),$$
    即
    $$
    (\alphabd_1,\alphabd_2,\cd,\alphabd_n) \left(
    \begin{array}{c}
      x_1\\
      x_2\\
      \vd\\
      x_n
    \end{array}
    \right)=\bb
    $$
    于是线性方程组可等价的表述为
    $$
    x_1\alphabd_1+x_2\alphabd_2+\cd+x_n\alphabd_n=\bb
    $$
    \begin{block}{注}
      向量$\bb$可由$\alphabd_1,\alphabd_2,\cd,\alphabd_n$线性表示,等价于方程组
      $$
      x_1\alphabd_1+x_2\alphabd_2+\cd+x_n\alphabd_n=\bb
      $$
      有解。
    \end{block}
  \end{footnotesize}
\end{frame}

\begin{frame}
  \begin{footnotesize}
    \begin{block}{定义(线性相关与线性无关)}
      若对$m$个向量$\alphabd_1,\alphabd_2,\cd,\alphabd_m\in F^n$,有$m$个不全为零的数$k_1,k_2,\cd,k_m\in F$,使
      \begin{equation}\label{def1}
        k_1\alphabd_1 + k_2\alphabd_2+\cd + k_m\alphabd_m = \zero        
      \end{equation}
      成立,则称\red{$\alphabd_1,\alphabd_2,\cd,\alphabd_m$线性相关};
      否则,称\red{$\alphabd_1,\alphabd_2,\cd,\alphabd_m$线性无关}。
    \end{block}

    \pause
    \begin{block}{注}
      向量组$\alphabd_1,\alphabd_2,\cd,\alphabd_m$线性无关,指的是
      \begin{itemize}
      \item 没有不全为零的数$k_1,k_2,\cd,k_m$使(\ref{def1})成立 \\[0.1in]
      \item 只有当$k_1,k_2,\cd,k_m$全为零时,才使(\ref{def1})成立 \\[0.1in]
      \item 若(\ref{def1})成立,则$k_1,k_2,\cd,k_m$必须全为零
      \end{itemize}
    \end{block}
  \end{footnotesize}
\end{frame}

\begin{frame}
  \begin{footnotesize}
    \begin{block}{定理3.1.1}
      \begin{itemize}
      \item  向量组$\alphabd_1,\alphabd_2,\cd,\alphabd_m$线性相关,等价于齐次方程组
        $$
        x_1\alphabd_1+x_2\alphabd_2+\cd+x_m\alphabd_m=\zero
        $$
        有非零解。
      \item  向量组$\alphabd_1,\alphabd_2,\cd,\alphabd_m$线性无关,等价于齐次方程组
        $$
        x_1\alphabd_1+x_2\alphabd_2+\cd+x_m\alphabd_m=\zero
        $$
        只有零解。
      \end{itemize}

    \end{block}
  \end{footnotesize}
\end{frame}


\begin{frame}
  \begin{footnotesize}
    对于只含有一个向量$\alphabd$的向量组,若存在不为零的数$k$使得
    $$
    k \alphabd = \zero,
    $$
    则
    $$
    \Rightarrow \alphabd = \frac1k \zero = \zero
    $$ \pause 
    若$\alphabd\ne \zero$,要使
    $$
    k \alphabd = \zero,
    $$
    必须$k=0$.
    \pause 
    \begin{block}{小结}
      \begin{itemize}
      \item 当$\alphabd=\zero$时,向量组$\alphabd$线性相关
      \item 当$\alphabd\ne \zero$时,向量组$\alphabd$线性无关
      \end{itemize}
    \end{block}
  \end{footnotesize}
\end{frame}


\begin{frame}
  \begin{footnotesize}
    \begin{block}{定理3.1.2}
      向量组$\alphabd_1,\alphabd_2,\cd,\alphabd_m(m\ge 2)$线性相关的充分必要条件是$\alphabd_1,\alphabd_2,\cd,\alphabd_m$中\red{至少有一个向量}可由其余$m-1$个向量线性表出。
    \end{block}
    \pause
    \proofname
    \red{($\Rightarrow$)} \quad
    若向量组$\alphabd_1,\alphabd_2,\cd,\alphabd_m(m\ge 2)$线性相关,则必存在不全为零的数$k_1,k_2,\cd,k_m$使得
    $$
    k_1\alphabd_1 + k_2\alphabd_2+\cd + k_m\alphabd_m = \zero,
    $$ \pause 
    不妨设$k_1\ne 0$,则
    $$
    \alphabd_1 =  -\frac{k_2}{k_1}\alphabd_2-\cd - \frac{k_m}{k_1}\alphabd_m,
    $$
    必要性得证。
    \pause\vspace{0.1in}

    \red{($\Leftarrow$)} \quad
    不妨设$\alphabd_1$可由$\alphabd_2,\cd,\alphabd_m$线性表示,即
    $$
    \alphabd_1 = l_2\alphabd_2+\cd+l_m\alphabd_m    
    $$ \pause 
    于是有
    $$
    \alphabd_1 - l_2\alphabd_2-\cd-l_m\alphabd_m=\zero,
    $$ \pause 
    显然$1,-l_2,\cd,-l_m$不全为零,故$\alphabd_1,\alphabd_2,\cd,\alphabd_m$线性相关。
      \end{footnotesize}
\end{frame}

\begin{frame}\frametitle{证明线性无关的方法}
  \begin{footnotesize}
    证明向量组$\alphabd_1,\alphabd_2,\cd,\alphabd_m$线性无关的基本方法为:
    \vspace{0.1in}

    
    说明齐次方程组
    $$
    x_1\alphabd_1+x_2\alphabd_2+\cd+x_m\alphabd_m=\zero
    $$
    只有零解。 \vspace{0.3in} \pause 

    也常常表述为:设
    $$
    x_1\alphabd_1+x_2\alphabd_2+\cd+x_m\alphabd_m=\zero
    $$
    然后说明上式成立,只能有唯一选择:
    $$
    x_1=x_2=\cd=x_m=0.
    $$
  \end{footnotesize}
\end{frame}



\begin{frame}
  \begin{footnotesize}
    \begin{exampleblock}{例1}
      设$n$维向量$\ee_i=(0,\cd,0,1,0,\cd,0)$,则$\ee_1,\ee_2,\cd,\ee_n$线性无关。
    \end{exampleblock}
    \pause
    \jiename 
    设存在$k_1,k_2,\cd,k_n$使得
    $$
    k_1\ee_1+k_2\ee_2+\cd+k_n\ee_n=\zero,
    $$
    即
    $$
    (k_1,k_2,\cd,k_n)=\zero,
    $$
    则必有$k_1=k_2=\cd=k_n=0$,故$\ee_1,\ee_2,\cd,\ee_n$线性无关。

    \pause 
    \begin{block}{注}
      $n$维向量$\ee_1,\ee_2,\cd,\ee_n$称为\red{基本向量}。$F^n$中任何向量$\alphabd=(a_1,a_2,\cd,a_n)$都可以由$\ee_1,\ee_2,\cd,\ee_n$线性表示,即
      $$
      \alphabd = a_1\ee_1+a_2\ee_2+\cd+a_n\ee_n.
      $$
    \end{block}

  \end{footnotesize}
\end{frame}


\begin{frame}
  \begin{footnotesize}
    \begin{exampleblock}{例2}
      包含零向量的向量组是线性相关的。
    \end{exampleblock}
    \pause
    \jiename
    设该向量组为$\alphabd_1,\alphabd_2,\cd,\alphabd_m$,其中$\alphabd_1=\zero$。则存在$m$个不全为零的数$1,0,\cd,0$使得
    $$
    1 \alphabd_1+0\alphabd_2+\cd+0\alphabd_m=\zero,
    $$
    故该向量组线性相关。

    %% \pause
    %% \begin{block}{注}
    %%   \begin{itemize}
    %%   \item 单个向量$\alphabd$线性相关,当且仅当$\alphabd$为零向量;\\[0.1in]
    %%   \item 单个向量$\alphabd$线性无关,当且仅当$\alphabd$为非零向量。        
    %%   \end{itemize}
    %% \end{block}
  \end{footnotesize}
\end{frame}


\begin{frame}
  \begin{footnotesize}
    \begin{exampleblock}{例3}
      如果向量组$\alphabd_1,\alphabd_2,\cd,\alphabd_m$中有一部分向量线性相关,则整个向量组也线性相关。
    \end{exampleblock}
    \pause
    \proofname
    不妨设$\alphabd_1,\alphabd_2,\cd,\alphabd_r(r<m)$线性相关,则存在$r$个不全为零的数$k_1,k_2,\cd,k_r$使得
    $$
    k_1\alphabd_1+k_2\alphabd_2+\cd+k_r\alphabd_r=\zero,
    $$
    从而有$m$个不全为零的数$k_1,k_2,\cd,k_r,0,\cd,0$,使得
    $$
    k_1\alphabd_1+k_2\alphabd_2+\cd+k_r\alphabd_r+0\alphabd_{r+1}+\cd+0\alphabd_m=\zero,
    $$
    故$\alphabd_1,\alphabd_2,\cd,\alphabd_m$线性相关。
  \end{footnotesize}
\end{frame}

\begin{frame}
  \begin{footnotesize}
    \begin{block}{注}
      \begin{itemize}
      \item 如果$\alphabd_1,\alphabd_2,\cd,\alphabd_m$线性无关,则其中任一部分向量组也线性无关。\\[0.1in]              
      \item     \red{部分相关,则整体相关;整体无关,则部分无关。}
      \end{itemize}
      
    \end{block}

    \pause 
    \begin{block}{注}
      该定理不能理解为:\blue{线性相关的向量组中,每一个向量都能由其余向量线性表示。} \pause 
      \vspace{0.1in}

      如$\alphabd_1=(0,1), ~~\alphabd_2=(0,-2), ~~\alphabd_3=(1,1)$线性相关(因为$\alphabd_1,~~\alphabd_2$线性相关),
      但$\alphabd_3$不能由$\alphabd_1,~~\alphabd_2$线性表示。
      
    \end{block}
    
  \end{footnotesize}
\end{frame}

%% \begin{frame}
%%   \begin{footnotesize}
%%     \begin{block}{定理3.1.3}
%%       设$\alphabd_1,\alphabd_2,\cd,\alphabd_r\in F^n$,其中
%%       $$
%%       \begin{array}{c}
%%         \alphabd_1 = (a_{11},~~a_{21},~~\cd,~~a_{n1})^T,\\[0.1in]
%%         \alphabd_2 = (a_{12},~~a_{22},~~\cd,~~a_{n2})^T,\\[0.1in]
%%         \cd\cd\cd\\[0.1in]
%%         \alphabd_r = (a_{1r},~~a_{2r},~~\cd,~~a_{nr})^T.
%%       \end{array}
%%       $$
%%       则向量组$\alphabd_1,\alphabd_2,\cd,\alphabd_r$线性相关的充分必要条件是齐次线性方程组
%%       \begin{equation}\label{ax}
%%         \A \xx = \zero
%%       \end{equation}
%%       有非零解,其中
%%       $$
%%       \A = (\alphabd_1,~\alphabd_2,~\cd,~\alphabd_r) = \left(
%%       \begin{array}{cccc}
%%         a_{11}&a_{12}&\cd&a_{1r}\\[0.05in]
%%         a_{21}&a_{22}&\cd&a_{2r}\\[0.05in]
%%         \vd&\vd&&\vd\\[0.05in]
%%         a_{n1}&a_{n2}&\cd&a_{nr}.
%%       \end{array}\right), \xx = \left(
%%       \begin{array}{c}
%%         x_1\\[0.05in]
%%         x_2\\[0.05in]
%%         \vd\\[0.05in]
%%         x_r
%%       \end{array}
%%       \right)
%%       $$
%%     \end{block}
%%   \end{footnotesize}
%% \end{frame}

%% \begin{frame}
%%   \begin{footnotesize}
%%     \proofname  \quad 
%%     设
%%     \begin{equation}\label{th3.1}
%%     x_1 \alphabd_1 + x_2 \alphabd_2 + \cd + x_r \alphabd_r = \zero,      
%%     \end{equation}
%%     即
%%     $$
%%     x_1 \left(
%%     \begin{array}{c}
%%       a_{11}\\
%%       a_{21}\\
%%       \vd \\
%%       a_{n1}
%%     \end{array}
%%     \right) + x_2 \left(
%%     \begin{array}{c}
%%       a_{12}\\
%%       a_{22}\\
%%       \vd \\
%%       a_{n2} 
%%     \end{array}
%%     \right)+ \cd + x_r \left(
%%     \begin{array}{c}
%%       a_{1r}\\
%%       a_{2r}\\
%%       \vd \\
%%       a_{nr}
%%     \end{array}
%%     \right) = \zero.
%%     $$
%%     此即齐次线性方程组(\ref{ax})。
%%     \pause 
%%     \begin{itemize}
%%     \item[\red{($\Rightarrow$)}]     若向量组$\alphabd_1,\alphabd_2,\cd,\alphabd_r$线性相关,
%%       则必有不全为零的数$x_1,x_2,\cd,x_r$使得(\ref{th3.1})成立,
%%       即齐次线性方程组(\ref{ax})有非零解。\pause  \\[0.1in]
%%     \item[\red{($\Leftarrow$)}]     若方程组(\ref{ax})有非零解,就是说有不全为零的数$x_1,x_2,\cd,x_r$使得(\ref{th3.1})成立,故向量组$\alphabd_1,\alphabd_2,\cd,\alphabd_r$线性相关。
      
%%     \end{itemize}\end{footnotesize}
%% \end{frame}


\begin{frame}
  \begin{footnotesize}
     \begin{block}{重要结论}
      对于齐次线性方程组,如果
      $$
      \red{\mbox{未知量个数} ~~>~~ \mbox{方程个数},}
      $$
      则它必有无穷多解,从而必有非零解。
    \end{block}   \end{footnotesize}
\end{frame}


\begin{frame}
  \begin{footnotesize}
    \begin{block}{定理3.1.3}      
      任意$n+1$个$n$维向量都是线性相关的。
    \end{block}
    \pause 
    \proofname
    对向量组$\alphabd_1,\alphabd_2,\cd,\alphabd_n,\alphabd_{n+1}\in F^n$,设
    $$
    x_1 \alphabd_1 + x_2 \alphabd_2 + \cd + x_n \alphabd_n + x_{n+1} \alphabd_{n+1} = \zero,  
    $$
    注意到此齐次线性方程组中,未知量个数为$n+1$,而方程个数为$n$,故方程组一定有无穷多个解,从而必有非零解。
    得证$\alphabd_1,\alphabd_2,\cd,\alphabd_n,\alphabd_{n+1}$线性相关。
    
    \pause 
    \begin{block}{注}
      \begin{itemize}
      \item    向量个数$~>~$向量维数 $~~\red{\Rightarrow}~~$ 向量组必线性相关。\pause 
      \item     在$\mathbb R^n$中,任意一组线性无关的向量最多只能含$n$个向量。
      \end{itemize}
    \end{block}
      
    
  \end{footnotesize}
\end{frame}


\begin{frame}
  \begin{footnotesize}
    \begin{block}{定理3.1.4}
      若向量组$\alphabd_1,\alphabd_2,\cd,\alphabd_r$线性无关,而$\red{\betabd},\alphabd_1,\alphabd_2,\cd,\alphabd_r$线性相关,则$\red{\betabd}$可由$\alphabd_1,\alphabd_2,\cd,\alphabd_r$线性表示,并且表示法惟一。
    \end{block}
    \pause
    \proofname 
    因为$\betabd,\alphabd_1,\alphabd_2,\cd,\alphabd_r$线性相关,故存在不全为零的数$k,k_1,k_2,\cd,k_r$使得
    $$
    k\betabd + k_1\alphabd_1+k_2\alphabd_2+\cd+k_r\alphabd_r=\zero,
    $$
    其中$k\ne 0 $\pause \red{(若$k=0$,则由$\alphabd_1,\alphabd_2,\cd,\alphabd_r$线性无关可知$k_1,k_2,\cd,k_r$全为零,这与$k,k_1,k_2,\cd,k_r$不全为零矛盾)}。 \pause 
    于是$\betabd$可由$\alphabd_1,\alphabd_2,\cd,\alphabd_r$线性表示为
    $$
    \betabd=-\frac{k_1}k\alphabd_1-\frac{k_2}k\alphabd_2-\cd-\frac{k_r}k\alphabd_r.
    $$
    \pause 
    \red{再证唯一性} \quad 设有两种表示法
    $$
    \betabd=l_1\alphabd_1+l_2\alphabd_2+\cd+l_r\alphabd_r,\quad
    \betabd=h_1\alphabd_1+h_2\alphabd_2+\cd+h_r\alphabd_r.
    $$ \pause
    于是
    $$
    (l_1-h_1)\alphabd_1+(l_2-h_2)\alphabd_2+\cd+(l_r-h_r)\alphabd_1=\zero,
    $$ \pause
    由$\alphabd_1,\alphabd_2,\cd,\alphabd_r$线性无关可知
    \red{$l_i-h_i=0, \quad \mbox{即} l_i=h_i.$}
    故$\betabd$由$\alphabd_1,\alphabd_2,\cd,\alphabd_r$线性表示的表示法惟一。
  \end{footnotesize}
\end{frame}



\begin{frame}
  \begin{footnotesize}
    \begin{block}{推论}
      如果$F^n$中的$n$个向量$\alphabd_1,\alphabd_2,\cd,\alphabd_n$线性无关,则$F^n$中的任一向量$\alphabd$可由$\alphabd_1,\alphabd_2,\cd,\alphabd_n$线性表示,且表示法惟一。
    \end{block}
    \pause
    \proofname
    由"任意$n+1$个$n$维向量线性相关''知,$\alphabd,\alphabd_1,\alphabd_2,\cd,\alphabd_n$线性相关,由前述定理可得结论成立。
  \end{footnotesize}
\end{frame}


\begin{frame}
  \begin{footnotesize}
    \begin{exampleblock}{例4}
      设$\alphabd_1=(1,-1,1),\alphabd_2=(1,2,0),\alphabd_3=(1,0,3),\alphabd_4=(2,-3,7)$.
      问:
      \begin{itemize}
      \item[(1)]$\alphabd_1,\alphabd_2,\alphabd_3$是否线性相关?\\[0.1in]
      \item[(2)]$\alphabd_4$可否由$\alphabd_1,\alphabd_2,\alphabd_3$线性表示?如能表示求出其表示式。
      \end{itemize}
    \end{exampleblock}
    \pause
    \jiename 
    \begin{itemize}
    \item[(1)]    考察
    $
    \A = (\alphabd_1^T, \alphabd_2^T, \alphabd_3^T) = \left(
    \begin{array}{rrr}
      1&1&1\\
      -1&2&0\\
      1&0&3
    \end{array}
    \right). 
    $ \quad
    由$|\A|=7$可知$\A$可逆,故$\A\xx=\zero$只有零解,从而$\alphabd_1,\alphabd_2,\alphabd_3$线性无关。\\[0.1in] \pause 
    \item[(2)] 根据推论,$\alphabd_4$可由$\alphabd_1,\alphabd_2,\alphabd_3$线性表示,且表示法惟一。 \pause 设
      $$
      x_1\alphabd_1+x_2\alphabd_2+x_3\alphabd_3=\alphabd_4 \pause  \Rightarrow
      x_1\alphabd_1^T+x_2\alphabd_2^T+x_3\alphabd_3^T=\alphabd_4^T       
      $$
      即$$
      \left(
      \begin{array}{ccc}
      \alphabd_1^T &\alphabd_2^T& \alphabd_3^T  
      \end{array}
      \right) \left(
      \begin{array}{c}
        x_1\\
        x_2\\
        x_3
      \end{array}
      \right)= 
       \left(
    \begin{array}{rrr}
      1&1&1\\
      -1&2&0\\
      1&0&3
    \end{array}
    \right) \left(
    \begin{array}{c}
      x_1\\
      x_2\\
      x_3
    \end{array}
    \right) =  \left(
    \begin{array}{r}
      2\\
      -3\\
      7
    \end{array}
    \right)
      $$ \pause 
    解此方程组得惟一解$x_1=1,x_2=-1,x_3=2$,故
    $
    \red{\alphabd_4=\alphabd_1-\alphabd_2+2\alphabd_3.}
    $
    \end{itemize}
  \end{footnotesize}
\end{frame}

\begin{frame}
  \begin{footnotesize}
    \begin{exampleblock}{例5}
      设向量组$\alphabd_1,\alphabd_2,\alphabd_3$线性无关,又$\betabd=\alphabd_1+\alphabd_2+2\alphabd_3$,
      $\betabd_2=\alphabd_1-\alphabd_2$,$\betabd_3=\alphabd_1+\alphabd_3$,证明$\betabd_1,\betabd_2,\betabd_3$线性相关。       
    \end{exampleblock}
    \pause
    \jiename
    设有数$x_1,x_2,x_3$使得
    \begin{equation}\label{li5}
      x_1\betabd_1+x_2\betabd_2+x_3\betabd_3=\zero
    \end{equation}    
    \pause
    即
    $$
    x_1(\alphabd_1+\alphabd_2+2\alphabd_3)+x_2(\alphabd_1-2\alphabd_2)+x_3(\alphabd_1+\alphabd_3)=\zero
    $$
    亦即
    $$
    (x_1+x_2)\alphabd_1+(x_1-2x_2)\alphabd_2+(x_1+x_3)\alphabd_3=\zero
    $$
    \pause
    因为$\alphabd_1,\alphabd_2,\alphabd_3$线性无关,故
    $$
    \left\{
    \begin{array}{rcrcrcrcr}
      x_1&+&x_2&&&=&0\\
      x_1&-&x_2&&&=&0\\
      2x_1&&&+&x_3&=&0.
    \end{array}
    \right.
    $$
    求解该方程组可得非零解$(-1,-1,2)$。因此,有不全为零的数$x_1,x_2,x_3$使得(\ref{li5})成立,从而$\betabd_1,\betabd_2,\betabd_3$线性相关。
  \end{footnotesize}
\end{frame}



%% \begin{frame}
%%   \begin{footnotesize}
%%     \begin{exampleblock}{例6}
%%       证明:$\alphabd_1+\alphabd_2,\alphabd_2+\alphabd_3,\alphabd_3+\alphabd_1$线性无关的充要条件是$\alphabd_1,\alphabd_2,\alphabd_3$线性无关。
%%     \end{exampleblock}
%%     \pause
%%     \proofname
%%     \red{($\Rightarrow$)} \quad
%%     假设$\alphabd_1,\alphabd_2,\alphabd_3$线性相关,则有不全为零的数$x_1+x_2,x_2+x_3,x_3+x_1$使得
%%     $$
%%     (x_1+x_2)\alphabd_1+(x_2+x_3)\alphabd_2+(x_3+x_1)\alphabd_3=\zero
%%     $$
%%     即
%%     $$
%%     x_1(\alphabd_1+\alphabd_2)+x_2(\alphabd_2+\alphabd_3)+x_3(\alphabd_3+\alphabd_1)=\zero
%%     $$
%%     \pause
%%     \vspace{0.1in}

%%     \red{($\Leftarrow$)} \quad
%%     设有$x_1,x_2,x_3$使得
%%     \begin{equation}\label{li6-1}
%%       x_1(\alphabd_1+\alphabd_2)+x_2(\alphabd_2+\alphabd_3)+x_3(\alphabd_3+\alphabd_1)=\zero
%%     \end{equation}
%%     即
%%     $$
%%     (x_1+x_3)\alphabd_1+(x_1+x_2)\alphabd_2+(x_2+x_3)\alphabd_3=\zero
%%     $$
%%     因为$\alphabd_1,\alphabd_2,\alphabd_3$线性无关,故
%%     $$
%%     x_1+x_3=0, \quad x_1+x_2=0, \quad x_2+x_3=0,
%%     $$
%%     该方程组只有零解。这说明若使(\ref{li6-1}),必有$x_1=x_2=x_3=0$,从而$\alphabd_1+\alphabd_2,\alphabd_2+\alphabd_3,\alphabd_3+\alphabd_1$线性无关。
%%   \end{footnotesize}
%% \end{frame}


\begin{frame}
  \begin{footnotesize}
    \begin{block}{定理3.1.5}
      \begin{itemize}
      \item[(1)] 如果一组$n$维向量$\alphabd_1,\alphabd_2,\cd,\alphabd_s$线性无关,那么把这些向量各任意添加$m$个分量所得的向量(\red{$n+m$维})组$\alphabd^*_1,\alphabd^*_2,\cd,\alphabd^*_s$也线性无关。
      \item[(2)] 如果$\alphabd_1,\alphabd_2,\cd,\alphabd_s$线性相关,那么它们各去掉相同的若干个分量所得到的新向量也线性相关。
      \end{itemize}
    \end{block}
  \end{footnotesize}
\end{frame}


\begin{frame}
  \begin{footnotesize}
    
    $$
    \left(
    \begin{array}{c}
      a_{11}\\
      a_{21}\\
      \vd\\
      a_{n1}\\
    \end{array}
    \right)
    \cd,
    \left(
    \begin{array}{c}
      a_{1s}\\
      a_{2s}\\
      \vd\\
      a_{ns}\\
    \end{array}
    \right) \mbox{线性无关} \pause ~~~\blue{\Rightarrow}~~~
        \left(
    \begin{array}{c}
      a_{11}\\
      a_{21}\\
      \vd\\
      a_{n1}\\
      \red{a_{n+1,1}}\\
      \vd\\
      \red{a_{n+m,1}}
    \end{array}
    \right),
    \cd,
    \left(
    \begin{array}{c}
      a_{1s}\\
      a_{2s}\\
      \vd\\
      a_{ns}\\
      \red{a_{n+1,s}}\\
      \vd\\
      \red{a_{n+m,s}}
    \end{array}
    \right) \mbox{线性无关}
    $$

    \pause 
    $$  
        \left(
    \begin{array}{c}
      a_{11}\\
      a_{21}\\
      \vd\\
      a_{n1}\\
      \red{a_{n+1,1}}\\
      \vd\\
      \red{a_{n+m,1}}
    \end{array}
    \right),
    \cd,
    \left(
    \begin{array}{c}
      a_{1s}\\
      a_{2s}\\
      \vd\\
      a_{ns}\\
      \red{a_{n+1,s}}\\
      \vd\\
      \red{a_{n+m,s}}
    \end{array}
    \right) \mbox{线性相关} \pause ~~~\blue{\Rightarrow}~~~
    \left(
    \begin{array}{c}
      a_{11}\\
      a_{21}\\
      \vd\\
      a_{n1}\\
    \end{array}
    \right)
    \cd,
    \left(
    \begin{array}{c}
      a_{1s}\\
      a_{2s}\\
      \vd\\
      a_{ns}\\
    \end{array}
    \right) \mbox{线性相关}
    $$
    
  \end{footnotesize}
\end{frame}


\begin{frame}
  \begin{footnotesize}

    \proofname
    两者互为逆否命题,证明第一个即可。\pause  \vspace{0.2in}

    向量组$\alphabd_1,\alphabd_2,\cd,\alphabd_s$线性无关,则方程组
    $$
    x_1\alphabd_1+x_2\alphabd_2+\cd+x_s\alphabd_s=\zero
    $$
    只有零解。\pause 设$\alphabd_i=(a_{1i},a_{2i},\cd,a_{ni})^T,~~ i=1,2,\cd,s$,即
    \begin{equation}\label{ls_ns}
    \left\{
    \begin{array}{rcrcrcrcr}
      a_{11}x_1&+&a_{12}x_2&+&\cd&+&a_{1s}x_s&=&0,\\[0.05in]
      a_{21}x_1&+&a_{22}x_2&+&\cd&+&a_{2s}x_s&=&0,\\[0.05in]
      &&&&\cd&&&&\\[0.05in]
      a_{n1}x_1&+&a_{n2}x_2&+&\cd&+&a_{ns}x_s&=&0.
    \end{array}
    \right.
    \end{equation}
    只有零解。
  \end{footnotesize}
\end{frame}

\begin{frame}
  \begin{footnotesize}
    不妨设每个向量增加了一个分量,即
    $$
    \alphabd_i^*= (a_{1i},a_{2i},\cd,a_{ni},\red{a_{n+1,i}})^T, ~~ ii=1,2,\cd,s.
    $$ \pause
    设
    $$
    x_1\alphabd_1^*+x_2\alphabd_2^*+\cd+x_s\alphabd_s^*=\zero
    $$
    即
    \begin{equation}\label{ls_ns1}
    \left\{
    \begin{array}{rcrcrcrcr}
      a_{11}x_1&+&a_{12}x_2&+&\cd&+&a_{1s}x_s&=&0,\\[0.05in]
      a_{21}x_1&+&a_{22}x_2&+&\cd&+&a_{2s}x_s&=&0,\\[0.05in]
      &&&&\cd&&&&\\[0.05in]
      a_{n1}x_1&+&a_{n2}x_2&+&\cd&+&a_{ns}x_s&=&0,\\[0.05in]
      \red{a_{n+1,1}x_1}&\red{+}&\red{a_{n+1,2}x_2}&\red{+}&\red{\cd}&\red{+}&\red{a_{n+1,s}x_s}&\red{=}&\red{0}.
    \end{array}
    \right.
    \end{equation}
    方程组(\ref{ls_ns1})的解全是方程组(\ref{ls_ns})的解。\pause \vspace{0.2in}

    而方程组(\ref{ls_ns})只有零解,故方程组(\ref{ls_ns1})也只有零解。故向量组$\alphabd^*_1,\alphabd^*_2,\cd,\alphabd^*_s$线性无关。
  \end{footnotesize}
\end{frame}

\begin{frame}
  \begin{footnotesize}
    \begin{block}{注}
      设向量组线性相关,若增加的分量全为零,则得到的新向量组也线性相关。
    \end{block}
    \pause 
    \proofname
    设$\alphabd_1,\alphabd_2,\cd,\alphabd_s$线性相关,把这些向量各任意添加$m$个全为零的分量,
    所得到的新向量组记为$\alphabd^*_1,\alphabd^*_2,\cd,\alphabd^*_s$。\pause 
    此时方程组
    $$
    x_1\alphabd_1+x_2\alphabd_2+\cd+x_s\alphabd_s=\zero
    $$ 
    与方程组
    $$
    x_1\alphabd_1^*+x_2\alphabd_2^*+\cd+x_s\alphabd_s^*=\zero
    $$
    完全相同。所以新向量组$\alphabd^*_1,\alphabd^*_2,\cd,\alphabd^*_s$也线性相关。
    
  \end{footnotesize}
\end{frame}

\begin{frame}
  \begin{footnotesize}
    \begin{block}{小结}
      \purple{对应位置全为零的向量,不影响向量组的线性相关性。}
    \end{block} \pause\vspace{0.2in}
    
    如
    $$  
    \left(
    \begin{array}{c}
      a_{11}\\
      0\\
      a_{21}\\
      \vd\\
      a_{n1}\\
      0\\
      \vd\\
      0
    \end{array}
    \right),
    \cd,
    \left(
    \begin{array}{c}
      a_{1s}\\
      0\\
      a_{2s}\\
      \vd\\
      a_{ns}\\
      0\\
      \vd\\
      0
    \end{array}
    \right) \mbox{~~与~~}
    \left(
    \begin{array}{c}
      a_{11}\\
      a_{21}\\
      \vd\\
      a_{n1}\\
    \end{array}
    \right),
    \cd,
    \left(
    \begin{array}{c}
      a_{1s}\\
      a_{2s}\\
      \vd\\
      a_{ns}\\
    \end{array}
    \right) 
    $$
    线性相关性一致。
  \end{footnotesize}
\end{frame}


\begin{frame}
  \begin{footnotesize}
    \begin{exampleblock}{例6}
      考察以下向量组的线性相关性:
      $$
      \left(
      \begin{array}{c}
        1\\
        0\\
        0\\
        2\\
        5
      \end{array}
      \right), \quad
      \left(
      \begin{array}{c}
        0\\
        1\\
        0\\
        6\\
        9
      \end{array}
      \right), \quad
      \left(
      \begin{array}{c}
        0\\
        0\\
        1\\
        4\\
        3
      \end{array}
      \right)
      $$
    \end{exampleblock}
    \pause
    \jiename
    去掉最后两个分量所得的向量组
      $$
      \left(
      \begin{array}{c}
        1\\
        0\\
        0
      \end{array}
      \right), \quad
      \left(
      \begin{array}{c}
        0\\
        1\\
        0
      \end{array}
      \right), \quad
      \left(
      \begin{array}{c}
        0\\
        0\\
        1
      \end{array}
      \right)
      $$
      线性无关,故原向量组线性无关。
  \end{footnotesize}
\end{frame}
