\section{齐次线性方程组有非零解的条件及解的结构}

\begin{frame}
  \begin{footnotesize}
    设$\mathbf A$为$m\times n$矩阵,考察以$\A$为系数矩阵的齐次线性方程组
    \begin{equation}\label{ax0}
      \A\xx=\zero.
    \end{equation}    
    \pause
    若将$\A$按列分块为
    $$
    \A = (\alphabd_1,~\alphabd_2,~\cd,~\alphabd_n),
    $$
    齐次方程组(\ref{ax0})可表示为
    $$
    x_1\alphabd_1+x_2\alphabd_2+\cd+x_n\alphabd_n=\zero.
    $$
    \pause
    齐次方程组(\ref{ax0})有\blue{非零解}的充分必要条件是$\alphabd_1,~\alphabd_2,~\cd,~\alphabd_n$\blue{线性相关},即
    $$
    \RR(\A) = \RR(\alphabd_1,~\alphabd_2,~\cd,~\alphabd_n) < n.
    $$
  \end{footnotesize}
\end{frame}

\begin{frame}
  \begin{footnotesize}
    \begin{block}{定理3.4.1}
      设$\A$为$m\times n$矩阵,则
      $$
      \blue{\underline{\A\xx=\zero\mbox{有非零解}}} ~~\Longleftrightarrow~~
      \blue{\underline{\RR(\A)<n}}.$$
    \end{block}
  \end{footnotesize}
\end{frame}

\begin{frame}
  \begin{footnotesize}
    \begin{block}{定理3.4.1的等价命题}
      设$\A$为$m\times n$矩阵,则
      $$
      \blue{\underline{\A\xx=\zero\mbox{只有零解}}} ~~\Longleftrightarrow~~
      \blue{\underline{\RR(\A)=n=\A\mbox{的列数}}}.
      $$
    \end{block}
  \end{footnotesize}
\end{frame}

\begin{frame}
  \begin{footnotesize}
    \begin{exampleblock}{例1}
      设$\A$为$n$阶矩阵,证明:存在$n\times s$矩阵$\B\ne \zero$,使得$\A\B=\zero$的充分必要条件是
      $$
      |\A|=0.
      $$      
    \end{exampleblock}
    \pause
    \proofname
    $|\A|=0~~\Longleftrightarrow~~\A\xx=\zero$有非零解。\pause 下证
    $$
    \purple{\mbox{存在}n\times s\mbox{矩阵}\B\ne 0\mbox{使得}\A\B=\zero
      ~~\Longleftrightarrow~~\A\xx=\zero\mbox{有非零解}}.
    $$     \pause
    \begin{itemize}
    \item[($\red{\Longrightarrow}$)]
      设$\A\B=\zero$,则$\B$的解向量为$\A\xx=\zero$的解。又$\B\ne\zero$,则$\B$至少有一个非零列向量,
      从而$\A\xx=\zero$至少有一个非零解。\\[0.2in] \pause 
    \item[($\red{\Longleftarrow}$)]
      设$\A\xx=\zero$有非零解,任取一个非零解$\betabd$,令
      $$
      \B=(\betabd,~\zero,~\cd,~\zero)
      $$
      则$\B\ne\zero$,且$\A\B=\zero$。
    \end{itemize}

  \end{footnotesize}
\end{frame}

\begin{frame}
  \begin{footnotesize}
    \begin{block}{定理3.4.2}
      若$\xx_1,~\xx_2$为齐次线性方程组$\A\xx=\zero$的两个解,则
      $$
      k_1\xx_1+k_2\xx_2\quad(k_1,~k_2\mbox{为任意常数})
      $$
      也是它的解。
    \end{block}
    \pause
    \proofname
    因为
    $$
    \A(\red{k_1\xx_1+k_2\xx_2}) = k_1\A\xx_1+k_2\A\xx_2 = k_1\zero+k_2\zero=\zero,
    $$
    故$k_1\xx_1+k_2\xx_2$也为$\A\xx=\zero$的解。
  \end{footnotesize}
\end{frame}

\begin{frame}
  \begin{footnotesize}
    \begin{block}{定义(基础解系)}
      设$\xx_1,~\xx_2,~\cd,~\xx_p$为$\A\xx=\zero$的解向量,若
      \begin{itemize}
      \item[(1)] $\xx_1,~\xx_2,~\cd,~\xx_p$线性无关
      \item[(2)] $\A\xx=\zero$的任一解向量可由$\xx_1,~\xx_2,~\cd,~\xx_p$线性表示。
      \end{itemize}
      则称$\xx_1,~\xx_2,~\cd,~\xx_p$为$\A\xx=\zero$的一个\blue{\underline{基础解系}}。
    \end{block}

    \pause
    \begin{block}{注}
      \begin{itemize}
      \item[(1)] 基础解系即全部解向量的\blue{极大无关组}。\\[0.1in] \pause 
      \item[(2)] 找到了基础解系,就找到了齐次线性方程组的全部解:
        $$
        k_1\xx_1+k_2\xx_2+\cd+k_p\xx_p \quad(k_1,k_2,\cd,k_p\mbox{为任意常数}).
        $$\pause 
      \item[(3)] 基础解系\blue{不唯一}。
      \end{itemize}
    \end{block}
  \end{footnotesize}
\end{frame}

\begin{frame}
  \begin{footnotesize}
    \begin{exampleblock}{例}
      求方程组
      $$
      x+y+z=0
      $$
      的全部解。
    \end{exampleblock}
    \pause
    \jiename
    \begin{itemize}
    \item[(1)] 选取$y,z$为自由未知量,则
      $$
      \left\{
      \begin{array}{cccccc}
        x&=&-&y&-&z\\
        y&=&&y&&\\
        z&=&&&&z
      \end{array}
      \right.
      $$
      则方程组的全部解为
      $$
      \left(
      \begin{array}{r}
        x\\y\\z
      \end{array}
      \right) = c_1      \left(
      \begin{array}{r}
        -1\\1\\0
      \end{array}
      \right) + c_2      \left(
      \begin{array}{r}
        -1\\0\\1
      \end{array}
      \right) \quad (c_1,c_2\mbox{为任意常数})
      $$
    \end{itemize}
  \end{footnotesize}
\end{frame}

\begin{frame}
  \begin{footnotesize}
    \begin{itemize}
    \item[(2)] 选取$x,z$为自由未知量,则
      $$
      \left\{
      \begin{array}{cccccc}
        x&=&&x&&\\
        y&=&-&x&-&z\\
        z&=&&&&z
      \end{array}
      \right.
      $$
      则方程组的全部解为
      $$
      \left(
      \begin{array}{r}
        x\\y\\z
      \end{array}
      \right) = c_1      \left(
      \begin{array}{r}
        1\\-1\\0
      \end{array}
      \right) + c_2      \left(
      \begin{array}{r}
        0\\-1\\1
      \end{array}
      \right) \quad (c_1,c_2\mbox{为任意常数})
      $$ \pause 
    \item[(3)] 选取$x,y$为自由未知量,则
      $$
      \left\{
      \begin{array}{cccccc}
        x&=&&x&&\\
        y&=&&&&y\\
        z&=&-&x&-&y
      \end{array}
      \right.
      $$
      则方程组的全部解为
      $$
      \left(
      \begin{array}{r}
        x\\y\\z
      \end{array}
      \right) = c_1      \left(
      \begin{array}{r}
        1\\0\\-1
      \end{array}
      \right) + c_2      \left(
      \begin{array}{r}
        0\\1\\-1
      \end{array}
      \right) \quad (c_1,c_2\mbox{为任意常数})
      $$
    \end{itemize}
  \end{footnotesize}
\end{frame}


\begin{frame}
  \begin{footnotesize}
    三个不同的基础解系为
    $$
    \left\{
    \left(
    \begin{array}{r}
      -1\\1\\0
    \end{array}
    \right),~~
    \left(
    \begin{array}{r}
      -1\\0\\1
    \end{array}
    \right)
    \right\},
    $$
    $$
    \left\{
    \left(
    \begin{array}{r}
      1\\-1\\0
    \end{array}
    \right),~~
    \left(
    \begin{array}{r}
      0\\-1\\1
    \end{array}
    \right)
    \right\},
    $$
    $$
    \left\{
    \left(
    \begin{array}{r}
      1\\0\\-1
    \end{array}
    \right),~~
    \left(
    \begin{array}{r}
      0\\1\\-1
    \end{array}
    \right)
    \right\}.
    $$
  \end{footnotesize}
\end{frame}

\begin{frame}
  \begin{footnotesize}
    \begin{block}{定理3.4.3}
      设$\A$为$m\times n$矩阵,若$\RR(\A)=r<n$,则齐次线性方程组$\A\xx=\zero$存在基础解系,
      且基础解系含$n-r$个解向量。
    \end{block}
    \pause 
    \begin{block}{注}
      \begin{itemize}
      \item $r$为$\A$的秩,也是$\A$的行阶梯形矩阵的非零行行数,是非自由未知量的个数。\pause 
      \item $n$为未知量的个数,故$n-r$为自由未知量的个数。\pause 有多少自由未知量,基础解系里就对应有多少个向量。
      \end{itemize}
    \end{block}
  \end{footnotesize}
\end{frame}

\begin{frame}
  \begin{footnotesize}
    \begin{exampleblock}{例2}
      求齐次线性方程组$\A\xx=\zero$的基础解系,其中
      $$
      \A = \left(
      \begin{array}{rrrr}
        1&-8&10&2\\
        2&4&5&-1\\
        3&8&6&-2
      \end{array}
      \right).
      $$
    \end{exampleblock}
    \pause
    \jiename
    $$
    \begin{array}{l}
      \left(
      \begin{array}{rrrr}
        1&-8&10&2\\
        2&4&5&-1\\
        3&8&6&-2
      \end{array}
      \right) \xlongrightarrow[r_3-3r_1]{r_2-2r_1}
      \left(
      \begin{array}{rrrr}
        1&-8&10&2\\
        0&20&-15&-5\\
        0&32&24&-8
      \end{array}
      \right)\\[0.3in]
      \xlongrightarrow[r_2\div4]{r_3\div8}
      \left(
      \begin{array}{rrrr}
        1&-8&10&2\\
        0&4&-3&-1\\
        0&4&-3&-1
      \end{array}
      \right) \xlongrightarrow[r_1+2r_2]{r_3-r_2}
      \left(
      \begin{array}{rrrr}
        1&0&4&0\\
        0&4&-3&-1\\
        0&0&0&0
      \end{array}
      \right) \\[0.3in]
      \xlongrightarrow[]{r_2\div4}
      \left(
      \begin{array}{rrrr}
        1&0&4&0\\
        0&1&-3/4&-1/4\\
        0&0&0&0
      \end{array}
      \right)
    \end{array}
    $$
  \end{footnotesize}
\end{frame}

\begin{frame}
  \begin{footnotesize}
    原方程等价于
    $$\left\{
    \begin{array}{rcrcrc}
      x_1&=&-4&x_3&&\\[0.1in]
      x_2&=&\frac34&x_3&+\frac14&x_4
    \end{array}
    \right. \pause \Leftrightarrow
    \left\{
    \begin{array}{rcrcrc}
      x_1&=&-4&x_3&&\\[0.1in]
      x_2&=&\frac34&x_3&+\frac14&x_4\\[0.1in]
      x_3&=&&x_3&&\\[0.1in]
      x_4&=&&&&x_4      
    \end{array}
    \right.
    $$
    \pause
    基础解系为
    $$
    \xibd_1 = \left(
    \begin{array}{r}
      -4\\[0.1in]
      \frac34\\[0.1in]
      1\\[0.1in]
      0
    \end{array}
    \right), \quad \xibd_2 = \left(
    \begin{array}{r}
      0\\[0.1in]
      \frac14\\[0.1in]
      0\\[0.1in]
      1
    \end{array}
    \right)
    $$
  \end{footnotesize}
\end{frame}

\begin{frame}
  \begin{footnotesize}
    \begin{exampleblock}{例3}
      求齐次线性方程组
      $$
      nx_1+(n-1)x_2+\cd+2x_{n-1}+x_n=0
      $$
      的基础解系。      
    \end{exampleblock}
    \pause
    \jiename
    原方程等价于$x_n=-nx_1-(n-1)x_2-\cd-2x_{n-1}$,\pause 即
    $$
    \left\{
    \begin{array}{rcrrrr}
      x_1&=&x_1&&&\\
      x_2&=&&x_2&&\\
      &\vd&&&&\\
      x_{n-1}&=&&&&x_{n-1}\\      
      x_n&=&-nx_1&-(n-1)x_2&\cd&-2x_{n-1}
    \end{array}    
    \right.
    $$
    \pause
    基础解系为
    $$
    (\xibd_1,\xibd_2,\cd,\xibd_{n-1})=\left(
    \begin{array}{rrrr}
      1&0&\cd&0\\
      0&1&\cd&0\\
      \vd&\vd&&\vd\\
      0&0&\cd&1\\
      -n&-n+1&\cd&-2
    \end{array}
    \right)
    $$
  \end{footnotesize}
\end{frame}

\begin{frame}
  \begin{footnotesize}
    \begin{exampleblock}{例3}
      设$\A$与$\B$分别为$m\times n$和$n\times s$矩阵,且$\A\B=\zero$。证明:
      $$
      \RR(\A)+\RR(\B)\le n.
      $$
    \end{exampleblock}
    \pause
    \proofname
    由$\A\B=\zero$知,$\B$的列向量是$\A\xx=\zero$的解。\pause
    故$\B$的列向量组的秩,不超过$\A\xx=\zero$的基础解系的秩,即
    $$
    \RR(\B) \le n-\RR(\A),
    $$
    即
    $$
    \RR(\A)+\RR(\B)\le n.
    $$
  \end{footnotesize}
\end{frame}

\begin{frame}
  \begin{footnotesize}
    \begin{exampleblock}{例}
      设$n$元齐次线性方程组$\A\xx=\zero$与$\B\xx=\zero$同解,证明
      $$
      \RR(\A)=\RR(\B).
      $$
    \end{exampleblock}
    \pause
    \jiename
    $\A\xx=\zero$与$\B\xx=\zero$同解,故它们有相同的基础解系,而基础解系包含的向量个数相等,即
    $$
    n-\RR(\A)=n-\RR(\B),
    $$
    从而
    $$
    \RR(\A)=\RR(\B).
    $$
  \end{footnotesize}
\end{frame}

\begin{frame}
  \begin{footnotesize}
    \begin{exampleblock}{例4}
      设$\A$为$m\times n$实矩阵,证明$\RR(\A^T\A)=\RR(\A)$。    
    \end{exampleblock}
    \pause
    \jiename
    只需证明$\A\xx=\zero$与$(\A^T\A)\xx=\zero$同解。
    \pause \vspace{0.1in}

    \begin{itemize}
    \item[(1)] 若$\xx$满足$\A\xx=\zero$,则有$(\A^T\A)\xx=\A^T(\A\xx)=\zero$。\pause 
    \item[(2)] 若$\xx$满足$\A^T\A\xx=\zero$,则
      $$
      \xx^T\A^T\A\xx=\zero,
      $$
      即
      $$
      (\A\xx)^T\A\xx=\zero,
      $$
      故$\A\xx=\zero$。
    \end{itemize}
    
  \end{footnotesize}
\end{frame}


