%%%%%%%%%%%%%%5
\section{行列式按行(列)展开}
\begin{frame}
  \begin{itemize}
  \item 目的:\\[0.2cm]
    将高阶行列式转换为低阶行列式,以简化行列式的计算
  \end{itemize}
\end{frame}

\begin{frame}
  \uncover<1->{
    \begin{block}{余子式与代数余子式}
      在$n$阶行列式中,把$a_{ij}$所在的第$i$行与第$j$列划去后,留下来的$n-1$阶行列式叫做$a_{ij}$的余子式,记为$M_{ij}$;
      记
      $$
      A_{ij} = (-1)^{i+j} M_{ij}
      $$
      为$a_{ij}$的代数余子式。
    \end{block}
  }
  \uncover<2->{
    \vspace{0.2cm}
    如
    \begin{center}
      \begin{tikzpicture}
        \uncover<3->{
          \matrix (D) [matrix of math nodes]  { 
            D_4 = \\
          };
          \matrix(A) [right=2pt of D, matrix of math nodes,nodes in empty cells,
            ampersand replacement=\&,left delimiter=|,right delimiter=|] {
            a_{11} \& a_{12} \& a_{13}  \& a_{14} \\
            a_{21} \& a_{22} \& a_{23}  \& a_{24} \\
            a_{31} \& a_{32} \& a_{33}  \& a_{34} \\
            a_{41} \& a_{42} \& a_{43}  \& a_{44} \\
          };
        }
        \uncover<4->{
          \draw[red, thick] (A-3-1.west) -- (A-3-4.east) (A-1-2.north) -- (A-4-2.south);
        }
      \end{tikzpicture}
      \uncover<5->{
        \begin{tikzpicture}
          \matrix (M) [matrix of math nodes]  { 
            M_{32} = \\
          };
          \matrix(MM) [right=2pt of M, matrix of math nodes,nodes in empty cells,
            ampersand replacement=\&,left delimiter=|,right delimiter=|] {
            a_{11} \& a_{13}  \& a_{14} \\
            a_{21} \& a_{23}  \& a_{24} \\
            a_{41} \& a_{43}  \& a_{44} \\
          };
        \end{tikzpicture}
      }
      \uncover<6->{
        \begin{tikzpicture}
          \matrix (A) [matrix of math nodes]  { 
            A_{32} = (-1)^{3+2} M_{32} = - M_{32} \\
          };
        \end{tikzpicture}
      }
    \end{center}
  }
\end{frame}


\begin{frame}
  \begin{footnotesize}
    \uncover<1->{
      \begin{block}{引例}
      一个$n$阶行列式,如果第$i$行所有元素除$(i,j)$元$a_{ij}$外都为0,
      那么这个行列式等于$a_{ij}$与它的代数余子式的乘积,即
      $$
      D = a_{ij} A_{ij}.
      $$
    \end{block}
    }
    \uncover<2->{
    \proofname
    先证$(i,j)=(1,1)$的情形,此时
    $$
    D = 
    \left|
    \begin{array}{cccc}
      a_{11} & 0      & \cd & 0 \\
      a_{21} & a_{22} & \cd & 0 \\
      \vd   & \vd    &     & \vd \\
      a_{n1} & a_{n2} & \cd & a_{nn} 
    \end{array}
    \right|
    $$
    由前面结论可知,
    $$
    D = a_{11} M_{11},
    $$
    又$A_{11} = (-1)^{1+1} M_{11} = M_{11}$,
    从而 $D = a_{11} A_{11}$。
    }
  \end{footnotesize}
\end{frame}


\begin{frame}
  \begin{footnotesize}
    \proofname(续) \\[0.2cm]
    再证一般情形,此时
    $$
    D = 
    \left|
    \begin{array}{ccccc}
      a_{11} & \cd    &  a_{1j} &  \cd & a_{1n} \\
      \vd   &        &  \vd    &      & \vd \\
       0    & \cd    &  a_{ij} &  \cd & 0 \\
       \vd   &        &  \vd    &      & \vd \\ 
      a_{n1} &        &  a_{nj} & \cd & a_{nn} 
    \end{array}
    \right|
    $$
    \begin{itemize}
    \item 
      把$D$的第$i$行依次与第$i-1$行、第$i-2$行、$\ldots$、第$1$行对调,$a_{ij}$调至$(1,j)$元,
      调换次数为$i-1$。\\[0.2cm]
    \item
      再把第$j$列依次与第$j-1$列、第$j-2$列、$\ldots$、第$1$列对调,$a_{ij}$调至$(1,1)$元,
      调换次数为$j-1$。\\[0.2cm]
    \item
      总之,经$i+j-2$次调换,$a_{ij}$调至$(1,1)$元,所得行列式$D_1= (-1)^{i+j-2}D = (-1)^{i+j}D$,
      而$D_1$中$(1,1)$元的余子式就是$D$中$(i,j)$元的余子式$M_{ij}$。从而
      $$
      D = (-1)^{i+j}D_1 = (-1)^{i+j} a_{ij} M_{ij} = a_{ij} A_{ij}.
      $$
    \end{itemize}
  \end{footnotesize}
\end{frame}


\begin{frame}
  \begin{footnotesize}
    \begin{block}{定理}
      行列式等于它的任一行(列)的各元素与其对应的代数余子式乘积之和,即
      $$
      \red{D = a_{i1} A_{i1} + a_{i2} A_{i2} + \cdots + a_{in} A_{in}, \ \  i = 1, 2, \cdots, n,}
      $$
      或
      $$
      \red{D = a_{1j} A_{1j} + a_{2j} A_{2j} + \cdots + a_{nj} A_{nj}, \ \ j = 1, 2, \cdots, n.}
      $$
    \end{block}
    \proofname
    $$ 
    \begin{array}{ll}
      D & = 
      \left|
      \begin{array}{cccc}
        a_{11}     &  a_{12} &  \cd & a_{1n} \\
        \vd           &  \vd    &      & \vd \\
        a_{i1}+0+\cd+0  & 0+ a_{i2}+ \cd + 0 &  \cd & 0 + \cd + 0 + a_{in} \\
        \vd           &  \vd    &      & \vd \\ 
        a_{n1}         &  a_{n2} & \cd & a_{nn} 
      \end{array}
      \right| \\[0.5cm]
      & = 
      \left|
      \begin{array}{cccc}
        a_{11}  &  a_{12} &  \cd & a_{1n} \\
        \vd     &  \vd    &      & \vd \\
        a_{i1}  &  0  &  \cd & 0  \\
        \vd     &  \vd    &      & \vd \\ 
        a_{n1}  &  a_{n2} & \cd & a_{nn} 
      \end{array}
      \right| + 
      \left|
      \begin{array}{ccccc}
        a_{11}   &  a_{12} &  \cd & a_{1n} \\
        \vd      &  \vd    &      & \vd \\
        0        &  a_{i2}  &  \cd & 0  \\
        \vd      &  \vd    &      & \vd \\ 
        a_{n1}   &   a_{n2} & \cd & a_{nn} 
      \end{array}
      \right|
      +
      \left|
      \begin{array}{cccc}
        a_{11}  &  a_{12} &  \cd & a_{1n} \\
        \vd     &  \vd    &      & \vd \\
        0       &  0  &  \cd & a_{in}  \\
        \vd     &  \vd    &      & \vd \\ 
        a_{n1}  &  a_{n2} & \cd & a_{nn} 
      \end{array}
      \right|
    \end{array}
    $$
  \end{footnotesize}
\end{frame}


\begin{frame}
  \begin{footnotesize}
    \proofname(续)\\[0.2cm]
    由引理可知
    $$
    D = a_{i1} A_{i1} + a_{i2} A_{i2} + \cdots a_{in} A_{in}, \ \ i = 1, 2, \cdots, n.
    $$
    类似地,按列可证明
    $$
    D = a_{1j} A_{1j} + a_{2j} A_{2j} + \cdots a_{nj} A_{nj}, \ \ j = 1, 2, \cdots, n.
    $$
    \qed

    该定理称为\red{行列式按行(列)展开法则}。 \\[0.3cm]
    
    利用这一法则并结合行列式的性质,可以简化行列式的性质。
    
  \end{footnotesize}
\end{frame}


\begin{frame}
  \begin{footnotesize}
    \uncover<1->{
      \begin{exampleblock}{例}
        计算
        $$
        D = \left|
        \begin{array}{rrrr}
          3  & 1 & -1 & 2  \\
          -5 & 1 &  3 & -4 \\
          2 & 0 &  1 & -1 \\
          1 & -5 & 3 & -3 
        \end{array}
        \right|
        $$
      \end{exampleblock}
    }
    \vspace{0.3cm}
    \textbf{解}:
    $$
    \begin{array}{ll}
      D&  \xlongequal[c_4+c_3]{c_1-2c_3} \left|
      \begin{array}{rrrr}
        5  & 1 & -1 & 1  \\
        -11 & 1 &  3 & -1 \\
        0 & 0 &  1 & 0 \\
        -5 & -5 & 3 & 0 
      \end{array}
      \right|\\[0.8cm]
      & = (-1)^{3+3} \left|
      \begin{array}{rrr}
        5 & 1 & 1\\
        -11 & 1 & -1 \\
        -5 & -5 & 0
      \end{array}
      \right| 
      \xlongequal{r_2+r_1}
      \left|
      \begin{array}{rrr}
        5 & 1 & 1\\
        -6 & 2 & 0 \\
        -5 & -5 & 0
      \end{array}
      \right|\\[0.8cm]
      & = (-1)^{1+3} \left|
      \begin{array}{rr}
        -6 &  2\\
        -5 & -5
      \end{array}
      \right| = 40.
    \end{array}
    $$    
  \end{footnotesize}
\end{frame}


\begin{frame}
  \begin{footnotesize}
    \uncover<1->{
      \begin{exampleblock}{例}
        计算行列式
        $$
        D_{20} = \left|
        \begin{array}{rrrrrrr}
          1   & 2    & 3    & \cd  & 18    & 19    &  20 \\ 
          2   & 1    & 2    & \cd  & 17    & 18    &  19 \\
          3   & 2    & 1    & \cd  & 16    & 17    &  18 \\
          \vd & \vd  & \vd  & \cd  & \vd   & \vd   &  \vd \\
          20  & 19   & 18   & \cd  & 3     & 2     &  1
        \end{array}
        \right|
        $$
      \end{exampleblock}
    }
    \uncover<2->{
      \vspace{0.3cm}
      \textbf{解:}
      $$
      \begin{array}{ll}
        D_{20} &  \xlongequal[i=19,\cd,1]{c_{i+1}-c_i} \left|
        \begin{array}{rrrrrrr}
          1   & 1    & 1    & \cd  & 1    & 1    &  1 \\ 
          2   & -1   & 1    & \cd  & 1    & 1    &  1 \\
          3   & -1   & -1   & \cd  & 1    & 1    &  1 \\
          \vd & \vd  & \vd  & \cd  & \vd  & \vd  &  \vd \\
         % 19  & -1   & -1   & \cd  & -1   & -1   &  1 \\
          20  & -1   & -1   & \cd  & -1   & -1   &  -1
        \end{array}
        \right| \\[0.5cm]
        &\xlongequal[i=2,\cd,20]{r_i+r_1} \left|
        \begin{array}{rrrrrrr}
          1   & 1    & 1    & \cd  & 1    & 1    &  1 \\ 
          3   & 0    & 2    & \cd  & 2    & 2    &  2 \\
          4   & 0    & 0    & \cd  & 2    & 2    &  2 \\
          \vd & \vd  & \vd  & \cd  & \vd  & \vd  &  \vd \\
         % 20  & 0    & 0    & \cd  & 0    & 0    &  2\\
          21  & 0    & 0    & \cd  & 0    & 0    &  0
        \end{array}
        \right|
        = 21 \times (-1)^{20+1} \times 2^{18} = -21 \times 2^{18}.
      \end{array}
      $$
    }
    
  \end{footnotesize}
\end{frame}


\begin{frame}
  \begin{footnotesize}
    \uncover<1->{
      \begin{exampleblock}{例 }
        计算元素为$a_{ij}=|i-j|$的$n$阶行列式
      \end{exampleblock}
    }
    \uncover<2->{
      \vspace{0.3cm}
      \textbf{解:}
      $$
      \begin{array}{ll}
        D_{n} & = \left|
        \begin{array}{rrrrrrr}
          0   & 1   & 2    & \cd & n-2  & n-1 \\ 
          1   & 0   & 1    & \cd & n-3  & n-2 \\
          %% 2   & 1   & 0    & \cd & n-4  & n-3 \\
          \vd & \vd & \vd  &     & \vd  & \vd \\
          n-2 & n-3 & n-4  & \cd & 0     & n-2 \\
          n-1 & n-2 & n-3  & \cd & n-4  & 0 
        \end{array}
        \right| \\[0.5cm]
        &\xlongequal[i=n-1,\cd,1]{c_{i+1}-c_i}
        \left|
        \begin{array}{rrrrrrr}
          0   & 1   & 1    & \cd & 1  & 1 \\ 
          1   & -1  & 1    & \cd & 1  & 1 \\
          %% 2   & -1  & -1   & \cd & 1  & 1 \\
          \vd & \vd & \vd  &     & \vd & \vd \\
          n-2 & -1  & -1   & \cd & -1 & 1 \\
          n-1 & -1  & -1   & \cd & -1 & -1 
        \end{array}
        \right| \\[0.5cm]
        & \xlongequal[i=2,\cd,n]{r_{i}+r_1}
        \left|
        \begin{array}{rrrrrrr}
          0   & 1   & 1   & \cd & 1   & 1   \\ 
          1   & 0   & 2   & \cd & 2   & 2   \\
          \vd & \vd & \vd &     & \vd & \vd \\
          n-2 & 0   & 0  & \cd  & 0   & 2 \\
          n-1 & 0   & 0  & \cd  & 0   & 0 
        \end{array}
        \right| = (-1)^{n-1}(n-1)2^{n-2}.
      \end{array}
      $$
    }
    
  \end{footnotesize}
\end{frame}



\begin{frame}
  \begin{footnotesize}
    \uncover<1->{
      \begin{exampleblock}{范德蒙德行列式}
        证明范德蒙德(Vandermonde)行列式
        $$
        D_n = \left|
        \begin{array}{cccc}
          1        &  1        & \cd &    1     \\                    
          x_1      &  x_2      & \cd &    x_n    \\ 
          x_1^2    &  x_2^2     & \cd &   x_n^2   \\ 
          \vd      &  \vd      &     &    \vd      \\
          x_1^{n-1} & x_2^{n-1} &  \cd &  x_n^{n-1}
        \end{array}
        \right|
        = \prod_{n \ge i \ge j \ge 1}(x_i-x_j).
        $$
      \end{exampleblock}
    }
    \uncover<2->{
      \proofname
      用数学归纳法证明。当$n=2$时,
      $$
      D_2 = \left|
      \begin{array}{cc}
        1 & 1 \\
        x_1 & x_2
      \end{array}
      \right|
      = x_2 - x_1 = \prod_{2 \ge i > j \ge 1} (x_i - x_j),
      $$
      结论成立。
    }

  \end{footnotesize}
\end{frame}

\begin{frame}
  \begin{footnotesize}
    \proofname(续) \\[0.2cm]
    现假设结论对$n-1$阶范德蒙德行列式成立,以下证明结论对$n$阶范德蒙德行列式也成立。
    $$
    D_n \xlongequal[i=n,\cdots, 2]{\red{r_i - x_1 r_{i-1}}}\left|
    \begin{array}{ccccc}  
      1     & 1                    & 1                       & \cd   & 1    \\
      0     & x_2 - x_1            & x_3 - x_1               &  \cd  & x_n - x_1 \\
      0     & x_2(x_2 - x_1)       & x_3(x_3 - x_1)          &  \cd  & x_n(x_n - x_1)\\
      \vd   & \vd                  & \vd                     &      & \vd   \\
      0     & x_2^{n-2}(x_2-x_1)    & x_3^{n-2}(x_3 - x_1)    &  \cd  & x_n^{n-2}(x_n - x_1) 
    \end{array}\right|
    $$
    按第1列展开,并把每列的公因子$(x_i-x_1)$提出,就有
    $$
    D_n = (x_2-x_1)(x_3-x_1)\cdots(x_n-x_1)\left|
    \begin{array}{cccc}  
      1            & 1          &  \cd  & 1 \\
      x_2          & x_3         &  \cd  & x_n\\
      \vd          & \vd         &      & \vd   \\
      x_2^{n-2}     & x_3^{n-2}    &  \cd  & x_n^{n-2}
    \end{array}\right|
    $$
    上式右端的行列式为$n-1$阶范德蒙德行列式,按归纳法假设,
    它等于所有$(x_i-x_j)$因子的乘积($n\ge i \ge j \ge 2$)。故
    $$
    D_n = (x_2-x_1)(x_3-x_1)\cdots(x_n-x_1) \prod_{n\ge i \ge j \ge 2}(x_i - x_j)
    = \prod_{n\ge i \ge j \ge 1}(x_i - x_j).
    $$
  \end{footnotesize}
\end{frame}

\begin{frame}
  \begin{footnotesize}
    \uncover<1->{
      \begin{exampleblock}{例}
        设$a,b,c$为互不相同的实数,证明:
        $$
        \left|
        \begin{array}{ccc}
          1   &   1   &   1\\
          a   &   b   &   c\\
          a^3 &   b^3 &   c^3
        \end{array}
        \right|=0
        $$
        的充要条件是$a+b+c=0$.
      \end{exampleblock}
    }
    \uncover<2->{
      \vspace{0.3cm}
      \textbf{解:}考察范德蒙德行列式
      $$
      \begin{array}{ll}
        D & = \left|
        \begin{array}{cccc}
          1   &   1   &   1   & 1\\
          a   &   b   &   c   & y\\
          a^2 &   b^2 &   c^2 & y^2\\
          a^3 &   b^3 &   c^3 & y^3\\
        \end{array}
        \right|
        = (a-b)(a-c)(b-c)(a-y)(b-y)(c-y) \\[0.8cm]
        & = -(a-b)(a-c)(b-c)(a+b+c)y^2 + \cd 
      \end{array}
      $$
      注意到行列式$
      \left|
      \begin{array}{cccc}
        1   &   1   &   1  \\ 
        a   &   b   &   c  \\
        a^3 &   b^3 &   c^3\\
      \end{array}
      \right|
      $为$y^2$的系数。
    }      
  \end{footnotesize}
\end{frame}

\begin{frame}
  \begin{footnotesize}
    \textbf{解:}(续)
    于是
    $$
    \left|
    \begin{array}{cccc}
      1   &   1   &   1  \\ 
      a   &   b   &   c  \\
      a^3 &   b^3 &   c^3\\
    \end{array}
    \right|
    = -(a-b)(a-c)(b-c)(a+b+c)
    $$
    于是结论得证。
  \end{footnotesize}
\end{frame}



\begin{frame}
  \begin{footnotesize}
    \uncover<1->{
      \begin{block}{推论}
        行列式某一行(列)的元素与另一行(列)的对应元素的代数余子式乘积之和等于0,即
        $$
        a_{i1} A_{j1} + a_{i2} A_{j2} + \cdots + a_{in} A_{jn} = 0,  \ \ i \ne j,
        $$
        或
        $$
        a_{1i} A_{1j} + a_{2i} A_{2j} + \cdots + a_{ni} A_{nj} = 0,  \ \ i \ne j.
        $$
      \end{block}
    }
    \uncover<2->{
      \proofname
      把行列式$D=det(a_{ij})$按第$j$行展开,有
      $$
      a_{j1} A_{j1} + a_{j2} A_{j2} + \cd + a_{jn} A_{jn} = 
      \left|
      \begin{array}{ccc}
        a_{11} & \cd & a_{1n} \\
        \vd   &     & \vd \\
        a_{i1} & \cd & a_{in} \\
        \vd   &     & \vd \\
        a_{j1} & \cd & a_{jn} \\
        \vd   &     & \vd \\
        a_{n1} & \cd & a_{nn}
      \end{array}
      \right|
      $$
    }

  \end{footnotesize}
\end{frame}



\begin{frame}
  \begin{footnotesize}
    \proofname(续)
    在上式中把$a_{jk}$换成$a_{ik}(k=1,\cdots,n)$得
    \begin{center}
      \begin{tikzpicture}
        \matrix (M) [matrix of math nodes]  { 
          a_{i1} A_{j1} + a_{i2} A_{j2} + \cdots + a_{in} A_{jn} = \\
        };
        \matrix(MM) [right=2pt of M, matrix of math nodes,nodes in empty cells,
          ampersand replacement=\&,left delimiter=|,right delimiter=|] {
          a_{11} \& \cd  \& a_{1n} \\
          \vd   \&     \& \vd \\       
          a_{i1} \& \cd  \& a_{in} \\
          \vd   \&     \& \vd \\
          a_{j1} \& \cd  \& a_{jn} \\
          \vd   \&     \& \vd \\       
          a_{n1} \& \cd  \& a_{nn} \\
        };
        \node[right=7pt  of MM-3-3, blue]  {第$i$行};
        \node[right=7pt  of MM-5-3, blue]  {第$j$行}; 
      \end{tikzpicture}
    \end{center}
    当$i\ne j$时,上式右端行列式中有两行对应元素相同,故行列式等于0,即得
    $$
    a_{i1} A_{j1} + a_{i2} A_{j2} + \cdots + a_{in} A_{jn} = 0, \ \ i\ne j.
    $$
    同理,按列可证
    $$
    a_{1i} A_{1j} + a_{2i} A_{2j} + \cdots + a_{ni} A_{nj} = 0, \ \ i\ne j.
    $$

  \end{footnotesize}
\end{frame}


\begin{frame}
  \begin{footnotesize}
    \begin{block}{代数余子式的性质}
      $$
      \sum_{k=1}^n a_{ki} A_{kj} = D \delta_{ij} = 
      \left \{
      \begin{array}{ll}
        D, & \mbox{当}i=j, \\[0.3cm]
        0, & \mbox{当}i\ne j;
      \end{array}
      \right.
      $$
      或
      $$
      \sum_{k=1}^n a_{ik} A_{jk} 
      = D \delta_{ij} = 
      \left \{
      \begin{array}{ll}
        D, & \mbox{当}i=j, \\[0.3cm]
        0, & \mbox{当}i\ne j;
      \end{array}
      \right.
      $$
      其中
      $$
      \delta_{ij} = \left \{
      \begin{array}{ll}
        1, & \mbox{当}i=j, \\[0.3cm]
        0, & \mbox{当}i\ne j.
      \end{array}
      \right.
      $$
    \end{block}
  \end{footnotesize}
\end{frame}


\begin{frame}
  \begin{footnotesize}
    \begin{block}{结论}
      $$
      \left|
      \begin{array}{ccc}
        a_{11} & \cd & a_{1n} \\
        \vd   &     & \vd \\
        a_{i-1,1} & \cd & a_{i-1,n} \\
        b_1   &  \cd   & b_n \\
        a_{i+1,1} & \cd & a_{i+1,n} \\
        \vd   &     & \vd \\
        a_{n1} & \cd & a_{nn}
      \end{array}
      \right|
      = b_1A_{i1} + b_2 A_{i2} + \cd + b_n A_{in} 
      $$
      及
      $$
      \left|
      \begin{array}{ccccccc}
        a_{11} & \cd & a_{1,j-1} & b_1 & a_{1,j+1} & \cd & a_{1n} \\
        \vd   &      & \vd      & \vd & \vd      &     & \vd \\
        a_{n1} & \cd & a_{n,j-1} & b_n & a_{n,j+1} & \cd & a_{nn} 
      \end{array}
      \right|
      = b_1A_{1j} + b_2 A_{2j} + \cd + b_n A_{nj}. 
      $$

    \end{block}
  \end{footnotesize}
\end{frame}


\begin{frame}
  \begin{footnotesize}
    \uncover<1->{
      \begin{exampleblock}{例}
        设
        $$
        D = \left|
        \begin{array}{rrrr}
          3    & -5  &  2 &  1 \\    
          1    &  1  &  0 & -5 \\
          -1   &  3  &  1 &  3 \\
          2    & -4  & -1 & -3
        \end{array}
        \right|
        $$
        求$A_{11}+A_{12}+A_{13}+A_{14}$及$M_{11}+M_{21}+M_{31}+M_{41}$
      \end{exampleblock}
    }
    \uncover<2->{
      \vspace{0.3cm}
      \textbf{解}:
      $$
      \begin{array}{ll}
        A_{11}+A_{12}+A_{13}+A_{14} & = 
        \left|
        \begin{array}{rrrr}
          \red{1} &  \red{1} & \red{1}   &  \red{1}\\
          1    &  1 & 0   &  -5\\
          -1   &  3 & 1   &  3\\
          2    & -4 & -1  & -3
        \end{array}
        \right| \xlongequal[r_3-r_1]{r_4+r_3}
        \left|
        \begin{array}{rrrr}
          1    &  1 & 1   &  1\\
          1    &  1 & 0   &  -5\\
          -2   &  2 & 0   &  2\\
          1    & -1 & 0   &  0
        \end{array}
        \right| \\[0.8cm]
        & = 
        \left|
        \begin{array}{rrrr}
          1    &  1 &  -5\\
          -2   &  2 &  2\\
          1    & -1 &  0
        \end{array}
        \right|
        \xlongequal{c_2+c_1}
        \left|
        \begin{array}{rrrr}
          1    &  2 &  -5\\
          -2   &  0 &  2\\
          1    & 0 &  0
        \end{array}
        \right|\\[0.8cm]
        & = \left|
        \begin{array}{rrrr}
           2 &  -5\\
           0 &  2\\
        \end{array}
        \right| = 4.
      \end{array}
      $$
    }
    
  \end{footnotesize}
\end{frame}


\begin{frame}
  \begin{footnotesize}
    \textbf{解:}(续)\\[0.2cm]
    $$
    \begin{array}{ll}
      M_{11}+M_{21}+M_{31}+M_{41} &= A_{11}-A_{21}+A_{31}-A_{41} \\[0.3cm]
      & = 
      \left|
      \begin{array}{rrrr}
        \red{1}    & -5 & 2   &   1\\
        \red{-1}   &  1 & 0   &  -5\\
        \red{1}    &  3 & 1   &  3\\
        \red{-1}   & -4 & -1  & -3
      \end{array}
      \right| \xlongequal{r_4+r_3}
      \left|
      \begin{array}{rrrr}
        1    & -5 & 2   &   1\\
        -1   &  1 & 0   &  -5\\
        1    &  3 & 1   &  3\\
        0    & -1 & 0   &  0
      \end{array}
      \right| \\[0.8cm]
      & = -
      \left|
      \begin{array}{rrrr}
        1    &  2 &  1\\
        -1   &  0 &  -5\\
        1    &  1 &  3
      \end{array}
      \right|
      \xlongequal{r_1-2r_3}
      \left|
      \begin{array}{rrrr}
        -1   &  0 &  -5\\
        -1   &  0 &  -5\\
        1    &  1 &  3
      \end{array}
      \right| = 0.
    \end{array}
    $$

    
  \end{footnotesize}
\end{frame}
