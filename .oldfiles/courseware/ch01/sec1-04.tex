%%%%%%
\section{对换}

\begin{frame}
  \begin{overprint}
    \onslide<1->
    \begin{center}
      \blue{对换} \vspace{0.5cm}
    \end{center}
  \end{overprint}
  

  \begin{overprint}
    \onslide<1>
    \begin{block}{对换的定义}
      在某个$n$阶排列中,任意对调两个元素的位置(如对调$p_s$与$p_t$的位置),其余元素不动,称为该排列的一个对换。
      可记为
      $$
      p_1 \cd p_s \cd p_t \cd p_n 
      \xlongrightarrow{(p_s, p_t)}
      p_1 \cd p_t \cd p_s \cd p_n 
      $$
    \end{block}

    
    %%%%%%%%%%
    \onslide<2>
    如下定理给出对换与排列奇偶性的关系: \\

    \begin{block}{定理}
      对换改变排列的奇偶性。
    \end{block}
    \vspace{0.3cm}
    如
    $$
    \underbrace{123}_{\mbox{奇}} \xlongrightarrow{(12)} \underbrace{213}_{\mbox{偶}}
    $$


    %%%%%
    \onslide<3>
    \proofname
    \begin{itemize}
    \item 相邻元素的对换\\[0.2cm]
      $$
      a_1 \cd a_l \red{a}  \blue{b} b_l \cd b_m
      \xlongrightarrow{(\red{a}, \blue{b})}
      \underbrace{a_1 \cd a_l}_{\mbox{不变}} \blue{b} \red{a} \underbrace{\blue{b_l \cd b_m}}_{\mbox{不变}}
      $$
      若$a<b$,经对换后,$b$的逆序数不变,$a$的逆序数加1 \\[0.2cm]
      若$a>b$,经对换后,$a$的逆序数不变,$b$的逆序数减1 \\[0.2cm]
      故两个排列奇偶性不同。
      
    \end{itemize}

    %%%%%
    \onslide<4>
    \proofname[续]
    \begin{itemize}
    \item 任意元素的对换\\[0.2cm]
      $$
      a_1 \cd a_l \red{a}  b_l \cd b_m \blue{b} c_1 \cdots c_n
      \xlongrightarrow{(a,b)}
      a_1 \cd a_l \blue{b}  b_l \cd b_m \red{a} c_1 \cdots c_n
      $$
      先作$m+1$次相邻元素对换:
      $$
      a_1 \cd a_l \red{a}   b_l \cd b_m \blue{b} c_1 \cdots c_n
      \xlongrightarrow{(a,b)}
      a_1 \cd a_l   b_l \cd b_m \blue{b} \red{a}  c_1 \cdots c_n
      $$
      再作$m+1$次相邻元素对换:
      $$
      a_1 \cd a_l   b_l \cd b_m \blue{b} \red{a}  c_1 \cdots c_n
      \xlongrightarrow{(a,b)}
      a_1 \cd a_l \blue{b}  b_l \cd b_m \red{a}  c_1 \cdots c_n
      $$
      共作$2m+1$次相邻元素对换,由前面结论可知,奇偶性改变。
    \end{itemize}


    %%%%%
    \onslide<5>
    \begin{block}{推论}
      奇排列变成自然排列的对换次数为奇数,偶排列变成自然排列的对换次数为偶数。
    \end{block}
    \vspace{0.3cm}
    如,把$32415$对换成自然排列
    $$
    32415 \xlongrightarrow{(1,3)} 12435 \xlongrightarrow{(3,4)} 12345
    $$
    说明$32415$是一个偶排列,但不一定成立$\tau(32415)=2$,事实上$\tau(32415)=4$。
    

    %%%%%
    \onslide<6>
    \begin{block}{推论}
      $n$阶行列式也可定义为
      $$
      D = \left|
      \begin{array}{ccccc}
        a_{11} & a_{12} & \cd & a_{1n} \\[0.2cm]
        a_{21} & a_{22} & \cd & a_{2n} \\[0.2cm]
        \vd   &  \vd  &     & \vd   \\[0.2cm]
        a_{n1} & a_{n2} & \cd & a_{nn} 
      \end{array}
      \right|
      = \sum_{n!} (-1)^{\tau} a_{p_1q_1}a_{p_2q_2}\cd a_{p_nq_n},
      $$      
      其中$p_1p_2\cdots p_n$、$q_1q_2\cdots q_n$为两个$n$阶排列,且
      $$
      \tau = \tau(p_1p_2\cdots p_n) + \tau(p_1p_2\cdots p_n).
      $$
    \end{block}


    %%%%%
    \onslide<7>
    \begin{block}{定理2}
      $n$阶行列式也可定义为
      $$
      D = \sum_{n!} (-1)^{ \tau(p_1p_2\cdots p_n) } a_{p_11}a_{p_22}\cd a_{p_nn}.
      $$      
    \end{block}
    
  \end{overprint}
\end{frame}

\begin{frame}
  \uncover<1->{
    \begin{exampleblock}{例9}
      试判断$a_{14}a_{23}a_{31}a_{42}a_{56}a_{65}$和$-a_{32}a_{43}a_{14}a_{51}a_{25}a_{66}$
      是否为6阶行列式中的项。
    \end{exampleblock}
  }
  \uncover<2->{
  \begin{itemize}
  \item 
    $\tau(431265) = 0 + 1 + 2 + 2 + 0 + 1 = 6$ \\[0.4cm]
  \item
    $\tau(341526) + \tau(234156) = (0+0+2+0+3+0) + (0+0+0+3+0+0) = 8$
  \end{itemize}
  }
\end{frame}


\begin{frame}
  
  \blue{小结}
  \begin{enumerate}
    \item 
      一个排列中的任意两个元素对换,都会改变排列的奇偶性。\\[0.4cm]
    \item
      行列式的三种表示方法
      $$
      \begin{array}{rl}
        D = & \disp \sum_{n!} (-1)^{ \tau(p_1p_2\cdots p_n) } a_{1p_1}a_{2p_2}\cd a_{np_n} \\[0.5cm]
        D = & \disp \sum_{n!} (-1)^{ \tau(p_1p_2\cdots p_n) } a_{p_11}a_{p_22}\cd a_{p_nn} \\[0.5cm]
        D = & \disp \sum_{n!} (-1)^{ \tau(p_1p_2\cdots p_n) + \tau(q_1q_2\cdots q_n) } a_{p_1q_1}a_{p_2q_2}\cd a_{p_nq_n} 
      \end{array}
      $$
      其中$p_1p_2\cdots p_n$、$q_1q_2\cdots q_n$为两个$n$阶排列
    \end{enumerate}
\end{frame}
