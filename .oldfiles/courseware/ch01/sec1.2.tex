\section{行列式的性质}
\begin{frame}
  \begin{block}{性质1}
    互换行列式的行与列,值不变,即
    \begin{equation}
      \left|
      \begin{array}{cccc}
        a_{11}  &  a_{12} & \cdots & a_{1n} \\
        a_{21}  &  a_{22} & \cdots & a_{2n} \\
        \vdots & \vdots & \ddots & \vdots\\  
        a_{n1}  &  a_{n2} & \cdots & a_{nn} 
      \end{array}
      \right|
      =
      \left|
      \begin{array}{cccc}
        a_{11}  &  a_{21} & \cdots & a_{n1} \\
        a_{12}  &  a_{22} & \cdots & a_{n2} \\
        \vdots & \vdots & \ddots & \vdots\\  
        a_{1n}  &  a_{2n} & \cdots & a_{nn} 
      \end{array}
      \right|
    \end{equation}

  \end{block}
\end{frame}


\begin{frame}
  \begin{block}{证明【数学归纳法】}
    将等式两端的行列式分别记为$D$和$D^\prime$,对阶数$n$用归纳法。\pause
    \begin{itemize}
    \item 当$n=2$时,$D=D^\prime$显然成立。\pause 
    \item 假设结论对于阶数小于$n$的行列式都成立,以下考虑阶数为$n$的情况。由定义可知,
      $$
      \begin{array}{c}
        D = a_{11} A_{11}+a_{12}A_{12}+\cdots+a_{1n}A_{1n}, \\[0.1in]
        D^\prime = a_{11} A^\prime_{11}+a_{21}A^\prime_{21}+\cdots+a_{n1}A^\prime_{n1}
      \end{array}
      $$
      显然,$A_{11}=A^\prime_{11}$。    \end{itemize}
  \end{block}
\end{frame}


\begin{frame}
  \begin{block}{证明【续】}
    于是
    $$
    \begin{array}{rcl}
      D^\prime &=& a_{11} A_{11}+(-1)^{1+2}a_{21}
      \left|
      \begin{array}{cccc}
        a_{12} & a_{32} & \cdots & a_{n2} \\
        a_{13} & a_{33} & \cdots & a_{n3} \\
        \vdots & \vdots & & \vdots \\
        a_{1n} & a_{3n} & \cdots & a_{nn} \\
      \end{array}
      \right| \\[0.4in]
      && + (-1)^{1+3}a_{31}
      \left|
      \begin{array}{ccccc}
        a_{12} & a_{22} & a_{42} & \cdots & a_{n2} \\
        a_{13} & a_{23} & a_{43} & \cdots & a_{n3} \\
        \vdots & \vdots & \vdots & & \vdots \\
        a_{1n} & a_{2n} & a_{4n} & \cdots & a_{nn} \\
      \end{array}
      \right|  \\[0.4in]
      && + \cdots + (-1)^{1+n} a_{n1} \left|
      \begin{array}{cccc}
        a_{12} & a_{22} & \cdots & a_{n-1,2} \\
        a_{13} & a_{23} & \cdots & a_{n-1,3} \\
        \vdots & \vdots & & \vdots \\
        a_{1n} & a_{2n} & \cdots & a_{n-1,n} \\
      \end{array}
      \right|
    \end{array}
    $$
  \end{block}
\end{frame}




\begin{frame}
  \begin{block}{证明【续】}
    对$n-1$个行列式按第一行展开,将含$a_{12}$的项进行合并,可得
    $$
    \begin{array}{ll}
      & (-1)^{1+2}a_{21} a_{12}
      \left|
      \begin{array}{ccc}       
        a_{33} & \cdots & a_{n3} \\
        \vdots  & & \vdots \\
        a_{3n} & \cdots & a_{nn} \\
      \end{array}
      \right| 
      + (-1)^{1+3}a_{31} a_{12}
      \left|
      \begin{array}{cccc}
        a_{23}  & a_{43} & \cdots & a_{n3} \\
        \vdots & \vdots & & \vdots \\
        a_{2n}  & a_{4n} & \cdots & a_{nn} \\
      \end{array}
      \right|  \\[0.4in]
      & + \cdots +(-1)^{1+n} a_{n1} a_{12}
      \left|
      \begin{array}{ccc}
        a_{23} & \cdots & a_{n-1,3} \\
        \vdots & & \vdots \\
        a_{2n} & \cdots & a_{n-1,n} \\
      \end{array}
      \right|
    \end{array}
    $$
  \end{block}
\end{frame}


\begin{frame}
  \begin{block}{证明【续】}
    $$
    \begin{array}{ll}       
      = & (-1)^{1+2} a_{12} \left(
      (-1)^{1+1} a_{21} 
      \left|
      \begin{array}{ccc}       
        a_{33} & \cdots & a_{n3} \\
        \vdots  & & \vdots \\
        a_{3n} & \cdots & a_{nn} \\
      \end{array}
      \right|  \right.\\[0.4in]
      & \left.
      + (-1)^{1+2}a_{31} 
      \left|
      \begin{array}{cccc}
        a_{23}  & a_{43} & \cdots & a_{n3} \\
        \vdots & \vdots & & \vdots \\
        a_{2n}  & a_{4n} & \cdots & a_{nn} \\
      \end{array}
      \right| + \cdots + \right.\\[0.4in]
      & \left. (-1)^{1+n-1} a_{n1}
      \left|
      \begin{array}{ccc}
        a_{23} & \cdots & a_{n-1,3} \\
        \vdots & & \vdots \\
        a_{2,n} & \cdots & a_{n-1,n} \\
      \end{array}
      \right|
      \right)
    \end{array}
    $$
  \end{block}
\end{frame}

\begin{frame}
  \begin{block}{证明【续】}
    $$
    \begin{array}{ll}       
      = & (-1)^{1+2} a_{12} \left(      
      \left|
      \begin{array}{cccc}       
        a_{21} & 0 & \cdots & 0 \\
        0 & a_{33} & \cdots & a_{n3} \\
        0 & \vdots  & & \vdots \\
        0 & a_{3n} & \cdots & a_{nn} \\
      \end{array}
      \right|  \right.\\[0.4in]
      & \left.
      + 
      \left|
      \begin{array}{ccccc}
        0 & a_{31} & 0 & \cdots & 0 \\
        a_{23}  & 0 & a_{43} & \cdots & a_{n3} \\
        \vdots & 0 & \vdots & & \vdots \\
        a_{2n}  & 0 & a_{4n} & \cdots & a_{nn} \\
      \end{array}
      \right| + \cdots + \right.\\[0.4in]
      & \left. 
      \left|
      \begin{array}{cccc}
        0 & \cdots & 0 & a_{n-1} \\
        a_{23} & \cdots & a_{n-1,3} & 0 \\
        \vdots & & \vdots & 0\\
        a_{2,n3} & \cdots & a_{n-1,n} & 0 \\
      \end{array}
      \right|
      \right)
    \end{array}
    $$
  \end{block}
\end{frame}




\begin{frame}
  \begin{block}{证明【续】}
    $$
    \begin{array}{ll}      
      = & (-1)^{1+2} a_{12}
      \left|
      \begin{array}{cccc}
        a_{21} & a_{31} & \cdots & a_{n1} \\
        a_{23} & a_{33} & \cdots & a_{n3} \\
        \vdots & \vdots & & \vdots \\
        a_{2n} & a_{3n} & \cdots & a_{nn}        
      \end{array}
      \right| \\[0.3in]
      = & (-1)^{1+2}a_{12} M_{12}^\prime  = a_{12} A_{12}^\prime = a_{12} A_{12}. 
    \end{array}
    $$ \pause 
    同理,含$a_{13}$的项合并后其值等于$a_{13}A_{13}$,$\cdots$,
    含$a_{1n}$的项合并后其值等于$a_{1n}A_{1n}$. 因此,$D=D^\prime$.

  \end{block}
  \pause 
  有了这个性质,行列式对行成立的性质都适用于列。
\end{frame}


\begin{frame}
  \begin{block}{性质2}
    行列式对任一行按下式展开,其值相等,即
    $$
    D = a_{i1} A_{i1} + a_{i2} A_{i2} + \cdots + a_{in}A_{in} = \sum_{j=1}^n a_{ij} A_{ij}, \quad
    i = 1, 2, \cdots, n,
    $$
    其中
    $$
    A_{ij} = (-1)^{i+j} M_{ij}
    $$
    而$M_{ij}$为$D$中划掉第$i$行第$j$列后其余元素按原顺序排成的$n-1$阶行列式,它称为$a_{ij}$的余子式,$A_{ij}$称为$a_{ij}$的代数余子式.
  \end{block}
\end{frame}


\begin{frame}
  \begin{block}{证明[数学归纳法]}
    \begin{itemize}
    \item 当$n=2$时,结论显然成立。
    \item 假设结论对阶数$\le n-1$的行列式成立,以下考虑阶数为$n$的情况。
    \end{itemize}
  \end{block}
\end{frame}


\begin{frame}
  \begin{block}{证明【续】}
    $$
    \begin{array}{rcl}
      D &=&  (-1)^{1+1} a_{11} \left|
      \begin{array}{cccc}
        a_{22}  & a_{23}   & \cdots & a_{2n}\\
        \vdots  & \vdots & & \vdots \\
        a_{i2}  & a_{i3}   & \cdots & a_{in}\\
        \vdots & \vdots  & & \vdots \\
        a_{n2}  & a_{n3}   & \cdots & a_{nn}\\
      \end{array}
      \right| \\[0.4in]
      &&+ (-1)^{1+2} a_{12} \left|
      \begin{array}{cccc}
        a_{21}  & a_{23}  & \cdots & a_{2n}\\
        \vdots & \vdots & & \vdots \\
        a_{i1}  & a_{i3}  & \cdots & a_{in}\\
        \vdots & \vdots & & \vdots \\
        a_{n1}  & a_{n3}  & \cdots & a_{nn}\\
      \end{array}
      \right|
    \end{array}
    $$
  \end{block}
\end{frame}


\begin{frame}
  \begin{block}{证明【续】}    
    $$
    \begin{array}{rcl}
      && + (-1)^{1+3} a_{13} \left|
      \begin{array}{ccccc}
        a_{21}  & a_{22} & a_{24}  & \cdots & a_{2n}\\
        \vdots & \vdots & \vdots & & \vdots \\
        a_{i1}  & a_{i2} & a_{24}  & \cdots & a_{in}\\
        \vdots & \vdots & \vdots & & \vdots \\
        a_{n1}  & a_{n2}  & a_{n4} & \cdots & a_{nn}\\
      \end{array}
      \right|\\[0.4in]
      &&
      + \cdots  
      + (-1)^{1+n} a_{1n} \left|
      \begin{array}{cccc}
        a_{21}  & a_{22}   & \cdots & a_{2,n-1}\\
        \vdots  & \vdots & & \vdots \\
        a_{i1}  & a_{i2}   & \cdots & a_{i,n-1}\\
        \vdots & \vdots  & & \vdots \\
        a_{n1}  & a_{n2}   & \cdots & a_{n,n-1}\\
      \end{array}
      \right|
    \end{array}
    $$
  \end{block}
\end{frame}


\begin{frame}
  \begin{block}{证明【续】}
    由归纳假设,按行展开后合并含$a_{i1}$的项可得
    $$
    \begin{array}{l}
      (-1)^{(i-1)+1}a_{i1} \left ( (-1)^{1+2} a_{12}  \left|
      \begin{array}{cccc}
        a_{23}  & a_{24}  & \cdots & a_{2n}\\
        \vdots & \vdots & & \vdots \\
        a_{i-1,3}  & a_{i-1,4}  & \cdots & a_{i-1,n}\\
        a_{i+1,3}  & a_{i+1,4}  & \cdots & a_{i+1,n}\\
        \vdots & \vdots & & \vdots \\
        a_{n,3}  & a_{n,4}  & \cdots & a_{nn}
      \end{array}
      \right| \right. \\[0.4in]
      \qquad\qquad\qquad\left. + (-1)^{1+3} a_{13}   \left|
      \begin{array}{cccc}
        a_{22} & a_{24}  & \cdots & a_{2n}\\
        \vdots & \vdots & & \vdots \\
        a_{i-1,2} & a_{i-1,4}  & \cdots & a_{i-1,n}\\
        a_{i+1,2} & a_{i+1,4}  & \cdots & a_{i+1,n}\\
        \vdots & \vdots & & \vdots \\
        a_{n2}  & a_{n4} & \cdots & a_{nn}
      \end{array}
      \right|
      \right.
    \end{array}
    $$
  \end{block}
\end{frame}


\begin{frame}
  \begin{block}{证明【续】}
    $$
    \begin{array}{l}     
      \qquad\qquad \left. + \cdots  +  (-1)^{1+n} a_{1n}  \left|
      \begin{array}{ccc}
        a_{22}   & \cdots & a_{2,n-1}\\
        \vdots & & \vdots \\
        a_{i-1,2}   & \cdots & a_{i-1,n-1}\\
        a_{i+1,2}   & \cdots & a_{i+1,n-1}\\
        \vdots  & & \vdots \\
        a_{n2}   & \cdots & a_{n,n-1}
      \end{array}
      \right| \right) \\[0.4in]
      = (-1)^{i+1}a_{i1} \left|
      \begin{array}{cccc}
        a_{12} & a_{13} & \cdots & a_{1n} \\
        a_{22} & a_{23} & \cdots & a_{2n}\\
        \vdots & \vdots & & \vdots \\
        a_{i-1,2} & a_{i-1,3}   & \cdots & a_{i-1,n}\\    
        a_{i-1,2} & a_{i-1,3}   & \cdots & a_{i-1,n}\\
        \vdots & \vdots & & \vdots \\
        a_{n2}  & a_{n3} & \cdots & a_{nn}
      \end{array}
      \right| = (-1)^{i+1}a_{i1} M_{i1} = a_{i1} A_{i1}.
    \end{array}
    $$
  \end{block}
\end{frame}


\begin{frame}
  \begin{block}{证明【续】}
    同理可证,含$a_{i2}$的项合并后其值为$a_{i2}A_{i2}$,$\cdots$,含$a_{in}$的项合并后其值为$a_{in}A_{in}$.  

  \end{block}
\end{frame}




\begin{frame}
  \begin{block}{(线性性质) }
    \begin{itemize}
    \item[1] 行列式的某一行(列)中所有的元素都乘以同一个数$k$,等于用数$k$乘以此行列式,即
      \begin{equation}\label{xz3-1}
        \left|
        \begin{array}{ccccc}
          a_{11}  & a_{12} & \cdots & a_{1n} \\
          \vdots & \vdots     &        & \vdots \\
          ka_{i1}  & ka_{i2} & \cdots & ka_{in} \\
          \vdots & \vdots     &        & \vdots \\
          a_{n1}  & a_{n2} & \cdots & a_{nn}
        \end{array}
        \right| = k
        \left|
        \begin{array}{ccccc}
          a_{11}  & a_{12} & \cdots & a_{1n} \\
          \vdots & \vdots     &        & \vdots \\
          a_{i1}  & a_{i2} & \cdots & a_{in} \\
          \vdots & \vdots     &        & \vdots \\
          a_{n1}  & a_{n2} & \cdots & a_{nn}
        \end{array}
        \right|
      \end{equation}
    \end{itemize}
  \end{block}
\end{frame}




\begin{frame}
  \begin{block}{(线性性质) }
    \begin{itemize}      
    \item[2] 若行列式的某一行(列)的元素都是两数之和,如
      \begin{equation}\label{xz3-2}
        \begin{array}{rcl}
          \left|
          \begin{array}{ccccc}
            a_{11} & \cdots & a_{1j}+b_{1j} & \cdots & a_{1n} \\
            a_{21} & \cdots & a_{2j}+b_{2j} & \cdots & a_{2n} \\
            \vdots&        & \vdots      &        & \vdots \\
            a_{n1} & \cdots & a_{nj}+b_{nj} & \cdots & a_{nn}
          \end{array}
          \right| & = & \left|
          \begin{array}{ccccc}
            a_{11} & \cdots & a_{1j} & \cdots & a_{1n} \\
            a_{21} & \cdots & a_{2j} & \cdots & a_{2n} \\
            \vdots&        & \vdots      &        & \vdots \\
            a_{n1} & \cdots & a_{nj} & \cdots & a_{nn}
          \end{array}
          \right| \\[0.4in]
          &+&
          \left|
          \begin{array}{ccccc}
            a_{11} & \cdots & b_{1j} & \cdots & a_{1n} \\
            a_{21} & \cdots & b_{2j} & \cdots & a_{2n} \\
            \vdots&        & \vdots      &        & \vdots \\
            a_{n1} & \cdots & b_{nj} & \cdots & a_{nn}
          \end{array}
          \right|
          
        \end{array}
      \end{equation}
    \end{itemize}
  \end{block}
\end{frame}


\begin{frame}
  \begin{block}{一些记号}
    \begin{itemize}
    \item $r_i\times k$($c_i\times k$):第$i$行(列)乘以$k$ \\[0.1in]
    \item $r_i\div k$($c_i\div k$):第$i$行(列)提取公因子$k$
    \end{itemize}
  \end{block}
\end{frame}

\begin{frame}
  \begin{exampleblock}{例}
    如果行列式$D=|a_{ij}|_{n}$的元素$a_{ij}=-a_{ji}(i,j=1,2,\cdots,n)$,就称$D$是反对称行列式(其中$a_{ii}=-a_{ii}\Rightarrow a_{ii}=0,i=1,2,\cdots,n$).

    \vspace{0,1in}
    证明:奇数阶反对称行列式的值为$0$.
  \end{exampleblock}

  \pause 
  \begin{block}{证明}
    
    $$
    \begin{array}{rcl}
      D &=& \left|
      \begin{array}{cccc}
        0 & a_{12} & \cd & a_{1n} \\
        -a_{12} & 0 & \cd & a_{2n} \\
        \vd & \vd & \dd & \vd \\
        -a_{1n} & -a_{2n} & \cd & 0
      \end{array}
      \right| \\[0.4in]
      
    \end{array}
    $$
  \end{block}
\end{frame}

\begin{frame}
  \begin{block}{证明【续】}
    $$
    \begin{array}{rcl}      
      & \pause \xlongequal[]{\text{性质1}}& \pause  \left|
      \begin{array}{cccc}
        0 & -a_{12} & \cd & -a_{1n} \\
        a_{12} & 0 & \cd & -a_{2n} \\
        \vd & \vd & \dd & \vd \\
        a_{1n} & a_{2n} & \cd & 0
      \end{array}
      \right| \\[0.4in]
      & \pause \xlongequal[\text{将每行提取公因子$-1$}]{\text{性质3-1}}& \pause 
      (-1)^n \left|
      \begin{array}{cccc}
        0 & a_{12} & \cd & a_{1n} \\
        -a_{12} & 0 & \cd & a_{2n} \\
        \vd & \vd & \dd & \vd \\
        -a_{1n} & -a_{2n} & \cd & 0
      \end{array}
      \right| = (-1)^n D.
    \end{array}  
    $$
    \pause 
    由于$n$为奇数,故$D=-D$,从而$D=0$.
  \end{block}
\end{frame}


\begin{frame}
  \begin{block}{推论1}
    若行列式的某行元素全为0,其值为0.
  \end{block}
  \pause 
  \begin{exampleblock}{例}
    $$
    \left|
    \begin{array}{ccc}
      1 & 2 & 3\\
      0 & 0 & 0\\
      2 & 5 & 1
    \end{array}
    \right| = 0.
    $$
  \end{exampleblock}
\end{frame}


\begin{frame}
  \begin{block}{性质4}
    若行列式有两行(列)完全相同,其值为$0$.
  \end{block}
  \pause 
  \begin{block}{证明}
    不妨设$D$的第$i$和$j$行元素全部相等,即对
    $$
    D = \left|
    \begin{array}{cccc}
      a_{11}  &  a_{12} & \cdots & a_{1n} \\
      \vdots & \vdots &  & \vdots\\  
      a_{i1}  &  a_{i2} & \cdots & a_{in} \\
      \vdots & \vdots &  & \vdots\\  
      a_{j1}  &  a_{j2} & \cdots & a_{jn} \\
      \vdots & \vdots &  & \vdots\\  
      a_{n1}  &  a_{n2} & \cdots & a_{nn} 
    \end{array}
    \right|,
    $$
    有$a_{il}=a_{jl}(i\ne j, l=1,2,\cdots,n)$.
  \end{block}
\end{frame}

\begin{frame}
  \begin{block}{证明【续】}
    对阶数$n$用数学归纳法。
    \begin{itemize}
    \item 当$n=2$时,结论显然成立。\pause 
    \item 假设结论对阶数为$n-1$的行列式成立,在$n$阶的情况下,对第$k(k\ne i, j)$行展开,有
      $$
      D = a_{k1} A_{k1} + a_{k2} A_{k2} + \cdots + a_{kn} A_{kn}. 
      $$ \pause 
      注意到余子式$M_{kl}(l=1,2,\cdots,n)$是$n-1$阶行列式,且其中有两行元素相同,故
      $$
      A_{kl} = (-1)^{k+l} M_{kl} = 0\quad (l=1,2,\cdots,n),
      $$
      从而$D=0$.
    \end{itemize}
  \end{block}
\end{frame}

\begin{frame}
  \begin{exampleblock}{例}
    $$
    \left|
    \begin{array}{ccc}
      1 & 2 & 3\\
      1 & 2 & 3\\
      2 & 3 & 4
    \end{array}
    \right| = 0.
    $$
  \end{exampleblock}
  \pause 
  \begin{block}{推论2}
    若行列式中有两行(列)元素成比例,则行列式的值为$0$.
  \end{block}
  \pause
  \begin{exampleblock}{例}
    $$
    \left|
    \begin{array}{ccc}
      2 & 0 & -2\\
      1 & 0 & -1\\
      2 & 3 & 4
    \end{array}
    \right| =2 \left|
    \begin{array}{ccc}
      1 & 0 & -1\\
      1 & 0 & -1\\
      2 & 3 & 4
    \end{array}
    \right| = 0.
    $$

  \end{exampleblock}
\end{frame}

\begin{frame}
  \begin{block}{性质5}
    把行列式的某一行(列)的各元素乘以同一个数然后加到另一行(列)对应的元素上去,行列式的值不变。
  \end{block}
  \pause
  \begin{block}{证明}
    将数$k$乘以第$j$行加到第$i$行,有
    $$
    \begin{array}{ll}
      & \left|
      \begin{array}{cccc}
        a_{11} & a_{12} & \cdots & a_{1n}\\
        \vdots & \vdots &  & \vdots \\
        a_{i1}+k a_{j1} & a_{i2}+k a_{j2} & \cdots & a_{in}+k a_{jn}\\
        \vdots & \vdots &  & \vdots \\
        a_{j1} & a_{j2} & \cdots & a_{jn}\\
        \vdots & \vdots &  & \vdots \\
        a_{n1} & a_{n2} & \cdots & a_{nn}
      \end{array}
      \right|
    \end{array}
    $$

  \end{block}
\end{frame}

\begin{frame}
  \begin{block}{证明【续】}
      $$
      \begin{array}{ll}
        \pause\xlongequal[]{\text{性质3-2}} & \pause
      \left|
      \begin{array}{cccc}
        a_{11} & a_{12} & \cdots & a_{1n}\\
        \vdots & \vdots &  & \vdots \\
        a_{i1} & a_{i2} & \cdots & a_{in}\\
        \vdots & \vdots &  & \vdots \\
        a_{j1} & a_{j2} & \cdots & a_{jn}\\
        \vdots & \vdots &  & \vdots \\
        a_{n1} & a_{n2} & \cdots & a_{nn}
      \end{array}
      \right| +
      \left|
      \begin{array}{cccc}
        a_{11} & a_{12} & \cdots & a_{1n}\\
        \vdots & \vdots &  & \vdots \\
        k a_{j1} & k a_{j2} & \cdots & k a_{jn}\\
        \vdots & \vdots &  & \vdots \\
        a_{j1} & a_{j2} & \cdots & a_{jn}\\
        \vdots & \vdots &  & \vdots \\
        a_{n1} & a_{n2} & \cdots & a_{nn}
      \end{array}
      \right|\\[0.2in]
      \pause \xlongequal[]{\text{推论2}}  & \pause
      \left|
      \begin{array}{cccc}
        a_{11} & a_{12} & \cdots & a_{1n}\\
        \vdots & \vdots &  & \vdots \\
        a_{i1} & a_{i2} & \cdots & a_{in}\\
        \vdots & \vdots &  & \vdots \\
        a_{j1} & a_{j2} & \cdots & a_{jn}\\
        \vdots & \vdots &  & \vdots \\
        a_{n1} & a_{n2} & \cdots & a_{nn}
      \end{array}
      \right|
    \end{array}
    $$

  \end{block}
\end{frame}

\begin{frame}
  \begin{block}{一些记号}
    \begin{itemize}
    \item $r_i + r_j\times k$:将第$j$行乘以$k$加到第$i$行 \\[0.1in]
    \item $c_i + c_j\times k$:将第$j$列乘以$k$加到第$i$列  
    \end{itemize}

  \end{block}
\end{frame}


\begin{frame}
  \begin{block}{性质6}
    互换行列式的两行(列),行列式变号。
  \end{block}
  \pause
  \begin{block}{证明}
    $$
    \begin{array}{rcl}
      D &=& \left|
      \begin{array}{cccc}
        a_{11} & a_{12} & \cdots & a_{1n}\\
        \vdots & \vdots &  & \vdots \\
        a_{i1} & a_{i2} & \cdots & a_{in}\\
        \vdots & \vdots &  & \vdots \\
        a_{j1} & a_{j2} & \cdots & a_{jn}\\
        \vdots & \vdots &  & \vdots \\
        a_{n1} & a_{n2} & \cdots & a_{nn}
      \end{array}
      \right| 
    \end{array}
    $$
  \end{block}
\end{frame}


\begin{frame}
  \begin{block}{证明【续】}
    $$
    \begin{array}{rcl}
      &\pause\xlongequal[r_i+r_j]{\text{性质5}} &\pause
      \left|
      \begin{array}{cccc}
        a_{11} & a_{12} & \cdots & a_{1n}\\
        \vdots & \vdots &  & \vdots \\
        a_{i1}+a_{j1} & a_{i2}+a_{j2} & \cdots & a_{in}+a_{jn}\\
        \vdots & \vdots &  & \vdots \\
        a_{j1} & a_{j2} & \cdots & a_{jn}\\
        \vdots & \vdots &  & \vdots \\
        a_{n1} & a_{n2} & \cdots & a_{nn}
      \end{array}
      \right|
    \end{array}
    $$
  \end{block}
\end{frame}


\begin{frame}
  \begin{block}{证明【续】}
    $$
    \begin{array}{rcl}
      &\pause\xlongequal[r_j-r_i]{\text{性质5}} &\pause
      \left|
      \begin{array}{cccc}
        a_{11} & a_{12} & \cdots & a_{1n}\\
        \vdots & \vdots &  & \vdots \\
        a_{i1}+a_{j1} & a_{i2}+a_{j2} & \cdots & a_{in}+a_{jn}\\
        \vdots & \vdots &  & \vdots \\
        -a_{i1} & -a_{i2} & \cdots & -a_{in}\\
        \vdots & \vdots &  & \vdots \\
        a_{n1} & a_{n2} & \cdots & a_{nn}
      \end{array}
      \right|
    \end{array}
    $$
  \end{block}
\end{frame}


\begin{frame}
  \begin{block}{证明【续】}
    $$
    \begin{array}{rcl}
      &\pause\xlongequal[r_i+r_j]{\text{性质5}} &\pause
      \left|
      \begin{array}{cccc}
        a_{11} & a_{12} & \cdots & a_{1n}\\
        \vdots & \vdots &  & \vdots \\
        a_{j1} & a_{j2} & \cdots & a_{jn}\\
        \vdots & \vdots &  & \vdots \\
        -a_{i1} & -a_{i2} & \cdots & -a_{in}\\
        \vdots & \vdots &  & \vdots \\
        a_{n1} & a_{n2} & \cdots & a_{nn}
      \end{array}
      \right| \pause\xlongequal[]{\text{性质3-1}} - D
    \end{array}
    $$
  \end{block}
\end{frame}


\begin{frame}
  \begin{block}{一些记号}
    \begin{itemize}
    \item $r_i \leftrightarrow r_j $:互换第$i,j$行 \\[0.1in]
    \item $c_i \leftrightarrow c_j $:互换第$i,j$列
    \end{itemize}
  \end{block}

  \begin{exampleblock}{例}
    $$
    \begin{array}{rcl}
      \left|
      \begin{array}{ccc}
        1 & 2  \\
        3 & 4  \\
      \end{array}
      \right|
      &\xlongequal[]{r_1\leftrightarrow r_2}&
      -\left|
      \begin{array}{ccc}
        3 & 4 \\
        1 & 2 
      \end{array}
      \right|    \\[0.2in]
      \left|
      \begin{array}{ccc}
        1 & 2  \\
        3 & 4  
      \end{array}
      \right|
      &\xlongequal[]{c_1\leftrightarrow c_2}&
      -\left|
      \begin{array}{ccc}
        2 & 1  \\
        4 & 3  
      \end{array}
      \right|    
    \end{array}
    $$

  \end{exampleblock}
\end{frame}

\begin{frame}
  \begin{block}{性质7}
    行列式某一行的元素乘以另一行对应元素的代数余子式之和等于$0$,即
    $$
    \sum_{k=1}^n a_{ik} A_{jk}  = 0 \quad (i\ne j).
    $$
  \end{block}
  \pause
  \begin{block}{证明}
    由性质2,对$D$的第$j$行展开得
    $$
    \left|
    \begin{array}{cccc}
      a_{11} & a_{12} & \cdots & a_{1n}\\
      \vdots & \vdots &  & \vdots \\
      a_{i1} & a_{i2} & \cdots & a_{in}\\
      \vdots & \vdots &  & \vdots \\
      a_{j1} & a_{j2} & \cdots & a_{jn}\\
      \vdots & \vdots &  & \vdots \\
      a_{n1} & a_{n2} & \cdots & a_{nn}
    \end{array}
    \right|   =  a_{j1} A_{j1} + a_{j2} A_{j2} + \cdots + a_{jn} A_{jn}
    $$

  \end{block}
\end{frame}

\begin{frame}
  \begin{block}{证明【续】}
    因此,将$D$中第$j$行的元素$a_{j1},a_{j2},\cdots,a_{jn}$换成$a_{i1},a_{i2},\cdots,a_{in}$后所得的行列式,
    其展开式就是$\sum_{k=1}^na_{ik}A_{jk}$,即
    $$
    \left|
    \begin{array}{cccc}
      a_{11}  &  a_{12} & \cdots & a_{1n} \\
      \vdots & \vdots &  & \vdots\\  
      a_{i1}  &  a_{i2} & \cdots & a_{in} \\
      \vdots & \vdots &  & \vdots\\  
      a_{i1}  &  a_{i2} & \cdots & a_{in} \\
      \vdots & \vdots &  & \vdots\\  
      a_{n1}  &  a_{n2} & \cdots & a_{nn} 
    \end{array}
    \right|
    \xlongequal[]{\text{性质4}}  0.
    $$  

  \end{block}
\end{frame}


\begin{frame}
  \begin{block}{结论}
    \begin{itemize}
    \item 对行列式$D$按行展开,有
      $$
      \sum_{k=1} a_{ik} A_{jk} = \delta_{ij} D,
      $$
      其中$\delta_{ij}$为克罗内克(Kronecker)记号,表示为
      $$
      \delta_{ij} = \left\{
      \begin{array}{ll}
        1 & i=j\\
        0 & i\ne j
      \end{array}
      \right..
      $$
    \item 对行列式$D$按列展开,有
      $$
      \sum_{k=1} a_{ki} A_{kj} = \delta_{ij} D,
      $$
    \end{itemize}
  \end{block}
\end{frame}
