\section{第五章~~特征值问题}

%%%%%%%%%%%%%%%%%%%%%%%%%%%%%%%
\subsection{知识点}


\begin{frame}\ft{特征值与特征向量}
  \begin{footnotesize}
    \begin{block}{定义(特征值与特征向量)}
      设$\A$为复数域$\CCC$上的$n$阶矩阵,如果存在数$\lambda\in\CCC$和非零的$n$维向量$\xx$使得
      $$
      \A\xx=\lambda\xx
      $$
      则称$\lambda$为矩阵$\A$的\blue{\underline{特征值}},$\xx$为$\A$的对应于特征值$\lambda$的\blue{\underline{特征向量}}。
    \end{block}  

    \begin{itemize}
    \item[(1)] 特征向量$\xx\ne\zero$;
    \item[(2)] 特征值问题是对方针而言的。 
    \end{itemize}
  \end{footnotesize}
\end{frame}


\begin{frame}\ft{特征值与特征向量}
  \begin{footnotesize}
    由定义,$n$阶矩阵$\A$的特征值,就是使齐次线性方程组
    $$
    (\A-\lambda\II)\xx=\zero
    $$
    有非零解的$\lambda$值,即满足方程
    $$
    \det(\A-\lambda\II)=0
    $$
    的$\lambda$都是矩阵$\A$的特征值。
    \pause


    
  \end{footnotesize}
\end{frame}


\begin{frame}\ft{特征值与特征向量}
  \begin{footnotesize}
    \begin{block}{定义(特征多项式、特征矩阵、特征方程)}
      设$n$阶矩阵$\A=(a_{ij})$,则
      $$
      f(\lambda)=\det(\A-\lambda\II)
      =\left|
      \begin{array}{cccc}
       a_{11}- \lambda&a_{12}&\cd&a_{1n}\\[0.2cm]
        a_{21}&a_{22}-\lambda&\cd&a_{2n}\\[0.2cm]
        \vd&\vd&&\vd\\[0.2cm]
        a_{n1}&a_{n2}&\cd&a_{nn}-\lambda
      \end{array}
      \right|
      $$
      称为矩阵$\A$的特征多项式,$\A-\lambda\II$称为$\A$的特征矩阵,$\det(\A-\lambda\II)=0$称为$\A$的特征方程。
    \end{block}
    
  \end{footnotesize}
\end{frame}

\begin{frame}\ft{特征值与特征向量}
  \begin{footnotesize}
    \begin{exampleblock}{例1}
      求矩阵
      $$
      \A=\left(
      \begin{array}{rrr}
        5&-1&-1\\
        3&1&-1\\
        4&-2&1
      \end{array}
      \right)
      $$
      的特征值与特征向量。
    \end{exampleblock}
    \pause\jiename
    $$
    \begin{array}{rl}
      \det(\A-\lambda\II)&=\left|\begin{array}{rrr}
      5-\lambda&-1&-1\\
        3&1-\lambda&-1\\
        4&-2&1-\lambda
      \end{array}\right| =-(\lambda-3)(\lambda-2)^2=0
    \end{array}
    $$
    故特征值为$\lambda_1=3,~\lambda_{2,3}=2\mbox{(二重特征值)}$。
  \end{footnotesize}
\end{frame}

\begin{frame}\ft{特征值与特征向量}
  \begin{footnotesize}
\begin{itemize}
\item对于特征值$\lambda_1=3$,齐次线性方程组$(\A-3\II)\xx=\zero$为
    $$
    \left(\begin{array}{rrr}
      2&-1&-1\\
      3&-2&-1\\
      4&-2&-2
      \end{array}\right)\left(
    \begin{array}{c}
      x_1\\
      x_2\\
      x_3
    \end{array}
    \right)=\left(
    \begin{array}{c}
      0\\
      0\\
      0
    \end{array}
    \right)
    $$
    基础解系为$\xx_1=(1,1,1)^T$,因此\red{$k_1\xx_1(k_1\ne 0)$}是$\A$对应于$\lambda_1=3$的全部特征向量。\\[0.1in]
    
\item 
    对于特征值$\lambda_{2,3}=2$,齐次线性方程组$(\A-2\II)\xx=\zero$为
    $$
    \left(\begin{array}{rrr}
      3&-1&-1\\
      3&-1&-1\\
      4&-2&-1
      \end{array}\right)\left(
    \begin{array}{c}
      x_1\\
      x_2\\
      x_3
    \end{array}
    \right)=\left(
    \begin{array}{c}
      0\\
      0\\
      0
    \end{array}
    \right)
    $$
    基础解系为$\xx_2=(1,1,2)^T$,因此\red{$k_2\xx_2(k_2\ne 0)$}是$\A$对应于$\lambda_{2,3}=2$的全部特征向量。

\end{itemize}•
    

  \end{footnotesize}
\end{frame}



\begin{frame}\ft{特征值与特征向量的性质}
  \begin{footnotesize}
    \begin{block}{定理}
      设$n$阶矩阵$\A=(a_{ij})$的$n$个特征值为$\lambda_1,\lambda_2,\cd,\lambda_n$,则
      \begin{itemize}
      \item[(1)] $\ds \sum_{i=1}^n\lambda_i=\sum_{i=1}^na_{ii}$
      \item[(2)] $\ds \prod_{i=1}^n\lambda_i=\det(\A)$         
      \end{itemize}
    \end{block}
 

    \begin{itemize}
    \item 当$\det(\A)\ne 0$,即$\A$为可逆矩阵时,其特征值全为非零数;
    \item 奇异矩阵$\A$至少有一个零特征值。      
    \end{itemize}
  \end{footnotesize}
\end{frame}


 


\begin{frame}\ft{特征值与特征向量的性质}
  \begin{footnotesize}
    \begin{block}{定理}
      一个特征向量不能属于不同的特征值。
    \end{block}
    
  \end{footnotesize}
\end{frame}

\begin{frame}\ft{特征值与特征向量的性质}
  \begin{footnotesize}
    \begin{block}{性质1}
      \begin{table}
        \caption{特征值与特征向量}
        
        \begin{tabular}{|c|c|c|}\hline
          &特征值&特征向量\\\hline
          \red{$\A$}&\red{$\lambda$}&\red{$\xx$}\\ \hline 
          \hline 
          $k\A$&$k\lambda$&$\xx$\\\hline
          $\A^m$&$\lambda^m$&$\xx$\\\hline
          $\A^{-1}$&$\lambda^{-1}$&$\xx$\\\hline
        \end{tabular}
      \end{table}
    \end{block}
%  \end{footnotesize}
%\end{frame}
%
%\begin{frame}\ft{特征值与特征向量的性质}
%  \begin{footnotesize}
    \begin{block}{性质2}
      矩阵$\A$与$\A^T$的特征值相同。
    \end{block}
     
  \end{footnotesize}
\end{frame}

 

%\begin{frame}\ft{特征值与特征向量的性质}
%  \begin{footnotesize}
%    \begin{exampleblock}{例}
%      设$\A=\left(
%      \begin{array}{rrr}
%        1&-1&1\\
%        2&-2&2\\
%        -1&1&-1
%      \end{array}
%      \right)$
%      \begin{itemize}
%      \item[(i)]求$\A$的特征值与特征向量
%      \item[(ii)] 求可逆矩阵$\PP$,使得$\PP^{-1}\A\PP$为对角阵。 
%      \end{itemize}
%    \end{exampleblock}
%    \pause\jiename
%    $$
%    \begin{array}{rl}
%      |\lambda\II-\A|&=\left|
%      \begin{array}{rrr}
%        \lambda-1&1&-1\\
%        -2&\lambda+2&-2\\
%        1&-1&\lambda+1
%      \end{array}
%      \right| = \lambda^2(\lambda+2)
%    \end{array}
%    $$
%    $\A$的特征值为$\lambda_1=\lambda_2=0$(二重特征值)和$\lambda=-2$.
%  \end{footnotesize}
%\end{frame}
%
%\begin{frame}\ft{特征值与特征向量的性质}
%  \begin{footnotesize}
%    \begin{itemize}
%    \item 当$\lambda_1=\lambda_2=0$时,
%      $$
%      \lambda_1\II-\A=\left(
%      \begin{array}{rrr}
%        -1&1&-1\\
%        -2&2&-2\\
%        1&-1&1
%      \end{array}
%      \right)\left(
%      \begin{array}{c}
%        x_1\\x_2\\x_3
%      \end{array}
%      \right)=\left(
%      \begin{array}{c}
%        0\\0\\0
%      \end{array}
%      \right)
%      $$
%      基础解系为$\xx_1=(1,1,0)^T$和$\xx_2=(-1,0,1)^T$,故$\A$对应于$\lambda_1=0$的全体特征向量为
%      $$
%      k_1\xx_1+k_2\xx_2 ~~~(k_1,k_2\mbox{为不全为零的任意常数})
%      $$
%    \end{itemize}
%  \end{footnotesize}
%\end{frame}
%
%
%\begin{frame}\ft{特征值与特征向量的性质}
%  \begin{footnotesize}
%    \begin{itemize}
%    \item 当$\lambda_3=-2$时,
%      $$
%      \lambda_1\II-\A=\left(
%      \begin{array}{rrr}
%        -3&1&-1\\
%        -2&0&-2\\
%        1&-1&-1
%      \end{array}
%      \right)\left(
%      \begin{array}{c}
%        x_1\\x_2\\x_3
%      \end{array}
%      \right)=\left(
%      \begin{array}{c}
%        0\\0\\0
%      \end{array}
%      \right)
%      $$
%      基础解系为$\xx_3=(-1,-2,1)^T$,故$\A$对应于$\lambda_3=-2$的全体特征向量为
%      $$
%      k_3\xx_3 ~~~(k_3\mbox{为非零的任意常数})
%      $$
%    \end{itemize}
%  \end{footnotesize}
%\end{frame}
%
%
%\begin{frame}\ft{特征值与特征向量的性质}
%  \begin{footnotesize}
%    $$
%    \A(\xx_1,~\xx_2,~\xx_3)=(\xx_1,~\xx_2,~\xx_3)\left(
%    \begin{array}{rrr}
%      \lambda_1&&\\
%      &\lambda_2&\\
%      &&\lambda_3
%    \end{array}
%    \right)
%    $$
%    取
%    $$
%    \PP=(\xx_1,~\xx_2,~\xx_3)=\left(
%    \begin{array}{rrr}
%      1&-1&-1\\
%      1&0&-2\\
%      0&1&1
%    \end{array}
%    \right), ~~ \Lambdabd=\left(
%    \begin{array}{rrr}
%      0&&\\
%      &0&\\
%      &&-2
%    \end{array}
%    \right)
%    $$
%    则$\PP$可逆(因为$|\PP|\ne 0$),且
%    $$
%    \PP^{-1}\A\PP=\Lambdabd
%    $$
%  \end{footnotesize}
%\end{frame}


\begin{frame}\ft{特征值与特征向量的性质}
  \begin{footnotesize}
    \begin{exampleblock}{例}
      对于下列矩阵$\A$的特征值,能做怎样的断言?
      \begin{itemize}
      \item[(1)] $\det(\II-\A^2)=0$
      \item[(2)] $\A^k=0$
      \item[(3)] $\A=k\II-\B$($\lambda_0$为$\B$的特征值)
      \end{itemize}
    \end{exampleblock}
  \end{footnotesize}
\end{frame}


\begin{frame}\ft{相似矩阵}
  \begin{footnotesize}
    \begin{block}{定义(相似矩阵)}
      对于方阵$\A$和$\B$,若存在可逆矩阵$\PP$,使得
      $$
      \PP^{-1}\A\PP=\B,
      $$
      就称$\A$相似于$\B$,记作$\A\sim\B$.
    \end{block}
     

%    
%  \end{footnotesize}
%\end{frame}
%
%
%\begin{frame}
%  \begin{footnotesize}
    \begin{block}{定理}
      相似矩阵的特征值相同。
    \end{block}
   
  \end{footnotesize}
\end{frame}



\begin{frame}\ft{矩阵可对角化的条件}

\red{矩阵可对角化,即矩阵与对角阵相似。 } 


  \begin{footnotesize}
    \begin{block}{定理}
      $\mbox{矩阵可对角化} ~~\Longleftrightarrow~~
      \mbox{$n$阶矩阵有$n$个线性无关的特征向量}$ 
    \end{block}
    
%  \end{footnotesize}
%\end{frame}
%
%
%\begin{frame}
%  \begin{footnotesize}
%    若$\A$与$\Lambdabd$相似,则$\Lambdabd$的主对角元都是$\A$的特征值。
%    若不计$\lambda_k$的排列次序,则$\Lambdabd$是唯一的,称$\Lambdabd$为$\A$的相似标准型。
%  \end{footnotesize}
%\end{frame}
%
%
%\begin{frame}
%  \begin{footnotesize}
    \begin{block}{定理}
      $\A$的属于不同特征值的特征向量是线性无关的。
    \end{block}
    
%  \end{footnotesize}
%\end{frame}
%
%\begin{frame}
%  \begin{footnotesize}
    \begin{block}{推论}
      若$\A$有$n$个互不相同的特征值,则$\A$与对角阵相似。
    \end{block}
  \end{footnotesize}
\end{frame}

\begin{frame}\ft{矩阵可对角化的条件}
  \begin{footnotesize}
    \begin{exampleblock}{例}
      设实对称矩阵
      $$
      \A=\left(
      \begin{array}{rrrr}
        1&-1&-1&-1\\
        -1&1&-1&-1\\
        -1&-1&1&-1\\
        -1&-1&-1&1
      \end{array}
      \right)
      $$
      问$\A$是否可对角化?若可对角化,求对角阵$\Lambdabd$及可逆矩阵$\PP$使得$\PP^{-1}\A\PP=\Lambdabd$,再求$\A^k$。
    \end{exampleblock}
    \pause\proofname
    $$
    |\A-\lambda\II|=\left|
      \begin{array}{rrrr}
        1-\lambda&-1&-1&-1\\
        -1&1-\lambda&-1&-1\\
        -1&-1&1-\lambda&-1\\
        -1&-1&-1&1-\lambda
      \end{array}
      \right|=(\lambda+2)(\lambda-2)^3
    $$
    故特征值为$\lambda_1=-2,~~\lambda_{2,3,4}=2$
  \end{footnotesize}
\end{frame}

\begin{frame}\ft{矩阵可对角化的条件}
  \begin{footnotesize}
\begin{itemize}
\item
对于特征值$\lambda_1=-2$,齐次线性方程组$(\A+2\II)\xx=\zero$为
      $$
      \left(
      \begin{array}{rrrr}
        3&-1&-1&-1\\
        -1&3&1&1\\
        -1&-1&3&1\\
        -1&-1&-1&3
      \end{array}
      \right)\left(
      \begin{array}{cccc}
        x_1\\x_2\\x_3\\x_4
      \end{array}
      \right)=\left(
      \begin{array}{cccc}
        0\\0\\0\\0
      \end{array}
      \right)
      $$
      基础解系为$$
      \xx_1=(1,1,1,1)^T,
      $$
      故对应于$\lambda_1=-2$的全部特征向量为
      $
     \red{ k_1\xx_1( k_1\ne 0)}.
      $
\pause 
 \item
对于特征值$\lambda_{2,3,4}=2$,齐次线性方程组$(\A-2\II)\xx=\zero$为
      $$
      (\A-\lambda_2\II)\xx= \left(
      \begin{array}{rrrr}
        -1&-1&-1&-1\\
        -1&-1&-1&-1\\
        -1&-1&-1&-1\\
        -1&-1&-1&-1
      \end{array}
      \right)\left(
      \begin{array}{cccc}
        x_1\\x_2\\x_3\\x_4
      \end{array}
      \right)=\left(
      \begin{array}{cccc}
        0\\0\\0\\0
      \end{array}
      \right)
      $$
      基础解系为
      $$
      \xx_2=(1,-1,0,0)^T,~~
      \xx_3=(1,0,-1,0)^T,~~
      \xx_4=(1,0,-1,0)^T,
      $$
      故对应于$\lambda_2=2$的全部特征向量为
      $\red{
      k_2\xx_2+k_3\xx_3+k_4\xx_4 (k_2,k_3,k_4\mbox{不全为零})}
      $
\end{itemize}•
      
 \end{footnotesize}
\end{frame}


\begin{frame}\ft{矩阵可对角化的条件}
  \begin{footnotesize}

由特征值问题定义可知
$$\red{\boxed{
\A(\xx_1,\xx_2,\xx_3,\xx_4)=(\xx_1,\xx_2,\xx_3,\xx_4)\left(
\begin{array}{cccc}
\lambda_1&&&\\
&\lambda_2&&\\
&&\lambda_3&\\
&&&\lambda_4
\end{array}
\right)}}
$$

取
$$	
    \PP=(\xx_1,\xx_2,\xx_3,\xx_4)=\left(
    \begin{array}{rrrr}
      1&1&1&1\\
      1&-1&0&0\\
      1&0&-1&0\\
      1&0&0&-1
    \end{array}
    \right)
    $$
    则$\A\PP=\PP\Lambdabd$,注意到$\det(\PP)\ne 0$,于是
	$$
\A=\PP\Lambdabd\PP^{-1}.
$$

  \end{footnotesize}
\end{frame}

\begin{frame}\ft{矩阵可对角化的条件}
  \begin{footnotesize}
    \begin{exampleblock}{例2}
      设$\A=(a_{ij})_{n\times n}$是主对角元全为$2$的上三角矩阵,且存在$a_{ij}\ne 0(i<j)$,问$\A$是否可对角化?
    \end{exampleblock}
\pause\proofname
    $$
    \A=\left(
    \begin{array}{cccc}
      2&*&\cd&*\\
      0&2&\cd&*\\
      \vd&\vd&\dd&\vd\\
      0&0&\cd&2
    \end{array}
    \right)~~\Longrightarrow~~
    \det(\A-\lambda\II)=(2-\lambda)^n
    ~~\Longrightarrow~~
    \lambda=2\mbox{为$\A$的$n$重特征值}
    $$ 
    $$
    \begin{array}{rl}
    \rr(2\II-\A)\ge 1 &
    \Longrightarrow~~
    (2\II-\A)\xx=\zero\mbox{的基础解系所含向量个数}\le n-1\\[0.1in] 
    &\Longrightarrow~~
    \A\mbox{的线性无关的特征向量的个数}\le n-1\\[0.1in]
    &\Longrightarrow~~
    \A\mbox{不与对角阵相似。}
    \end{array}
    $$
  \end{footnotesize}
\end{frame}



\begin{frame}\ft{实对称矩阵的对角化}
  \begin{footnotesize}
    \begin{block}{定理}
      实对称矩阵$\A$的任一特征值都是实数。
    \end{block}
   
%  \end{footnotesize}
%\end{frame}
%
%
%\begin{frame}
%  \begin{footnotesize}
    \begin{block}{定理}
      实对称矩阵$\A$对应于不同特征值的特征向量\red{正交}。
    \end{block}
    
  \end{footnotesize}
\end{frame}



\begin{frame}\ft{实对称矩阵的对角化}
  \begin{footnotesize}
    \begin{block}{定理}
      对于$n$阶实对称矩阵$\A$,存在$n$阶正交矩阵$\T$,使得
      $$
      \T^{-1}\A\T=\Lambdabd
      $$
    \end{block}
    
  \end{footnotesize}
\end{frame}



\begin{frame}\ft{实对称矩阵的对角化}
  \begin{footnotesize}
    \begin{exampleblock}{例}
      设
      $$
      \A=\left(
      \begin{array}{rrr}
        1&-2&2\\
        -2&-2&4\\
        2&4&-2
      \end{array}
      \right)
      $$
      求正交阵$\T$,使$\T^{-1}\A\T$为对角阵。      
    \end{exampleblock}
    \pause\jiename
    $$
    |\A-\lambda\II|=\left|
    \begin{array}{rrr}
      1-\lambda&-2&2\\
        -2&-2-\lambda&4\\
        2&4&-2-\lambda
    \end{array}
    \right| = -(\lambda-2)^2(\lambda+7)
    $$
    特征值为$\lambda_{1,2}=2$(二重)和$\lambda_3=-7$。
   \end{footnotesize}
\end{frame}


\begin{frame}\ft{实对称矩阵的对角化}
  \begin{footnotesize}
\begin{itemize}
\item对于特征值$\lambda_{1,2}=2$,齐次线性方程组$(\A-2\II)\xx=\zero$为
    $$
    \left(
      \begin{array}{rrr}
        -1&-2& 2\\
        -2&-3& 4\\
         2& 4&-4
      \end{array}
      \right)\left(
      \begin{array}{r}
        x_1\\
        x_2\\
        x_3
      \end{array}
      \right)=\left(
      \begin{array}{r}
        0\\
        0\\
        0
      \end{array}
      \right)
    $$
      得特征向量$\xx_1=(2,-1,0)^T,~~\xx_2=(2,0,1)^T$。 \\[0.1in]
\item 
对于特征值$\lambda_2=-7$,齐次线性方程组$(\A-\lambda_2\II)\xx=\zero$为
    $$
    \left(
    \begin{array}{rrr}
      8&-2&2\\
      -2&5&4\\
      2&4&5
    \end{array}
    \right)\left(
    \begin{array}{r}
      x_1\\
      x_2\\
      x_3
      \end{array}
      \right)=\left(
      \begin{array}{r}
        0\\
        0\\
        0
      \end{array}
      \right)
    $$
      得特征向量$\xx_3=(1,2,-2)^T$。
\end{itemize}•
    


  
  \end{footnotesize}
\end{frame}


\begin{frame}\ft{实对称矩阵的对角化}
  \begin{footnotesize}
\begin{itemize}
\item 对特征向量$\xx_1=(2,-1,0)^T,~~\xx_2=(2,0,1)^T$,先用\red{施密特正交化过程}正交化,然后单位化。
\item[] 先正交化得
      $$
      \begin{array}{rl}
        \betabd_1&=\xx_1,\\[0.2cm]
        \betabd_2&\ds =\xx_2-\frac{(\xx_2,\betabd_1)}{(\betabd_1,\betabd_1)}\betabd_1\\[0.2in]
        &\ds=\left(
        \begin{array}{rrr}
          2\\0\\1
        \end{array}
        \right)-\frac45\left(
        \begin{array}{rrr}
          2\\-1\\0
        \end{array}
        \right)=\frac15\left(
        \begin{array}{rrr}
          2\\4\\5
        \end{array}
        \right)
      \end{array}
      $$
      \pause
      再单位化得
      $$
      \yy_1=\left(\frac{2\sqrt{5}}{5},~-\frac{\sqrt{5}}{5},~0\right)^T,~~
      \yy_2=\left(\frac{2\sqrt{5}}{15},~-\frac{4\sqrt{5}}{15},~\frac{\sqrt{5}}3\right)^T
      $$


\item
 对特征向量$\xx_3=(1,2,-2)^T$单位化,得$\ds \yy_3=\left(\frac13,~~\frac23,~~-\frac23\right)^T$。
\end{itemize}
 \end{footnotesize}
\end{frame}


\begin{frame}\ft{实对称矩阵的对角化}
  \begin{footnotesize}
取正交矩阵
      $$
      \T=(\yy_1,~~\yy_2,~~\yy_3)=\left(
      \begin{array}{ccc}
        \ds\frac{2\sqrt{5}}{5}&\ds\frac{2\sqrt{5}}{15}&\ds\frac13\\[0.1in]
        \ds-\frac{\sqrt{5}}{5}&\ds-\frac{4\sqrt{5}}{15}&\ds\frac23\\[0.1in]
        0&\ds\frac{\sqrt{5}}3&\ds-\frac23
      \end{array}
      \right)
      $$
      则
      $$
      \T^{-1}\A\T=\mathrm{diag}(2,2,-7).
      $$

  \end{footnotesize}
\end{frame}


\begin{frame}\ft{实对称矩阵的对角化}
  \begin{footnotesize}
    \begin{exampleblock}{例}
      设实对称矩阵$\A$和$\B$是相似矩阵,证明:存在正交矩阵$\PP$,使$\PP^{-1}\A\PP=\B$。
    \end{exampleblock}
    \pause\proofname
    $$
    \begin{array}{rl}
      \A\sim\B & \Longrightarrow~~
      \A,~\B\mbox{有相同的特征值$\lambda_1,\lambda_2,\cd,\lambda_n$}\\[0.1in]
      & \Longrightarrow~~ \exists\mbox{正交阵}\PP_1,\PP_2,~~ s.t.~~
      \PP_1^{-1}\A\PP_1=\mathrm{diag}(\lambda_1,\lambda_2,\cd,\lambda_n)=\PP_2^{-1}\A\PP_2\\[0.1in]
      & \Longrightarrow~~
      \PP_2\PP_1^{-1}\A\PP_1 \PP_2^{-1}=\B
    \end{array}
    $$
    取$\PP=\PP_1\PP_2^{-1}$,则$\PP$为正交阵,且
    $$
    \PP^{-1}\A\PP=\B
    $$
  \end{footnotesize}
\end{frame}

 


\begin{frame}\ft{实对称矩阵的对角化}
  \begin{footnotesize}
    \begin{exampleblock}{例}
      设$\A,\B$都是$n$阶实对称矩阵,若存在正交矩阵$\T$使$\T^{-1}\A\T,~\T^{-1}\B\T$都是对角阵,则$\A\B$是实对称矩阵。
    \end{exampleblock}
    \pause\proofname
    $$
    \begin{array}{rl}
    \left.
    \begin{array}{rl}
      \T^{-1}\A\T=\Lambdabd_1\\[0.1in]
      \T^{-1}\B\T=\Lambdabd_2
    \end{array}
    \right\} & \Longrightarrow~~
    (\T^{-1}\A\T)(\T^{-1}\B\T)=\Lambdabd_1\Lambdabd_2=\Lambdabd_2\Lambdabd_1=(\T^{-1}\B\T)(\T^{-1}\A\T)      \\[0.2in]
    & \Longrightarrow~~ \T^{-1}\A\B\T=\T^{-1}\B\A\T\\[0.2in]
    & \Longrightarrow~~ \A\B=\B\A\\[0.2in]
    & \Longrightarrow~~ (\A\B)^T=\B^T\A^T=\B\A=\A\B
    \end{array}
    $$
  \end{footnotesize}
\end{frame}


\begin{frame}\ft{实对称矩阵的对角化}
  \begin{scriptsize}
    \begin{exampleblock}{例$\bigstar\bigstar\bigstar$}
      三阶实对称矩阵$\A$的特征值为$\lambda_1=-1,\lambda_2=\lambda_3=1$,对应于$\lambda_1=-1$的特征向量为$\alphabd_1=(0,1,1)^T$,求$\A$。
    \end{exampleblock}\pause\proofname
    $$
    \A \sim \diag(-1,~1,~1)
    $$ 
    注意\red{不同特征值对应的特征向量正交},在与$\alphabd_1$正交的平面上取两个线性无关的向量,如$\alphabd_2=(1,0,0)^T,\alphabd_3=(0,1,-1)^T$,则
    $$
    \A(\alphabd_1,~\alphabd_2,~\alphabd_3)=(\alphabd_1,~\alphabd_2,~\alphabd_3)\left(
    \begin{array}{ccc}
      -1&&\\
      &1&\\
      &&1
    \end{array}
    \right)
    $$ 
    注意到$\alphabd_1,~\alphabd_2,~\alphabd_3$正交,单位化即得标准正交向量组
    $$
    \betabd_1=\frac1{\sqrt{2}}(0,1,1)^T,~~
    \betabd_2=(1,0,0)^T,~~
    \betabd_3=\frac1{\sqrt{2}}(0,1,-1)^T.
    $$ 
    令$\PP=(\betabd_1,~\betabd_2,~\betabd_3)$,则
    $$
    \A=\PP\Lambdabd\PP^{-1}=\PP\Lambdabd\PP^{T}
    = \left(
    \begin{array}{rrr}
      1&0&0\\
      0&0&-1\\
      0&-1&0
    \end{array}
    \right)
    $$
  \end{scriptsize}
\end{frame}


%%%%%%%%%%%%%%%%%%%%%%%%%%%%%%%
\subsection{往年试题}

\begin{frame} 
  \begin{footnotesize}
    \begin{exampleblock}{05-06上}
     设二阶方阵$\A$满足$\A^2-3\A+2\II=\zero$,求$\A$所有可能的特征值。
    \end{exampleblock}

\begin{exampleblock}{05-06下}
     设三阶方阵$\A$有三个实特征值$\lambda_1,\lambda_2,\lambda_3$,且$\lambda_1=\lambda_2\ne\lambda_3$,如果$\lambda_1$对应两个线性无关的特征向量$\alphabd_1$和$\alphabd_2$,$\lambda_3$对应一个特征向量$\alphabd_3$,证明$\alphabd_1,\alphabd_2,\alphabd_3$线性无关。
    \end{exampleblock}

\begin{exampleblock}{05-06下}
     设$\A=\left(
\begin{array}{ccc}
0&0&1\\
1&1&-1\\
x^2&0&0
\end{array}
\right)$,$x$为实数,试讨论$x$为何值时,$\A$可与对角阵相似?
    \end{exampleblock}
  \end{footnotesize}
\end{frame}


\begin{frame} 
  \begin{footnotesize}
    \begin{exampleblock}{06-07上,08-09上}     
     设$\A=\left(
\begin{array}{ccc}
1&k&1\\
1&0&1\\
0&1&0
\end{array}
\right)$,
\begin{itemize}
\item 当$k=1$时,是否存在正交矩阵$\QQ$,使得$\QQ^T\A\QQ$为对角阵?如果存在,是否唯一?
\item 当$k=0$时,$\A$能否与对角阵相似?

\end{itemize}•
    \end{exampleblock}


 \begin{exampleblock}{07-08上}     
     设$\A=\left(
\begin{array}{ccc}
2&0&0\\
1&2&-1\\
1&0&1
\end{array}
\right)$,
\begin{itemize}
\item 求$\A$的特征值和特征向量;
\item 求$\A^k$及其特征值和特征向量;
\end{itemize}•
    \end{exampleblock}
  \end{footnotesize}
\end{frame}


\begin{frame} 
  \begin{footnotesize}
 


 \begin{exampleblock}{07-08下}     
     已知$1,1,-1$是三阶实对称矩阵$\A$的三个特征值,向量$\alphabd_1=(1,1,1)^T,\alphabd_2=(2,2,1)^T$是$\A$的对应于$\lambda_1=\lambda_2=1$的特征向量。
\begin{itemize}
\item[(1)] 能否求出$\A$的属于$\lambda_3=-1$的特征向量?如能,试求出该特征向量,若不能,请说明理由;
\item[(2)]能否由此求得$\A$?若能,试求之,若不能请说明理由。

\end{itemize}•
    \end{exampleblock}



 \begin{exampleblock}{08-09上}     
     已知$\A$是三阶方阵,且$\A^2\ne \zero, \A^3=\zero$。
\begin{itemize}
\item[(1)] 能否求出$\A$的特征值?如能,试求出该特征值,若不能,请说明理由;
\item[(2)] $\A$能否对角化?若能,试求之,若不能请说明理由。
\item[(3)] 已知$\B=\A^3-5\A^2+3\II$,能否求得$\det(\B)$,若能,试求之,若不能请说明理由。

\end{itemize}•
    \end{exampleblock}



  \end{footnotesize}
\end{frame}



\begin{frame} 
  \begin{footnotesize}
 


 \begin{exampleblock}{09-10下}     
     设$\alphabd$是$n$维非零实列向量,$\A=\II-\frac2{\alphabd^T\alphabd}\alphabd\alphabd^T$,
\begin{itemize}
\item[(1)] 计算$\A^T$,并回答$k\II-\A$能否对角化?请说明理由,其中$k$为常数;
\item[(2)] 计算$\A^2$,并回答$k\II-\A$是否可逆? 请说明理由,其中$k\ne\pm 1$为常数;
\item[(3)] 给出$\II-2\alphabd\alphabd^T$为正交矩阵的充分必要条件。
\end{itemize}•
    \end{exampleblock}



 \begin{exampleblock}{08-09上}     
     已知$\A$是三阶方阵,且$\A^2\ne \zero, \A^3=\zero$。
\begin{itemize}
\item[(1)] 能否求出$\A$的特征值?如能,试求出该特征值,若不能,请说明理由;
\item[(2)] $\A$能否对角化?若能,试求之,若不能请说明理由。
\item[(3)] 已知$\B=\A^3-5\A^2+3\II$,能否求得$\det(\B)$,若能,试求之,若不能请说明理由。

\end{itemize}•
    \end{exampleblock}

 \end{footnotesize}
\end{frame}



\begin{frame} 
  \begin{footnotesize}

 \begin{exampleblock}{12-13下}     
     已知$\A$是三阶实对称阵,且$\A^2+2\A= \zero$,已知$\rr(\A)=2$。
\begin{itemize}
\item[(1)] 求$\A$的全部特征值? 
\item[(2)] 计算$\det(\A+4\II)$
\item[(3)] 当$k$为何值时,$\A+k\II$正定。

\end{itemize}•
    \end{exampleblock}



\begin{exampleblock}{12-13下}     
     已知三阶矩阵$\A$的特征值为$1,2,3$,求$\det(\A^3-5\A^2+7\A)$
    \end{exampleblock}

\begin{exampleblock}{12-13下}     
     证明:设$\A$为$n$阶非零实对称矩阵,则存在$n$维列向量$\xx$使得$\xx^T\A\xx\ne 0$.
    \end{exampleblock}


 \end{footnotesize}
\end{frame}



\begin{frame} 
  \begin{footnotesize}
\begin{exampleblock}{13-14上}     
     设$\alphabd=(a_1,a_2,a_3)^T,\betabd=(b_1,b_2,b_3)^T$,且$\alphabd^T\betabd=2,\A=\alphabd\betabd^T$,
\begin{itemize}
\item[(1)] 求$\A$的特征值;
\item[(2)] 求可逆阵$\PP$及对角阵$\Lambdabd$使得$\PP^{-1}\A\PP=\Lambdabd$。

\end{itemize}•
    \end{exampleblock}

  \end{footnotesize}
\end{frame}
