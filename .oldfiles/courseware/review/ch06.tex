\section{第六章~~二次型}

\subsection{知识点}

\begin{frame}\ft{二次型的定义和矩阵表示}
  \begin{footnotesize}
    \begin{block}{定义1(二次型)}
      $n$元变量$x_1,x_2,\cd,x_n$的二次齐次多项式
      $$
      \begin{array}{rcccccccc}
        f(x_1,x_2,\cd,x_n) &=& \\[0.1in]
        a_{11}x_1^2&+&2a_{12}x_1x_2&+&2a_{13}x_1x_3&+&\cd&+&2a_{1n}x_1x_n\\[0.1in]
        &+&a_{22}x_2^2&+&2a_{23}x_2x_3&+&\cd&+&2a_{2n}x_2x_n\\[0.1in]
        &&&&\cd&&\cd\\[0.1in]
        &&&&&&&+&a_{nn}x_n^2\\[0.1in]
      \end{array}
      $$
  其矩阵形式为
 $$
f(x_1,x_2,\cd,x_n)=   (x_1,x_2,\cd,x_n)\left(
      \begin{array}{cccc}
        a_{11}&a_{12}&\cd&a_{1n}\\
        a_{21}&a_{22}&\cd&a_{2n}\\
        \vd&\vd&&\vd\\
        a_{n1}&a_{n2}&\cd&a_{nn}\\
      \end{array}
      \right)\left(
      \begin{array}{c}
        x_1\\
        x_2\\
        \vd\\
        x_n
      \end{array}\right) = \xx^T\A\xx
$$
    \end{block}
  \end{footnotesize}
\end{frame}

 


\begin{frame}\ft{二次型的定义和矩阵表示}
  \begin{footnotesize}
    \begin{exampleblock}{例}
      设$f(x_1,x_2,x_3,x_4)=2x_1^2+x_1x_2+2x_1x_3+4x_2x_4+x_3^2+5x_4^2$,则它的矩阵为
      $$
      \A=\left(
      \begin{array}{cccc}
        2&\ds1/2&1&0\\[0.2cm]
        \ds1/2&0&0&2\\[0.2cm]
        1&0&1&0\\[0.2cm]
        0&2&0&5
      \end{array}
      \right)
      $$
    \end{exampleblock}
  \end{footnotesize}
\end{frame}
 

\begin{frame}\ft{二次型的定义和矩阵表示}
  \begin{footnotesize}
    设$\alphabd$在两组基$\{\epsilonbd_1,\epsilonbd_2,\cd,\epsilonbd_n\}$和$\{\etabd_1,\etabd_2,\cd,\etabd_n\}$下的坐标向量分别为
    $$
    \xx=(x_1,x_2,\cd,x_n)^T\mbox{~~和~~}\yy=(y_1,y_2,\cd,y_n)^T
    $$
    又
    $$
    (\etabd_1,\etabd_2,\cd,\etabd_n)=(\epsilonbd_1,\epsilonbd_2,\cd,\epsilonbd_n)\C
    $$
    故
    $$
    \xx=\C\yy
    $$
    从而
    $$
    f(\alphabd)=\xx^T\A\xx=\yy^T(\C^T\A\C)\yy
    $$ \pause


    \blue{
      二次型$f(\alphabd)$在两组基$\{\epsilonbd_1,\epsilonbd_2,\cd,\epsilonbd_n\}$和$\{\etabd_1,\etabd_2,\cd,\etabd_n\}$下所对应的矩阵分别为
      $$
      \A \mbox{~~和~~} \C^T\A\C     
      $$
    }
  \end{footnotesize}
\end{frame}


 
 


\begin{frame}\ft{矩阵的合同}
  \begin{footnotesize}
    \begin{block}{定义2(矩阵的合同)}
      对于两个矩阵$\A$和$\B$,若存在可逆矩阵$\C$,使得
      $$
      \C^T\A\C=\B,
      $$
      就称$\A$合同于$\B$,记作$\A\simeq\B$。
    \end{block}
  \end{footnotesize}
\end{frame}


\begin{frame}\ft{用正交变换法将二次型化为标准型}
  \begin{footnotesize}
    \begin{itemize}
    \item 含平方项而不含混合项的二次型称为\red{标准二次型}。\\[0.2cm]
    \item 化二次型为标准型,就是对实对称矩阵$\A$,寻找可逆阵$\C$,使$\C^T\A\C$成为对角矩阵。
    \end{itemize}
  \end{footnotesize}
\end{frame}

 


 

\begin{frame}\ft{用正交变换法将二次型化为标准型}
  \begin{footnotesize}
    \begin{block}{定理1(主轴定理)}
      对于任一个$n$元二次型
      $$
      f(x_1,x_2,\cd,x_n)=\xx^T\A\xx,
      $$
      存在正交变换$\xx=\QQ\yy$~($\QQ$为正交阵),使得
      $$
      \xx^T\A\xx=\yy^T(\QQ^T\A\QQ)\yy=\lambda_1y_1^2+\lambda_2y_2^2+\cd+\lambda_ny_n^2,
      $$
      其中$\lambda_1,\lambda_2,\cd,\lambda_n$为$\A$的$n$个特征值,
      $\QQ$的$n$个列向量$\alphabd_1,\alphabd_2,\cd,\alphabd_n$是$\A$对应于$\lambda_1,\lambda_2,\cd,\lambda_n$的标准正交特征向量。
    \end{block}
  \end{footnotesize}
\end{frame}



\begin{frame}\ft{用正交变换法将二次型化为标准型}
  \begin{footnotesize}
    \begin{exampleblock}{例$\bigstar\bigstar\bigstar\bigstar\bigstar$}
      用正交变换法,将二次型
      $$
      f(x_1,x_2,x_3)=2x_1^2+5x_2^2+5x_3^2+4x_1x_2-4x_1x_3-8x_2x_3
      $$
      化为标准型。
    \end{exampleblock}
    \pause\jiename
    对应方程为
    $$
    \A=\left(
    \begin{array}{rrr}
      2&2&-2\\
      2&5&-4\\
      -2&-4&5
    \end{array}
    \right)
    $$
    \pause
    其特征多项式为
    $$
    \det(\A-\lambda\II)=-(\lambda-1)^2(\lambda-10)
    $$
    得特征值$\lambda_{1,2}=1$和$\lambda_3=10$.
  \end{footnotesize}
\end{frame}


\begin{frame}\ft{用正交变换法将二次型化为标准型}
  \begin{footnotesize}
    $$
    \begin{array}{rl}
      (\A-\II)\xx=\zero & \Rightarrow~~
      \left(
      \begin{array}{rrr}
        1&2&-2\\
      2&4&-4\\
      -2&-4&4
      \end{array}
      \right)\left(
      \begin{array}{c}
        x_1\\
        x_2\\
        x_3
      \end{array}
      \right)=\zero\\[0.3in]  
      & \Rightarrow~~
      \xx_1=(-2,1,0)^T, \quad
      \xx_2=(2,0,1)^T. \\[0.2in] 
      (\A-10\II)\xx=\zero & \Rightarrow~~
      \left(
      \begin{array}{rrr}
       -8&2&-2\\
      2&-5&-4\\
      -2&-4&-5
      \end{array}
      \right)\left(
      \begin{array}{c}
        x_1\\
        x_2\\
        x_3
      \end{array}
      \right)=\zero\\[0.3in]  
      & \Rightarrow~~
      \xx_3=(1,2,-2)^T.
    \end{array}
    $$ \pause 

    对$\xx_1,\xx_2$用施密特正交化过程先正交化,再单位化,得
    $$
    \xibd_1=\left(-\frac{2\sqrt{5}}5,\frac{2\sqrt{5}}5,0\right)^T,~~~~
    \xibd_2=\left(\frac{2\sqrt{5}}{15},\frac{4\sqrt{5}}{15},\frac{\sqrt{5}}3\right)^T
    $$ 
    对$\xx_3$单位化,得
    $$
    \xibd_3=\left(\frac13,\frac23,-\frac23\right)^T
    $$
  \end{footnotesize}
\end{frame}


\begin{frame}\ft{用正交变换法将二次型化为标准型}
  \begin{footnotesize}
    取正交矩阵
    $$
    \QQ=(\xibd_1,\xibd_2,\xibd_3)=\left(
    \begin{array}{rrr}
      \ds-\frac{2\sqrt{5}}5&\ds\frac{2\sqrt{5}}{15}&\ds\frac13\\[0.2cm]
      \ds \frac{2\sqrt{5}}5&\ds\frac{4\sqrt{5}}{15}&\ds\frac23\\[0.2cm]
      \ds 0&\ds\frac{\sqrt{5}}3&\ds-\frac23
    \end{array}
    \right)
    $$
    则
    $$
    \QQ^{-1}\A\QQ=\QQ^{T}\A\QQ=\diag(1,1,10).
    $$ \pause 
    令$\xx=(x_1,x_2,x_3)^T,\yy=(y_1,y_2,y_3)^T$,\red{做正交变换$\xx=\QQ\yy$,原二次型就化成标准型}
    $$
    \xx^T\A\xx=\yy^T(\QQ^T\A\QQ)\yy=y_1^2+y_2^2+10y_3^2.
    $$
  \end{footnotesize}
\end{frame}



 




\begin{frame}\ft{惯性定理和二次型的规范形}
  \begin{footnotesize}
    \begin{block}{惯性定理}
      对于一个$n$元二次型$\xx^T\A\xx$,不论做怎样的坐标变换使之化为标准形,其中正平方项的项数$p$和负平方项的项数$q$都是唯一确定的。或者说,对一个$n$阶实对称矩阵$\A$,不论取怎样的可逆矩阵$\C$,只要使
      $$
      \C^T\A\C=\left(
      \begin{array}{cccccccccc}
        d_1&&&&&&&&\\
        &\dd&&&&&&&\\
        &&d_p&&&&&&\\
        &&&-d_{p+1}&&&&&\\
        &&&&\dd&&&&\\
        &&&&&-d_{p+q}&&&\\
        &&&&&&0&&\\
        &&&&&&&\dd&\\
        &&&&&&&&0
      \end{array}
      \right)
      $$
      其中$d_i>0(i=1,2,\cd,p+q),p+q\le n$成立,则$p$和$q$是由$\A$唯一确定的。
    \end{block}
  \end{footnotesize}
\end{frame}

 

\begin{frame}\ft{惯性定理和二次型的规范形}
  \begin{footnotesize}
    \begin{block}{定义}
      二次型$\xx^T\A\xx$的标准形中,
      \begin{itemize}
      \item  正平方项的项数(与$\A$合同的对角阵中正对角元的个数),称为二次型(或$\A$)的\red{正惯性指数};
      \item  负平方项的项数(与$\A$合同的对角阵中负对角元的个数),称为二次型(或$\A$)的\red{负惯性指数};
      \item
        正、负惯性指数的差称为符号差;
      \item 矩阵$\A$的秩也成为\red{二次型$\xx^T\A\xx$的秩}。
      \end{itemize}
    \end{block}
    \pause
    设$\rr(\A)=r$,正惯性指数为$p$,则
    \begin{itemize}
    \item 负惯性指数为$q=r-p$
    \item 符号差为$p-q=2p-r$
    \item与$\A$合同的对角阵的零对角元个数为$n-r$。
    \end{itemize}
    
  \end{footnotesize}
\end{frame}

\begin{frame}\ft{惯性定理和二次型的规范形}
  \begin{footnotesize}
    \begin{block}{推论}
      设$\A$为$n$阶实对称矩阵,若$\A$的正、负惯性指数分别为$p$和$q$,则
      $$\blue{
      \A\simeq\underbrace{\diag(\underbrace{1,\cd,1}_{p\mbox{个}},\underbrace{-1,\cd,-1}_{q\mbox{个}},\underbrace{0,\cd,0}_{n-p-q\mbox{个}})}_{\red{\ds\A\mbox{的合同规范形}}}
      }
      $$ \pause 
      或者说,对于二次型$\xx^T\A\xx$,存在坐标变换$\xx=\C\yy$,使得
      $$
      \blue{
        \xx^T\A\xx=\underbrace{y_1^2+\cd+y_p^2-y_{p+1}^2-\cd-y_{p+q}^2}_{\red{\ds\xx^T\A\xx\mbox{的规范形}}}.
      }
      $$
    \end{block}
  \end{footnotesize}
\end{frame}
 



\begin{frame}\ft{正定二次型和正定矩阵}
  \begin{footnotesize}
    \begin{block}{定义}
      如果对于任意的非零向量$\xx=(x_1,x_2,\cd,x_n)^T$,恒有
      $$
      \xx^T\A\xx=\sum_{i=1}^n\sum_{j=1}^na_{ij}x_ix_j>0,
      $$
      就称$\xx^T\A\xx$为正定二次型,称$\A$为正定矩阵。
    \end{block}
    \pause\vspace{0.1in}

    
    注:正定矩阵是针对对称矩阵而言的。
    
  \end{footnotesize}
\end{frame}

\begin{frame}
  \begin{footnotesize}
    \begin{block}{结论1}
      二次型$f(y_1,y_2,\cd,y_n)=d_1y_1^2+d_2y_2^2+\cd+d_ny_n^2$正定
      $~~~\Longleftrightarrow~~~d_i>0~~(i=1,2,\cd,n)$
    \end{block} 
    \begin{block}{结论2}
      一个二次型$\xx^T\A\xx$,经过非退化线性变换$\xx=\C\yy$,化为$\yy^T(\C^T\A\C)\yy$,其正定性保持不变。即当
      $$\xx^T\A\xx~~~\xLongleftrightarrow[]{\ds \xx=\C\yy}~~~\yy^T(\C^T\A\C)\yy\quad (\C\mbox{可逆})$$
      时,等式两端的二次型有相同的正定性。
    \end{block} 
  \end{footnotesize}
\end{frame}


\begin{frame}
  \begin{footnotesize}
    \begin{block}{定理}
      若$\A$是$n$阶实对称矩阵,则以下命题等价:
      \begin{itemize}
      \item[(1)]$\A$正定;
      \item[(2)]$\A$的正惯性指数为$n$,即$\A\simeq\II$;
      \item[(3)]存在可逆矩阵$\PP$使得$\A=\PP^T\PP$;
      \item[(4)]$\A$的$n$个特征值$\lambda_1,\lambda_2,\cd,\lambda_n$全大于零。
      \item[(5)]$\A$的$n$个顺序主子式全大于零。
      \end{itemize}
    \end{block}
     
    \begin{block}{定理}
      $$
      \A\mbox{正定}~~\Longrightarrow~~
      a_{ii}>0(i=1,2,\cd,n) \mbox{~~且~~}
      \det(\A)>0
      $$
    \end{block}
  \end{footnotesize}
\end{frame}

\begin{frame}
  \begin{footnotesize}
    \begin{exampleblock}{例}
      $\A\mbox{正定} ~~\Longrightarrow~~ \A^{-1}\mbox{正定}$
    \end{exampleblock}
%  \end{footnotesize}
%\end{frame}
%
%\begin{frame}
%  \begin{footnotesize}
    \begin{exampleblock}{例}
      判断二次型
      $$
      f(x_1,x_2,x_3)=x_1^2+2x_2^2+3x_3^2+2x_1x_2-2x_2x_3
      $$
      是否为正定二次型。
    \end{exampleblock}
%  \end{footnotesize}
%\end{frame}
%
%\begin{frame}
%  \begin{footnotesize}
    \begin{exampleblock}{例}
      判断二次型
      $$
      f(x_1,x_2,x_3)=3x_1^2+x_2^2+3x_3^2-4x_1x_2-4x_1x_3+4x_2x_3
      $$
      是否为正定二次型。
    \end{exampleblock}
  \end{footnotesize}
\end{frame}






%%%%%%%%%%%%%%%%%%%%%%%%%%%%%
\subsection{典型例题}


 \begin{frame}
   \begin{footnotesize}
    \begin{exampleblock}{2005-2006第一学期}
      求二次型
      $
      f(x_1,x_2,x_3)=(x_1+x_2)^2+(x_2-x_3)^2+(x_3+x_1)^2
      $
      的秩。
    \end{exampleblock}

   \begin{exampleblock}{2005-2006第一学期}
      设二次型
      $
      f(x_1,x_2,x_3)=x_1^2+x_2^2+x_3^2-2x_1x_2-2x_2x_3-2x_3x_1,
      $
      \begin{itemize}
\item[(1)] 求二次型$f$的矩阵$\A$的全部特征值;
\item[(2)] 求可逆矩阵$\PP$,使得$\PP^{-1}\A\PP$为对角阵;
\item[(3)] 计算$\det(\A^m)$.
\end{itemize}•
    \end{exampleblock}
  \end{footnotesize}
\end{frame}


 \begin{frame}
   \begin{footnotesize}
    \begin{exampleblock}{2005-2006第二学期}
      判断二次型
      $
      f(x_1,x_2,x_3)=x_1^2+2x_2^2+6x_3^2+2x_1x_2+2x_1x_2+6x_2x_3
      $
      的正定性。
    \end{exampleblock}

   \begin{exampleblock}{2006-2007第一学期}
      设二次型
      $
      f(x_1,x_2,x_3)=x_1^2+4x_2^2+2\lambda  x_1x_2-2x_1x_3+4x_2x_3,
      $
      试求该二次型的矩阵,并指出$\lambda$取何值时,$f$正定?
    \end{exampleblock}
  \end{footnotesize}
\end{frame}


 \begin{frame}
   \begin{footnotesize}
    \begin{exampleblock}{2006-2007第二学期}
      判断二次型
      $
      f(x,y,z)=3x^2+2y^2+2z^2+2xy+2xz
      $
       \begin{itemize}
\item[(1)] 用正交变换化二次型$f$为标准型,并写出相应的正交阵;
\item[(2)]  求$f(x,y,z)$在单位球面$x^2+y^2+z^2=1$上的最大值和最小值。
\end{itemize}•
    \end{exampleblock}

   \begin{exampleblock}{2006-2007第二学期}
      设二次型
      $
      f(x_1,x_2,x_3)=2x_1x_3+2x_1x_3-2x_2x_3,
      $
\begin{itemize}
\item[(1)] 写出二次型$f$的矩阵$\A$;
\item[(2)]  求出$\A$的全部特征值和特征向量;
\item[(3)] 化$f$为标准型;
\item[(4)] 判断$f$是否正定.
\end{itemize}•
    \end{exampleblock}
  \end{footnotesize}
\end{frame}


 \begin{frame}
   \begin{footnotesize}
    \begin{exampleblock}{2007-2008第一学期,2009-2010第一学期}
      对于二次型
      $
      f(x_1,x_2,x_3)=ax_1^2+2x_2^2-2x_3^2+2bx_1x_3(b>0),
      $其中二次型的矩阵$\A$的特征值之和为$1$,特征值之积为$-12$.
       \begin{itemize}
\item[(1)] 求$a,b$;
\item[(2)]  化$f$为标准型,并写出所用的正交变换和正交矩阵。
\end{itemize}•
    \end{exampleblock}

   \begin{exampleblock}{2007-2008第二学期}
      设二次型的矩阵为
      $
      \left(
\begin{array}{ccc}
5&-a&2b-1\\
a-b&c&2-c\\
c-2&-3&3
\end{array}
\right), a,b,c
      $为常数,则
\begin{itemize}
\item[(1)] 写出二次型$f$的具体形式;
\item[(2)]  求出$\A$的全部特征值和特征向量;
\item[(3)] 求正交变换$\xx=\PP\yy$,化$f$为标准型;
\item[(4)] 在$\|\xx\|=1$的条件下,求$f$的最大值和最小值.
\end{itemize}•
    \end{exampleblock}
  \end{footnotesize}
\end{frame}


 \begin{frame}
   \begin{footnotesize}
    \begin{exampleblock}{2008-2009第一学期}
      设二次型
      $
      f(x_1,x_2,x_3)=x_1^2+x_2^2+x_3^2+2ax_1x_2+2bx_2x_3+2x_1x_3,
      $经正交变换$\xx=\PP\yy$化为标准型$f=y_2^2+2y_3^2$,试求$a,b$。      
    \end{exampleblock}

   \begin{exampleblock}{2008-2009第一学期}
      设二次型$f(x_1,x_2,x_3)=x_1^2+x_2^2+x_3^2-2x_1x_2-2x_2x_3-2x_3x_1$,      
\begin{itemize}
\item[(1)] 求出$\A$的全部特征值和特征向量;
\item[(2)] 求正交变换$\xx=\PP\yy$,化$f$为标准型;
\item[(3)] 计算$\det(\A^m)$
\end{itemize}•
    \end{exampleblock}
  \end{footnotesize}
\end{frame}


 \begin{frame}
   \begin{footnotesize}
    \begin{exampleblock}{2009-2010第二学期}
      设二次型
      $
      f(x_1,x_2,x_3)=2x_1x_3+x_2^2
      $,\begin{itemize}
\item[(1)] 求出$\A$的全部特征值和特征向量;
\item[(2)] 求正交变换$\xx=\PP\yy$,化$f$为标准型。
\end{itemize}      
    \end{exampleblock}

   \begin{exampleblock}{2010-2011第一学期}
      设二次型$f(x_1,x_2,x_3)=4x_2^2-3x_3^2+4x_1x_2-4x_1x_3+8x_2x_3$,      
\begin{itemize}
\item[(1)] 写出$\A$;
\item[(2)] 求正交变换$\xx=\PP\yy$,化$f$为标准型。
\end{itemize}•
    \end{exampleblock}
  \end{footnotesize}
\end{frame}


\begin{frame}
   \begin{footnotesize}

    \begin{exampleblock}{2010-2011第二学期}
      设二次型
      $
      f(x_1,x_2,x_3)=x_1^2+2x_2^2+x_3^2+2tx_1x_2+2x_1x_3
      $的矩阵是奇异阵,
\begin{itemize}
\item[(1)] 写出$\A$并求$t$的值;
\item[(2)] 根据所求$t$的值,求一个可逆矩阵$\PP$和一个对角阵$\Lambdabd$,使得$\PP^{-1}\A\PP=\Lambdabd$;
\item[(3)] 求$\A^n(n\ge 2)$.
\end{itemize}      
    \end{exampleblock}

   \begin{exampleblock}{2011-2012第二学期}
      在正交变换$\xx=\QQ\yy$将二次型$f(x_1,x_2,x_3)=2x_1x_2+2x_1x_3+2x_2x_3$化为标准型。
    \end{exampleblock}
  \end{footnotesize}
\end{frame}

\begin{frame}
   \begin{footnotesize}

    \begin{exampleblock}{2012-2013第二学期}
      已知二次型
      $
      f(x_1,x_2,x_3)=(1-a)x_1^2+(1-a)x_2^2+2x_3^2+2(1+a)x_1x_2 
      $的秩为2,
\begin{itemize}
\item[(1)] 求$a$;
\item[(2)] 求正交变换$\xx=\PP\yy$,将$f$化为标准型.
\end{itemize}      
    \end{exampleblock}

   \begin{exampleblock}{2012-2013第二学期}
已知二次型
      $
      f(x_1,x_2,x_3)=2x_1^2+3x_2^2+3x_3^2+4x_2x_3 
      $的秩为2,
\begin{itemize}
\item[(1)] 把$f$写成$f=\xx^T\A\xx$的形式;
\item[(2)] 求$\A$的特征值和特征向量;
\item[(3)] 求正交变换$\xx=\PP\yy$,将$f$化为标准型.
\end{itemize}      
    \end{exampleblock}
  \end{footnotesize}
\end{frame}

\begin{frame}
   \begin{footnotesize}

    \begin{exampleblock}{2013-2014第一学期}
      用正交变换化二次型
      $
      f(x_1,x_2,x_3)=x_1^2+x_2^2+2x_3^2-2x_1x_2+4x_1x_3+4x_2x_3$为标准型.
    \end{exampleblock}

  
  \end{footnotesize}
\end{frame}






