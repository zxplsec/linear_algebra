\subsection{知识点}

\begin{frame}\ft{线性相关性}
  \begin{footnotesize}
    设$m\times n$矩阵$\A$按列分块为
    $$\A=(\alphabd_1,\alphabd_2,\cd,\alphabd_n),$$
    其中$\alphabd_i(i=1,2,\cd,n)$为$m$维列向量,则线性组合
    $$
    x_1\alphabd_1+x_2\alphabd_2+\cd+x_n\alphabd_n
    $$
    可表示为矩阵形式
    $$ 
    (\alphabd_1,\alphabd_2,\cd,\alphabd_n)\left(
    \begin{array}{c}
      x_1\\x_2\\\vd\\x_n
    \end{array}
    \right)
    $$
  \end{footnotesize}
\end{frame}


\begin{frame}\ft{线性相关性与齐次线性方程组的解}
  \begin{footnotesize}
    \begin{block}{重要结论1}
      向量组$\alphabd_1,\alphabd_2,\cd,\alphabd_n$线性相关,等价于齐次方程组
        $$
        (\alphabd_1,\alphabd_2,\cd,\alphabd_n)\left(
        \begin{array}{c}
          x_1\\x_2\\\vd\\x_n
        \end{array}
        \right)=\zero
        $$
        有非零解,也等价于
        $$
        \rr(\A)=\rr(\alphabd_1,\alphabd_2,\cd,\alphabd_n)<n=\mbox{矩阵$\A$的列数}.
        $$
    \end{block}
\end{footnotesize}
\end{frame}

\begin{frame}\ft{线性相关性与齐次线性方程组的解}
  \begin{footnotesize}
    \begin{block}{重要结论2}  向量组$\alphabd_1,\alphabd_2,\cd,\alphabd_n$线性无关,等价于齐次方程组
        $$
        (\alphabd_1,\alphabd_2,\cd,\alphabd_n)\left(
        \begin{array}{c}
          x_1\\x_2\\\vd\\x_n
        \end{array}
        \right)=\zero
        $$
        只有零解,也等价于
        $$
        \rr(\A)=\rr(\alphabd_1,\alphabd_2,\cd,\alphabd_n)=n=\mbox{矩阵$\A$的列数}.
        $$

    \end{block}
  \end{footnotesize}
\end{frame}


\begin{frame}\ft{线性表示与非齐次线性方程组的解}
  \begin{footnotesize}
    \begin{block}{重要结论3} 
      向量$\bb$可由$\alphabd_1,\alphabd_2,\cd,\alphabd_n$线性表示,等价于方程组
      $$
      (\alphabd_1,\alphabd_2,\cd,\alphabd_n)\left(
        \begin{array}{c}
          x_1\\x_2\\\vd\\x_n
        \end{array}
        \right)=\bb
      $$
      有解,也等价于
      $$
      \rr(\A,\bb)=\rr(\A).
      $$ 
    \end{block}
  \end{footnotesize}
\end{frame}


\begin{frame}\ft{线性表示与非齐次线性方程组的解}
  \begin{footnotesize}

    \begin{block}{重要结论4} 
      向量$\bb$可由$\alphabd_1,\alphabd_2,\cd,\alphabd_n$惟一地线性表示,等价于方程组
      $$
      (\alphabd_1,\alphabd_2,\cd,\alphabd_n)\left(
        \begin{array}{c}
          x_1\\x_2\\\vd\\x_n
        \end{array}
        \right)=\bb
      $$
      有惟一解,也等价于
      $$
      \rr(\A,\bb)=\rr(\A)=\A\mbox{的列数}.
      $$ 
\end{block}
  \end{footnotesize}
\end{frame}


\begin{frame}\ft{线性相关性}
  \begin{scriptsize}
    \begin{block}{重要结论5} 
      关于向量组的线性相关性,有如下结论:
      \begin{itemize}
      \item 部分相关,则整体相关;整体无关,则部分无关。
      \item
        $$
        \begin{aligned}
          \left(
          \begin{array}{c}
            a_{11}\\
            \vd\\
            a_{n1}\\
          \end{array}
          \right)
          \cd
          \left(
          \begin{array}{c}
            a_{1s}\\
            \vd\\
            a_{ns}\\
          \end{array}
          \right) \mbox{无关}  ~~~\blue{\Rightarrow}~~~
          \left(
          \begin{array}{c}
            a_{11}\\
            \vd\\
            a_{n1}\\
            \red{a_{n+1,1}}\\
            \vd\\
            \red{a_{n+m,1}}
          \end{array}
          \right)
          \cd
          \left(
          \begin{array}{c}
            a_{1s}\\
            \vd\\
            a_{ns}\\
            \red{a_{n+1,s}}\\
            \vd\\
            \red{a_{n+m,s}}
          \end{array}
          \right) \mbox{无关} \\          
          \left(
          \begin{array}{c}
            a_{11}\\
            \vd\\
            a_{n1}\\
            \red{a_{n+1,1}}\\
            \vd\\
            \red{a_{n+m,1}}
          \end{array}
          \right)
          \cd
          \left(
          \begin{array}{c}
            a_{1s}\\
            \vd\\
            a_{ns}\\
            \red{a_{n+1,s}}\\
            \vd\\
            \red{a_{n+m,s}}
          \end{array}
          \right) \mbox{相关}  ~~~\blue{\Rightarrow}~~~
          \left(
          \begin{array}{c}
            a_{11}\\
            \vd\\
            a_{n1}\\
          \end{array}
          \right)
          \cd
          \left(
          \begin{array}{c}
            a_{1s}\\
            \vd\\
            a_{ns}\\
          \end{array}
          \right) \mbox{相关}
        \end{aligned}
        $$

      \end{itemize}
    \end{block}

  \end{scriptsize}
\end{frame}

\begin{frame}\ft{向量组和矩阵的秩}
  \begin{footnotesize}
    \begin{block}{向量组的秩}
      设有向量组$\alphabd_1,\alphabd_2,\cd,\alphabd_s$。
      如果能从其中选出$r$个向量$\alphabd_1,\alphabd_2,\cd,\alphabd_{\red{r}}$,满足
      \begin{itemize}
      \item 向量组$\alphabd_1,\alphabd_2,\cd,\alphabd_{\red{r}}$线性无关;
      \item 向量组$\alphabd_1,\alphabd_2,\cd,\alphabd_s$中任意$r+1$个向量都线性相关,
      \end{itemize}
      则称向量组$\alphabd_1,\alphabd_2,\cd,\alphabd_{\red{r}}$为原向量组的一个\red{极大线性无关组},简称\red{极大无关组}。
      \vspace{0.1in}

      \blue{\underline{极大线性无关组所含向量的个数$\red{r}$}},称为原向量组的\red{秩}。     
    \end{block}

    \begin{block}{矩阵的秩}
      矩阵的行秩或列秩的数值,称为矩阵的秩。
    \end{block}

  \end{footnotesize}
\end{frame}


\begin{frame}\ft{向量组的秩}
  \begin{footnotesize}
    \begin{block}{重要结论6}
      设
      $$\blue{\rr(\alphabd_1,\alphabd_2,\cd,\alphabd_s)=p,~~\rr(\betabd_1,\betabd_2,\cd,\betabd_t)=r},$$
      如果向量组$\blue{B:~\betabd_1,\betabd_2,\cd,\betabd_t}$可由$\blue{A:~\alphabd_1,\alphabd_2,\cd,\alphabd_s}$线性表示,则
      $$\red{r\le p.}$$
    \end{block}

    以上结论中,向量组$\blue{B:~\betabd_1,\betabd_2,\cd,\betabd_t}$可看作是向量组$\blue{A:~\alphabd_1,\alphabd_2,\cd,\alphabd_s}$的一个线性组合。由此可知
    \begin{block}{$\bigstar\bigstar\bigstar\bigstar\bigstar$}      
      \red{对向量组进行线性组合,秩不变或减少。}
    \end{block}
  \end{footnotesize}
\end{frame}


\begin{frame}\ft{矩阵的秩}
  \begin{footnotesize}
    \begin{block}{性质1}
      $$
      \red{\max\{\rr(\A),~\rr(\B)\}~~\le~~ \rr(\A,~\B) ~~\le~~ \rr(\A) + \rr(\B).}
      $$
      
      $$
      \blue{\rr(\A)~~\le~~\rr(\A,~\bb)~~\le~~\rr(\A)+1.}
      $$
    \end{block} 
    
    设$\rr(\A)=p, ~\rr(\B)=q$,
    $\A$和$\B$的列向量组的极大无关组分别为
    $$
    \alphabd_1,~\cd,~\alphabd_p \mbox{~~和~~}
    \betabd_1,~\cd,~\betabd_q. 
    $$ 
    显然$(\A,\B)$的列向量组可由向量组$\alphabd_1,~\cd,~\alphabd_p,~
    \betabd_1,~\cd,~\betabd_q$线性表示。

   \end{footnotesize}
\end{frame}


\begin{frame}\ft{矩阵的秩}
  \begin{footnotesize}
    \begin{block}{注}
      \begin{itemize}
      \item         
        $$
        \red{\min\{\rr(\A),~\rr(\B)\} ~~\le~~ \rr(\A,~\B)}
        $$
        意味着:\blue{在$\A$的右侧添加新的列,只有可能使得秩在原来的基础上得到增加;当$\B$的列向量能被$\A$的列向量线性表示时,等号成立。}\\[0.1in]
      \item 
        $$
        \red{\rr(\A,~\B) ~~\le~~ \rr(\A)+\rr(\B)}
        $$
        意味着:\blue{对$(\A,~\B)$,有可能$\A$的列向量与$\B$的列向量出现线性相关,合并为$(\A,~\B)$的秩一般会比$\rr(\A)+\rr(\B)$要小。}
      \end{itemize}
    \end{block}
  \end{footnotesize}
\end{frame}


\begin{frame}\ft{矩阵的秩}
  \begin{footnotesize}
     \begin{block}{性质2}
      $$
      \red{\rr(\A+\B) \le \rr(\A)+\rr(\B).}
      $$
    \end{block}

     
    设$\rr(\A)=p, ~\rr(\B)=q$,
    $\A$和$\B$的列向量组的极大无关组分别为
    $$
    \alphabd_1,~\cd,~\alphabd_p \mbox{~~和~~}
    \betabd_1,~\cd,~\betabd_q. \pause 
    $$ 
    显然$\A+\B$的列向量组可由向量组$\alphabd_1,~\cd,~\alphabd_p,~
    \betabd_1,~\cd,~\betabd_q$线性表示。


    \begin{block}{注}
      将矩阵$\A$和$\B$合并、相加,秩不变或减小。
    \end{block}
  \end{footnotesize}
\end{frame}


\begin{frame}\ft{矩阵的秩}
  \begin{footnotesize}
    \begin{block}{性质3}
      $$
      \red{\rr(\A\B) \le \min(\rr(\A),~\rr(\B)).}
      $$
    \end{block}
    \pause
    \proofname
    设$\A,\B$分别为$m\times n, n\times s$矩阵,将$\A$按列分块,则
    $$
    \A\B = (\alphabd_1,~\cd,~\alphabd_n) \left(
    \begin{array}{cccc}
      b_{11}&b_{12}&\cd&b_{1s}\\
      b_{21}&b_{22}&\cd&b_{2s}\\
      \vd&\vd&&\vd\\
      b_{n1}&b_{n2}&\cd&b_{ns}
    \end{array}
    \right).
    $$  
    由此可知,$\A\B$的列向量组可由$\alphabd_1,~\alphabd_2,~\cd,~\alphabd_n$线性表示,故
    $$
    \rr(\A\B) = \A\B\mbox{的列秩} \le \A\mbox{的列秩} = \rr(\A).
    $$
    \pause
    类似地,将$\B$按行分块,可得$$\rr(\A\B)\le \rr(\B).$$
      \end{footnotesize}
\end{frame}

\begin{frame}\ft{矩阵的秩}
  \begin{footnotesize}
    \begin{block}{性质4}
      设$\A$为$m\times n$矩阵,$\PP,\QQ$分别为$m$阶、$n$阶可逆矩阵,则
      $$
      \rr(\A) = \rr(\PP\A) = \rr(\A\QQ)  = \rr(\PP\A\QQ).
      $$
    \end{block}
  \end{footnotesize}
\end{frame}



\begin{frame}\ft{齐次线性方程组解的结构}
  \begin{footnotesize}
    \begin{block}{定义(基础解系)}
      设$\xx_1,~\xx_2,~\cd,~\xx_p$为$\A\xx=\zero$的解向量,若
      \begin{itemize}
      \item[(1)] $\xx_1,~\xx_2,~\cd,~\xx_p$线性无关
      \item[(2)] $\A\xx=\zero$的任一解向量可由$\xx_1,~\xx_2,~\cd,~\xx_p$线性表示。
      \end{itemize}
      则称$\xx_1,~\xx_2,~\cd,~\xx_p$为$\A\xx=\zero$的一个\blue{\underline{基础解系}}。
    \end{block}

  \begin{block}{$\bigstar\bigstar\bigstar\bigstar\bigstar$}
      设$\A$为$m\times n$矩阵,若$\rr(\A)=r<n$,则齐次线性方程组$\A\xx=\zero$存在基础解系,
      且基础解系含$n-r$个解向量。
    \end{block}


  齐次线性方程组的全部解可由基础解系给出:
  $$
  k_1\xx_1+k_2\xx_2+\cd+k_p\xx_p \quad(k_1,k_2,\cd,k_p\mbox{为任意常数}).
  $$
  \end{footnotesize}
\end{frame}

\begin{frame}\ft{非齐次线性方程组}
  \begin{footnotesize}
    \begin{block}{非齐次线性方程组解的结构}
      “$\A\xx=\bb$的通解” =  “$\A\xx=\zero$的通解” + “$\A\xx=\bb$的特解”
    \end{block}
  \end{footnotesize}
\end{frame}



\begin{frame}\ft{几类重要的矩阵}
  \begin{footnotesize}
    \begin{block}{阶梯形矩阵}
      若矩阵$\A$满足
      \begin{itemize}
      \item[(1)] 零行在最下方;
      \item[(2)] 非零行首元的列标号随行标号的增加而严格递增,
      \end{itemize}
      则称$\A$为\red{阶梯形矩阵}。
    \end{block}
    \pause
    \begin{exampleblock}{例}
      $$
      \left(
      \begin{array}{rrrr}
        2&0&2&1\\
        0&5&2&-2\\
        0&0&3&2\\
        0&0&0&0
      \end{array}
      \right)
      $$
    \end{exampleblock}
  \end{footnotesize}
\end{frame}


\begin{frame}\ft{几类重要的矩阵}
  \begin{footnotesize}
    \begin{block}{行简化阶梯形矩阵}
      若矩阵$\A$满足
      \begin{itemize}
      \item[(1)] 它是阶梯形矩阵;
      \item[(2)] 非零行首元所在的列除了非零行首元外,其余元素全为零,
      \end{itemize}
      则称$\A$为\red{行简化阶梯形矩阵}。
    \end{block}
    \pause
    \begin{exampleblock}{例}
      $$
      \left(
      \begin{array}{rrrr}
        2&0&0&1\\
        0&5&0&-2\\
        0&0&3&2\\
        0&0&0&0
      \end{array}
      \right)
      $$
    \end{exampleblock}
  \end{footnotesize}
\end{frame}

