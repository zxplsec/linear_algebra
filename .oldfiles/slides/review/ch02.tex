\section{矩阵}
\subsection{矩阵的定义}

\begin{frame}
  对于一般的线性方程组
  $$
  \left\{
    \begin{array}{c}
      a_{11}x_1 + a_{12}x_2 + \cd + a_{1n}x_n = b_1\\[0.2cm]
      a_{21}x_1 + a_{22}x_2 + \cd + a_{2n}x_n = b_2\\[0.2cm]
      \vd\\[0.2cm]
      a_{m1}x_1 + a_{m2}x_2 + \cd + a_{mn}x_n = b_m
    \end{array}
  \right.
  $$

  其增广矩阵为
  \begin{figure}[htbp]
    \centering
    \begin{tikzpicture}
      \matrix(MM) [matrix of math nodes,nodes in empty cells,
      column sep=1ex,row sep=.5ex,ampersand replacement=\&,left delimiter=(,right delimiter=)] {
        a_{11} \&  a_{12} \&  \cd \& a_{1n} \& \&  b_1\\
        a_{21} \&  a_{22} \&  \cd \& a_{2n} \& \&  b_2\\
        \vd   \&  \vd   \&      \&  \vd  \& \& \vd \\
        a_{m1} \&  a_{m2} \&  \cd \& a_{mn} \& \&  b_m\\
      };
      \draw[dashed] (MM-1-5.north) -- (MM-4-5);        
    \end{tikzpicture}      
  \end{figure}
\end{frame}

\begin{frame}
  \begin{small}
    对于以上增广矩阵,总是可以经过一系列的变换将其化成
    \begin{figure}[htbp]
      \centering
      \begin{tikzpicture}
        \matrix(MM) [matrix of math nodes,nodes in empty cells,
        column sep=0.6ex,row sep=.5ex,ampersand replacement=\&,left delimiter=(,right delimiter=)] {
          a_{11} \&  a_{12} \&  \cd \& a_{1n} \& \&  b_1\\
          a_{21} \&  a_{22} \&  \cd \& a_{2n} \& \&  b_2\\
          \vd   \&  \vd   \&      \&  \vd  \& \& \vd \\
          a_{m1} \&  a_{m2} \&  \cd \& a_{mn} \& \&  b_m\\
        };
        \draw[dashed] (MM-1-5.north) -- (MM-4-5);
        
        \matrix (M2) [right=.05in of MM,matrix of math nodes]  { 
          \Rightarrow\\
        };
        
        \matrix(MM) [right=.05in of M2,matrix of math nodes,nodes in empty cells,
        column sep=0.6ex,row sep=.5ex,ampersand replacement=\&,left delimiter=(,right delimiter=)] {
          c_{11} \&   0 \& \cd \&  0 \& c_{1,r+1} \& \cd \& c_{1n} \& \& d_1\\
          0   \& c_{22} \& \cd \&  0 \& c_{2,r+1} \& \cd \& c_{2n} \& \& d_2\\
          \vd \& \vd \& \dd \&\vd \& \vd     \&     \& \vd   \& \& \vd\\
          0   \&  0  \& \cd \& c_{rr} \& c_{r,r+1} \& \cd \& c_{rn} \& \& d_r\\
          0   \&  0  \& \cd \& 0 \& 0 \& \cd \& 0 \&  \& d_{r+1}\\
          0   \&  0  \& \cd \& 0 \& 0 \& \cd \& 0 \&  \& 0\\          
          \vd \& \vd \& \dd \&\vd \& \vd     \&     \& \vd   \& \& \vd\\
          0   \&  0  \& \cd \& 0 \& 0 \& \cd \& 0 \&  \& 0\\
        };
        \draw[dashed] (MM-1-8.north) -- (MM-8-8);
        \filldraw[opacity=0.5,red!50] (MM-5-9) circle (0.3cm);
      \end{tikzpicture}      
    \end{figure}
    其中$c_{ii}=1~(i=1,2,\cd,r)$。
  \end{small}

\end{frame}

\begin{frame}
对应线性方程组解的情况如下:   
\begin{itemize}
\item[1] 线性方程组有解$ \Leftrightarrow \red{d_{r+1}=0}$;\\[0.3cm]
\item[2] 在有解的情况下:
  \begin{itemize}
  \item 当$r=n$时,有唯一解$x_1=d_1,~x_2=d_2,~\cd,~x_n=d_n$;
  \item 当$r<n$时,有无穷多解
    $$
    \left\{
      \begin{array}{ccl}
        x_1 &=& d_1 - c_{1,r+1}k_1 - \cd - c_{1n}k_{n-r}, \\[0.1cm]
        x_2 &=& d_2 - c_{2,r+1}k_1 - \cd - c_{2n}k_{n-r}, \\[0.1cm]
        \vd & & \vd \\[0.1cm]
        x_r &=& d_r - c_{r,r+1}k_1 - \cd - c_{rn}k_{n-r}, \\[0.1cm]
        x_{r+1} &=& k_1, \\[0.1cm]
        \vd && \vd \\[0.1cm]
        x_{n} &=& k_{n-r},
      \end{array}
    \right.
    $$
    其中$k_1,k_2,\cd,k_{n-r}$为任意常数。
  \end{itemize}
  \end{itemize}
\end{frame}

\subsection{矩阵的加法、数乘及乘法}

\begin{frame}
  \begin{dingyi}[矩阵的加法]
  设有两个$m\times n$矩阵$\A=(a_{ij})$和$\B=(b_{ij})$,则矩阵$\A$与$\B$之和记为$\A+\B$,规定为
  $$
  \A + \B = 
  \left(
    \begin{array}{cccc}
      a_{11} + b_{11}  & a_{12} + b_{12}  & \cd & a_{1n} + b_{1n}  \\[0.2cm]
      a_{21} + b_{21}  & a_{22} + b_{22}  & \cd & a_{2n} + b_{2n}  \\[0.2cm]
      \cd            & \cd            &     & \cd            \\[0.2cm]
      a_{n1} + b_{n1}  & a_{n2} + b_{n2}  & \cd & a_{nn} + b_{nn}  
    \end{array}
  \right)
  $$
\end{dingyi}

\begin{zhu*}
  \red{只有当两个矩阵同型时才能进行加法运算。}
\end{zhu*}
\end{frame}

\begin{frame}
\begin{dingyi}[矩阵的数乘]
  数$k$与矩阵$\A$的乘积记作$k \A$或$\A k$,规定为
  $$
  k \A = 
  \left(
    \begin{array}{cccc}
      k a_{11}   & k a_{12}   & \cd & k a_{1n}  \\[0.2cm]
      k a_{21}   & k a_{22}   & \cd & k a_{2n}  \\[0.2cm]
      \cd     & \cd     &     & \cd    \\[0.2cm]
      k a_{m1}   & k a_{m2}   & \cd & k a_{mn}  
    \end{array}
  \right)
  $$
\end{dingyi}

\begin{zhu*}
  \red{用数$k$乘一个矩阵,需要把数$k$乘矩阵的每一个元素,这与行列式的数乘性质不同。}
\end{zhu*}
\end{frame}

\begin{frame}
\begin{dingyi}[矩阵乘法]
  设$A$为$m\times n$矩阵,$B$为$n\times s$矩阵,即
  $$
  A = \left(
    \begin{array}{cccc}
      a_{11} & a_{12} & \cd & a_{1n}\\
      a_{21} & a_{22} & \cd & a_{2n}\\
      \vd   & \vd   &     & \vd \\
      a_{m1} & a_{m2} & \cd & a_{mn}
    \end{array}
  \right), ~~
  B = \left(
    \begin{array}{cccc}
      b_{11} & b_{12} & \cd & b_{1s}\\
      b_{21} & b_{22} & \cd & b_{2s}\\
      \vd   & \vd   &     & \vd \\
      b_{n1} & b_{n2} & \cd & b_{ns}
    \end{array}
  \right)
  $$
  则$A$与$B$之乘积$AB$(记为$C=(c_{ij})$)为$m\times s$矩阵,且
  $$
  c_{ij} = c_{i1}b_{1j} + c_{i2}b_{2j} + \cd + c_{in}b_{nj} = \sum_{k=1}^na_{ik}b_{kj}.
  $$
\end{dingyi}
\begin{zhu*}
  两个矩阵$A$与$B$相乘有意义的前提是\red{$A$的列数等于$B$的行数}。
\end{zhu*}
\end{frame}

\begin{frame}
\begin{li}
  设
  $$
  \A = \left(
    \begin{array}{c}
      a_1\\
      a_2\\
      \vd\\
      a_n
    \end{array}
  \right), ~~
  \B = \left(
    \begin{array}{cccc}
      b_1 & b_2 & \cd & b_n
    \end{array}
  \right)
  $$
  计算$\A\B$与$\B\A$.
\end{li}
\begin{jie}
  $$
  \A\B = \left(
    \begin{array}{c}
      a_1\\
      a_2\\
      \vd\\
      a_n
    \end{array}
  \right)\left(
    \begin{array}{cccc}
      b_1 & b_2 & \cd & b_n
    \end{array}
  \right) 
  = \left(
    \begin{array}{cccc}
      a_1b_1 & a_1b_2 & \cd & a_1b_n\\
      a_2b_1 & a_2b_2 & \cd & a_2b_n\\
      \vd & \vd & & \vd\\
      a_nb_1 & a_nb_2 & \cd & a_nb_n
    \end{array}
  \right)
  $$  
  $$
  \B\A = \left(
    \begin{array}{cccc}
      b_1 & b_2 & \cd & b_n
    \end{array}
  \right)\left(
    \begin{array}{c}
      a_1\\
      a_2\\
      \vd\\
      a_n
    \end{array}
  \right)  
  = a_1b_1+a_2b_2+\cd+a_nb_n.
  $$
\end{jie}
\end{frame}

\begin{frame}
 矩阵乘法不满足交换律。
\begin{itemize}
\item 若$\A\B\ne\B\A$,则称\red{$\A$与$\B$不可交换}。
\item 若$\A\B=\B\A$,则称\red{$\A$与$\B$可交换}。  
\end{itemize}
\end{frame}

\begin{frame}
  \begin{li}[$\bigstar$]
    已知矩阵$\left[
    \begin{array}{ccc}
      1&2&3\\
      a_1&a_2&a_3\\
      b_1&b_2&b_3
    \end{array}\right]
    $与$\left[
    \begin{array}{ccc}
      0&1&0\\
      0&0&1\\
      1&0&0
    \end{array}\right]
    $可交换。试求$$
    \left|
    \begin{array}{ccc}
      1&2&3\\
      a_1&a_2&a_3\\
      b_1&b_2&b_3
    \end{array}
    \right|.
    $$
  \end{li}
  \pause
  \begin{jie}
    由两矩阵可交换知$a_2=b_3=1, b_1=a_3=2, a_1=b_2=3$. 于是所求行列式为
    $$
    \left|
    \begin{array}{ccc}
      1&2&3\\
      3&1&2\\
      2&3&1
    \end{array}
    \right|=6\left|
    \begin{array}{ccc}
      1&2&3\\
      1&1&2\\
      1&3&1
    \end{array}
    \right|=6\left|
    \begin{array}{ccc}
      1&2&3\\
      0&-1&-1\\
      0&1&-2
    \end{array}
    \right|=18.
    $$
    
  \end{jie}
\end{frame}

\begin{frame}
\begin{dingyi}[单位矩阵与数量矩阵]
  \begin{itemize}
  \item[1] 主对角元全为1,其余元素全为零的$n$阶方阵,称为$n$阶\red{单位矩阵},记为$\II_n, \II, \E$
    $$
    \II_n = \left(
      \begin{array}{cccc}
        1 & & &\\
          & 1 & & \\
          & & \dd & \\
          & & & 1
      \end{array}
    \right)
    $$ 
  \item[2] 主对角元全为非零数$k$,其余元素全为零的$n$阶方阵,称为$n$阶\red{数量矩阵},记为$k\II_n, k\II, k\E$
    $$
    k\II_n = \left(
      \begin{array}{cccc}
        k & & &\\
          & k & & \\
          & & \dd & \\
          & & & k
      \end{array}
    \right)~(k\ne 0)
    $$
  \end{itemize}
\end{dingyi}
\end{frame}

\begin{frame}
\begin{dingyi}[对角矩阵]
  非对角元皆为零的$n$阶方阵称为$n$阶\red{对角矩阵},记作$\Lambdabd$,即
  $$
  \Lambdabd = \left(
    \begin{array}{cccc}
      \lambda_1 & & &\\
                & \lambda_2 & & \\
                & & \dd & \\
                & & & \lambda_n
    \end{array}
  \right)
  $$
  或记作$\mathrm{diag}(\lambda_1,\lambda_2,\cd,\lambda_n)$.
\end{dingyi}
\end{frame}

\begin{frame}
\begin{zhu*}
  \begin{itemize}
  \item[1] 用对角阵$\Lambdabd$左乘$\A$,就是用$\lambda_i(i=1,\cd,n)$乘$\A$中第$i$行的每个元素;
    \begin{footnotesize}
    $$
    \left(
      \begin{array}{cccc}
        \lambda_1 & & &\\
                  & \lambda_2 & & \\
                  & & \dd & \\
                  & & & \lambda_n
      \end{array}
    \right)
    \left(
      \begin{array}{cccc}
        a_{11} & a_{12} & \cd & a_{1n}\\
        a_{21} & a_{22} & \cd & a_{2n}\\
        \vd & \vd &  & \vd\\
        a_{n1} & a_{n2} & \cd & a_{nn}
      \end{array}
    \right) = 
    \left(
      \begin{array}{cccc}
        \lambda_1 a_{11} & \lambda_1a_{12} & \cd & \lambda_1a_{1n}\\
        \lambda_2a_{21} & \lambda_2a_{22} & \cd & \lambda_2a_{2n}\\
        \vd & \vd &  & \vd\\
        \lambda_na_{n1} & \lambda_na_{n2} & \cd & \lambda_na_{nn}
      \end{array}
    \right)
    $$
  \end{footnotesize} \vspace{.1in}
  
  \item[2] 用对角阵$\Lambdabd$右乘$\A$,就是用$\lambda_i(i=1,\cd,n)$乘$\A$中第$i$列的每个元素,即
    \begin{footnotesize}
    $$
    \left(
      \begin{array}{cccc}
        a_{11} & a_{12} & \cd & a_{1n}\\
        a_{21} & a_{22} & \cd & a_{2n}\\
        \vd & \vd &  & \vd\\
        a_{n1} & a_{n2} & \cd & a_{nn}
      \end{array}
    \right)  
    \left(
      \begin{array}{cccc}
        \lambda_1 & & &\\
                  & \lambda_2 & & \\
                  & & \dd & \\
                  & & & \lambda_n
      \end{array}
    \right)
    = 
    \left(
      \begin{array}{cccc}
        \lambda_1a_{11} & \lambda_2a_{12} & \cd & \lambda_na_{1n}\\
        \lambda_1a_{21} & \lambda_2a_{22} & \cd & \lambda_na_{2n}\\
        \vd & \vd &  & \vd\\
        \lambda_1a_{n1} & \lambda_2a_{n2} & \cd & \lambda_na_{nn}
      \end{array}
    \right)
    $$
    \end{footnotesize}
  \end{itemize}
\end{zhu*}

\end{frame}

\begin{frame}
\begin{dingyi}[三角矩阵]
  \begin{itemize}
  \item[1] 主对角线以上的元素全为零的$n$阶方阵称为\red{上三角矩阵}($a_{ij}=0, ~i>j$)
    $$
    \left(
      \begin{array}{cccc}
        a_{11} & a_{12} & \cd & a_{1n} \\
               & a_{22} & \cd & a_{2n} \\
               &       & \dd & \vd   \\
               &       &     & a_{nn}
      \end{array}
    \right)
    $$
  \item[2] 主对角线以下的元素全为零的$n$阶方阵称为\red{下三角矩阵}($a_{ij}=0, ~i<j$)
    $$
    \left(
      \begin{array}{cccc}
        a_{11} &       &     &       \\
        a_{21} & a_{22} &     &  \\
        \vd   & \vd   & \dd &    \\
        a_{n1} & a_{n2} & \cd & a_{nn}
      \end{array}
    \right)
    $$
  \end{itemize}
\end{dingyi}
\end{frame}

\begin{frame}
\begin{jielun}
  两个上三角矩阵的乘积仍为上三角矩阵;两个下三角矩阵的乘积仍为下三角矩阵。
\end{jielun}
\end{frame}

\begin{frame}
\begin{dingli}
  设$\A,\B$是两个$n$阶方阵,则
  $$
  |\A\B| = |\A||\B|.
  $$
\end{dingli}
\end{frame}

\begin{frame}
\begin{li}
  设$A=\left[
    \begin{array}{rrr}
      1&0&0\\
      -1&3&0\\
      5&4&2
    \end{array}
  \right]$,求$|(4E-A)^T(4E-A)|$.
\end{li}
\pause
\begin{jie}
  因$|(4E-A)^T(4E-A)|=|(4E-A)^T||(4E-A)|=|4E-A|^2$,而
  $$
  |4E-A|=\left|
    \begin{array}{rrr}
      3&0&0\\
      1&1&0\\
      -5&-4&2
    \end{array}
  \right|=6,
  $$
  故$|(4E-A)^T(4E-A)|=36.$
\end{jie}
\end{frame}

\begin{frame}
\begin{dingyi}[矩阵幂]
  设$\A$是$n$阶矩阵,$k$个$\A$的连乘积称为$\A$的$k$次幂,记作$\A^k$,即
  $$
  \A^k = \underbrace{\A~ \A~ \cd ~\A}_{k}
  $$
\end{dingyi}

矩阵幂的运算律:
\begin{itemize}
\item[1] 当$m,k$为正整数时,
  $$
  \A^m \A^k = \A^{m+k}, \quad
  (\A^m)^k = \A^{mk}.
  $$  
\item[2]
  当$\A\B$不可交换时,一般情况下,
  $$
  (\A\B)^k \ne \A^k\B^k 
  $$  
\item[3]
  当$\A\B$可交换时,
  $$
  (\A\B)^k = \A^k\B^k =  \B^k\A^k. 
  $$
\end{itemize}

\end{frame}

\begin{frame}
\begin{dingyi}[矩阵多项式]
  设$f(x)=a_kx^k+a_{k-1}x^{k-1}+\cd+a_1x+a_0$是$x$的$k$次多项式,$\A$是$n$阶矩阵,则
  $$
  f(\A)=a_k\A^k+a_{k-1}\A^{k-1}+\cd+a_1\A+a_0\II
  $$
  称为矩阵$\A$的$k$次多项式。
\end{dingyi}
% 
\begin{zhu*}
  \begin{itemize}
  \item[1] 若$f(x), g(x)$为多项式,$\A,\B$皆是$n$阶矩阵,则
    $$
    f(\A)g(\A) = g(\A)f(\A).
    $$
  \item[2] 当$\A\B$不可交换时,一般
    $$f(\A)g(\B)\ne g(\B)f(\A)$$
  \end{itemize}
\end{zhu*}
\end{frame}

\subsection{矩阵的转置}

\begin{frame}
  \begin{dingyi}[转置矩阵]
  把一个$m\times n$矩阵
  $$
  \A = \left(
    \begin{array}{cccc}
      a_{11} & a_{12} & \cd & a_{1n} \\
      a_{21} & a_{22} & \cd & a_{2n} \\
      \vd   & \vd &  & \vd \\
      a_{m1} & a_{m2} & \cd & a_{mn} 
    \end{array}
  \right)
  $$
  的行列互换得到的一个$n\times m$矩阵,称之为$\A$的\red{转置矩阵},记为$\A^T$或$\A^\prime$,即
  $$
  \A^\prime = \left(
    \begin{array}{cccc}
      a_{11} & a_{21} & \cd & a_{m1} \\
      a_{12} & a_{22} & \cd & a_{m2} \\
      \vd   & \vd &  & \vd \\
      a_{1n} & a_{2n} & \cd & a_{mn} 
    \end{array}
  \right).
  $$  
\end{dingyi}
\end{frame}

\begin{frame}
\begin{dingli}[矩阵转置的运算律]
  \begin{itemize}
  \item[(i)] $(\A^T)^T=\A$
  \item[(ii)] $(\A+\B)^T=\A^T+\B^T$
  \item[(iii)] $(k\A)^T= k\A^T$
  \item[(iv)] $(\A\B)^T=\B^T\A^T$
  \end{itemize}
\end{dingli}
\end{frame}

\begin{frame}
\begin{dingyi}[对称矩阵、反对称矩阵]
  设
  $$
  \A = \left(
    \begin{array}{cccc}
      a_{11} & a_{12} & \cd & a_{1n} \\
      a_{21} & a_{22} & \cd & a_{2n} \\
      \vd   & \vd &  & \vd \\
      a_{n1} & a_{n2} & \cd & a_{nn} 
    \end{array}
  \right)
  $$
  是一个$n$阶矩阵。
  \begin{itemize}
  \item[1]
    如果
    $$
    a_{ij} = a_{ji},
    $$
    则称$\A$为\red{对称矩阵};
  \item[2]
    如果
    $$
    a_{ij} = -a_{ji},
    $$
    则称$\A$为\red{反对称矩阵}。
  \end{itemize}      
\end{dingyi}
\end{frame}

\begin{frame}
\begin{zhu*}
  关于对称矩阵与反对称矩阵,有如下性质:
  \begin{enumerate}
  \item $\A$为对称矩阵的充分必要条件是$\A^T=\A$;
  \item $\A$为反对称矩阵的充分必要条件是$\A^T=-\A$;
  \item 反对称矩阵的主对角元全为零。 
  \item 奇数阶反对称矩阵的行列式为零。
  \item 任何一个方阵都可表示成一个对称矩阵与一个反对称矩阵的和。
  \item[]  设$\A$为一$n$阶方阵,则
    $$
    \A = \frac{\A+\A^T}2 + \frac{\A-\A^T}2
    $$
    容易验证$\frac{\A+\A^T}2$为对称阵,$\frac{\A-\A^T}2$为反对称阵。 
  \item 对称矩阵的乘积不一定为对称矩阵。
  \item[]  \red{若$\A$与$\B$均为对称矩阵,则$\A\B$对称的充分必要条件是$\A\B$可交换。}
  \end{enumerate}
\end{zhu*}
\end{frame}


\subsection{矩阵的逆}
\begin{frame}
\begin{dingyi}[逆矩阵]
  对于$n$阶矩阵$\A$,如果有一个$n$阶矩阵$\B$,使
  $$
  \red{
    \A\B = \B\A = \II.
  }
  $$
  则称$\A$是\red{可逆}的,并把$\B$称为$\A$的\red{逆矩阵}。
\end{dingyi}
\end{frame}

\begin{frame}
\begin{dingli}
  若$\A$可逆,则$\A$的逆阵惟一。
\end{dingli}

\begin{dingli}
  若$\A$可逆,则$|\A|\ne 0$.
\end{dingli}

\end{frame}

\begin{frame}
\begin{dingyi}{代数余子式矩阵,伴随矩阵}
  设$\A=(a_{ij})_{n\times n}$,$A_{ij}$为行列式$|\A|$中元素$a_{ij}$的代数余子式,称
  $$
  \mathrm{coef}~ \A = (A_{ij})_{n\times n}
  $$
  为$\A$的\red{代数余子式矩阵},并称$\mathrm{coef}~ \A$的转置矩阵为$\A$的\red{伴随矩阵},记为$\A^*$,
  即
  $$\red{
    \A^* = (\mathrm{coef}~\A)^T = \left(
      \begin{array}{cccc}
        A_{11} & A_{21} & \cd & A_{n1} \\
        A_{12} & A_{22} & \cd & A_{n2} \\
        \vd   & \vd   &     & \vd   \\
        A_{1n} & A_{2n} & \cd & A_{nn} \\
      \end{array}
    \right)
  }
  $$
\end{dingyi}
\end{frame}

\begin{frame}
\begin{dingli}
  若$|\A|\ne 0$,则$\A$可逆,且
  $$
  \A^{-1} = \frac1{|\A|} \A^*
  $$
\end{dingli}
\end{frame}

\begin{frame}
\begin{dingyi}[奇异阵与非奇异阵]
  当$|\A|=0$时,$\A$称为\red{奇异矩阵},否则称为\red{非奇异矩阵}。
\end{dingyi}

\begin{zhu*}
  \red{可逆矩阵就是非奇异矩阵。}
\end{zhu*}

\end{frame}

\begin{frame}

\begin{dingli}可逆矩阵有如下运算规律:
  \begin{enumerate}
  \item[1] 若$\A$可逆,则$\A^{-1}$亦可逆,且
    \red{$(\A^{-1})^{-1}=\A.$}\\[.1in]
  \item[2] 若$\A$可逆,$k\ne 0$,则$k\A$可逆,且
    \red{$(k\A)^{-1}= k^{-1}A^{-1}.$}\\[.1in]
  \item[3] 若$\A, ~\B$为同阶矩阵且均可逆,则$\A\B$可逆,且
    \red{$(\A\B)^{-1} = \B^{-1}\A^{-1}.$}\\[.1in]
  \item[] 若$\A_1,\A_2,\cd,\A_m$皆可逆,则
    \red{$
    (\A_1\A_2\cd\A_m)^{-1}=\A_m^{-1}\cd\A_2^{-1}\A_1^{-1}.
    $}\\[.1in]
  \item[4] 若$\A$可逆,则$\A^T$亦可逆,且
    \red{$(\A^T)^{-1}=(\A^{-1})^T.$ }\\[.1in]
  \item[5] 若$\A$可逆,则
    \red{$|\A^{-1}|=|\A|^{-1}.$}\\[.1in]
  \end{enumerate}
\end{dingli}
\end{frame}

\begin{frame}
  \begin{li}[$\bigstar$]
    设$A=\left[
      \begin{array}{rrr}
        1&2&-1\\
        -2&1&3\\
        0&5&1
      \end{array}
    \right]$,问$A$是否可逆?如可逆求$A^{-1}$,如不可逆,求$A$的伴随矩阵$A^*$.
  \end{li}
  \pause
  \begin{jie}
    因$$|A|=\left|
      \begin{array}{rrr}
        1&2&-1\\
        -2&1&3\\
        0&5&1
      \end{array}
    \right|=0,$$
    故$A$不可逆。而
    $$A^*=\left|
      \begin{array}{rrr}
        -14&-7&7\\
        2&1&-1\\
        -10&-5&5
      \end{array}
    \right|.$$
  \end{jie}
\end{frame}

\begin{frame}
\begin{li}
设$\A$与$\B$可交换,且$\A$可逆,$\A^*$为$\A$的伴随矩阵,试证明$\A^*$与$\B$可交换。
\end{li}
\pause
\begin{proof}
  由$\A^*=|\A|\A^{-1}, \A\B=\B\A$可得$\A^{-1}B=\B\A^{-1}$,故$\A^{*}B=\B\A^{*}.$
\end{proof}
\end{frame}

\begin{frame}
  \begin{li}
  设方阵$\A$满足方程
  $$
  \A^2 - 3\A - 10 \II = \zero,
  $$
  证明:$\A, \A-4\II$都可逆,并求它们的逆矩阵。      
\end{li} \pause
\begin{proof}
$$
\A^2-3\A-10\II=\zero ~\Rightarrow~ \A(\A-3\II) = 10\II 
~\Rightarrow~ \A\left[\frac1{10}(\A-3\II)\right] = \II
$$  
故$\A$可逆,且\red{$\ds \A^{-1} = \frac1{10}(\A-3\II)$}.\pause 

$$
\A^2-3\A-10\II=\zero ~\Rightarrow~ (\A+\II)(\A-4\II) = 6\II 
~\Rightarrow~ \frac1{6}(\A+\II)(\A-4\II) = \II
$$     
故$\A-4\II$可逆,且\red{$\ds (\A-4\II)^{-1} = \frac1{6}(\A+\II)$}.
\end{proof}
\end{frame}


\subsection{矩阵的初等变换与初等矩阵}



\begin{frame}

\begin{dingyi}[初等矩阵]
  将单位矩阵$\II$做一次初等变换所得的矩阵称为\textcolor{acolor3}{初等矩阵}。
  对应于$3$类初等行、列变换,有$3$种类型的初等矩阵。
\end{dingyi}
\end{frame}


\begin{frame}
以下介绍三种初等矩阵:
\begin{enumerate}
\item 初等对调矩阵;
\item 初等倍乘矩阵;
\item 初等倍加矩阵。
\end{enumerate}
\end{frame}


\begin{frame}
1、 对调$\II$的两行或两列(\textcolor{acolor3}{初等对调矩阵})
  \begin{figure}[htbp]
    \centering
    \begin{tikzpicture}[scale=0.8]
      \matrix (M) [matrix of math nodes]  { 
        \E_{ij} = \\
      };
      \matrix(MM) [right=2pt of M, matrix of math nodes,nodes in empty cells, ampersand replacement=\&,left delimiter=(,right delimiter=)] {
        1 \&     \&   \&   \&     \&   \&   \& \& \\
        \& \dd \&   \&   \&     \&   \&   \& \& \\
        \&     \& 0 \&   \& \cd \&   \& 1 \& \& \\
        \&     \&   \& 1 \&     \&   \&   \& \& \\
        \&     \&\vd\&   \& \dd \&   \&\vd\& \& \\
        \&     \&   \&   \&     \& 1 \&   \& \& \\
        \&     \& 1 \&   \& \cd \&   \& 0 \& \& \\
        \&     \&   \&   \&     \&   \&   \& \dd \& \\
        \&     \&   \&   \&     \&   \&   \& \& 1\\
      };
      \node[right=16pt  of MM-3-9, blue]  {第$i$行};
      \node[right=16pt  of MM-7-9, blue]  {第$j$行};
      \node[below=5pt  of MM-9-3, blue]  {第$i$列};
      \node[below=5pt  of MM-9-7, blue]  {第$j$列};
    \end{tikzpicture}
  \end{figure}
\end{frame}


\begin{frame}
a、用$m$阶初等矩阵$\E_{ij}$左乘$\A=(a_{ij})_{m\times n}$,得
    \begin{figure}[htbp]
      \centering
      \begin{tikzpicture}
        \matrix (M) [matrix of math nodes]  { 
          \E_{ij}\A = \\
        };
        \matrix(MM) [right=2pt of M, matrix of math nodes,nodes in empty cells,
        ampersand replacement=\&,left delimiter=(,right delimiter=)] {
          a_{11} \& a_{12}    \& \cd   \&  a_{1n} \\
          \vd   \& \vd      \&   \&  \vd \\          
          a_{j1} \& a_{j2}    \& \cd  \&  a_{jn} \\
          \vd   \& \vd      \&   \&  \vd \\
          a_{i1} \& a_{i2}    \& \cd   \&  a_{in} \\
          \vd   \& \vd      \&   \&  \vd \\
          a_{m1} \& a_{m2}    \& \cd  \&  a_{mn} \\
        };
        \node[right=12pt  of MM-3-4, blue]  {第$i$行};
        \node[right=12pt  of MM-5-4, blue]  {第$j$行};
      \end{tikzpicture}
    \end{figure}
    相当于
    \textcolor{acolor3}{把$\A$的第$i$行与第$j$行对调($r_i \leftrightarrow r_j$).}
\end{frame}


\begin{frame}
b、用$n$阶初等矩阵$\E_{ij}$右乘$\A$,得
    \begin{figure}[htbp]
      \centering
      \begin{tikzpicture}
        \matrix (M) [matrix of math nodes]  { 
          \A \E_{ij}= \\
        };
        \matrix(MM) [right=2pt of M, matrix of math nodes,nodes in empty cells,
        ampersand replacement=\&,left delimiter=(,right delimiter=)] {
          a_{11} \& \cd \&a_{1j}    \& \cd \&a_{1i}    \& \cd  \&  a_{1n} \\
          a_{21} \& \cd \&a_{2j}    \& \cd \&a_{2i}    \& \cd  \&  a_{jn} \\
          \vd    \&     \&\vd       \&     \&\vd       \&      \&  \vd \\
          a_{m1} \& \cd \&a_{mj}    \& \cd \&a_{mi}    \& \cd  \&  a_{mn} \\
        };
        \node[below=12pt  of MM-4-3, blue]  {第$i$列};
        \node[below=12pt  of MM-4-5, blue]  {第$j$列};
      \end{tikzpicture}
    \end{figure}

    相当于\textcolor{acolor3}{把$\A$的第$i$列与第$j$列对调($c_i \leftrightarrow c_j$).}
\end{frame}


\begin{frame}
2、 以非零常数$k$乘$\II$的某行或某列(\textcolor{acolor3}{初等倍乘矩阵})
\begin{figure}[htbp]
  \centering
    \begin{tikzpicture}
      \matrix (M) [matrix of math nodes]  { 
        \E_{i}(k) = \\
      };
      \matrix(MM) [right=2pt of M, matrix of math nodes,nodes in empty cells,
      ampersand replacement=\&,left delimiter=(,right delimiter=)] {
        1 \&     \&   \&   \&     \&   \& \\
        \& \dd \&   \&   \&     \&   \& \\
        \&     \& 1 \&   \&     \&   \& \\
        \&     \&   \& k \&     \&   \& \\
        \&     \&   \&   \& 1   \&   \& \\
        \&     \&   \&   \&     \& \dd \& \\
        \&     \&   \&   \&     \&   \& 1\\
      };
      \node[right=12pt  of MM-4-7, blue]  {第$i$行};
      \node[below=12pt  of MM-7-4, blue]  {第$i$列};
    \end{tikzpicture}
  \end{figure}
\end{frame}


\begin{frame}
a、以$m$阶初等矩阵$\E_i(k)$左乘$\A$,得
    \begin{figure}[htbp]
      \centering
      \begin{tikzpicture}
        \matrix (M) [matrix of math nodes]  { 
          \E_{i}(k)\A = \\
        };
        \matrix(MM) [right=2pt of M, matrix of math nodes,nodes in empty cells,
        ampersand replacement=\&,left delimiter=(,right delimiter=)] {
          a_{11} \& a_{12}    \& \cd   \&  a_{1n} \\
          \vd   \& \vd      \&   \&  \vd \\          
          ka_{i1} \& ka_{i2}    \& \cd  \&  ka_{in} \\
          \vd   \& \vd      \&   \&  \vd \\
          a_{m1} \& a_{m2}    \& \cd  \&  a_{mn} \\
        };
        \node[right=12pt  of MM-3-4, blue]  {第$i$行};
      \end{tikzpicture}
    \end{figure} 
  相当于\textcolor{acolor3}{以数$k$乘$\A$的第$i$行($r_i\times k$)};
\end{frame}


\begin{frame}
b、 以$n$阶初等矩阵$\E_i(k)$右乘$\A$,得
    \begin{figure}[htbp]
      \centering
      \begin{tikzpicture}
        \matrix (M) [matrix of math nodes]  { 
          \A \E_{i}(k)= \\
        };
        \matrix(MM) [right=2pt of M, matrix of math nodes,nodes in empty cells,
        ampersand replacement=\&,left delimiter=(,right delimiter=)] {
          a_{11} \& \cd \&ka_{1i}      \& \cd  \&  a_{1n} \\
          a_{21} \& \cd \&ka_{2i}      \& \cd  \&  a_{jn} \\
          \vd    \&     \&\vd         \&      \&  \vd \\
          a_{m1} \& \cd \&ka_{mi}      \& \cd  \&  a_{mn} \\
        };
        \node[below=12pt  of MM-4-3, blue]  {第$i$列};
      \end{tikzpicture}
    \end{figure}
    
  相当于\textcolor{acolor3}{以数$k$乘$\A$的第$i$列($c_i\times k$)}。
\end{frame}


\begin{frame}
3、将非零常数$k$乘$\II$的某行再加到另一行上(\textcolor{acolor3}{初等倍加矩阵})
\begin{figure}[htbp]
  \centering
  \begin{tikzpicture}
    \matrix (M) [matrix of math nodes]  { 
      \E_{ij}(k) = \\
    };
    \matrix(MM) [right=2pt of M, matrix of math nodes,nodes in empty cells,
    ampersand replacement=\&,left delimiter=(,right delimiter=)] {
      1 \&     \&   \&   \&     \&   \& \\
      \& \dd \&   \&   \&     \&   \& \\
      \&     \& 1 \&\cd\& k    \&   \& \\
      \&     \&   \&\dd\& \vd  \&   \& \\
      \&     \&   \&   \& 1   \&   \& \\
      \&     \&   \&   \&     \& \dd \& \\
      \&     \&   \&   \&     \&   \& 1\\
    };
    \node[right=12pt  of MM-3-7, blue]  {第$i$行};
    \node[right=12pt  of MM-5-7, blue]  {第$j$行};
  \end{tikzpicture}
\end{figure} 
\end{frame}


\begin{frame}
a、 以$m$阶初等矩阵$\E_{ij}(k)$左乘$\A$,得
    \begin{figure}[htbp]
      \centering
      \begin{tikzpicture}
        \matrix (M) [matrix of math nodes]  { 
          \E_{ij}\A = \\
        };
        \matrix(MM) [right=2pt of M, matrix of math nodes,nodes in empty cells,
        ampersand replacement=\&,left delimiter=(,right delimiter=)] {
          a_{11} \& a_{12}    \& \cd   \&  a_{1n} \\
          \vd   \& \vd      \&   \&  \vd \\          
          a_{i1}+ka_{j1} \& a_{i2}+ka_{j2}    \& \cd  \&  a_{in}+ka_{jn} \\
          \vd   \& \vd      \&   \&  \vd \\
          a_{j1} \& a_{j2}    \& \cd   \&  a_{jn} \\
          \vd   \& \vd      \&   \&  \vd \\
          a_{m1} \& a_{m2}    \& \cd  \&  a_{mn} \\
        };
        \node[right=12pt  of MM-3-4, blue]  {第$i$行};
        \node[right=28pt  of MM-5-4, blue]  {第$j$行};
      \end{tikzpicture}
    \end{figure}
  相当于\textcolor{acolor3}{把$\A$的第$j$行乘以数$k$加到第$i$行上($r_i+r_j\times k$)};
\end{frame}


\begin{frame}
b、 以$n$阶初等矩阵$\E_{ij}(k)$右乘$\A$,得
    \begin{figure}[htbp]
      \centering
      \begin{tikzpicture}
        \matrix (M) [matrix of math nodes]  { 
          \A \E_{ij}= \\
        };
        \matrix(MM) [right=2pt of M, matrix of math nodes,nodes in empty cells,
        ampersand replacement=\&,left delimiter=(,right delimiter=)] {
          a_{11} \& \cd \&a_{1i}    \& \cd \&a_{1j}+ka_{1i}  \& \cd  \&  a_{1n} \\
          a_{21} \& \cd \&a_{2i}    \& \cd \&a_{2j}+ka_{2i}    \& \cd  \&  a_{jn} \\
          \vd    \&     \&\vd       \&     \&\vd       \&      \&  \vd \\
          a_{m1} \& \cd \&a_{mi}    \& \cd \&a_{mj}+ka_{mi}    \& \cd  \&  a_{mn} \\
        };
        \node[below=12pt  of MM-4-3, blue]  {第$i$列};
        \node[below=12pt  of MM-4-5, blue]  {第$j$列};
      \end{tikzpicture}
    \end{figure}

  相当于\textcolor{acolor3}{把$\A$的第$i$列乘以数$k$加到第$j$列上($c_j+c_i\times k$)}。
 
\end{frame}


\begin{frame}
%
\begin{dingli}
  设$\A$为一个$m\times n$矩阵,
  \begin{itemize}
  \item 
    对$\A$施行一次初等行变换,相当于在$\A$的左边乘以相应的$m$阶初等矩阵;
  \item
    对$\A$施行一次初等列变换,相当于在$\A$的右边乘以相应的$n$阶初等矩阵。
  \end{itemize}
\end{dingli}
\end{frame}




\begin{frame}
由初等变换可逆,可知初等矩阵可逆。  
\begin{itemize}
\item[(i)] 由\textcolor{acolor1}{变换$r_i\leftrightarrow r_j$的逆变换为其本身}可知
  $$
  \textcolor{acolor3}{\E_{ij}^{-1} = \E_{ij}}
  $$ 
\item[(ii)] 由\textcolor{acolor1}{变换$r_i\times k$的逆变换为$\ds r_i\div k$}可知
  $$
  \textcolor{acolor3}{\E_{i}(k)^{-1} = \E_{i}(k^{-1})}
  $$ 
\item[(iii)] 由\textcolor{acolor1}{变换$r_i+r_j\times k$的逆变换为$\ds r_i-r_j\times k$}可知
  $$
  \textcolor{acolor3}{\E_{ij}(k)^{-1} = \E_{ij}(-k)}
  $$ 
\end{itemize}
\end{frame}


\begin{frame}
以上结论也可总结为
  $$ \textcolor{acolor3}{
    \E_{ij}\E_{ij}=\II, \quad
    \E_{i}(k)\E_{i}(k^{-1}) = \II, \quad
    \E_{ij}(k)\E_{ij}(-k) = \II.
  }
  $$      
\end{frame}

\begin{frame}
  \begin{li}[$\bigstar$]
    设$A=\left[
      \begin{array}{rrrr}
        3&-1&1&-1\\
        1&-2&-2&1\\
        -2&3&-3&1
      \end{array}
    \right], B=\left[
      \begin{array}{rrrr}
        4&2&2\\
        2&3&1\\
        1&2&1\\
        3&3&2
      \end{array}
    \right], C=\left[
      \begin{array}{rrrr}
        -2&3&1\\
        1&-2&3\\
        3&1&-2
      \end{array}
    \right]$,求$(AB+2C)E(2,3)$,其中$E(2,3)$是交换单位矩阵的2,3行(列)所得的三阶初等方阵。
  \end{li}
  \pause
  \begin{jie}
    经计算$$
    AB+2C=\left[
      \begin{array}{rrrr}
        1&5&4\\
        2&-9&6\\
        4&4&-6
      \end{array}
    \right]
    $$
    故
    $$
    (AB+2C)E(2,3)=\left[
      \begin{array}{rrrr}
        1&4&5\\
        2&6&-9\\
        4&-6&4
      \end{array}
    \right]
    $$
  \end{jie}
\end{frame}



 

\begin{frame}
 
\begin{dingli}
  可逆矩阵可以经过若干次初等行变换化为单位矩阵。
\end{dingli}

\begin{tuilun}
  可逆矩阵$\A$可以表示为若干个初等矩阵的乘积。
\end{tuilun}

\begin{tuilun}
  如果对可逆矩阵$\A$与同阶单位矩阵$\II$做同样的初等行变换,那么当$\A$变为单位阵时,
  $\II$就变为$\A^{-1}$,即
  $$\textcolor{acolor3}{
    \left(
      \begin{array}{cc}
        \A & \II
      \end{array}
    \right) \xrightarrow[]{\mbox{初等行变换}} \left(
      \begin{array}{cc}
        \II & \A^{-1}
      \end{array}
    \right)
  } 
  $$
  同理,
$$\textcolor{acolor3}{
  \left(
    \begin{array}{c}
      \A\\
      \II
    \end{array}
  \right) \xrightarrow[]{\mbox{初等列变换}} \left(
    \begin{array}{c}
      \II \\
      \A^{-1}
    \end{array}
  \right)
} 
$$
\end{tuilun}
\pause

\begin{zhu}
\textcolor{acolor1}{该推论给出了求可逆矩阵的逆的一种有效方法,请大家熟练掌握。}
\end{zhu}
\end{frame}


\begin{frame}
\begin{li}
  求$
  \A=\left(
    \begin{array}{rrr}
      0&2&-1\\
      1&1&2\\
      -1&-1&-1
    \end{array}
  \right)
  $
  的逆矩阵。
\end{li}
\end{frame}


\begin{frame}
\begin{jie}

$$
\begin{aligned}
&\left(
  \begin{array}{c|c}
    \A & \II
  \end{array}
\right)=\left(
  \begin{array}{rrr|rrr}
    0 &  2 & -1 &  1 & 0 & 0\\
    1 &  1 &  2 &  0 & 1 & 0\\
    -1 & -1 & -1 &  0 & 0 & 1\\              
  \end{array}
\right) \\ 
&\xrightarrow[]{r_1\leftrightarrow r_2}\left(
  \begin{array}{rrr|rrr}
    1 &  1 &  2 &  0 & 1 & 0\\
    0 &  2 & -1 &  1 & 0 & 0\\
    -1 & -1 & -1 &  0 & 0 & 1\\          
  \end{array}
\right) 
\xrightarrow[]{r_3+ r_1}\left(
  \begin{array}{rrr|rrr}
    1 &  1 &  2 & 0 & 1 & 0\\
    0 &  2 & -1 & 1 & 0 & 0\\
    0 &  0 &  1 & 0 & 1 & 1\\          
  \end{array}
\right)\\
&\xrightarrow[r_2+r_3]{r_1+ r_3\times(-2)}\left(
  \begin{array}{rrr|rrr}
    1 &  1 &  0  & 0 &-1 &-2\\
    0 &  2 &  0  & 1 & 1 & 1\\
    0 &  0 &  1  & 0 & 1 & 1\\    
  \end{array}
\right)\\
&\xrightarrow[r_2\times \frac12]{r_1+ r_2\times(-\frac12)}\left(
  \begin{array}{rrr|rrr}
    1 &  0 &  0  & -\frac12 &-\frac32 &-\frac52\\[.1in]
    0 &  1 &  0  & \frac12 & \frac12 & \frac12 \\[.1in]
    0 &  0 &  1  & 0 & 1 & 1                   
  \end{array}
\right)    
\end{aligned}
$$
\end{jie}
\end{frame}


\begin{frame}
\begin{li}
  已知$\A\B\A^T=2\B\A^T+\II$,求$\B$,其中$
  \A = \left(
    \begin{array}{ccc}
      1&0&0\\
      0&1&2\\
      0&0&1
    \end{array}
  \right)
  $
\end{li}
\end{frame}


\begin{frame}
\begin{jie}
$$
\A\B\A^T=2\B\A^T+\II ~~ \Rightarrow ~~ (\A-2\II)\B\A^T=\II 
~~ \Rightarrow ~~ \B\A^T = (\A-2\II)^{-1}
$$

故
$$
\B = (\A-2\II)^{-1} (\A^T)^{-1}  = [\A^T(\A-2\II)]^{-1} 
=(\A^T\A-2\A^T)^{-1}
$$
而
$$
\A^T\A-2\A^T = \left(
  \begin{array}{rrr}
    -1&0&0\\
    0&-1&2\\
    0&-2&3
  \end{array}
\right)
$$ 
可求得
$$
\B = \left(
  \begin{array}{rrr}
    -1&0&0\\
    0&3&-2\\
    0&2&-1
  \end{array}
\right)
$$
\end{jie}
\end{frame}


\begin{frame}

\begin{tuilun}
  对于$n$个未知数$n$个方程的线性方程组
  $
  \A\xx=\bb,
  $
  如果增广矩阵
  $$
  \textcolor{acolor3}{(\A,~\bb)~~\overset{r}{\sim}~~(\II,\xx)},
  $$
  则$\A$可逆,且$\xx=\A^{-1}\bb$为惟一解。  
\end{tuilun}
\end{frame}


\begin{frame}
\begin{li}
  设
  $$
  \A = \left(
    \begin{array}{rrr}
      2&1&-3\\
      1&2&-3\\
      -1&3&2
    \end{array}
  \right),
  ~~
  \bb_1=\left(
    \begin{array}{r}
      1\\
      2\\
      -2
    \end{array}
  \right),
  ~~
  \bb_2=\left(
    \begin{array}{r}
      -1\\
      0\\
      5
    \end{array}
  \right),
  $$
  求$\A\xx=\bb_1$与$\A\xx=\bb_2$的解。
\end{li}
\end{frame}


\begin{frame}
\begin{jie}
  $$
  \begin{aligned}
    & (\A~~\textcolor{acolor3}{\bb_1}~~\textcolor{acolor3}{\bb_2})
    = \left(
      \begin{array}{rrrrr}
        2 & 1 & 3 &\textcolor{acolor3}{ 1} & \textcolor{acolor3}{-1}\\
        1 & 2 &-2 &\textcolor{acolor3}{ 2} & \textcolor{acolor3}{ 0}\\
        -1 & 3 & 2 &\textcolor{acolor3}{-2} & \textcolor{acolor3}{ 5}        
      \end{array}\right) \\ 
      & \overset{{r_1\leftrightarrow r_2 \atop r_2-2r_1}\atop  r_3+r_1}{\sim}
                               \left(
                               \begin{array}{rrrrr}
                                 1 & 2 &-2 & \textcolor{acolor3}{ 2} & \textcolor{acolor3}{ 0}\\
                                 0 &-3 & 1 & \textcolor{acolor3}{-3} & \textcolor{acolor3}{-1}\\
                                 0 & 5 & 0 & \textcolor{acolor3}{ 0} & \textcolor{acolor3}{ 5}        
                               \end{array}
                                                        \right) \\ 
    & \overset{{r_3\leftrightarrow r_2 \atop r_2\div5}\atop  r_3+3r_2}{\sim}
      \left(
      \begin{array}{rrrrr}
        1 & 2 &-2 &\textcolor{acolor3}{ 2} &  \textcolor{acolor3}{0}\\
        0 & 1 & 0 &\textcolor{acolor3}{ 0} &  \textcolor{acolor3}{1}\\
        0 & 0 & 1 &\textcolor{acolor3}{-3} &  \textcolor{acolor3}{2}        
      \end{array}
                              \right)   \overset{r_1-2r_2+2r_3}{\sim}
                              \left(
                              \begin{array}{rrrrr}
                                1 & 0 & 0 & \textcolor{acolor3}{-4} & \textcolor{acolor3}{2}\\
                                0 & 1 & 0 & \textcolor{acolor3}{0} &  \textcolor{acolor3}{1}\\
                                0 & 0 & 1 & \textcolor{acolor3}{-3} &  \textcolor{acolor3}{2}        
                              \end{array}
                                                       \right) 
  \end{aligned}
  $$
\end{jie}

\end{frame}


\begin{frame}

\begin{li}[$\bigstar$]
  求解矩阵方程$\A\XX=\A+\XX$,其中
  $
  \A = \left(
    \begin{array}{ccc}
      2&2&0\\
      2&1&3\\
      0&1&0
    \end{array}
  \right)
  $
\end{li}
\end{frame}


\begin{frame}
\begin{jie}
原方程等价于
$$
(\A-\II)\XX=\A
$$

而
$$
\begin{aligned}
&  (\A-\II ~~\textcolor{acolor3}{\A}) =\left(
                       \begin{array}{rrrrrr}
                         1&2& 0&\textcolor{acolor3}{2}&\textcolor{acolor3}{2}&\textcolor{acolor3}{0}\\
                         2&0& 3&\textcolor{acolor3}{2}&\textcolor{acolor3}{1}&\textcolor{acolor3}{3}\\
                         0&1&-1&\textcolor{acolor3}{0}&\textcolor{acolor3}{1}&\textcolor{acolor3}{0}
                       \end{array}
                                                 \right)\\
                                                 &\overset{r_2-2r_1\atop r_2\leftrightarrow r_3}{\sim}
                                                 \left(
                                                 \begin{array}{rrrrrr}
                                                   1& 2& 0&\textcolor{acolor3}{ 2}&\textcolor{acolor3}{ 2}&\textcolor{acolor3}{0}\\
                                                   0& 1&-1&\textcolor{acolor3}{ 0}&\textcolor{acolor3}{ 1}&\textcolor{acolor3}{0}\\
                                                   0&-4& 3&\textcolor{acolor3}{-2}&\textcolor{acolor3}{-3}&\textcolor{acolor3}{3}
                                                 \end{array}
                                                                              \right)     
                       \overset{r_3+4r_2\atop r_3\div(-1)}{\sim}
                       \left(
                       \begin{array}{rrrrrr}
                         1&2& 0&\textcolor{acolor3}{2}&\textcolor{acolor3}{ 2}&\textcolor{acolor3}{ 0}\\
                         0&1&-1&\textcolor{acolor3}{0}&\textcolor{acolor3}{ 1}&\textcolor{acolor3}{ 0}\\
                         0&0& 1&\textcolor{acolor3}{2}&\textcolor{acolor3}{-1}&\textcolor{acolor3}{-3}
                       \end{array}
                                                  \right) \\ 
                     &  
                       \overset{r_3+4r_2\atop r_3\div(-1)}{\sim}
                       \left(
                       \begin{array}{rrrrrr}
                         1&0&0&\textcolor{acolor3}{-2}&\textcolor{acolor3}{2}&\textcolor{acolor3}{6}\\
                         0&1&0&\textcolor{acolor3}{2}&\textcolor{acolor3}{0}&\textcolor{acolor3}{-3}\\
                         0&0&1&\textcolor{acolor3}{2}&\textcolor{acolor3}{-1}&\textcolor{acolor3}{-3}
                       \end{array}
                                                 \right)
\end{aligned}
$$
\end{jie}
\end{frame}

\begin{frame}

\begin{li}[$\bigstar$]
  求解矩阵方程$\A\XX=\B+\XX$,其中
  $$
  \A = \left(
    \begin{array}{ccc}
      1&1&1\\
      1&1&1\\
      -3&2&1
    \end{array}
  \right), \B = \left(
    \begin{array}{ccc}
      1&3&1\\
      2&-1&1\\
      -3&1&-1
    \end{array}
  \right),
  $$
  求$\XX$。
\end{li}
\end{frame}


% \subsection{矩阵分块*}
% \begin{frame}
% 矩阵
% \begin{figure}[htbp]
%   \centering
%   \begin{tikzpicture}
%     \matrix(A) [matrix of math nodes,nodes in empty cells,ampersand replacement=\&,left delimiter=(,right delimiter=)] {
%       a_{11} \& a_{12} \& a_{13}  \& a_{14} \\
%       a_{21} \& a_{22} \& a_{23}  \& a_{24} \\
%       a_{31} \& a_{32} \& a_{33}  \& a_{34} \\
%     };
%     \draw[red, dashed, very thick] (A-1-2.north east) -- (A-3-2.south east) (A-2-1.south west) -- (A-2-4.south east);;
%   \end{tikzpicture}
% \end{figure}
% 可记为
% $$
% \left(
%   \begin{array}{cc}
%     \A_{11} &  \A_{12}\\
%     \A_{21} &  \A_{22}
%   \end{array}
% \right)
% $$
% 其中
% $$
% \begin{array}{ll}
%   \A_{11} = 
%   \left(
%   \begin{array}{cc}
%     a_{11} &  a_{12}\\
%     a_{21} &  a_{22}
%   \end{array} \right),
%            &
%              \A_{12} = 
%              \left(
%              \begin{array}{cc}
%                a_{13} &  a_{14}\\
%                a_{23} &  a_{24}
%              \end{array}
%                         \right)\\ [0.3cm]
%   \A_{21} = 
%   \left(
%   \begin{array}{cc}
%     a_{31} &  a_{32}
%   \end{array}\right) ,
%            &
%              \A_{22} = 
%              \left(
%              \begin{array}{cc}
%                a_{33} &  a_{34}
%              \end{array}
%                         \right)    
% \end{array}
% $$
% \end{frame}

% \begin{frame}
% \begin{dingyi}[矩阵的按行分块]
%   $$
%   \A = \left(
%     \begin{array}{cccc}
%       a_{11} & a_{12} & \cd & a_{1n}\\
%       a_{21} & a_{22} & \cd & a_{2n}\\
%       \vd & \vd & & \vd \\
%       a_{m1} & a_{m2} & \cd & a_{mn}
%     \end{array}
%   \right)
%   = \left(
%     \begin{array}{c}
%       \aa_1\\
%       \aa_2\\
%       \vd \\
%       \aa_m
%     \end{array}
%   \right)
%   $$
%   其中
%   $$
%   \aa_i = (a_{i1}, ~a_{i2}, ~\cd, ~a_{in})
%   $$
% \end{dingyi}
% \end{frame}

% \begin{frame}
% \begin{dingyi}[矩阵的按列分块]
%   $$
%   \B = \left(
%     \begin{array}{cccc}
%       b_{11} & b_{12} & \cd & b_{1s}\\
%       b_{21} & b_{22} & \cd & b_{2s}\\
%       \vd & \vd & & \vd \\
%       b_{n1} & b_{n2} & \cd & b_{ns}
%     \end{array}
%   \right)
%   = \left(
%     \begin{array}{c}
%       \bb_1, ~ \bb_2, ~ \cd, \bb_s
%     \end{array}
%   \right)
%   $$
%   其中
%   $$
%   \bb_j = \left(
%     \begin{array}{c}
%       b_{1j}\\
%       b_{2j}\\
%       \vd\\
%       b_{nj}
%     \end{array}
%   \right)
%   $$
% \end{dingyi}
% \end{frame}

% \begin{frame}
% 当$n$阶矩阵$\A$中非零元素都集中在主对角线附近,有时可分块成如下\textcolor{acolor3}{对角块矩阵}
% $$
% \A = \left(
%   \begin{array}{cccc}
%     \A_1 & & &\\
%     &\A_2&&\\
%     &&\dd&\\
%     &&&\A_m
%   \end{array}
% \right)
% $$
% 其中$\A_i$为$r_i$阶方阵($i=1,2,\cd,m$),且
% $$
% \sum_{i=1}^m r_i = n.
% $$

% \end{frame}

% \begin{frame}
% 如
% \begin{figure}
%   \centering
%   \begin{tikzpicture}       
%     [column 1/.style={anchor=base east},
%     column 2/.style={anchor=base east},
%     column 3/.style={anchor=base east},
%     column 4/.style={anchor=base east},
%     column 5/.style={anchor=base east},
%     column 6/.style={anchor=base east}]
%     \matrix(A) [matrix of math nodes,nodes in empty cells,ampersand replacement=\&,left delimiter=(,right delimiter=)] {
%     0 \& -1 \& 0 \& 0 \& 0 \& 0\\
%     1 \&  2 \& 0 \& 0 \& 0 \& 0\\
%     0 \&  0 \& 1 \&-1 \& 0 \& 0\\
%     0 \&  0 \&-1 \& 1 \& 2 \& 0\\
%     0 \&  0 \& 0 \& 2 \&-2 \& 0\\
%     0 \&  0 \& 0 \& 0 \& 0 \& 3\\
%   };
%     \draw[red, dashed, very thick] 
%     (A-1-2.north east) -- (A-6-2.south east)
%     (A-1-5.north east) -- (A-6-5.south east) 
%     (A-2-1.south west) -- (A-2-6.south east)
%     (A-5-1.south west) -- (A-5-6.south east);

%   \end{tikzpicture}
% \end{figure}

% \end{frame}

% \begin{frame}



% \begin{dingyi}[分块矩阵的加法]
%   设$\A, \B$为同型矩阵,采用相同的分块法,有
%   $$
%   \A = \left(
%     \begin{array}{ccc}
%       \A_{11} & \cd & \A_{1r} \\
%       \vd   &     & \vd   \\
%       \A_{s1} & \cd & \A_{sr}
%     \end{array}
%   \right), \ \ 
%   \B = \left(
%     \begin{array}{ccc}
%       \B_{11} & \cd & \B_{1r} \\
%       \vd   &     & \vd   \\
%       \B_{s1} & \cd & \B_{sr}
%     \end{array}
%   \right),
%   $$
%   其中$\A_{ij}$与$\B_{ij}$为同型矩阵,则
%   $$
%   A = \left(
%     \begin{array}{ccc}
%       \A_{11} + \B_{11}  & \cd & \A_{1r} + \B_{1r} \\
%       \vd   &     & \vd   \\
%       \A_{s1} + \B_{s1}  & \cd & \A_{sr} + \B_{sr}
%     \end{array}
%   \right).
%   $$
% \end{dingyi}
% \end{frame}

% \begin{frame}

% \begin{dingyi}[分块矩阵的数乘]
%   $$
%   \lambda \A = \left(
%     \begin{array}{ccc}
%       \lambda \A_{11} & \cd & \lambda \A_{1r} \\
%       \vd   &     & \vd   \\
%       \lambda \A_{s1} & \cd & \lambda \A_{sr}
%     \end{array}
%   \right)
%   $$    
% \end{dingyi}
% \end{frame}

% \begin{frame}

% \begin{dingyi}[分块矩阵的乘法]
%   设$\A$为$m\times n$矩阵, $\B$为$n \times s$矩阵,
%   $$
%   \A = \left(
%     \begin{array}{ccc}
%       \A_{11} & \cd & \A_{1s} \\
%       \vd   &     & \vd   \\
%       \A_{r1} & \cd & \A_{rs}
%     \end{array}
%   \right), \ \ 
%   \B = \left(
%     \begin{array}{ccc}
%       \B_{11} & \cd & \B_{1t} \\
%       \vd   &     & \vd   \\
%       \B_{s1} & \cd & \B_{st}
%     \end{array}
%   \right),
%   $$
%   其中\textcolor{acolor3}{$\A_{i1}, \A_{i2}, \cd, A_{is}$的列数分别等于$\B_{1j}, \B_{2j}, \cd, \B_{sj}$的行数},则
%   $$
%   \A \B = \left(
%     \begin{array}{ccc}
%       \C_{11}   & \cd & \C_{1t}  \\
%       \vd   &     & \vd   \\
%       \C_{r1}   & \cd & \C_{rt}
%     \end{array}
%   \right),
%   $$
%   其中
%   $$
%   \C_{ij} = \sum_{k=1}^s \A_{ik} \B_{kj}.
%   $$
% \end{dingyi}
% \end{frame}

% \begin{frame}


% \begin{li} 
%   用分块矩阵的乘法计算$\A\B$,其中
%   $$
%   \A = \left(
%     \begin{array}{rrrrr}
%       1&0&0&0&0\\
%       0&1&0&0&0\\
%       -1&2&1&0&0\\
%       1&1&0&1&0\\
%       -2&0&0&0&1
%     \end{array}
%   \right), \quad
%   \B = \left(
%     \begin{array}{rrrrr}
%       3&2&0&1&0\\
%       1&3&0&0&1\\
%       -1&0&0&0&0\\
%       0&-1&0&0&0\\
%       0&0&-1&0&0
%     \end{array}
%   \right)
%   $$
% \end{li}
% \end{frame}

% \begin{frame}
% \begin{center}
%   \begin{tikzpicture}       
%     [column 1/.style={anchor=base east},
%     column 2/.style={anchor=base east},
%     column 3/.style={anchor=base east},
%     column 4/.style={anchor=base east},
%     column 5/.style={anchor=base east}]
%     \matrix(A1) [matrix of math nodes]{
%     \A = \\
%   };
%   \matrix(A2) [right=.1in of A1,matrix of math nodes,nodes in empty cells,inner sep=0.2cm,ampersand replacement=\&,left delimiter=(,right delimiter=)] {
%     1 \& 0 \& 0 \& 0 \& 0 \\
%     0 \& 1 \& 0 \& 0 \& 0 \\
%     -1 \& 2 \& 1 \& 0 \& 0 \\
%     1 \& 1 \& 0 \& 1 \& 0 \\
%     -2 \& 0 \& 0 \& 0 \& 1 \\
%   };
%     \draw[red, dashed, very thick] 
%     (A2-1-2.north east) -- (A2-5-2.south east)
%     (A2-2-1.south west) -- (A2-2-5.south east);
%     \matrix(A3) [right=.1in of A2,matrix of math nodes]{
%     = \\
%   };
%     \matrix(A4) [right=.1in of A3,matrix of math nodes,nodes in empty cells,inner sep=0.2cm,ampersand replacement=\&,left delimiter=(,right delimiter=)] {
%     \II_2 \& \zero_{2\times 3} \\
%     \A_1 \& \II_3\\
%   };
%   \end{tikzpicture}
% \end{center}


% \begin{center}
%   \begin{tikzpicture}       
%     [column 1/.style={anchor=base east},
%     column 2/.style={anchor=base east},
%     column 3/.style={anchor=base east},
%     column 4/.style={anchor=base east},
%     column 5/.style={anchor=base east}]
%     \matrix(A1) [matrix of math nodes]{
%     \B = \\
%   };
%     \matrix(A2) [right=.1in of A1,matrix of math nodes,nodes in empty cells,inner sep=0.2cm,ampersand replacement=\&,left delimiter=(,right delimiter=)] {
%     3\&2\&0\&1\&0\\
%     1\&3\&0\&0\&1\\
%     -1\&0\&0\&0\&0\\
%     0\&-1\&0\&0\&0\\
%     0\&0\&-1\&0\&0\\
%   };
%     \draw[red, dashed, very thick] 
%     (A2-1-3.north east) -- (A2-5-3.south east)
%     (A2-2-1.south west) -- (A2-2-5.south east);
%     \matrix(A3) [right=.1in of A2,matrix of math nodes]{
%     = \\
%   };
%     \matrix(A4) [right=.1in of A3,matrix of math nodes,nodes in empty cells,inner sep=0.2cm,ampersand replacement=\&,left delimiter=(,right delimiter=)] {
%     \B_1 \& \II_2 \\
%     -\II_3 \& \zero_{3\times 2}\\
%   };
%   \end{tikzpicture}
% \end{center}
% \end{frame}

% \begin{frame}
% 则
% $$
% \A\B = \left(
%   \begin{array}{cc}
%     \II_2 & \zero\\
%     \A_1 & \II_3
%   \end{array}
% \right)\left(
%   \begin{array}{cc}
%     \B_1 & \II_2\\
%     -\II_3 & \zero
%   \end{array}
% \right) = \left(
%   \begin{array}{cc}
%     \B_1 & \II_2\\
%     \A_1\B_1-\II_3 & \A_1
%   \end{array}
% \right)
% $$
% 其中
% $$
% \A_1\B_1-\II_3 = \left(
%   \begin{array}{rr}
%     -1&2\\
%     1&1\\
%     -2&0
%   \end{array}
% \right)\left(
%   \begin{array}{rrr}
%     3&2&0\\
%     1&3&0
%   \end{array}
% \right)-\left(
%   \begin{array}{ccc}
%     1&0&0\\
%     0&1&0\\
%     0&0&1
%   \end{array}
% \right)=\left(
%   \begin{array}{rrr}
%     -2&4&0\\
%     4&4&0\\
%     -6&-4&-1
%   \end{array}
% \right)
% $$
% \end{frame}

% \begin{frame}

% \begin{li}
%   设$\A$为$m\times n$矩阵,$\B$为$n\times s$矩阵,$\B$按列分块成$1\times s$分块矩阵,
%   将$\A$看成$1\times 1$分块矩阵,则
%   $$
%   \A\B=\A(\bb_1,\bb_2,\cd,\bb_s)=(\A\bb_1,\A\bb_2,\cd,\A\bb_s)      
%   $$
%   若已知$\A\B=0$,则显然
%   $$
%   \A\bb_j=0, \quad j=1,2,\cd,s.
%   $$
%   因此,$\B$的每一列$\bb_j$都是线性方程组$\A\xx=0$的解。
% \end{li}    
% \end{frame}

% \begin{frame}
% \begin{li}
%   设$\A^T\A=\zero$,证明$\A=\zero$.
% \end{li}

% \begin{proof}
% 设$\A=(a_{ij})_{m\times n}$,把$\A$用列向量表示为$\A=(\aa_1, ~\aa_2,~\cd,~\aa_n)$,则
% $$
% \A^T\A = \left(
%   \begin{array}{c}
%     \aa_1^T\\
%     \aa_2^T\\
%     \cd \\
%     \aa_n^T
%   \end{array}
% \right) (\aa_1, ~\aa_2,~\cd,~\aa_n) = \left(
%   \begin{array}{cccc}
%     \aa_1^T\aa_1 & \aa_1^T\aa_2 & \cd & \aa_1^T\aa_n\\
%     \aa_2^T\aa_1 & \aa_2^T\aa_2 & \cd & \aa_2^T\aa_n\\
%     \vd & \vd & & \vd \\
%     \aa_n^T\aa_1 & \aa_n^T\aa_2 & \cd & \aa_n^T\aa_n
%   \end{array}
% \right)
% $$

% 因为$\A^T\A=\zero$,故
% $$
% \aa_i^T \aa_j = 0, \quad i,j=1,2,\cd,n.
% $$

% 特别地,有
% $$
% \aa_j^T \aa_j = 0, \quad j=1,2,\cd,n,
% $$
% 即
% $$
% a_{1j}^2+a_{2j}^2+\cd+a_{mj}^2=0  ~\Rightarrow~ a_{1j}=a_{2j}=\cd=a_{mj}=0 ~\Rightarrow~ \A = \zero.
% $$
% \end{proof}
% \end{frame}

% \begin{frame}

% \begin{li}
%   若$n$阶矩阵$\C,\D$可以分块成同型对角块矩阵,即
%   $$
%   \C = \left(
%     \begin{array}{cccc}
%       \C_1&&&\\
%       &\C_2&&\\
%       &&\cd&\\
%       &&&\C_m
%     \end{array}
%   \right),\quad
%   \D = \left(
%     \begin{array}{cccc}
%       \D_1&&&\\
%       &\D_2&&\\
%       &&\cd&\\
%       &&&\D_m
%     \end{array}
%   \right)
%   $$
%   其中$\C_i$和$\D_i$为同阶方阵($i=1,2,\cd,m$),则
%   $$
%   \C\D = \left(
%     \begin{array}{cccc}
%       \C_1\D_1&&&\\
%       &\C_2\D_2&&\\
%       &&\cd&\\
%       &&&\C_m\D_m
%     \end{array}
%   \right)
%   $$
% \end{li}

% \end{frame}

% \begin{frame}





% \begin{li}
%   可用矩阵的分块来证明:若方阵$\A$为可逆的上三角阵,则$\A^{-1}$也为上三角阵。
% \end{li}
% \end{frame}


% \begin{frame}



% \begin{dingyi}[分块矩阵的转置]
%   分块矩阵$\A=(\A_{kl})_{s\times t}$的转置矩阵为
%   $$
%   \A^T = (\B_{lk})_{t\times s},
%   $$
%   其中$\B_{lk}=\A_{kl}$.
% \end{dingyi}

% \begin{li}
%   $$
%   \A = \left(
%     \begin{array}{ccc}
%       \A_{11} & \A_{12} & \A_{13}\\
%       \A_{21} & \A_{22} & \A_{23}
%     \end{array}
%   \right) ~\Rightarrow~
%   \A = \left(
%     \begin{array}{cc}
%       \A_{11}^T & \A_{21}^T \\[0.2cm]
%       \A_{12}^T & \A_{22}^T \\[0.2cm]
%       \A_{13}^T & \A_{23}^T
%     \end{array}
%   \right)
%   $$

%   $$
%   \B \xlongequal[]{\mbox{按行分块}} \left(
%     \begin{array}{c}
%       \bb_1\\
%       \bb_2\\
%       \vd\\
%       \bb_m
%     \end{array}
%   \right) ~\Rightarrow~
%   \B^T = \left(
%     \begin{array}{cccc}
%       \bb_1^T & \bb_2^T & \cd & \bb_m^T
%     \end{array}
%   \right)
%   $$
% \end{li}



% \end{frame}

% \begin{frame}


% \begin{dingyi}[可逆分块矩阵的逆矩阵]
%   对角块矩阵(准对角矩阵)
%   $$
%   \A = \left(
%     \begin{array}{cccc}
%       \A_1&&&\\
%           &\A_2&&\\
%           &&\dd&\\
%           &&&\A_m
%     \end{array}
%   \right)
%   $$
%   的行列式为$|\A|=|\A_1||\A_2|\cd|\A_m|$,因此,$\A$可逆的充分必要条件为
%   $$
%   |\A_i|\ne 0, \quad i=1,2,\cd, m.
%   $$

%   其逆矩阵为
%   $$
%   \A^{-1} = \left(
%     \begin{array}{cccc}
%       \A_1^{-1}&&&\\
%                &\A_2^{-1}&&\\
%                &&\dd&\\
%                &&&\A_m^{-1}
%     \end{array}
%   \right)
%   $$
% \end{dingyi}
% \end{frame}

% \begin{frame}
% 分块矩阵的作用:
% \begin{itemize}
% \item   用分块矩阵求逆矩阵,可将高阶矩阵的求逆转化为低阶矩阵的求逆。
% \item   一个$2\times 2$的分块矩阵求逆,可以根据逆矩阵的定义,用解矩阵方程的方法解得。
% \end{itemize}
% \end{frame}

% \begin{frame}
% \begin{li}
%   设$\A=\left(
%     \begin{array}{cc}
%       \B&\zero\\
%       \C&\D
%     \end{array}
%   \right)$,其中$\B,\D$皆为可逆矩阵,证明$\A$可逆并求$\A^{-1}$.
% \end{li}
% \end{frame}

% \begin{frame}
% \begin{jie}
%   因$|\A|=|\B||\D|\ne 0$,故$\A$可逆。 设$\A^{-1}=\left(
%     \begin{array}{cc}
%       \X&\Y\\
%       \Z&\T
%     \end{array}
%   \right)$,则
%   $$
%   \left(
%     \begin{array}{cc}
%       \B&\zero\\
%       \C&\D
%     \end{array}
%   \right) \left(
%     \begin{array}{cc}
%       \X&\Y\\
%       \Z&\T
%     \end{array}
%   \right)=\left(
%     \begin{array}{cc}
%       \B\X&\B\Y\\
%       \C\X+\D\Z&\C\Y+\D\T
%     \end{array}
%   \right) = \left(
%     \begin{array}{cc}
%       \II & \zero\\
%       \zero & \II
%     \end{array}
%   \right)
%   $$

%   由此可知
%   $$
%   \begin{array}{ll}
%     \B\X = \II   & \Rightarrow ~ \X = \B^{-1}\\[0.2cm]
%     \B\Y = \zero & \Rightarrow ~ \Y = \zero\\[0.2cm]
%     \C\X+\D\Z = \zero & \Rightarrow ~ \Z = -\D^{-1}\C\B^{-1}\\[0.2cm]
%     \C\Y+\D\T = \II & \Rightarrow ~ \T = \D^{-1}
%   \end{array}
%   $$

%   故
%   $$
%   \A^{-1} = \left(
%     \begin{array}{cc}
%       \B^{-1} & \zero\\
%       -\D^{-1}\C\B^{-1} & \D^{-1}
%     \end{array}
%   \right).
%   $$
% \end{jie}
% \end{frame}

% \begin{frame}




% \begin{dingyi}[分块矩阵的初等变换与分块初等矩阵]
%   对于分块矩阵
%   $$
%   \A = \left(
%     \begin{array}{cc}
%       \A_{11} & \A_{12}\\
%       \A_{21} & \A_{22}
%     \end{array}
%   \right)
%   $$
%   同样可以定义它的3类初等行变换与列变换,并相应地定义3类分块矩阵:
%   \begin{itemize}
%   \item[(i)] 分块倍乘矩阵($\C_1,\C_2$为可逆阵)
%     $$
%     \left(
%       \begin{array}{cc}
%         \C_1 & \zero\\
%         \zero & \II_n
%       \end{array}
%     \right) ~~\mbox{或}~~
%     \left(
%       \begin{array}{cc}
%         \II_m & \zero\\
%         \zero & \C_2
%       \end{array}
%     \right)
%     $$
%   \item[(ii)] 分块倍加矩阵
%     $$
%     \left(
%       \begin{array}{cc}
%         \II_m & \zero\\
%         \C_3 & \II_n
%       \end{array}
%     \right) ~~\mbox{或}~~
%     \left(
%       \begin{array}{cc}
%         \II_m & \C_4\\
%         \zero & \II_n
%       \end{array}
%     \right)
%     $$
%   \item[(iii)] 分块对换矩阵
%     $$
%     \left(
%       \begin{array}{cc}
%         \zero & \II_n\\
%         \II_m & \zero
%       \end{array}
%     \right)
%     $$
%   \end{itemize}
% \end{dingyi}
% \end{frame}

% \begin{frame}

% \begin{li}
%   设$n$阶矩阵$\A$分块表示为
%   $$
%   \A = \left(
%     \begin{array}{cc}
%       \A_{11} & \A_{12}\\
%       \A_{21} & \A_{22}
%     \end{array}
%   \right)
%   $$
%   其中$\A_{11},\A_{22}$为方阵,且$\A$与$\A_{11}$可逆。证明:$\A_{22}-\A_{21}\A_{11}^{-1}\A_{12}$可逆,并求$\A^{-1}$。
% \end{li}
% \end{frame}

% \begin{frame}
% \begin{jie}
%   构造分块倍加矩阵
%   $$
%   \PP_1 = \left(
%     \begin{array}{cc}
%       \II_1 & \zero\\
%       -\A_{21}\A_{11}^{-1} & \II_2
%     \end{array}
%   \right)
%   $$
%   则$$
%   \PP_1\A = \left(
%     \begin{array}{cc}
%       \A_{11} & \A_{12} \\
%       \zero & \A_{22}-\A_{21}\A_{11}^{-1}\A_{12}
%     \end{array}
%   \right)
%   $$
%   两边同时取行列式可知
%   $$
%   |\A| = |\PP_1\A| = |\A_{11}|\cdot |\A_{22}-\A_{21}\A_{11}^{-1}\A_{12}|
%   $$
%   故$\A_{22}-\A_{21}\A_{11}^{-1}\A_{12}$可逆。
% \end{jie}
% \end{frame}

% \begin{frame}
% \begin{jie}[续]
%   $$
%   \PP_1\A = \left(
%     \begin{array}{cc}
%       \A_{11} & \A_{12} \\
%       \zero & \A_{22}-\A_{21}\A_{11}^{-1}\A_{12}
%     \end{array}
%   \right)\xlongequal[]{\textcolor{acolor3}{\ds \QQ=\A_{22}-\A_{21}\A_{11}^{-1}\A_{12}}}
%   \left(
%     \begin{array}{cc}
%       \A_{11} & \A_{12} \\
%       \zero & \QQ
%     \end{array}
%   \right)
%   $$ 
%   构造分块倍加矩阵
%   $$
%   \PP_2 = \left(
%     \begin{array}{cc}
%       \II_1 & -\A_{12}\QQ^{-1}\\
%       \zero & \II_2
%     \end{array}
%   \right)
%   $$ 
%   则
%   $$
%   \PP_2\PP_1\A = \left(
%     \begin{array}{cc}
%       \A_{11} & \zero\\
%       \zero & \QQ
%     \end{array}
%   \right)
%   $$ 
%   于是
%   $$
%   \begin{array}{rl}
%     \A^{-1} & = \left(
%       \begin{array}{cc}
%         \A_{11}^{-1} & \zero\\
%         \zero & \QQ^{-1}
%       \end{array}
%     \right)\left(
%       \begin{array}{cc}
%         \II_1 & -\A_{12}\QQ^{-1}\\
%         \zero & \II_2
%       \end{array}
%     \right)\left(
%       \begin{array}{cc}
%         \II_1 & \zero\\[0.2cm]
%         -\A_{21}\A_{11}^{-1} & \II_2
%       \end{array}
%     \right) \\[0.3in]
%     & = \left(
%       \begin{array}{cc}
%         \A_{11}^{-1} & \zero\\[0.2cm]
%         \zero & \QQ^{-1}
%       \end{array}
%     \right)\left(
%       \begin{array}{cc}
%         \II_1+ \A_{12}\QQ^{-1}\A_{21}\A_{11}^{-1}& -\A_{12}\QQ^{-1}\\[0.2cm]
%         -\A_{21}\A_{11}^{-1} & \II_2
%       \end{array}
%     \right)\\[0.3in]
%     & = \left(
%       \begin{array}{cc}
%         \A_{11}^{-1}+ \A_{11}^{-1}\A_{12}\QQ^{-1}\A_{21}\A_{11}^{-1}& -\A_{11}^{-1}\A_{12}\QQ^{-1}\\[0.2cm]
%         -\QQ^{-1}\A_{21}\A_{11}^{-1} & \QQ^{-1}
%       \end{array}
%     \right)
%   \end{array}
%   $$
% \end{jie}
% \end{frame}

% \begin{frame}

%   \begin{li}
%     设$\QQ=\left(
%       \begin{array}{cc}
%         \A&\B\\
%         \C&\D
%       \end{array}
%     \right)$,且$\A$可逆,证明:
%     $$
%     |\QQ| = |\A| \cdot |\D-\C\A^{-1}\B|
%     $$
%   \end{li}
% \begin{proof}
%   构造分块倍加矩阵
%   $$
%   \PP_1 = \left(
%     \begin{array}{cc}
%       \II_1 & \zero\\
%       -\C\A^{-1} & \II_2
%     \end{array}
%   \right)
%   $$ 
%   则
%   $$
%   \PP_1 \QQ = \left(
%     \begin{array}{cc}
%       \A & \B\\
%       \zero & \D-\C\A^{-1}\B
%     \end{array}
%   \right)
%   $$

%   两边同时取行列式得
%   $$
%   |\QQ| = |\PP_1\QQ| = |\A|\cdot |\D-\C\A^{-1}\B|.
%   $$
% \end{proof}
% \end{frame}

% \begin{frame}
%   \begin{li}
%     设$\A$与$\B$均为$n$阶分块矩阵,证明
%     $$
%     \left|
%       \begin{array}{cc}
%         \A&\B\\
%         \B&\A
%       \end{array}
%     \right| = |\A+\B|~|\A-\B|
%     $$
%   \end{li}
% \end{frame}

% \begin{frame}
% \begin{proof}

%   将分块矩阵$
%   \PP = 
%   \left(
%     \begin{array}{cc}
%       \A&\B\\
%       \B&\A
%     \end{array}
%   \right)$的第一行加到第二行,得
%   $$
%   \left(
%     \begin{array}{cc}
%       \II & \zero\\
%       \II & \II
%     \end{array}
%   \right) \left(
%     \begin{array}{cc}
%       \A&\B\\
%       \B&\A
%     \end{array}
%   \right) = \left(
%     \begin{array}{cc}
%       \A&\B\\
%       \A+\B&\A+\B
%     \end{array}
%   \right)
%   $$
%   再将第一列减去第二列,得
%   $$
%   \left(
%     \begin{array}{cc}
%       \A&\B\\
%       \A+\B&\A+\B
%     \end{array}
%   \right) \left(
%     \begin{array}{cc}
%       \II&\zero\\
%       -\II&\II
%     \end{array}
%   \right) = \left(
%     \begin{array}{cc}
%       \A-\B & \B\\
%       \zero & \A+\B
%     \end{array}
%   \right)
%   $$
%   总之有
%   $$
%   \left(
%     \begin{array}{cc}
%       \II & \zero\\
%       \II & \II
%     \end{array}
%   \right) \left(
%     \begin{array}{cc}
%       \A&\B\\
%       \B&\A
%     \end{array}
%   \right) 
%   \left(
%     \begin{array}{cc}
%       \II&\zero\\
%       -\II&\II
%     \end{array}
%   \right) = \left(
%     \begin{array}{cc}
%       \A-\B & \B\\
%       \zero & \A+\B
%     \end{array}
%   \right)
%   $$
%   两边同时取行列式即得结论。
% \end{proof}
% \end{frame}
