\section{特征值问题}
\subsection{矩阵的特征值与特征向量~~相似矩阵}


\begin{frame}[fragile]\ft{\subsecname}


  \begin{dingyi}[特征值与特征向量]
    设$\A$为复数域$\mathbb C$上的$n$阶矩阵,如果存在数$\lambda\in\mathbb C$和\red{非零}的$n$维向量$\xx$使得
    $$
    \A\xx=\lambda\xx
    $$
    则称$\lambda$为矩阵$\A$的\blue{\underline{特征值}},$\xx$为$\A$的对应于特征值$\lambda$的\blue{\underline{特征向量}}。
  \end{dingyi} 
  
  \begin{zhu}
    \begin{itemize}
  \item[(1)] 特征向量$\xx\ne\zero$;
  \item[(2)] 特征值问题是对方针而言的。 
  \end{itemize}
  \end{zhu}
\end{frame}


\begin{frame}[fragile]

  \begin{dingyi}[特征多项式、特征矩阵、特征方程]
    设$n$阶矩阵$\A=(a_{ij})$,则
    $$
    f(\lambda)=\det(\A-\lambda\II)
    =\left|
      \begin{array}{cccc}
                    a_{11}-\lambda&a_{12}&\cd&a_{1n}\\[0.2cm]
                    a_{21}&a_{22}-\lambda&\cd&a_{2n}\\[0.2cm]
                    \vd&\vd&&\vd\\[0.2cm]
                    a_{n1}&a_{n2}&\cd&a_{nn}-\lambda
                                                 \end{array}
                                                                                  \right|
                                                                                  $$
                                                                                  称为矩阵$\A$的特征多项式,$\A-\lambda\II$称为$\A$的特征矩阵,$\det(\A-\lambda\II)=0$称为$\A$的特征方程。
                                                                                  \end{dingyi}

\end{frame}


\begin{frame}[fragile]
  \begin{li}
    求矩阵
    $$
    \A=\left(
      \begin{array}{rrr}
        5&-1&-1\\
        3&1&-1\\
        4&-2&1
      \end{array}
    \right)
    $$
    的特征值与特征向量。
  \end{li}
\end{frame}

\begin{frame}[fragile,allowframebreaks]
  \begin{jie}
    $$
    \begin{array}{rl}
      \det(\A-\lambda\II)
      &=\left|
        \begin{array}{rrr}
          5-\lambda&-1&-1\\
          3&1-\lambda&-1\\
          4&-2&1-\lambda
        \end{array}
                \right| = \left|
                \begin{array}{rrr}
                  3-\lambda&-1&-1\\
                  3-\lambda&1-\lambda&-1\\
                  3-\lambda&-2&1-\lambda
                \end{array}
                                \right|\\[0.3in]
      &= (3-\lambda)\left|
        \begin{array}{rrr}
          1&-1&-1\\
          1&1-\lambda&-1\\
          1&-2&1-\lambda
        \end{array}
                \right|= (3-\lambda)\left|
                \begin{array}{rrr}
                  1&-1&-1\\
                  0&2-\lambda&0\\
                  0&-1&2-\lambda
                \end{array}
                        \right|\\
      &=(3-\lambda)(\lambda-2)^2=0
    \end{array}
    $$
    故$\A$的特征值为$\lambda_1=3,~\lambda_2=2\mbox{(二重特征值)}$。

    当$\lambda_1=3$时,由$(\A-\lambda\II)\xx=\zero$,即
    $$
    \left(
      \begin{array}{rrr}
        2&-1&-1\\
        3&-2&-1\\
        4&-2&-2
      \end{array}
    \right)
    \left(
      \begin{array}{c}
        x_1\\
        x_2\\
        x_3
      \end{array}
    \right)=
    \left(
      \begin{array}{c}
        0\\
        0\\
        0
      \end{array}
    \right)
    $$
    得其基础解系为$\xx_1=(1,1,1)^T$,因此$k_1\xx_1$($k_1$为非零任意常数)是$\A$对应于$\lambda_1=3$的全部特征向量。
    \vspace{0.1in}

    当$\lambda_2=2$时,由$(\A-\lambda_2\II)\xx=\zero$,即
    $$
    \left(
      \begin{array}{rrr}
        3&-1&-1\\
        3&-1&-1\\
        4&-2&-1
      \end{array}
    \right)
    \left(
      \begin{array}{c}
        x_1\\
        x_2\\
        x_3
      \end{array}
    \right)=
    \left(
      \begin{array}{c}
        0\\
        0\\
        0
      \end{array}
    \right)
    $$
    得其基础解系为$\xx_2=(1,1,2)^T$,因此$k_2\xx_2$($k_2$为非零任意常数)是$\A$对应于$\lambda_2=2$的全部特征向量。
  \end{jie}

\end{frame}




\begin{frame}[fragile]
\begin{dingli}
  若$\xx_1$和$\xx_2$都是$\A$的对应于特征值$\lambda_0$的特征向量,则$k_1\xx_1+k_2\xx_2$也是$\A$的对应于特征值$\lambda_0$的特征向量(其中$k_1,k_2$为任意常数,但$k_1\xx_1+k_2\xx_2\ne 0$)。
\end{dingli}
\end{frame}


\begin{frame}[fragile]
 在$(\A-\lambda\II)\xx=0$的解空间中,除零向量以外的全体解向量就是$\A$的属于特征值$\lambda$的全体特征向量。因此,$(\A-\lambda\II)\xx=0$的解空间也称为$\A$关于特征值$\lambda$的特征子空间,记作$V_\lambda$。$n$阶矩阵$\A$的特征子空间就是$n$维向量空间的子空间,其维数为
 $$
 dim V_\lambda = n - rank(\A-\lambda \II).
 $$
\end{frame}


\begin{frame}[fragile]

\begin{dingli}
  设$n$阶矩阵$\A=(a_{ij})$的$n$个特征值为$\lambda_1,\lambda_2,\cd,\lambda_n$,则
  \begin{itemize}
  \item[(1)] $\ds \sum_{i=1}^n\lambda_i=\sum_{i=1}^na_{ii}$;
  \item[(2)] $\ds \prod_{i=1}^n\lambda_i=\det(\A)$,         
  \end{itemize}
  其中$\sum_{i=1}^na_{ii}$是$\A$的主对角元之和,称为$\A$的迹(trace),记为$tr(\A)$。
\end{dingli}
\end{frame}

\begin{frame}
  \begin{li}[\red{$\bigstar$}]
    设矩阵
    $\A=\left[
      \begin{array}{rrr}
        1&2&-3\\
        -1&4&-3\\
        1&a&5
      \end{array}\right]
    $的全部特征值之积为$24$. \begin{enumerate}
    \item 求$a$的值;
    \item 讨论$\A$能否对角化,若能,求一个可逆矩阵$\PP$使$\PP^{-1}\A\PP=\Lambdabd$为对角阵.
    \end{enumerate}                  
  \end{li}\pause
  \begin{jie}
    \begin{enumerate}
    \item 因$|\A|=\lambda_1\lambda_2\lambda_3=24$,故$a=-2$;\pause
    \item 由$|\A-\lambda\II|=\left|
        \begin{array}{rrr}
          1-\lambda&2&-3\\
          -1&4-\lambda&-3\\
          1&-2&5-\lambda
        \end{array}\right|=(\lambda-2)^2(\lambda-6),
      $故特征值为$\lambda_1=\lambda_2=2, \lambda_3=6$.当$\lambda_1=\lambda_2=2$时,$\rank(\A-2\II)=1$,这说明$\A$由$3$个线性无关的特征向量,故$\A$可对角化.
    \end{enumerate}
  \end{jie}
\end{frame}

\begin{frame}
  当$\lambda=2$时,解方程组$(\A-2\II)\xx=0$,得基础解系
  $$
  \xx_1=(2,1,0)^T, ~~ \xx_2=(-3,0,1)^T,
  $$
  当$\lambda=6$时,解方程组$(\A-6\II)\xx=0$,得基础解系
  $$
  \xx_3=(1,1,-1)^T,
  $$
  取可逆矩阵$\PP=\left[
      \begin{array}{rrr}
        2&-3&1\\
        1&0&1\\
        0&1&-1
      \end{array}\right]
    $,使$\PP^{-1}\A\PP=\diag(2,2,6)$为对角阵.
\end{frame}

\begin{frame}[fragile]

\begin{dingli}
  一个特征向量不能属于不同的特征值。
\end{dingli}

\end{frame}

\begin{frame}[fragile]



\begin{xingzhi}
  若$\lambda$是矩阵$\A$的特征值,$\xx$是$\A$属于$\lambda$的特征向量,则
  \begin{itemize}
  \item[(i)] $k\lambda$是$k\A$的特征值;
  \item[(ii)] $\lambda^m$是$\A^m$的特征值;
  \item[(iii)] 当$\A$可逆时,$\lambda^{-1}$是$\A^{-1}$的特征值;
  \end{itemize}
  且$\xx$仍是矩阵$k\A,\A^m,\A^{-1}$分别对应于$k\lambda,\lambda^m,\lambda^{-1}$的特征向量。
\end{xingzhi}
\end{frame}


\begin{frame}[fragile]

\begin{zhu*}
  若$\lambda$是$\A$的特征值,则$\varphi(\lambda)$是$\varphi(\A)$的特征值,其中
  $$
  \begin{aligned}
    \varphi(\lambda)=a_0+a_1\lambda+\cd+a_m\lambda^m,\\
    \varphi(\A)=a_0\II+a_1\A+\cd+a_m\A^m.
  \end{aligned}
  $$
\end{zhu*}
\end{frame}

\begin{frame}[fragile]

\begin{li}
  设$3$阶矩阵$\A$的特征值为$1,-1,2$,求$|\A^*+3\A-2\II|$.
\end{li} \pause 
\begin{jie}
  因$\A$的特征值全不为零,故$\A$可逆,从而$\A^*=|\A|\A^{-1}$。又因$|\A|=\lambda_1\lambda_2\lambda_3=-2$,故
  $$
  \A^*+3\A-2\II=-2\A^{-1}+3\A-2\II.
  $$
  令$\varphi(\lambda)=-\frac2\lambda+3\lambda-2$,则$\varphi(\lambda)$为上述矩阵的特征值,分别为$\varphi(1)=-1,\varphi(-1)=-3,\varphi(2)=3$,于是
  $$
  |\A^*+3\A-2\II|=(-1)\cdot(-3)\cdot3=9.
  $$
\end{jie}
\end{frame}

\begin{frame}[fragile]

\begin{xingzhi}
  矩阵$\A$与$\A^T$的特征值相同。
\end{xingzhi}


\end{frame}

\begin{frame}[fragile]


\begin{li}
  设$\A=\left(
    \begin{array}{rrr}
      1&-1&1\\
      2&-2&2\\
      -1&1&-1
    \end{array}
  \right)$
  \begin{itemize}
  \item[(i)]求$\A$的特征值与特征向量
  \item[(ii)] 求可逆矩阵$\PP$,使得$\PP^{-1}\A\PP$为对角阵。 
  \end{itemize}
\end{li}
\end{frame}

\begin{frame}[fragile,allowframebreaks]

\begin{jie}
  由
  $$
  \begin{aligned}
    |\A-\lambda\II|&=\left|
      \begin{array}{rrr}
        1-\lambda&-1&1\\
        2&-2-\lambda&2\\
        -1&1&-1-\lambda
      \end{array}
  \right|=\left|
    \begin{array}{rrr}
      1-\lambda&0&1\\
        2&-\lambda&2\\
        -1&-\lambda&-1-\lambda
    \end{array}
  \right|\\
  &=\left|
    \begin{array}{rrr}
      1-\lambda&0&1\\
      2&-\lambda&2\\
      -3&0&-3-\lambda
    \end{array}
  \right|=-\lambda[(\lambda-1)(\lambda+3)+3]=-\lambda^2(\lambda+2),
\end{aligned}
$$
知$\A$的特征值为$\lambda_1=\lambda_2=0$和$\lambda_3=-2$.

当$\lambda_{1,2}=0$时,由$(\A-0\II)\xx=0$,即$\A\xx=0$得基础解系
$$
\xx_1=(1,1,0)^T, \quad \xx_2=(-1,0,1)^T,
$$
故$\A$对应于$\lambda_{1,2}=0$的全体特征向量为
$$
k_1\xx_1+k_2\xx_2=k_1(1,1,0)^T+k_2(-1,0,1)^T,
$$
其中$k_1,k_2$为不全为零的任意常数。


当$\lambda_{3}=-2$时,由$(\A-\lambda_{3}\II)\xx=0$,即
$$
\left(
  \begin{array}{rrr}
    3&-1&1\\
    2&0&2\\
    -1&1&1
  \end{array}
\right)\left(
  \begin{array}{c}
    x_1\\x_2\\x_3
  \end{array}
\right)=\left(
  \begin{array}{c}
    0\\0\\0
  \end{array}
\right)
$$得基础解系
$$
\xx_3=(-1,-2,1)^T,
$$
故$\A$对应于$\lambda_{3}=-2$的全体特征向量为
$$
k_3\xx_3=k_3(-1,-2,1)^T,
$$
其中$k_3$为非零的任意常数。


将$\A\xx_i=\lambda_i\xx_i(i=1,2,3)$表示成
$$
\A(\xx_1,\xx_2,\xx_3)=(\xx_1,\xx_2,\xx_3)\left(
  \begin{array}{ccc}
    \lambda_1&&\\
    &\lambda_2\\
    &&\lambda_3\\
  \end{array}
\right)
$$
取
$$
\PP=(\xx_1,\xx_2,\xx_3)=\left(
  \begin{array}{rrr}
    1&-1&-1\\
    1& 0&-2\\
    0& 1& 1
  \end{array}
\right), \quad \Lambdabd=\left(
  \begin{array}{rrr}
    0&&\\
     &0&\\
     & &-2
  \end{array}
\right),
$$
则$\A\PP=\PP\Lambdabd$,且$|\PP|=2\ne 0$,故得
$$
\PP^{-1}\A\PP=\Lambdabd
$$
为对角阵。
\end{jie}
\end{frame}

\begin{frame}[fragile]

\begin{dingli}
  设$\lambda_1,\lambda_2,\cd,\lambda_m$是方阵$\A$的$m$个特征值,$\xx_1,\xx_2,\cd,\xx_m$依次是与之对应的特征向量,若$\lambda_1,\lambda_2,\cd,\lambda_m$互不相等,则$\xx_1,\xx_2,\cd,\xx_m$线性无关。
\end{dingli}
\end{frame}


\subsection{相似矩阵}

\begin{frame}[fragile]\ft{\subsecname}
\begin{dingyi}
  设$\A,\B$为$n$阶矩阵,若存在可逆矩阵$\PP$,使得
  $$
  \PP^{-1}\A\PP=\B,
  $$
  则称\blue{$\B$是$\A$的相似矩阵,或$\A$与$\B$相似,记为$\A\sim \B$。对$\A$进行运算$\PP^{-1}\A\PP$称为对$\A$进行相似变化,可逆矩阵$\PP$称为把$\A$变成$\B$的相似变换矩阵。}
\end{dingyi}
\end{frame}

\begin{frame}[fragile]\ft{\subsecname}
\begin{dingli}
若$\A\sim \B$,则$\A$与$\B$的特征多项式相同,从而$\A$与$\B$的特征值相同。
\end{dingli}

\begin{tuilun}
  若$\A$与对角阵
  $$
  \Lambdabd=\left(
    \begin{array}{cccc}
      \lambda_1&&&\\
               &\lambda_2&&\\
               &&\dd&\\
               &&&\lambda_n
    \end{array}
  \right)
  $$
  相似,则$\lambda_1,\lambda_2,\cd,\lambda_n$为$\A$的$n$个特征值。
\end{tuilun}
\end{frame}

\subsection{矩阵可对角化的条件}
\begin{frame}[fragile]\ft{\subsecname}
矩阵可对角化,即矩阵与对角阵相似。    

\begin{dingli}
  $\mbox{矩阵可对角化} ~~\Longleftrightarrow~~
  \mbox{$n$阶矩阵有$n$个线性无关的特征向量}$ 
\end{dingli}
\end{frame}

\begin{frame}[fragile]\ft{\subsecname}


\begin{dingli}
  $\A$的属于不同特征值的特征向量是线性无关的。
\end{dingli} 

\begin{tuilun}
  若$\A$有$n$个互不相同的特征值,则$\A$与对角阵相似。
\end{tuilun}


\end{frame}

\begin{frame}[fragile]\ft{\subsecname}



\begin{li}
  设实对称矩阵
  $$
  \A=\left(
    \begin{array}{rrrr}
      1&-1&-1&-1\\
      -1&1&-1&-1\\
      -1&-1&1&-1\\
      -1&-1&-1&1
    \end{array}
  \right)
  $$
  问$\A$是否可对角化?若可对角化,求对角阵$\Lambdabd$及可逆矩阵$\PP$使得$\PP^{-1}\A\PP=\Lambdabd$,再求$\A^k$。
\end{li}
\end{frame}

\begin{frame}[fragile,allowframebreaks]\ft{\subsecname}
\begin{jie}
  由
  $$
  \begin{aligned}
  |\A-\lambda\II|&=
  \left|
    \begin{array}{rrrr}
      1-\lambda&-1&-1&-1\\
      -1&1-\lambda&-1&-1\\
      -1&-1&1-\lambda&-1\\
      -1&-1&-1&1-\lambda
    \end{array}
  \right|\\
  &=-(\lambda+2)
  \left|
    \begin{array}{rrrr}
      1&-1&-1&-1\\
      1&1-\lambda&-1&-1\\
      1&-1&1-\lambda&-1\\
      1&-1&-1&1-\lambda
    \end{array}
  \right|\\
  &=-(\lambda+2)
  \left|
    \begin{array}{rrrr}
      1&-1&-1&-1\\
      0&2-\lambda&0&0\\
      0&0&2-\lambda&0\\
      0&0&0&2-\lambda
    \end{array}
  \right|=(\lambda+2)(\lambda-2)^3,
  \end{aligned}
  $$
  故$\A$的特征值为$\lambda_1=-2$(单根),$\lambda_2=2$(三重根)。

  由$(\A-\lambda_1\II)\xx=0$,即
  $$
  \left(
    \begin{array}{rrrr}
      3&-1&-1&-1\\
      -1&3&-1&-1\\
      -1&-1&3&-1\\
      -1&-1&-1&3
    \end{array}
  \right)
  \left(
    \begin{array}{c}
      x_1\\x_2\\x_3\\x_4
    \end{array}
  \right)=
  \left(
    \begin{array}{c}
      0\\0\\0\\0
    \end{array}
  \right)
  $$
  得$\lambda_1$对应的特征向量为$\{k_1\xx_1|\xx_1=(1,1,1,1)^T, k_1\ne 0\}$。

  由$(\A-\lambda_2\II)\xx=0$,即
  $$
  \left(
    \begin{array}{rrrr}
      -1&-1&-1&-1\\
      -1&-1&-1&-1\\
      -1&-1&-1&-1\\
      -1&-1&-1&-1
    \end{array}
  \right)
  \left(
    \begin{array}{c}
      x_1\\x_2\\x_3\\x_4
    \end{array}
  \right)=
  \left(
    \begin{array}{c}
      0\\0\\0\\0
    \end{array}
  \right)
  $$
  得基础解系:
  $$
  \xx_{21}=(1,-1,0,0)^T, \quad
  \xx_{22}=(1,0,-1,0)^T, \quad
  \xx_{23}=(1,0,0,-1)^T.
  $$

  因$\A$有$4$个线性无关的特征向量,故$\A\sim \Lambdabd$。

  取
  $$
  \PP=(\xx_1,\xx_{21},\xx_{22},\xx_{23})=
  \left(
    \begin{array}{rrrr}
      1& 1& 1&-1\\
      1&-1& 0& 0\\
      1& 0&-1& 0\\
      1& 0& 0&-1
    \end{array}
  \right),
  $$
  则
  $$
  \PP^{-1}\A\PP=\left(
    \begin{array}{rrrr}
      -2&&&\\
        &2&&\\
        &&2\\
        &&&2
    \end{array}
  \right)=\Lambdabd.
  $$
  再由$\A=\PP\Lambdabd\PP^{-1}$得
  $$
  \begin{aligned}
    &\A^k=(\PP\Lambdabd\PP^{-1})^k=\PP\Lambdabd^k\PP^{-1}\\
    &=\left(
    \begin{array}{rrrr}
      1& 1& 1&-1\\
      1&-1& 0& 0\\
      1& 0&-1& 0\\
      1& 0& 0&-1
    \end{array}
  \right)
  \left(
    \begin{array}{rrrr}
      (-2)^k&&&\\
        &2^k&&\\
        &&2^k\\
        &&&2^k
    \end{array}
  \right)
  \frac14
  \left(
    \begin{array}{rrrr}
      1& 1& 1& 1\\
      1&-3& 1& 1\\
      1& 1&-3& 1\\
      1& 1& 1&-3
    \end{array}
  \right)\\
  &=\left\{
    \begin{array}{ll}
      2^k \II_4, & k\mbox{ even}, \\
      2^{k-1} \A, & k\mbox{ odd}.
    \end{array}
  \right.
  \end{aligned}
  $$
\end{jie}

\end{frame}

\begin{frame}
  \begin{li}[\red{$\bigstar$}]
    设$\A$的特征值为$1,2,-3$,矩阵$\B=\A^3-7\A+5\E$,求$\B$.
  \end{li} \pause
  \begin{jie}
    因$\A$的特征值互不相同,故存在可逆阵$\PP$使得$\PP^{-1}\A\PP=\Lambdabd=\left[
      \begin{array}{rrr}
        1&&\\
        &2&\\
        &&-3
      \end{array}
    \right]$. 而
    $$
    \PP^{-1}\B\PP=\PP^{-1}(\A^3-7\A+5\E)\PP=\Lambdabd^3-7\Lambdabd+5\E=-\E,
    $$
    故$\B=\PP(-\E)\PP^{-1}=\E$.
  \end{jie}
\end{frame}

\begin{frame}
  \begin{li}[\red{$\bigstar$}]
    已知三阶矩阵$\A$的特征值为$-1,1,2, \B=\A^2+3\A-2\E$,求$\B$的特征值及行列式$|\B|$.
  \end{li} \pause
  \begin{proof}
    设$\lambda$为$\A$的特征值,则$\varphi(\lambda)=\lambda^2+3\lambda-2$为$\B$的特征值,故$\B$的特征值分别为$-4,2,8$,从而$|\B|=(-4)\times2\times8=-64$.
  \end{proof}
\end{frame}

\begin{frame}[fragile]\ft{\subsecname}
\begin{li}
  设
  $$
  \A=\left(
    \begin{array}{ccc}
      0&0&1\\
      1&1&x\\
      1&0&0
    \end{array}
  \right)
  $$
  问$x$为何值时,矩阵$\A$能对角化?
\end{li}
\end{frame}

\begin{frame}[fragile]\ft{\subsecname}

\begin{jie}
  由
  $$
  |\A-\lambda\II|=
  \left|
    \begin{array}{ccc}
      -\lambda&0&1\\
      1&1-\lambda&x\\
      1&0&-\lambda
    \end{array}
  \right|
  =(1-\lambda)
  \left|
    \begin{array}{ccc}
      -\lambda&1\\
      1&-\lambda
    \end{array}
  \right|=-(\lambda-1)^2(\lambda+1),
  $$
  即$\lambda_1=-1,\lambda_2=\lambda_3=1$。

  对应于单根$\lambda_1=-1$,可求得线性无关的特征向量恰有$1$个,故$\A$可对角化的充分必要条件是对应重根$\lambda_2=\lambda_3=1$,有$2$个线性无关的特征向量,即$(\A-\II)\xx=0$有两个线性无关的解,亦即$\A-\II$的秩$\rank(\A-\II)=1$。由
  $$
  \A-\II=\left(
    \begin{array}{ccc}
      -1&0&1\\
      1&0&x\\
      1&0&-1
    \end{array}
  \right)\sim \left(
    \begin{array}{ccc}
      1&0&-1\\
      0&0&x+1\\
      0&0&0
    \end{array}
  \right)
  $$
  欲使$\rank(\A-\II)=1$,须有$x+1=0$,即$x=-1$。因此当$x=-1$时,矩阵$\A$能对角化。
\end{jie}

\end{frame}

\begin{frame}[fragile]\ft{\subsecname}

\begin{li}
  设$\A=(a_{ij})_{n\times n}$是主对角元全为$2$的上三角矩阵,且存在$a_{ij}\ne 0(i<j)$,问$\A$是否可对角化?
\end{li}

\end{frame}

\begin{frame}[fragile]\ft{\subsecname}
\begin{jie}
  设
  $$
  A=\left(
    \begin{array}{cccc}
      2&*&\cd&*\\
       &2&\cd&*\\
       &&\dd&\vd\\
       &&&2
    \end{array}
  \right)
  $$
  其中$*$为不全为零的任意常数,则
  $$
  |\A-\lambda\II|=(2-\lambda)^n,
  $$
  即$\lambda=2$为$\A$的$n$重特征根,而$\rank(\A-2\II)\ge 1$,故$(\A-2\II)\xx=0$的基础解系所含向量个数$\le n-1$个,即$\A$的线性无关的特征向量的个数$\le n-1$个,因此$\A$不与对角阵相似。
\end{jie}
\end{frame}

\subsection{实对称矩阵的对角化}

\begin{frame}
  
  \begin{dingli}
    实对称矩阵$\A$的任一特征值都是实数。
  \end{dingli}
  
  \begin{dingli}
    实对称矩阵$\A$对应于不同特征值的特征向量是正交的。
  \end{dingli}

  \begin{dingli}
  设$\A$为$n$阶对称阵,则必有正交阵$\QQ$,使得$\QQ^{-1}\A\QQ=\QQ^T\A\QQ=\Lambdabd$,其中$\Lambdabd$是以$\A$的$n$个特征值为对角元的对角阵。
  \end{dingli}

  \begin{tuilun}
    设$\A$为$n$阶对称阵,$\lambda$为$\A$的特征方程的$k$重根,则矩阵$\A-\lambda\II$的秩$\rank(\A-\lambda\II)=n-k$,从而对应特征值$\lambda$恰有$k$个u线性无关的特征向量。
  \end{tuilun} 
\end{frame}

\begin{frame}
  \red{将对称阵$\A$对角化的步骤:}
  \begin{enumerate}
  \item 求出$\A$的全部互不相等的特征值$\lambda_1,\cd,\lambda_s$,它们的重数依次为$k_1,\cd,k_s(k_1+\cd+k_s=n)$;\\[0.1in]
  \item 对每个$k_i$重特征值$\lambda_i$,求$(\A-\lambda_i\II)\xx=0$的基础解系,得$k_i$个线性无关的特征向量;\\[0.1in]
  \item 再把它们正交化、单位化,得$k_i$个两两正交的单位特征向量。因$k_1+\cd+k_s=n$,故总共可得$n$个两两正交的单位特征向量;\\[0.1in]
  \item 将这$n$个两两正交的单位特征向量构成正交阵$\QQ$,便有$\QQ^{-1}\A\QQ=\Lambdabd$。
  \end{enumerate}
\end{frame}

\begin{frame}
  \begin{li}[\red{$\bigstar$}]
    设$3$阶实对称矩阵的三个特征值为$5,5,-4$,属于特征值$-4$的特征向量为$(1,1,-4)^T$,求$\A$.
  \end{li} \pause
  \begin{jie}
    设$\A$属于特征值$\lambda=5$的特征向量为$\betabd_1,\betabd_2$,它们必与$(1,1,-4)^T$正交,即$x_1+x_2-4x_3=0$,可解得$\betabd_1=(1,-1,0)^T, \betabd_2=(2,2,1)^T$.利用施密特正交化过程得正交矩阵
    $$
    \QQ=\frac1{3\sqrt2}\left[
      \begin{array}{rrr}
        3&2\sqrt2&1\\
        -3&2\sqrt2&1\\
        0&\sqrt2&-4
      \end{array}
    \right]
    $$
    于是
    $$
    \A=\QQ\left[
      \begin{array}{rrr}
        5&&\\
        &5&\\
        &&-4
      \end{array}
    \right]\QQ^T=\frac12\left[
      \begin{array}{rrr}
        9&-1&4\\
        -1&9&4\\
         4&4&-6
      \end{array}
    \right]
    $$
  \end{jie}
\end{frame}