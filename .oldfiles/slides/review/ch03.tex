\section{向量的线性相关性及线性方程组的解}
%%%%% 
\subsection{$n$维向量及其线性相关性}


\begin{frame}
  \begin{dingyi}[向量空间]
    数域$\R$上的$n$维向量,在其中定义了上述加法与数乘运算,就称之为$\R$上的$n$维向量空间,仍记为$\R^n$。
  \end{dingyi}
\end{frame}

\begin{frame}
  \begin{dingyi}[线性表示]
    设$\alphabd_i\in \R^n, k_i \in \R (i=1,2,\cd,m)$,则向量
    $$
    \sum_{i=1}^m k_i\alphabd_i = k_1\alphabd_1 + k_2\alphabd_2+\cd + k_m\alphabd_m
    $$
    称为向量组$\alphabd_1,\alphabd_2,\cd,\alphabd_m$在数域$\R$上的一个\red{线性组合}。  如果记
    $$\betabd=\sum_{i=1}^m k_i\alphabd_i,$$
    则称$\betabd$可由$\alphabd_1,\alphabd_2,\cd,\alphabd_m$\red{线性表示}(或\red{线性表出})。
  \end{dingyi}
\end{frame}

\begin{frame}
  设有线性方程组$\A\xx=\bb$,其中$\A$为$m\times n$矩阵。记
  $$\A=(\alphabd_1,\alphabd_2,\cd,\alphabd_n),$$
  即
  $$
  (\alphabd_1,\alphabd_2,\cd,\alphabd_n) \left(
    \begin{array}{c}
      x_1\\
      x_2\\
      \vd\\
      x_n
    \end{array}
  \right)=\bb
  $$
  于是线性方程组可等价的表述为
  $$
  x_1\alphabd_1+x_2\alphabd_2+\cd+x_n\alphabd_n=\bb
  $$
\end{frame}

\begin{frame}
  \begin{zhu*}
    向量$\bb$可由$\alphabd_1,\alphabd_2,\cd,\alphabd_n$线性表示,等价于方程组
    $$
    x_1\alphabd_1+x_2\alphabd_2+\cd+x_n\alphabd_n=\bb
    $$
    有解。
  \end{zhu*}
\end{frame}

\begin{frame}
  \begin{dingyi}[线性相关与线性无关]
    若对$m$个向量$\alphabd_1,\alphabd_2,\cd,\alphabd_m\in \R^n$,有$m$个不全为零的数$k_1,k_2,\cd,k_m\in \R$,使
    \begin{equation}\label{def1}
      k_1\alphabd_1 + k_2\alphabd_2+\cd + k_m\alphabd_m = \zero        
    \end{equation}
    成立,则称\red{$\alphabd_1,\alphabd_2,\cd,\alphabd_m$线性相关};
    否则,称\red{$\alphabd_1,\alphabd_2,\cd,\alphabd_m$线性无关}。
  \end{dingyi}
\end{frame}

\begin{frame}
  \begin{zhu*}
    向量组$\alphabd_1,\alphabd_2,\cd,\alphabd_m$线性无关,指的是
    \begin{itemize}
    \item 没有不全为零的数$k_1,k_2,\cd,k_m$使(\ref{def1})成立 ;\\[0.1in]
    \item 只有当$k_1,k_2,\cd,k_m$全为零时,才使(\ref{def1})成立 ;\\[0.1in]
    \item 若(\ref{def1})成立,则$k_1,k_2,\cd,k_m$必须全为零。
    \end{itemize}
  \end{zhu*}
\end{frame}


\begin{frame}
  \begin{dingli}
    以下两组等价关系成立:
    \begin{itemize}
    \item  向量组$\alphabd_1,\alphabd_2,\cd,\alphabd_m$线性相关,等价于齐次方程组
      $$
      x_1\alphabd_1+x_2\alphabd_2+\cd+x_m\alphabd_m=\zero
      $$
      有非零解。
    \item  向量组$\alphabd_1,\alphabd_2,\cd,\alphabd_m$线性无关,等价于齐次方程组
      $$
      x_1\alphabd_1+x_2\alphabd_2+\cd+x_m\alphabd_m=\zero
      $$
      只有零解。
    \end{itemize}
  \end{dingli}
\end{frame}

\begin{frame}
  \begin{li}[$\bigstar$]
    下列命题是否正确?如正确,请证明,若不正确请举反例:\blue{向量组$a_1,a_2,\cd,a_s(s\ge 2)$线性无关的充分必要条件是存在一组不全为零的常数$k_1,k_2,\cd,k_s$使得$k_1a_1+k_2a_2+\cd+k_sa_s\ne 0$.}
  \end{li}
  \pause
  \begin{jie}
    不正确。如$a_1=(1,2,3), a_2=(2,4,6)$,存在$k_1=k_2=1\ne 0$使得$k_1a_1+k_2a_2\ne0$,但$a_1,a_2$线性相关。
  \end{jie}
\end{frame}


\begin{frame}
  \blue{证明向量组$\alphabd_1,\alphabd_2,\cd,\alphabd_m$线性无关}的基本方法为:

  设
  $$
  x_1\alphabd_1+x_2\alphabd_2+\cd+x_m\alphabd_m=\zero
  $$
  然后说明上式成立,只能有唯一选择:
  $$
  x_1=x_2=\cd=x_m=0.
  $$
\end{frame}


\begin{frame}
  对于只含一个向量的向量组,
  \begin{itemize}
  \item 当$\alphabd=\zero$时,向量组$\alphabd$线性相关
  \item 当$\alphabd\ne \zero$时,向量组$\alphabd$线性无关
  \end{itemize}
\end{frame}




\begin{frame}
  \begin{dingli}
    向量组$\alphabd_1,\alphabd_2,\cd,\alphabd_m(m\ge 2)$线性相关的充分必要条件是$\alphabd_1,\alphabd_2,\cd,\alphabd_m$中\red{至少有一个向量}可由其余$m-1$个向量线性表出。
  \end{dingli}

\end{frame}




\begin{frame}
  \begin{jielun}
    如果向量组$\alphabd_1,\alphabd_2,\cd,\alphabd_m$中有一部分向量线性相关,则整个向量组也线性相关。
  \end{jielun}

  \begin{zhu*}
    \begin{itemize}
    \item 如果$\alphabd_1,\alphabd_2,\cd,\alphabd_m$线性无关,则其中任一部分向量组也线性无关。              
    \item     \red{部分相关,则整体相关;整体无关,则部分无关。}
    \end{itemize}
    
  \end{zhu*}


\end{frame}

\begin{frame}
  \begin{dingli}
    设$\alphabd_1,\alphabd_2,\cd,\alphabd_r\in \R^n$,其中
    $$
    \begin{aligned}
      \alphabd_1 &= (a_{11},~a_{21},~\cd,~a_{n1})^T,\\
      \alphabd_2 &= (a_{12},~a_{22},~\cd,~a_{n2})^T,\\
      &\cd,\\
      \alphabd_r &= (a_{1r},~a_{2r},~\cd,~a_{nr})^T,
    \end{aligned}
    $$
    则向量组$\alphabd_1,\alphabd_2,\cd,\alphabd_r$线性相关的充分必要条件是齐次线性方程组
    \begin{equation}\label{ax}
      \A \xx = \zero
    \end{equation}
    有非零解,其中
    $$
    \A = (\alphabd_1,~\alphabd_2,~\cd,~\alphabd_r) = \left(
      \begin{array}{cccc}
        a_{11}&a_{12}&\cd&a_{1r}\\[0.05in]
        a_{21}&a_{22}&\cd&a_{2r}\\[0.05in]
        \vd&\vd&&\vd\\[0.05in]
        a_{n1}&a_{n2}&\cd&a_{nr}.
      \end{array}\right), \xx = \left(
      \begin{array}{c}
        x_1\\[0.05in]
        x_2\\[0.05in]
        \vd\\[0.05in]
        x_r
      \end{array}
    \right)
    $$
  \end{dingli}
\end{frame}  


\begin{frame}
  \begin{jielun}
    对于齐次线性方程组,如果
    $$
    \red{\mbox{未知量个数} ~~>~~ \mbox{方程个数},}
    $$
    则它必有无穷多解,从而必有非零解。
  \end{jielun}   

\end{frame}




\begin{frame}
  \begin{dingli}
    任意$n+1$个$n$维向量都是线性相关的。
  \end{dingli}

  \begin{zhu*}
    \begin{itemize}
    \item    向量个数$~>~$向量维数 $~~\red{\Rightarrow}~~$ 向量组必线性相关。 
    \item     在$\mathbb R^n$中,任意一组线性无关的向量最多只能含$n$个向量。
    \end{itemize}
  \end{zhu*}
\end{frame}

\begin{frame}
  \begin{dingli}
    若向量组$\alphabd_1,\alphabd_2,\cd,\alphabd_r$线性无关,而$\red{\betabd},\alphabd_1,\alphabd_2,\cd,\alphabd_r$线性相关,则$\red{\betabd}$可由$\alphabd_1,\alphabd_2,\cd,\alphabd_r$线性表示,并且表示法惟一。
  \end{dingli} \vspace{0.1in}

  \begin{tuilun}
    如果$\R^n$中的$n$个向量$\alphabd_1,\alphabd_2,\cd,\alphabd_n$线性无关,则$\R^n$中的任一向量$\alphabd$可由$\alphabd_1,\alphabd_2,\cd,\alphabd_n$线性表示,且表示法惟一。
  \end{tuilun}
\end{frame}

\begin{frame}
  \begin{li}
    设$\alphabd_1=(1,-1,1),\alphabd_2=(1,2,0),\alphabd_3=(1,0,3),\alphabd_4=(2,-3,7)$.  问:
    \begin{itemize}
    \item[(1)]$\alphabd_1,\alphabd_2,\alphabd_3$是否线性相关?
    \item[(2)]$\alphabd_4$可否由$\alphabd_1,\alphabd_2,\alphabd_3$线性表示?如能表示求出其表示式。
    \end{itemize}
  \end{li}
  % \pause 
  % \begin{jie}
  %   \begin{itemize}
  %   \item[(1)]    考察
  %     $
  %     \A = (\alphabd_1^T, \alphabd_2^T, \alphabd_3^T) = \left(
  %       \begin{array}{rrr}
  %         1&1&1\\
  %         -1&2&0\\
  %         1&0&3
  %       \end{array}
  %     \right). 
  %     $ \quad
  %     由$|\A|=7$可知$\A$可逆,故$\A\xx=\zero$只有零解,从而$\alphabd_1,\alphabd_2,\alphabd_3$线性无关。  \pause 
  %   \item[(2)] 根据推论,$\alphabd_4$可由$\alphabd_1,\alphabd_2,\alphabd_3$线性表示,且表示法惟一。  设
  %     $$
  %     x_1\alphabd_1+x_2\alphabd_2+x_3\alphabd_3=\alphabd_4   \Rightarrow
  %     x_1\alphabd_1^T+x_2\alphabd_2^T+x_3\alphabd_3^T=\alphabd_4^T       
  %     $$
  %     即$$
  %     \left(
  %       \begin{array}{ccc}
  %         \alphabd_1^T &\alphabd_2^T& \alphabd_3^T  
  %       \end{array}
  %     \right) \left(
  %       \begin{array}{c}
  %         x_1\\
  %         x_2\\
  %         x_3
  %       \end{array}
  %     \right)= 
  %     \left(
  %       \begin{array}{rrr}
  %         1&1&1\\
  %         -1&2&0\\
  %         1&0&3
  %       \end{array}
  %     \right) \left(
  %       \begin{array}{c}
  %         x_1\\
  %         x_2\\
  %         x_3
  %       \end{array}
  %     \right) =  \left(
  %       \begin{array}{r}
  %         2\\
  %         -3\\
  %         7
  %       \end{array}
  %     \right)
  %     $$  
  %     解此方程组得惟一解$x_1=1,x_2=-1,x_3=2$,故
  %     $
  %     \red{\alphabd_4=\alphabd_1-\alphabd_2+2\alphabd_3.}
  %     $
  %   \end{itemize}
  % \end{jie}
\end{frame}

\begin{frame}
  \begin{li}
    设向量组$\alphabd_1,\alphabd_2,\alphabd_3$线性无关,又$\betabd=\alphabd_1+\alphabd_2+2\alphabd_3$,
    $\betabd_2=\alphabd_1-\alphabd_2$,$\betabd_3=\alphabd_1+\alphabd_3$,证明$\betabd_1,\betabd_2,\betabd_3$线性相关。       
  \end{li}
  % \pause 
  % \begin{jie}
  %   设有数$x_1,x_2,x_3$使得
  %   \begin{equation}\label{li5}
  %     x_1\betabd_1+x_2\betabd_2+x_3\betabd_3=\zero
  %   \end{equation}    
  %   即
  %   $$
  %   x_1(\alphabd_1+\alphabd_2+2\alphabd_3)+x_2(\alphabd_1-2\alphabd_2)+x_3(\alphabd_1+\alphabd_3)=\zero
  %   $$
  %   亦即
  %   $$
  %   (x_1+x_2)\alphabd_1+(x_1-2x_2)\alphabd_2+(x_1+x_3)\alphabd_3=\zero
  %   $$
  %   因为$\alphabd_1,\alphabd_2,\alphabd_3$线性无关,故
  %   $$
  %   \left\{
  %     \begin{array}{rcrcrcrcr}
  %       x_1&+&x_2&&&=&0\\
  %       x_1&-&x_2&&&=&0\\
  %       2x_1&&&+&x_3&=&0.
  %     \end{array}
  %   \right.
  %   $$
  %   求解该方程组可得非零解$(-1,-1,2)$。因此,有不全为零的数$x_1,x_2,x_3$使得(\ref{li5})成立,从而$\betabd_1,\betabd_2,\betabd_3$线性相关。
  % \end{jie}
\end{frame}

\begin{frame}
  \begin{li}
    证明:$\alphabd_1+\alphabd_2,\alphabd_2+\alphabd_3,\alphabd_3+\alphabd_1$线性无关的充要条件是$\alphabd_1,\alphabd_2,\alphabd_3$线性无关。
  \end{li}
  % \begin{proof}
  %   \red{($\Rightarrow$)} \quad
  %   假设$\alphabd_1,\alphabd_2,\alphabd_3$线性相关,则有不全为零的数$x_1+x_2,x_2+x_3,x_3+x_1$使得
  %   $$
  %   (x_1+x_2)\alphabd_1+(x_2+x_3)\alphabd_2+(x_3+x_1)\alphabd_3=\zero
  %   $$
  %   即
  %   $$
  %   x_1(\alphabd_1+\alphabd_2)+x_2(\alphabd_2+\alphabd_3)+x_3(\alphabd_3+\alphabd_1)=\zero
  %   $$
  %   \pause 
  %   \red{($\Leftarrow$)} \quad
  %   设有$x_1,x_2,x_3$使得
  %   \begin{equation}\label{li6-1}
  %     x_1(\alphabd_1+\alphabd_2)+x_2(\alphabd_2+\alphabd_3)+x_3(\alphabd_3+\alphabd_1)=\zero
  %   \end{equation}
  %   即
  %   $$
  %   (x_1+x_3)\alphabd_1+(x_1+x_2)\alphabd_2+(x_2+x_3)\alphabd_3=\zero
  %   $$
  %   因为$\alphabd_1,\alphabd_2,\alphabd_3$线性无关,故
  %   $$
  %   x_1+x_3=0, \quad x_1+x_2=0, \quad x_2+x_3=0,
  %   $$
  %   该方程组只有零解。这说明若使(\ref{li6-1}),必有$x_1=x_2=x_3=0$,从而$\alphabd_1+\alphabd_2,\alphabd_2+\alphabd_3,\alphabd_3+\alphabd_1$线性无关。
  % \end{proof}
\end{frame}

\begin{frame}
  \begin{dingli}
    如果一组$n$维向量$\alphabd_1,\alphabd_2,\cd,\alphabd_s$线性无关,那么把这些向量各任意添加$m$个分量所得的向量(\red{$n+m$维})组$\alphabd^*_1,\alphabd^*_2,\cd,\alphabd^*_s$也线性无关。亦即
    $$
    \left(
      \begin{array}{c}
        a_{11}\\
        a_{21}\\
        \vd\\
        a_{n1}\\
      \end{array}
    \right)
    \cd,
    \left(
      \begin{array}{c}
        a_{1s}\\
        a_{2s}\\
        \vd\\
        a_{ns}\\
      \end{array}
    \right) \mbox{线性无关}  ~~~\blue{\Rightarrow}~~~
    \left(
      \begin{array}{c}
        a_{11}\\
        a_{21}\\
        \vd\\
        a_{n1}\\
        \red{a_{n+1,1}}\\
        \vd\\
        \red{a_{n+m,1}}
      \end{array}
    \right),
    \cd,
    \left(
      \begin{array}{c}
        a_{1s}\\
        a_{2s}\\
        \vd\\
        a_{ns}\\
        \red{a_{n+1,s}}\\
        \vd\\
        \red{a_{n+m,s}}
      \end{array}
    \right) \mbox{线性无关}
    $$
  \end{dingli}
\end{frame}

\begin{frame}
  \begin{dingli}
    如果$\alphabd_1,\alphabd_2,\cd,\alphabd_s$线性相关,那么它们各去掉相同的若干个分量所得到的新向量也线性相关,亦即
    $$  
    \left(
      \begin{array}{c}
        a_{11}\\
        a_{21}\\
        \vd\\
        a_{n1}\\
        \red{a_{n+1,1}}\\
        \vd\\
        \red{a_{n+m,1}}
      \end{array}
    \right),
    \cd,
    \left(
      \begin{array}{c}
        a_{1s}\\
        a_{2s}\\
        \vd\\
        a_{ns}\\
        \red{a_{n+1,s}}\\
        \vd\\
        \red{a_{n+m,s}}
      \end{array}
    \right) \mbox{线性相关}  ~~~\blue{\Rightarrow}~~~
    \left(
      \begin{array}{c}
        a_{11}\\
        a_{21}\\
        \vd\\
        a_{n1}\\
      \end{array}
    \right)
    \cd,
    \left(
      \begin{array}{c}
        a_{1s}\\
        a_{2s}\\
        \vd\\
        a_{ns}\\
      \end{array}
    \right) \mbox{线性相关}
    $$
  \end{dingli}
\end{frame}


\begin{frame}
  \begin{tuilun}
    设向量组线性相关,若增加的分量全为零,则得到的新向量组也线性相关。
  \end{tuilun}

  \purple{对应位置全为零的向量,不影响向量组的线性相关性。}

  如
  $$  
  \left(
    \begin{array}{c}
      a_{11}\\
      0\\
      a_{21}\\
      \vd\\
      a_{n1}\\
      0\\
      \vd\\
      0
    \end{array}
  \right),
  \cd,
  \left(
    \begin{array}{c}
      a_{1s}\\
      0\\
      a_{2s}\\
      \vd\\
      a_{ns}\\
      0\\
      \vd\\
      0
    \end{array}
  \right) \mbox{~~与~~}
  \left(
    \begin{array}{c}
      a_{11}\\
      a_{21}\\
      \vd\\
      a_{n1}\\
    \end{array}
  \right),
  \cd,
  \left(
    \begin{array}{c}
      a_{1s}\\
      a_{2s}\\
      \vd\\
      a_{ns}\\
    \end{array}
  \right) 
  $$
  线性相关性一致。
\end{frame}

\begin{frame}
  \begin{li}
    考察以下向量组的线性相关性:
    $$
    \left(
      \begin{array}{c}
        1\\
        0\\
        0\\
        2\\
        5
      \end{array}
    \right), \quad
    \left(
      \begin{array}{c}
        0\\
        1\\
        0\\
        6\\
        9
      \end{array}
    \right), \quad
    \left(
      \begin{array}{c}
        0\\
        0\\
        1\\
        4\\
        3
      \end{array}
    \right)
    $$
  \end{li} \pause 
  \begin{jie}
    去掉最后两个分量所得的向量组
    $$
    \left(
      \begin{array}{c}
        1\\
        0\\
        0
      \end{array}
    \right), \quad
    \left(
      \begin{array}{c}
        0\\
        1\\
        0
      \end{array}
    \right), \quad
    \left(
      \begin{array}{c}
        0\\
        0\\
        1
      \end{array}
    \right)
    $$
    线性无关,故原向量组线性无关。
  \end{jie}
\end{frame}


\subsection{向量组的秩及其极大线性无关组}

\begin{frame}
  \begin{dingyi}[向量组的秩]
    向量组$\alphabd_1,\alphabd_2,\cd,\alphabd_s$中,若
    \begin{itemize}
    \item 存在$r$个\blue{\underline{线性无关}}的向量,
    \item 且其中\blue{\underline{任一向量}}可由这$r$个线性无关的向量线性表示, 
    \end{itemize}
    则数$r$称为\red{向量组的秩(rank)},记作
    $$
    \rank(\alphabd_1,\alphabd_2,\cd,\alphabd_s)=r
    $$
    或
    $$
    \mathrm{rank}(\alphabd_1,\alphabd_2,\cd,\alphabd_s)=r
    $$
  \end{dingyi}

\end{frame}

\begin{frame}
  \begin{itemize}
  \item 若$\alphabd_1,\alphabd_2,\cd,\alphabd_s$线性无关,
    则$\rank(\alphabd_1,\alphabd_2,\cd,\alphabd_s)=s$;
  \item 只含零向量的向量组的秩为零。
  \item 只含一个非零向量的向量组的秩为1。
  \end{itemize}
\end{frame}

\begin{frame}
  \begin{dingyi}
    若向量组$B:~\betabd_1,\betabd_2,\cd,\betabd_t$中每个向量可由向量组$A:~\alphabd_1,\alphabd_2,\cd,\alphabd_s$线性表示,
    就称\red{向量组$B:~\betabd_1,\betabd_2,\cd,\betabd_t$可由向量组$A:~\alphabd_1,\alphabd_2,\cd,\alphabd_s$线性表示}。 
    \vspace{0.1in}
    
    如果两个向量组可以互相线性表示,则称这两个向量组是\red{等价}的。
  \end{dingyi}
\end{frame}

\begin{frame}
  \begin{li}
    设向量$\betabd$可由向量组$\alphabd_1,\cd,\alphabd_r$线性表出,但不能由$\alphabd_1,\cd,\alphabd_{r-1}$线性表出,证明:向量组$U: \alphabd_1,\cd,\alphabd_r$与向量组$V: \alphabd_1,\cd,\alphabd_{r-1},\betabd$等价。
  \end{li}
  \pause
  \begin{proof}
    由已知条件知$V$可由$U$线性表出,而$\alphabd_1,\cd,\alphabd_{r-1}$可由$V$线性表出,故只需证明$\alphabd_r$可由$V$线性表出。设
    $$
    \betabd=k_1\alphabd_1+\cd+k_{r-1}\alphabd_{r-1}+k_r\alphabd_r
    $$
    则$k_r\ne 0$.
    否则$\betabd$可由$\alphabd_1,\cd,\alphabd_{r-1}$线性表出,与已知矛盾,故$k_r\ne 0$。 于是
    $$
    \alphabd_r=-\frac1{k_r}(\betabd-k_1\alphabd_1-\cd-k_{r-1}\alphabd_{r-1})
    $$
    即$\alphabd_r$可由$V$线性表出。
  \end{proof}
\end{frame}


\begin{frame}
  \begin{dingli}
    若向量组$\blue{B:~\betabd_1,\betabd_2,\cd,\betabd_t}$可由向量组$\blue{A:~\alphabd_1,\alphabd_2,\cd,\alphabd_s}$线性表示,且$\blue{t>s}$,
    则$\blue{B:~\betabd_1,\betabd_2,\cd,\betabd_t}$线性相关。
  \end{dingli}

\end{frame}

\begin{frame}
  \begin{dingyi}[向量组的秩的等价定义 \& 极大线性无关组]
    设有向量组$\alphabd_1,\alphabd_2,\cd,\alphabd_s$。
    如果能从其中选出$r$个向量$\alphabd_1,\alphabd_2,\cd,\alphabd_{\red{r}}$,满足
    \begin{itemize}
    \item 向量组$\alphabd_1,\alphabd_2,\cd,\alphabd_{\red{r}}$线性无关;
    \item 向量组$\alphabd_1,\alphabd_2,\cd,\alphabd_s$中任意$r+1$个向量都线性相关,
    \end{itemize}
    则称向量组$\alphabd_1,\alphabd_2,\cd,\alphabd_{\red{r}}$为原向量组的一个\red{极大线性无关组},简称\red{极大无关组}。 
    \vspace{0.1in}

    \blue{\underline{极大线性无关组所含向量的个数$\red{r}$}},称为原向量组的\red{秩}。
  \end{dingyi}
\end{frame}

\begin{frame}
  \begin{zhu}
    \begin{itemize}
    \item   秩为$r$的向量组中,任一个线性无关的部分组最多含有$r$个向量;
    \item 一般情况下,极大无关组不惟一;
    \item 不同的极大无关组所含向量个数相同;
    \item 极大无关组与原向量组是等价的;
    \item 极大无关组是原向量组的\red{全权代表}。
    \end{itemize}
  \end{zhu}
\end{frame}

\begin{frame}
  \begin{tuilun}
    设$\blue{\rank(\alphabd_1,\alphabd_2,\cd,\alphabd_s)=p,~~\rank(\betabd_1,\betabd_2,\cd,\betabd_t)=r}$,
    如果向量组$\blue{B:~\betabd_1,\betabd_2,\cd,\betabd_t}$可由$\blue{A:~\alphabd_1,\alphabd_2,\cd,\alphabd_s}$线性表示,则
    $$\red{r\le p.}$$
  \end{tuilun}

\end{frame}


\begin{frame}
  \begin{tuilun}
    \red{等价向量组的秩相等}。
  \end{tuilun}
\end{frame}

\subsection{矩阵的秩}
\begin{frame}
  \begin{dingyi}[行秩 \& 列秩]
    \begin{itemize}
    \item
      对于矩阵$\A$,把它的每一行称为$\A$的一个\red{行向量}。
      把$\A$的行向量组的秩,称为矩阵$\A$的\red{行秩}。
    \item
      对于矩阵$\A$,把它的每一列称为$\A$的一个\red{列向量}。
      把$\A$的列向量组的秩,称为矩阵$\A$的\red{列秩}。
    \end{itemize}      
  \end{dingyi}    
  对于$m\times n$阶矩阵$\A$,
  \begin{itemize}
  \item $\A$的行秩$~\le~ m$;
  \item $\A$的列秩$~\le~ n$。
  \end{itemize}
\end{frame}

\begin{frame}
  \begin{dingyi}[阶梯形矩阵]
    若矩阵$\A$满足
    \begin{itemize}
    \item[(1)] 零行在最下方;
    \item[(2)] 非零行首元的列标号随行标号的增加而严格递增,
    \end{itemize}
    则称$\A$为\red{阶梯形矩阵}。
  \end{dingyi}

  \begin{li*}
    $$
    \left(
      \begin{array}{rrrr}
        2&0&2&1\\
        0&5&2&-2\\
        0&0&3&2\\
        0&0&0&0
      \end{array}
    \right)
    $$
  \end{li*}
\end{frame}


\begin{frame}
  \begin{dingyi}{行简化阶梯形矩阵}
    若矩阵$\A$满足
    \begin{itemize}
    \item[(1)] 它是阶梯形矩阵;
    \item[(2)] 非零行首元所在的列除了非零行首元外,其余元素全为零,
    \end{itemize}
    则称$\A$为\red{行简化阶梯形矩阵}。
  \end{dingyi}

  \begin{li*}
    $$
    \left(
      \begin{array}{rrrr}
        2&0&0&1\\
        0&5&0&-2\\
        0&0&3&2\\
        0&0&0&0
      \end{array}
    \right)
    $$
  \end{li*}
\end{frame}

\begin{frame}
  \begin{dingyi}{行简化阶梯形矩阵}
    若矩阵$\A$满足
    \begin{itemize}
    \item[(1)] 它是行简化阶梯形矩阵;
    \item[(2)] 非零行首元全为$1$,
    \end{itemize}
    则称$\A$为\red{行最简阶梯形矩阵}。
  \end{dingyi}

  \begin{li*}
    $$
    \left(
      \begin{array}{rrrr}
        1&0&0&1\\
        0&1&0&-2\\
        0&0&1&2\\
        0&0&0&0
      \end{array}
    \right)
    $$
  \end{li*}
\end{frame}



\begin{frame}
  阶梯形矩阵
  \begin{figure}
    \centering
    \begin{tikzpicture}
      \matrix (M) [matrix of math nodes]  { 
        \A = \\
      };
      \matrix(MM) [right=.1in of M, matrix of math nodes,nodes in empty cells,
      column sep=3ex,row sep=1ex,ampersand replacement=\&,left delimiter=(,right delimiter=)] {
        a_{11} \& a_{12} \& a_{13}  \& a_{14} \& a_{15}\\
        0 \& 0 \& a_{23}  \& a_{24} \& a_{25}\\
        0 \& 0 \& 0  \& a_{34} \& a_{35}\\
        0 \& 0 \& 0  \& 0 \& 0\\
      };
      \draw[thick,red,dashed] (MM-2-1.north west)--(MM-2-2.north east)
      --(MM-3-2.north east)--(MM-3-3.north east)
      --(MM-4-3.north east)--(MM-4-5.north east);
    \end{tikzpicture}
  \end{figure}
  其中$a_{11}\ne0, a_{23}\ne 0, a_{34}\ne 0$。
  \red{验证$\A$的行秩$=3$,列秩$=3$}。
\end{frame}


\begin{frame}
  \begin{jielun}
    阶梯形矩阵的行秩等于列秩,其值等于阶梯形矩阵的非零行的行数。
  \end{jielun}
\end{frame}

\begin{frame}
  \begin{dingli}
    初等行变换不改变矩阵的行秩和列秩。
  \end{dingli}
  \begin{dingli}
    初等变换不改变矩阵的行秩与列秩。
  \end{dingli}

\end{frame}

\begin{frame}
  \begin{li}[$\bigstar$]
    设向量组
    $$
    \alphabd_1=\left(
      \begin{array}{r}
        -1\\-1\\0\\0
      \end{array}
    \right),~~ \alphabd_2=\left(
      \begin{array}{r}
        1\\2\\1\\-1
      \end{array}
    \right),~~ \alphabd_3=\left(
      \begin{array}{r}
        0\\1\\1\\-1
      \end{array}
    \right),~~ \alphabd_4=\left(
      \begin{array}{r}
        1\\3\\2\\1
      \end{array}
    \right),~~ \alphabd_5=\left(
      \begin{array}{r}
        2\\6\\4\\-1
      \end{array}
    \right)
    $$
    求向量组的秩及其一个极大无关组,并将其余向量用该极大无关组线性表示。
  \end{li}
\end{frame}

\begin{frame}[allowframebreaks]
  \begin{jie}
    作矩阵$\A=(\alphabd_1,\alphabd_2,\alphabd_3,\alphabd_4,\alphabd_5)$,由
    $$
    \begin{array}{rl}
      \A &= \left(
           \begin{array}{rrrrr}
             -1&1&0&1&2\\
             -1&2&1&3&6\\
             0&1&1&2&4\\
             0&-1&-1&1&-1
           \end{array}
                        \right) \xlongrightarrow[r_2+r_1]{ r_1\times(-1)}
                        \left(
                        \begin{array}{rrrrr}
                          1&-1&0&-1&-2\\
                          0&1&1&2&4\\
                          0&1&1&2&4\\
                          0&-1&-1&1&-1
                        \end{array}
                                     \right)\\[0.4in]
         &\xlongrightarrow[r_4+r_2]{r_3- r_2}
           \left(
           \begin{array}{rrrrr}
             1&-1&0&-1&-2\\
             0&1&1&2&4\\
             0&0&0&0&0\\
             0&0&0&3&3
           \end{array}
                      \right) \xlongrightarrow[r_3\leftrightarrow r_4]{r_4\div 3}
                      \left(
                      \begin{array}{rrrrr}
                        1&-1&0&-1&-2\\
                        0&1&1&2&4\\
                        0&0&0&1&1\\
                        0&0&0&0&0
                      \end{array}
                                 \right)
    \end{array}
    $$

    $$
    \begin{array}{rl}
      & \xlongrightarrow[r_2 -2 r_3]{r_1+r_3}
        \left(
        \begin{array}{rrrrr}
          1&-1&0&0&-1\\
          0&1&1&0&2\\
          0&0&0&1&1\\
          0&0&0&0&0
        \end{array}
                   \right) \xlongrightarrow[]{r_1+r_2}
                   \left(
                   \begin{array}{rrrrr}
                     1&0&1&0&1\\
                     0&1&1&0&2\\
                     0&0&0&1&1\\
                     0&0&0&0&0
                   \end{array}
                              \right) = \B
    \end{array}
    $$
    将最后一个阶梯矩阵$\B$记为$(\betabd_1,\betabd_2,\betabd_3,\betabd_4,\betabd_5)$

    易知$\betabd_1,\betabd_2,\betabd_4$为$\B$的列向量组的一个极大无关组,故$\alphabd_1,\alphabd_2,\alphabd_4$也为$\A$的列向量组的一个极大无关组,故
    $$
    \rank(\alphabd_1,\alphabd_2,\alphabd_3,\alphabd_4,\alphabd_5)=3,
    $$
    且
    $$
    \begin{array}{l}
      \alphabd_3=\alphabd_1+\alphabd_2,\\
      \alphabd_5=\alphabd_1+2\alphabd_2+\alphabd_4,\\
    \end{array}
    $$
  \end{jie}
\end{frame}


\begin{frame}
  \begin{dingli}
    矩阵的行秩等于其列秩。
  \end{dingli}
  \begin{dingyi}[矩阵的秩]
    矩阵的行秩或列秩的数值,称为\red{矩阵的秩}。记作
    $$
    \rank(\A)  \quad \mbox{或}  \quad
    \mathrm{rank} (\A)
    $$
  \end{dingyi}

  \begin{dingyi}[满秩矩阵]
    对于$n$阶方阵,若
    $$
    \rank(\A) = n,
    $$
    则称$\A$为\red{满秩矩阵}。
  \end{dingyi}
\end{frame}

\begin{frame}
  \begin{dingli}
    对于$n$阶方阵,下列表述等价:
    \begin{itemize}
    \item[(1)] $\A$为满秩矩阵。
    \item[(2)] $\A$为可逆矩阵。
    \item[(3)] $\A$为非奇异矩阵。
    \item[(4)] $\det(\A)\ne 0$。
    \end{itemize}
  \end{dingli} 
\end{frame}

\begin{frame}
  \begin{dingyi}[子式与主子式]
    对矩阵$\A=(a_{ij})_{m\times n}$,任意挑选$k$行($i_1,i_2,\cd,i_k$行)与$k$列($j_1,j_2,\cd,j_k$列),
    其交点上的$k^2$个元素按原顺序排成的$k$阶行列式
    \begin{equation}\label{subdet}
      \left|
        \begin{array}{cccc}
          a_{i_1j_1} & a_{i_1j_2} & \cd & a_{i_1j_k}\\
          a_{i_2j_1} & a_{i_2j_2} & \cd & a_{i_2j_k}\\
          \vd & \vd && \vd\\
          a_{i_kj_1} & a_{i_kj_2} & \cd & a_{i_kj_k}\\
        \end{array}
      \right|
    \end{equation}
    称为$\A$的\red{$k$阶子行列式},简称$\A$的\red{$k$阶子式}。 
    \begin{itemize}
    \item 当(\ref{subdet})等于零时,称为\red{$k$阶零子式};
    \item 当(\ref{subdet})不等于零时,称为\red{$k$阶非零子式};
    \item 当(\ref{subdet})的$j_1=i_1,~j_2=i_2,~\cd,~j_k=i_k$,称为$\A$的\red{$k$阶主子式}。
    \end{itemize}
  \end{dingyi}
\end{frame}

\begin{frame}
  \begin{zhu*}
    \blue{若$\A$存在$r$阶非零子式,而所有$r+1$阶子式(如果有)都等于零,则矩阵$\A$的非零子式的最高阶数为$r$。}
  \end{zhu*}
  事实上,由行列式的按行展开可知,若所有$r+1$阶子式都等于零,可得到所有更高阶的子式都等于零。

\end{frame}


\begin{frame}
  \begin{dingli}
    $\rank(\A)=r$的充分必要条件是$\A$的非零子式的最高阶数为$r$。
  \end{dingli} 
\end{frame}

\begin{frame}
  关于矩阵的秩的基本结论
  \begin{itemize}
  \item[(1)]  $\red{\rank(\A)=\A\mbox{的行秩}=\A\mbox{的列秩}=\A\mbox{的非零子式的最高阶数}}$
  \item[(2)]  \red{初等变换不改变矩阵的秩}
  \end{itemize}
\end{frame}

\begin{frame}
  \begin{xingzhi}
    $$
    \red{\max\{\rank(\A),~\rank(\B)\}~~\le~~ \rank(\A,~\B) ~~\le~~ \rank(\A) + \rank(\B).}
    $$
    特别地,当$\B=\bb$为非零向量时,有
    $$
    \red{\rank(\A)~~\le~~\rank(\A,~\bb)~~\le~~\rank(\A)+1.}
    $$
  \end{xingzhi}
\end{frame}

\begin{frame}
  $$
  \rank(\A,\bb) = \left\{
    \begin{array}{ll}
      \rank(\A) & \Longleftrightarrow~~ \bb\mbox{可以被}\A\mbox{的列向量线性表示}\\[0.1in]
      \rank(\A)+1 & \Longleftrightarrow~~ \bb\mbox{不能被}\A\mbox{的列向量线性表示}
    \end{array}
  \right.
  $$
\end{frame}



\begin{frame}
  设$$\A=\left(
    \begin{array}{cc}
      1&0\\
      0&1\\
      0&0
    \end{array}
  \right),~~
  \bb_1=\left(
    \begin{array}{cc}
      1\\
      2\\
      0
    \end{array}
  \right), ~~
  \bb_2=\left(
    \begin{array}{cc}
      0\\
      0\\
      1
    \end{array}
  \right)  
  $$
  \begin{itemize}
  \item[(1)] 因
    $$
    (\A,~\bb_1) = \left(
      \begin{array}{ccc}
        1&0&\red{1}\\
        0&1&\red{2}\\
        0&0&\red{0}
      \end{array}
    \right) \xlongrightarrow[]{c_3-(c_1+2c_2)}
    \left(
      \begin{array}{ccc}
        1&0&\red{0}\\
        0&1&\red{0}\\
        0&0&\red{0}
      \end{array}
    \right) = (\A, \zero),
    $$
    故$\rank(\A,\bb_1)=\rank(\A,\zero)=\rank(\A)$,从而$\bb_1$可由$\A$的列向量线性表示。\\[0.1in]  
  \item[(2)] 因
    $$
    (\A,~\bb_2) = \left(
      \begin{array}{ccc}
        1&0&\red{0}\\
        0&1&\red{0}\\
        0&0&\red{1}
      \end{array}
    \right),
    $$
    故$\rank(\A,\bb)=\rank(\A)+1$,
    从而$\bb$不能由$\A$的列向量线性表示。  
  \end{itemize}
\end{frame}


\begin{frame}
  \begin{zhu*}
    \begin{itemize}
    \item 不等式
      $$
      \min\{\rank(\A),~\rank(\B)\} ~~\le~~ \rank(\A,~\B)
      $$
      意味着:在$\A$的右侧添加新的列,只有可能使得秩在原来的基础上得到增加;当$\B$的列向量能被$\A$的列向量线性表示时,等号成立。\\[0.1in]  \pause 
    \item 不等式
      $$
      \rank(\A,~\B) ~~\le~~ \rank(\A)+\rank(\B)
      $$
      意味着:对$(\A,~\B)$,有可能$\A$的列向量与$\B$的列向量出现线性相关,合并为$(\A,~\B)$的秩一般会比$\rank(\A)+\rank(\B)$要小。
    \end{itemize}
  \end{zhu*}
\end{frame}


\begin{frame}
  \begin{xingzhi}
    $$
    \red{\rank(\A+\B) \le \rank(\A)+\rank(\B).}
    $$
  \end{xingzhi}

  \begin{zhu*}
    将矩阵$\A$和$\B$合并、相加,只可能使得秩减小。
  \end{zhu*}

\end{frame}

\begin{frame}
  \begin{xingzhi}
    $$
    \red{\rank(\A\B) \le \min(\rank(\A),~\rank(\B)).}
    $$
  \end{xingzhi}


  \red{该性质告诉我们,对一个向量组进行线性组合,可能会使向量组的秩减小。}
\end{frame}

\begin{frame}
  \begin{xingzhi}
    设$\A$为$m\times n$矩阵,$\PP,\QQ$分别为$m$阶、$n$阶可逆矩阵,则
    $$
    \rank(\A) = \rank(\PP\A) = \rank(\A\QQ)  = \rank(\PP\A\QQ).
    $$
  \end{xingzhi}
\end{frame}

\begin{frame}
  \begin{li}
    设$\A$是$m\times n$矩阵,且$m<n$,证明:$|\A^T\A|=0$.
  \end{li}
  \vspace{.1in}\pause 

  \begin{jie}
    由于$\rank(\A)=\rank(\A^T)\le \min\{m,n\}<n$,根据性质2,有
    $$
    \rank(\A^T\A) \le \min\{\rank(\A^T),~\rank(\A)\} < n,
    $$
    而$\A^T\A$为$n$阶矩阵,故$|\A^T\A|=0$。
  \end{jie}
\end{frame}




\begin{frame}
  \begin{dingli}
    若$\A$为$m\times n$矩阵,且$\rank(\A)=r$,则一定存在可逆的$m$阶矩阵$\PP$和$n$阶矩阵$\QQ$使得
    $$
    \PP\A\QQ=\left(
      \begin{array}{cc}
        \II_r&\zero\\
        \zero&\zero
      \end{array}
    \right)_{m\times n} = \U.
    $$
  \end{dingli}
\end{frame}


\begin{frame}
  \begin{li}[$\bigstar$]
    设
    $$
    \begin{aligned}
      \alphabd_1=(1,3,1,2), &~\alphabd_2=(2,5,3,3), \\
      \alphabd_3=(0,1,-1,a),& ~\alphabd_4=(3,10,k,4),
    \end{aligned}
    $$
    试求向量组$\alphabd_1,~\alphabd_2,~\alphabd_3,~\alphabd_4$的秩,并将$\alphabd_4$用$\alphabd_1,~\alphabd_2,~\alphabd_3$线性表示。
  \end{li}
\end{frame}

\begin{frame}[allowframebreaks]
  \begin{jie}
    将4个向量按列排成一个矩阵$\A$,做初等变换将其化为阶梯形矩阵$\U$,即
    $$
    \A=\left(
      \begin{array}{rrrr}
        1&2&0&3\\
        3&5&1&10\\
        1&3&-1&k\\
        2&3&a&4
      \end{array}
    \right) \xlongrightarrow[]{\mbox{初等行变换}}
    \left(
      \begin{array}{rrcc}
        1&2&0&3\\
        0&-1&1&1\\
        0&0&a-1&-3\\
        0&0&0&k-2
      \end{array}
    \right)
    $$
    \begin{itemize}
    \item[(1)] 当$a=1$或$k=2$时,$\U$只有3个非零行,故
      $$
      \red{\rank(\alphabd_1,\alphabd_2,\alphabd_3,\alphabd_4)=\rank(\A)=3.}
      $$ 
    \item[(2)]  当$a\ne1$且$k\ne2$时,
      $\red{\rank(\alphabd_1,\alphabd_2,\alphabd_3,\alphabd_4)=\rank(\A)=4.}$
    \item[(3)] 当$k=2$且$a\ne1$时,$\alphabd_4$可由$\alphabd_1,~\alphabd_2,~\alphabd_3$线性表示,
      且
      $$
      \alphabd_4=-\frac{1+5a}{1-a}\alphabd_1+\frac{2+a}{1-a}\alphabd_2+\frac{3}{1-a}\alphabd_3.
      $$
    \item[(4)]  当$k\ne2$或$a=1$时,$\alphabd_4$不能由$\alphabd_1,~\alphabd_2,~\alphabd_3$线性表示。
    \end{itemize}
  \end{jie}
\end{frame}

\begin{frame}
  \begin{li}
    设
    $$
    \A=\left(
      \begin{array}{rrr}
        1&2&1\\
        2&2&-2\\
        -1&t&5\\
        1&0&-3
      \end{array}
    \right)
    $$
    已知$\rank(\A)=2$,求$t$。
  \end{li}\pause 
  \begin{jie}
    $$
    \A \xlongrightarrow[]{\mbox{初等行变换}} \left(
      \begin{array}{ccr}
        1&2&1\\
        0&-2&-4\\
        0&2+t&6\\
        0&0&0
      \end{array}
    \right)=\B
    $$ 
    由于$\rank(\B)=\rank(\A)$,故$\B$中第2、3行必须成比例,即
    $$
    \frac{-2}{2+t}=\frac{-4}6,
    $$
    即得$t=1$。
  \end{jie}
\end{frame}


\subsection{齐次线性方程组有非零解的条件及解的结构}
\begin{frame}
  设$\A$为$m\times n$矩阵,考察以$\A$为系数矩阵的齐次线性方程组
  \begin{equation}\label{ax0}
    \A\xx=\zero.
  \end{equation}    
  若将$\A$按列分块为
  $$
  \A = (\alphabd_1,~\alphabd_2,~\cd,~\alphabd_n),
  $$
  齐次方程组(\ref{ax0})可表示为
  $$
  x_1\alphabd_1+x_2\alphabd_2+\cd+x_n\alphabd_n=\zero.
  $$
  而齐次方程组(\ref{ax0})有\blue{非零解}的充分必要条件是$\alphabd_1,~\alphabd_2,~\cd,~\alphabd_n$\blue{线性相关},即
  $$
  \rank(\A) = \rank(\alphabd_1,~\alphabd_2,~\cd,~\alphabd_n) < n.
  $$
\end{frame}


\begin{frame}
  \begin{dingli}
    设$\A$为$m\times n$矩阵,则
    $$
    \blue{\underline{\A\xx=\zero\mbox{有非零解}}} ~~\Longleftrightarrow~~
    \blue{\underline{\rank(\A)<n}}.$$
  \end{dingli}


  \begin{dingli}[定理1的等价命题]
    设$\A$为$m\times n$矩阵,则
    $$
    \blue{\underline{\A\xx=\zero\mbox{只有零解}}} ~~\Longleftrightarrow~~
    \blue{\underline{\rank(\A)=n=\A\mbox{的列数}}}.
    $$
  \end{dingli}
\end{frame}



\begin{frame}
  \begin{dingyi}[基础解系]
    设$\xx_1,~\xx_2,~\cd,~\xx_p$为$\A\xx=\zero$的解向量,若
    \begin{itemize}
    \item[(1)] $\xx_1,~\xx_2,~\cd,~\xx_p$线性无关;
    \item[(2)] $\A\xx=\zero$的任一解向量可由$\xx_1,~\xx_2,~\cd,~\xx_p$线性表示。
    \end{itemize}
    则称$\xx_1,~\xx_2,~\cd,~\xx_p$为$\A\xx=\zero$的一个\blue{\underline{基础解系}}。
  \end{dingyi}
\end{frame}

\begin{frame}
  \begin{zhu*}
    关于基础解系,请注意以下几点:
    \begin{itemize}
    \item[(1)] 基础解系即全部解向量的\blue{极大无关组}。\\[0.1in]  
    \item[(2)] 找到了基础解系,就找到了齐次线性方程组的全部解:
      $$
      k_1\xx_1+k_2\xx_2+\cd+k_p\xx_p \quad(k_1,k_2,\cd,k_p\mbox{为任意常数}).
      $$ 
    \item[(3)] 基础解系\blue{不唯一}。
    \end{itemize}
  \end{zhu*}
\end{frame}




\begin{frame}
  \begin{dingli}
    设$\A$为$m\times n$矩阵,若$\rank(\A)=r<n$,则齐次线性方程组$\A\xx=\zero$存在基础解系,
    且基础解系含$n-r$个解向量。
  \end{dingli}

  \begin{zhu*}
    注意以下两点:
    \begin{itemize}
    \item $r$为$\A$的秩,也是$\A$的行阶梯形矩阵的非零行行数,是非自由未知量的个数。 
    \item $n$为未知量的个数,故$n-r$为自由未知量的个数。 有多少自由未知量,基础解系里就对应有多少个向量。
    \end{itemize}
  \end{zhu*}
\end{frame}

\begin{frame}
  \begin{li}[$\bigstar$]
    求齐次线性方程组$\A\xx=\zero$的基础解系,其中
    $$
    \A = \left(
      \begin{array}{rrrr}
        1&-8&10&2\\
        2&4&5&-1\\
        3&8&6&-2
      \end{array}
    \right).
    $$
  \end{li} 
\end{frame}

\begin{frame}[allowframebreaks]
  \begin{jie}
    $$
    \begin{array}{l}
      \left(
      \begin{array}{rrrr}
        1&-8&10&2\\
        2&4&5&-1\\
        3&8&6&-2
      \end{array}
               \right) \xlongrightarrow[r_3-3r_1]{r_2-2r_1}
               \left(
               \begin{array}{rrrr}
                 1&-8&10&2\\
                 0&20&-15&-5\\
                 0&32&24&-8
               \end{array}
                          \right)\\[0.3in]
      \xlongrightarrow[r_2\div4]{r_3\div8}
      \left(
      \begin{array}{rrrr}
        1&-8&10&2\\
        0&4&-3&-1\\
        0&4&-3&-1
      \end{array}
                \right) \xlongrightarrow[r_1+2r_2]{r_3-r_2}
                \left(
                \begin{array}{rrrr}
                  1&0&4&0\\
                  0&4&-3&-1\\
                  0&0&0&0
                \end{array}
                         \right) \\[0.3in]
      \xlongrightarrow[]{r_2\div4}
      \left(
      \begin{array}{rrrr}
        1&0&4&0\\
        0&1&-3/4&-1/4\\
        0&0&0&0
      \end{array}
               \right)
    \end{array}
    $$
    原方程等价于
    $$\left\{
      \begin{array}{rcrcrc}
        x_1&=&-4&x_3&&\\[0.1in]
        x_2&=&\frac34&x_3&+\frac14&x_4
      \end{array}
    \right.  \Leftrightarrow
    \left\{
      \begin{array}{rcrcrc}
        x_1&=&-4&x_3&&\\[0.1in]
        x_2&=&\frac34&x_3&+\frac14&x_4\\[0.1in]
        x_3&=&&x_3&&\\[0.1in]
        x_4&=&&&&x_4      
      \end{array}
    \right.
    $$
    基础解系为
    $$
    \xibd_1 = \left(
      \begin{array}{r}
        -4\\[0.1in]
        \frac34\\[0.1in]
        1\\[0.1in]
        0
      \end{array}
    \right), \quad \xibd_2 = \left(
      \begin{array}{r}
        0\\[0.1in]
        \frac14\\[0.1in]
        0\\[0.1in]
        1
      \end{array}
    \right)
    $$
  \end{jie}
\end{frame}

\begin{frame}
  \begin{li}[$\bigstar$]
    求齐次线性方程组
    $$
    nx_1+(n-1)x_2+\cd+2x_{n-1}+x_n=0
    $$
    的基础解系。      
  \end{li}
  \pause 
  \begin{jie}
    原方程等价于$x_n=-nx_1-(n-1)x_2-\cd-2x_{n-1}$, 即
    $$
    \left\{
      \begin{array}{rcrrrr}
        x_1&=&x_1&&&\\
        x_2&=&&x_2&&\\
           &\vd&&&&\\
        x_{n-1}&=&&&&x_{n-1}\\      
        x_n&=&-nx_1&-(n-1)x_2&\cd&-2x_{n-1}
      \end{array}    
    \right.
    $$
    基础解系为
    $$
    (\xibd_1,\xibd_2,\cd,\xibd_{n-1})=\left(
      \begin{array}{rrrr}
        1&0&\cd&0\\
        0&1&\cd&0\\
        \vd&\vd&&\vd\\
        0&0&\cd&1\\
        -n&-n+1&\cd&-2
      \end{array}
    \right)
    $$
  \end{jie}
\end{frame}

\begin{frame}
  \begin{li}
    设$\A$与$\B$分别为$m\times n$和$n\times s$矩阵,且$\A\B=\zero$。证明:
    $$
    \rank(\A)+\rank(\B)\le n.
    $$
  \end{li}
  \pause 
  \begin{proof}
    由$\A\B=\zero$知,$\B$的列向量是$\A\xx=\zero$的解。
    故$\B$的列向量组的秩,不超过$\A\xx=\zero$的基础解系的秩,即
    $$
    \rank(\B) \le n-\rank(\A),
    $$
    即
    $$
    \rank(\A)+\rank(\B)\le n.
    $$
  \end{proof}
\end{frame}

\begin{frame}

  \begin{li}
    设$n$元齐次线性方程组$\A\xx=\zero$与$\B\xx=\zero$同解,证明
    $$
    \rank(\A)=\rank(\B).
    $$
  \end{li}
  \pause 
  \begin{jie}
    $\A\xx=\zero$与$\B\xx=\zero$同解,故它们有相同的基础解系,而基础解系包含的向量个数相等,即
    $$
    n-\rank(\A)=n-\rank(\B),
    $$
    从而
    $$
    \rank(\A)=\rank(\B).
    $$
  \end{jie}
\end{frame}


\begin{frame}
  \begin{li}
    设$\A$为$m\times n$实矩阵,证明$\rank(\A^T\A)=\rank(\A)$。    
  \end{li}
  \pause 
  \begin{proof}
    只需证明$\A\xx=\zero$与$(\A^T\A)\xx=\zero$同解。
    \begin{itemize}
    \item[(1)] 若$\xx$满足$\A\xx=\zero$,则有$(\A^T\A)\xx=\A^T(\A\xx)=\zero$。 
    \item[(2)] 若$\xx$满足$\A^T\A\xx=\zero$,则
      $$
      \xx^T\A^T\A\xx=\zero,
      $$
      即
      $$
      (\A\xx)^T\A\xx=\zero,
      $$
      故$\A\xx=\zero$。
    \end{itemize}
  \end{proof}
\end{frame}


\subsection{非齐次线性方程组有解的条件及解的结构}
\begin{frame}
  \begin{dingli}
    对于非齐次线性方程组$\A\xx=\bb$,以下命题等价:
    \begin{itemize}
    \item[(i)] $\A\xx=\bb$有解;
    \item[(ii)] $\bb$可由$\A$的列向量组线性表示;
    \item[(iii)] $\rank(\A,\bb)=\rank(\A)$。
    \end{itemize}
  \end{dingli}
\end{frame}


\begin{frame}
  \begin{zhu*}
    \blue{$\rank(\A,\bb)=\rank(\A)+1$会导致矛盾方程的出现。}
  \end{zhu*}
  记$\rank(\A)=r$,若$\rank(\A,\bb)=\rank(\A)+1$,则增广矩阵$(\A,\bb)$经过初等行变换所得的行最简阶梯形矩阵形如
  \begin{center}
    \begin{tikzpicture}
      \matrix(MM) [matrix of math nodes,nodes in empty cells,ampersand replacement=\&,left delimiter=(,right delimiter=)] {
        1\&\cd\&0\&c_{1,r+1}\&\cd\&c_{1n}\&\&d_1\\        
        \vd\&\&\vd\&\vd\&\&\vd\&\&\vd\\
        0\&\cd\&1\&c_{r,r+1}\&\cd\&c_{rn}\&\&d_r\\
        0\&\cd\&0\&0\&\cd\&0\&\&\red{d_{r+1}}\\
        0\&\cd\&0\&0\&\cd\&0\&\&0\\        
        \vd\&\&\vd\&\vd\&\&\vd\&\&\vd\\
        0\&\cd\&0\&0\&\cd\&0\&\&0\\
      };  
      \draw[thick,dashed] (MM-1-7.north)--(MM-7-7.south);
    \end{tikzpicture}
  \end{center}
  其中$d_{r+1}\ne 0$(否则$\rank(\A,\bb)=r$)。这意味着出现了矛盾方程
  $$
  0 = \red{d_{r+1}}.
  $$    
\end{frame}

\begin{frame}
  \begin{tuilun}
    $$
    \A\xx=\bb\mbox{有唯一解} ~~\Longleftrightarrow~~
    \rank(\A,\bb)=\rank(\A)=\A\mbox{的列数}.
    $$
  \end{tuilun}
  \begin{center}
    \begin{tikzpicture}
      \matrix(MM) [matrix of math nodes,nodes in empty cells,ampersand replacement=\&,left delimiter=(,right delimiter=)] {
        1\&\cd\&0\&\&d_1\\        
        \vd\&\&\vd\&\&\vd\\
        0\&\cd\&1\&\&d_r\\
        0\&\cd\&0\&\&0\\
        0\&\cd\&0\&\&0\\
        \vd\&\&\vd\&\&\vd\\
        0\&\cd\&0\&\&0\\
      };  
      \draw[thick,dashed] (MM-1-4.north)--(MM-7-4.south);
    \end{tikzpicture}
  \end{center}
\end{frame}


\begin{frame}
  \begin{dingli}
    若$\A\xx=\bb$有解,则其通解为
    $$
    \xx=\xx_0+\widetilde {\xx}
    $$
    其中$\xx_0$是$\A\xx=\bb$的一个特解,而
    $$
    \widetilde{\xx}=k_1\xx_1+k_2\xx_2+\cd+k_p\xx_p
    $$
    为$\A\xx=\zero$的通解。
  \end{dingli}

\end{frame}


\begin{frame}
  \begin{zhu*}
    “$\A\xx=\bb$的通解” =  “$\A\xx=\zero$的通解” + “$\A\xx=\bb$的特解”
  \end{zhu*}
\end{frame}

\begin{frame}
  \begin{li}[$\bigstar$]
    求非齐次线性方程组$\A\xx=\bb$的一般解,其中增广矩阵为
    $$
    (\A,\bb) = \left(
      \begin{array}{rrrrr}
        1&-1&-1& 1&\red{0}\\
        1&-1& 1&-3&\red{1}\\
        1&-1&-2& 3&\red{-\frac12}
      \end{array}
    \right)
    $$
  \end{li}
\end{frame}

\begin{frame}[allowframebreaks]
  \begin{jie}
    $$
    \begin{array}{rl}
      \left(
      \begin{array}{rrrrr}
        1&-1&-1& 1&\red{0}\\
        1&-1& 1&-3&\red{1}\\
        1&-1&-2& 3&\red{-\frac12}
      \end{array}
                    \right)
                    \xlongrightarrow[r_3-r_1]{r_2-r_1} &
                                                         \left(
                                                         \begin{array}{rrrrr}
                                                           1&-1&-1& 1&\red{0}\\
                                                           0& 0& 2&-4&\red{1}\\
                                                           0& 0&-1& 2&\red{-\frac12}
                                                         \end{array}
                                                                       \right) \\[0.4in]
      \xlongrightarrow[r_2\div2]{r_1-r_3,r_3+\frac12r_2} &
                                                           \left(
                                                           \begin{array}{rrrrr}
                                                             1&-1&-1& 1&\red{0}\\
                                                             0& 0& 1&-2&\red{\frac12}\\
                                                             0& 0& 0& 0&\red{0}
                                                           \end{array}
                                                                         \right)\\[0.4in]
      \xlongrightarrow[r_2\div2]{r_1+r_2} &
                                            \left(
                                            \begin{array}{rrrrr}
                                              1&-1& 0&-1&\red{\frac12}\\
                                              0& 0& 1&-2&\red{\frac12}\\
                                              0& 0& 0& 0&\red{0}
                                            \end{array}
                                                          \right)
    \end{array}
    $$
    同解方程为
    $$
    \left\{
      \begin{array}{rcrcrcr}
        x_1&=&x_2&+&x_4&+&\frac12\\[0.1in]
        x_3&=&&&2x_4&+&\frac12
      \end{array}
    \right.
    $$
    亦即
    $$
    \left\{
      \begin{array}{rcrcrcr}
        x_1&=&x_2&+&x_4&+&\frac12\\[0.1in]
        x_2&=&x_2&&&&\\[0.1in]
        x_3&=&&&2x_4&+&\frac12\\[0.1in]
        x_4&=&&&x_4&&
      \end{array}
    \right.
    $$
    故通解为
    $$
    \left(
      \begin{array}{c}
        x_1\\x_2\\x_3\\x_4
      \end{array}
    \right) = c_1    \left(
      \begin{array}{c}
        1\\1\\0\\0
      \end{array}
    \right)+c_2    \left(
      \begin{array}{c}
        1\\0\\2\\1
      \end{array}
    \right)+    \left(
      \begin{array}{c}
        1/2\\0\\1/2\\0
      \end{array}
    \right) \quad c_1,c_2\in\mathbb R
    $$
  \end{jie}
\end{frame}

\begin{frame}
  \begin{li}[$\bigstar$]
    设有线性方程组
    $$
    \left\{
      \begin{array}{rrrcr}
        (1+\lambda)x_1&+x_2&+x_3&=&0\\[0.05in]
        x_1&+(1+\lambda)x_2&+x_3&=&3\\[0.05in]
        x_1&+x_2&+(1+\lambda)x_3&=&\lambda
      \end{array}
    \right.
    $$
    问$\lambda$取何值时,此方程组
    \begin{itemize}
    \item[(1)]有唯一解?
    \item[(2)]无解? 
    \item[(3)]有无穷多解? 并在有无穷多解时求其通解。
    \end{itemize}
  \end{li}
\end{frame}

\begin{frame}[allowframebreaks]
  \begin{jie}
    $$
    |\A|=\left|
      \begin{array}{ccc}
        1+\lambda&1&1\\
        1&1+\lambda&1\\
        1&1&1+\lambda
      \end{array}
    \right| = (3+\lambda)\lambda^2.
    $$
    故当$\lambda\ne0$且$\lambda\ne-3$时,有唯一解。
    当$\lambda=0$时,原方程组为
    $$
    \left\{
      \begin{array}{l}
        x_1+x_2+x_3=0,\\
        x_1+x_2+x_3=3,\\
        x_1+x_2+x_3=0      
      \end{array}
    \right.
    $$
    它为矛盾方程组,故无解。 \vspace{0.1in}

    当$\lambda=-3$时,增广矩阵为
    $$
    \left(
      \begin{array}{rrrr}
        -2&1&1&\red{0}\\
        1&-2&1&\red{3}\\
        1&1&-2&\red{-3}
      \end{array}
    \right) \xlongrightarrow[]{\mbox{初等行变换}}
    \left(
      \begin{array}{rrrr}
        1&0&-1&\red{-1}\\
        0&1&-1&\red{-2}\\
        0&0&0&\red{0}
      \end{array}
    \right)
    $$
    得同解方程组为
    $$
    \left\{
      \begin{array}{l}
        x_1=x_3-1\\[0.05in]
        x_2=x_3-2\\[0.05in]
        x_3=x_3
      \end{array}
    \right.
    $$
    通解为
    $$
    \left(
      \begin{array}{c}
        x_1\\x_2\\x_3
      \end{array}
    \right) = c\left(
      \begin{array}{c}
        1\\1\\1
      \end{array}
    \right)+\left(
      \begin{array}{r}
        -1\\-2\\0
      \end{array}
    \right) \quad c\in\mathbb R
    $$
  \end{jie}
\end{frame}




\begin{frame}
  \begin{li}
    设四元齐次线性方程组
    $$
    (I):\left\{
      \begin{array}{l}
        x_1+x_2=0,\\
        x_2-x_4=0;
      \end{array}
    \right. \quad
    (II):\left\{
      \begin{array}{l}
        x_1-x_2+x_3=0,\\
        x_2-x_3+x_4=0.
      \end{array}
    \right.
    $$
    求
    \begin{itemize}
    \item[(1)] 方程组$(I)$与$(II)$的基础解系
    \item[(2)] 方程组$(I)$与$(II)$的公共解        
    \end{itemize}
  \end{li}
\end{frame}

