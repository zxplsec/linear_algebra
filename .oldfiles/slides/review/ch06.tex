\section{二次型}

\subsection{二次型的定义与矩阵表示}

\begin{frame}
  
  \begin{dingyi}[二次型]
    $n$元变量$x_1,x_2,\cd,x_n$的二次齐次多项式
    \begin{equation}\label{qf}
    \begin{array}{rcccccccc}
      f(x_1,x_2,\cd,x_n) &=& \\[0.1in]
      a_{11}x_1^2&+&2a_{12}x_1x_2&+&2a_{13}x_1x_3&+&\cd&+&2a_{1n}x_1x_n\\[0.1in]
                         &+&a_{22}x_2^2&+&2a_{23}x_2x_3&+&\cd&+&2a_{2n}x_2x_n\\[0.1in]
                         &&&&\cd&&\cd\\[0.1in]
                         &&&&&&&+&a_{nn}x_n^2
    \end{array}
    \end{equation}
    当系数属于数域$F$时,称为数域$F$上的一个\blue{\underline{$n$元二次型}}。
  \end{dingyi}
  
\end{frame}

\begin{frame}
  
  \begin{itemize}
  \item
    对于任意一个二次型,总可以写成对称形式
    $$
    f(x_1,x_2,\cd,x_n)=\xx^T\A\xx
    $$
    其中$\A$为对称矩阵。\\[0.1in]\pause
  \item
    若$\A,\B$为对称矩阵,且
    $$
    f(x_1,x_2,\cd,x_n)=\xx^T\A\xx=\xx^T\B\xx
    $$
    则必有$\A=\B$。\\[0.1in]\pause
  \item 二次型和它的矩阵是相互唯一确定的,因此研究二次型的性质就转化为研究对称矩阵$\A$所具有的性质。\\[0.1in]\pause
  \item \blue{对于二次型$f=\xx^T\A\xx$,对称阵$\A$称为二次型$f$的矩阵,$f$称为对称阵$\A$的二次型,而$\A$的秩称为二次型$f$的秩。}
  \end{itemize}

  
\end{frame}

\begin{frame}
  
  \begin{li}
    设$f(x_1,x_2,x_3,x_4)=2x_1^2+x_1x_2+2x_1x_3+4x_2x_4+x_3^2+5x_4^2$,则它的矩阵为
    $$
    \A=\left(
      \begin{array}{cccc}
        2&\frac12&1&0\\[0.2cm]
        \frac12&0&0&2\\[0.2cm]
        1&0&1&0\\[0.2cm]
        0&2&0&5
      \end{array}
    \right)
    $$
  \end{li}
  
\end{frame}

\begin{frame}
  
  \begin{itemize}
  \item 对于二次型,我们讨论的主要问题是:寻求可逆的线性变换$\xx=\C\yy$
    使得
    $$
    f=\xx^T\A\xx=\yy^T(\C^T\A\C)\yy=\yy^T\Lambdabd\yy=\blue{\lambda_1y_1^2+\lambda_2y_2^2+\cd+\lambda_ny_n^2.}
    $$
    这种只含平方项的二次型,称为\red{二次型的标准形}。\\[0.2in]

    \item 若标准形的系数$\lambda_1,\lambda_2,\cd,\lambda_n$只在$1,-1,0$中取值,也就是用\eqref{lt}代入\eqref{qf},能使
    $$
    f=y_1^2+\cd+y_p^2-y_{p+1}^2-\cd-y_r^2.
    $$
    则称上式为\red{二次型的规范形}。
  \end{itemize}
\end{frame}



\begin{frame}
  
  \begin{dingyi}[矩阵的合同]
    对于两个矩阵$\A$和$\B$,若存在可逆矩阵$\C$,使得
    $$
    \C^T\A\C=\B,
    $$
    就称\red{$\A$合同于$\B$}.
  \end{dingyi}
  
\end{frame}

\begin{frame}
  \begin{jielun}

    经过可逆变换$\xx=\C\yy$后,二次型$f$的矩阵由$\A$变为与$\A$合同的矩阵$\C^T\A\C$,且二次型的秩不变。
  \end{jielun}
\end{frame}


\subsection{化二次型为标准型}

\begin{frame}
  
  \red{化二次型为标准型,就是对实对称矩阵$\A$,寻找可逆阵$\C$,使$\C^T\A\C$成为对角矩阵。}
    
  
\end{frame}


\subsubsection{正交变换法}
\begin{frame}
  \begin{center}
    \red{\Huge 正交变换法}
  \end{center}
\end{frame}

\begin{frame}
  
    \begin{jielun}
      对于实对称矩阵$\A$,存在正交阵$\QQ$,使得
      $$
      \QQ^{-1}\A\QQ=\Lambdabd,
      $$
      由于$\QQ^{-1}=\QQ^T$,故
      $$
      \QQ^{T}\A\QQ=\Lambdabd.
      $$
    \end{jielun}
  
\end{frame}

\begin{frame}
  
    \begin{dingli}[主轴定理]
      对于任一个$n$元二次型
      $$
      f(x_1,x_2,\cd,x_n)=\xx^T\A\xx,
      $$
      存在正交变换$\xx=\QQ\yy$($\QQ$为正交阵),使得
      $$
      \xx^T\A\xx=\yy^T(\QQ^T\A\QQ)\yy=\lambda_1y_1^2+\lambda_2y_2^2+\cd+\lambda_ny_n^2,
      $$
      其中$\lambda_1,\lambda_2,\cd,\lambda_n$为$\A$的$n$个特征值,
      $\QQ$的$n$个列向量$\alphabd_1,\alphabd_2,\cd,\alphabd_n$是$\A$对应于$\lambda_1,\lambda_2,\cd,\lambda_n$的标准正交特征向量。
    \end{dingli} \vspace{.2in}

    \begin{tuilun}
      任给二次型$f(\xx)=\xx^T\A\xx$,总有可逆变换$\xx=\C\zz$,使得$f(\C\zz)$为规范形。
    \end{tuilun}
\end{frame}



\begin{frame}
  
    \begin{li}[\red{$\bigstar$}]
      用正交变换法,将二次型
      $$
      f(x_1,x_2,x_3)=2x_1^2+5x_2^2+5x_3^2+4x_1x_2-4x_1x_3-8x_2x_3
      $$
      化为标准型。
    \end{li} \pause
    
    \begin{jie}
    对应矩阵为
    $$
    \A=\left(
    \begin{array}{rrr}
      2&2&-2\\
      2&5&-4\\
      -2&-4&5
    \end{array}
    \right)
    $$
    \pause
    其特征多项式为
    $$
    |\A-\lambda\E|=-(\lambda-1)^2(\lambda-10)
    $$
    得特征值$\lambda_{1,2}=1$和$\lambda_3=10$.
    \end{jie}
  
\end{frame}


\begin{frame}
  
    $$
    \begin{array}{rl}
      (\A-\lambda_1\E)\xx=\zero & \Rightarrow~~
      \left(
      \begin{array}{rrr}
        1&2&-2\\
        2&4&-4\\
        -2&-4&4
      \end{array}
      \right)\left(
      \begin{array}{c}
        x_1\\
        x_2\\
        x_3
      \end{array}
      \right)=\zero\\[0.3in] \pause
      & \Rightarrow~~
      \xx_1=(-2,1,0)^T, \quad
      \xx_2=(2,0,1)^T. \\[0.2in]\pause
      (\A-\lambda\E)\xx=\zero & \Rightarrow~~
      \left(
      \begin{array}{rrr}
        -8&2&-2\\
        2&-5&-4\\
        -2&-4&-5
      \end{array}
      \right)\left(
      \begin{array}{c}
        x_1\\
        x_2\\
        x_3
      \end{array}
      \right)=\zero\\[0.3in] \pause
      & \Rightarrow~~
      \xx_3=(1,2,-2)^T.
    \end{array}
    $$ \pause 

    对$\xx_1,\xx_2$用施密特正交化方法得
    $$
    \xibd_1=\left(-\frac{2\sqrt{5}}5,\frac{2\sqrt{5}}5,0\right)^T,~~~~
    \xibd_2=\left(\frac{2\sqrt{5}}{15},\frac{4\sqrt{5}}{15},\frac{\sqrt{5}}3\right)^T
    $$
    再将$\xx_3$单位化为
    $$
    \xibd_3=\left(\frac13,\frac23,-\frac23\right)^T
    $$
  
\end{frame}


\begin{frame}
  
    取正交矩阵
    $$
    \QQ=(\xibd_1,\xibd_2,\xibd_3)=\left(
    \begin{array}{rrr}
      \ds-\frac{2\sqrt{5}}5&\ds\frac{2\sqrt{5}}{15}&\ds\frac13\\[0.2cm]
      \ds \frac{2\sqrt{5}}5&\ds\frac{4\sqrt{5}}{15}&\ds\frac23\\[0.2cm]
      \ds 0&\ds\frac{\sqrt{5}}3&\ds-\frac23
    \end{array}
    \right)
    $$
    则
    $$
    \QQ^{-1}\A\QQ=\QQ^{T}\A\QQ=\mathrm{diag}(1,1,10).
    $$ \pause 
    \blue{在正交变换$\xx=\QQ\yy$的作用下,原二次型就化成标准型
    $$
    \xx^T\A\xx=\yy^T(\QQ^T\A\QQ)\yy=y_1^2+y_2^2+10y_3^2.
    $$}
  
\end{frame}


\subsubsection{配方法}

\begin{frame}
  \begin{center}
    \red{\Huge 配方法}
  \end{center}
\end{frame}

\begin{frame}
  
    \begin{li}
      用配方法把三元二次型
      $$
      f(x_1,x_2,x_3)=2x_1^2+3x_2^2+x_3^2+4x_1x_2-4x_1x_3-8x_2x_3
      $$
      化为标准型。
    \end{li}
    \begin{jie}
    先按$x_1^2$和含$x_1$的混合项配成完全平方,即
    $$
    \begin{array}{rl}
      f&=2[x_1^2+2x_1(x_2-x_3)+(x_2-x_3)^2]-2(x_2-x_3)^2+3x_2^2+x_3^2-8x_2x_3\\[0.1in]
      &=2(x_1+x_2-x_3)^2+x_2^2-x_3^2-4x_2x_3
    \end{array}
    $$\pause
    再按$x_2^2-4x_2x_3$配成完全平方,得
    $$
    f(x_1,x_2,x_3)=2(x_1+x_2-x_3)^2+(x_2-2x_3)^2-5x_3^2.
    $$
    \pause
    令
    $$
    \left\{
    \begin{array}{rcrcrcr}
      y_1&=&x_1&+&x_2&-&x_3\\
      y_2&=& &&x_2&-&2x_3\\
      y_3&=&&&&&x_3
    \end{array}
    \right. \pause ~~\Longrightarrow~~
    \left(
    \begin{array}{c}
      x_1\\
      x_2\\
      x_3
    \end{array}
    \right)=
    \left(
    \begin{array}{rrr}
      1&-1&-1\\
      0&1&2\\
      0&0&1
    \end{array}
    \right)
    \left(
    \begin{array}{c}
      y_1\\
      y_2\\
      y_3
    \end{array}
    \right)
    $$ \pause 
    则
    $$
    f(x_1,x_2,x_3)=2y_1^2+y_2^2-5y_3^2.
    $$

    \end{jie}
  
\end{frame}


\begin{frame}
  
    \begin{li}
      用配方法化二次型
      $$
      f(x_1,x_2,x_3)=2x_1x_2+4x_1x_3
      $$
      为标准型。
    \end{li}
    \pause
    \begin{jie}
    对$x_1x_2$利用平方差公式,令
    $$
    \left\{
    \begin{array}{l}
      x_1=y_1+y_2\\
      x_2=y_1-y_2\\
      x_3=y_3
    \end{array}
    \right.
    $$
    则
    $$
    f(x_1,x_2,x_3)=2(y_1+y_2)(y_1-y_2)+4(y_1+y_2)y_3=2y_1^2-2y_2^2+4y_1y_3+4y_2y_3
    $$
    \pause
    先对含$y_1$的项配完全平方,得
    $$
    f(x_1,x_2,x_3)=2(y_1^2+2y_1y_3+y_3^2)-2y_2^2-2y_3^2+4y_2y_3
    $$
    再对含$y_2$的项配完全平方,得
    $$
    f(x_1,x_2,x_3)=2(y_1+y_3)^2-2(y_2-y_3)^2
    $$
    \end{jie}
  
\end{frame}

\begin{frame}
  
    令
    $$
    \left\{
    \begin{array}{l}
      z_1=y_1+y_3\\
      z_2=y_2-y_3\\
      z_3=y_3
    \end{array}
    \right. ~~\Longleftrightarrow~~
    \left\{
    \begin{array}{l}
      y_1=z_1-z_3\\
      y_2=z_2+z_3\\
      y_3=z_3
    \end{array}
    \right.
    $$
    则
    $$
    f(x_1,x_2,x_3)=2z_1^2-2z_2^2.
    $$\pause
    坐标变换记为
    $$
    \xx=\C_1\yy, \quad  \yy=\C_2\zz, \quad \xx=\C_1\C_2\zz=\C\zz
    $$
    其中
    $$
    \begin{array}{c}
      \C_1=\left(
      \begin{array}{rrr}
        1&1&0\\
        1&-1&0\\
        0&0&1
      \end{array}
      \right),
      \quad\C_2=\left(
      \begin{array}{rrr}
        1&0&-1\\
        0&1&1\\
        0&0&1
      \end{array}
      \right)
      \\[0.4in]
      \C=\C_1\C_2=\left(
      \begin{array}{rrr}
        1&1&0\\
        1&-1&-2\\
        0&0&1
      \end{array}
      \right)      
    \end{array}
    $$
  
\end{frame}

\begin{frame}
  
    \begin{table}
      \caption{}
      \begin{tabular}{|c|c|}\hline
        二次型&对应矩阵\\\hline
        $2x_1x_2+4x_1x_3$ & $\A=\left(
        \begin{array}{ccc}
          0&1&2\\
          1&0&0\\
          2&0&0
        \end{array}
        \right)$\\\hline
        $2z_1^2-2z_2^2$ & $\Lambdabd=\left(
        \begin{array}{ccc}
          2&&\\
          &-2&\\
          &&0
        \end{array}
        \right)$ \\\hline
      \end{tabular}      
    \end{table}
    易验证
    $$
    \C^T\A\C=\diag(2,-2,0)    
    $$
  
\end{frame}

\begin{frame}
  
    任何$n$元二次型都可用配方法化为标准型,相应的变换矩阵为主对角元为1的上三角阵和对角块矩阵,或者是这两类矩阵的乘积。
  
\end{frame}


\subsection{正定二次型和正定矩阵}

\begin{frame}
  二次型的标准形是不唯一的,但标准形中所含项数(即二次型的秩)是确定的。不仅如此,在限定变换为实变换时,标准形中正系数的个数是不变的,从而负系数的个数也是不变的。
\end{frame}

\begin{frame}
  \begin{dingli}[惯性定理]
    设有二次型$f=\xx^T\A\xx$,它的秩为$r$,有两个可逆变换
    $$
    \xx = \C\yy, \quad \xx = \QQ \zz,
    $$
    使得
    $$
    \begin{aligned}
      f & = k_1y_1^2+k_2y_2^2+\cd+k_ry_r^2, ~~ k_i\ne 0, \\[0.1in]
      f & = \lambda_1y_1^2+\lambda_2y_2^2+\cd+\lambda_ry_r^2, ~~ \lambda_i\ne 0,
    \end{aligned}
    $$
    则$k_1,k_2,\cd,k_r$中正数的个数与$\lambda_1,\lambda_2,\cd,\lambda_r$中正数的个数相等。
  \end{dingli}
\end{frame}

\begin{frame}
  二次型的标准形中,正系数的个数称为二次型的\blue{正惯性指数},负系数的个数称为\blue{负惯性指数}。若二次型$f$的正惯性指数为$p$,秩为$r$,则$f$的规范形便可确定为
  $$
  f=y_1^2+\cd+y_p^2-y_{p+1}^2-\cd-y_r^2.
  $$
\end{frame}

\begin{frame}
  
    \begin{dingyi}
      如果对于任意的非零向量$\xx=(x_1,x_2,\cd,x_n)^T$,恒有
      $$
      \xx^T\A\xx=\sum_{i=1}^n\sum_{j=1}^na_{ij}x_ix_j>0,
      $$
      就称$\xx^T\A\xx$为\red{正定二次型},称$\A$为\red{正定矩阵}。
    \end{dingyi}
    \pause\vspace{0.1in}

    
    类似地,
    \begin{itemize}
    \item 若$\forall \xx \ne 0$, 恒有$\xx^T\A\xx\ge 0$,则称$\A$为\red{半正定矩阵};
    \item 若$\forall \xx \ne 0$, 恒有$\xx^T\A\xx < 0$,则称$\A$为\red{负定矩阵};
    \item 若$\forall \xx \ne 0$, 恒有$\xx^T\A\xx \le 0$,则称$\A$为\red{半负定矩阵}.
    \end{itemize}
    
    
  
\end{frame}


\begin{frame}
  
    \begin{dingli}
      若$\A$是$n$阶实对称矩阵,则以下命题等价:
      \begin{itemize}
      \item[(1)]$\xx^T\A\xx$是正定二次型($\A$是正定矩阵);
      \item[(2)]$\A$的正惯性指数为$n$,即$\A$合同于单位阵$\E$;
      \item[(3)]存在可逆矩阵$\PP$使得$\A=\PP^T\PP$;
      \item[(4)]$\A$的$n$个特征值$\lambda_1,\lambda_2,\cd,\lambda_n$全大于零;
      \item[(5)]$\A$的各阶顺序主子式全大于零.
      \end{itemize}
    \end{dingli}
  
\end{frame}

\begin{frame}
  
    \begin{dingli}
      $$
      \A\mbox{正定}~~\Longrightarrow~~
      a_{ii}>0(i=1,2,\cd,n) \mbox{~~且~~}
      |\A|>0
      $$
    \end{dingli}
\end{frame}


\begin{frame}
  
    \begin{li}[\red{$\bigstar$}]
      判断二次型
      $$
      f(x_1,x_2,x_3)=x_1^2+2x_2^2+3x_3^2+2x_1x_2-2x_2x_3
      $$
      是否为正定二次型。
    \end{li}
    \begin{jie}
      用特征值或顺序主子式判定.
    \end{jie}
\end{frame}

\begin{frame}
  
    \begin{li}[\red{$\bigstar$}]
      判断二次型
      $$
      f(x_1,x_2,x_3)=3x_1^2+x_2^2+3x_3^2-4x_1x_2-4x_1x_3+4x_2x_3
      $$
      是否为正定二次型。
    \end{li}
    \begin{jie}
      用特征值或顺序主子式判定。
    \end{jie}
\end{frame}

\begin{frame}
\begin{li}[\red{$\bigstar$}]
  判断二次型$f=-5x^2-6y^2-4z^2+4xy+4xz$的正定性。
\end{li}
\pause
\begin{jie}
  $f$的矩阵为
  $$
  \A=\left(
    \begin{array}{rrr}
      -5&2&2\\
      2&-6&0\\
      2&0&-4\\
    \end{array}
  \right)
  $$
  因
  $$
  a_{11}=-5<0, ~~\left|
    \begin{array}{rr}
      -5&2\\
      2&-6
    \end{array}
  \right|=26>0, ~~\left|
    \begin{array}{rrr}
      -5&2&2\\
      2&-6&0\\
      2&0&-4\\
    \end{array}
  \right|=-80<0
  $$
  故$f$负定。
\end{jie}
\end{frame}
