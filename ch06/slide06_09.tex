\section{选择哪种循环}
\begin{frame}[fragile]\ft{\secname}
先确定是入口条件循环还是退出条件循环。通常采用入口条件循环,原因在于

\begin{itemize}
\item
一般原则:在循环之前判断要比之后好;\\[0.1in]
\item 
在循环开始时进行判断,程序的可读性更强;\\[0.1in]
\item
很多应用中,若一开始就不满足判断条件,那么跳过整个循环是非常重要的。
\end{itemize}
\end{frame}

\begin{frame}[fragile]\ft{\secname}
要使for循环更像while循环,可以去掉第一个和第三个表达式:
\begin{lstlisting}
for (; condition; )
\end{lstlisting}
与
\begin{lstlisting}
while (condition)
\end{lstlisting}
等效。
\end{frame}

\begin{frame}[fragile]\ft{\secname}
要使while循环更像for循环,可以在前面使用初始化语句并包含更新语句:
\begin{lstlisting}
initialization;
while (condition) {
  body;
  update;
}
\end{lstlisting}
与
\begin{lstlisting}
for (initialization; condition; update){
  body;
}
\end{lstlisting}
等效。
\end{frame}

\begin{frame}[fragile]\ft{\secname}
\begin{itemize}
\item 在循环中涉及到初始化和更新变量时,使用for循环较为适当,而在其他条件下使用while循环更好一些。while循环对以下条件比较自然:
\begin{lstlisting}
while (scanf("%ld", &num) == 1){
  ...
}
\end{lstlisting}

\item 对那些涉及到用索引计数的循环,使用for循环是一个更自然的选择。
\begin{lstlisting}
for (count = 1; count <= 100; count++){
  ...
}
\end{lstlisting}
\end{itemize}

\end{frame}
