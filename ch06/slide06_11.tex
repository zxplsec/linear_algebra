\section{数组}
\begin{frame}[fragile]\ft{\secname}
数组是线性存储的一系列相同类型的值。

整个数组有一个单一的名字,每个元素使用一个整数索引来进行访问。
\end{frame}

\begin{frame}[fragile]\ft{\secname:声明}
\begin{lstlisting}[language=c,backgroundcolor=\color{red!10}]
float debts[20];
\end{lstlisting}

\begin{itemize}
\item
声明 \lstinline|debts| 是一个具有20个元素的数组,每个元素都是一个类型为 \lstinline|float| 的值。\\[0.1in]
\item
该数组的第一个元素为 \lstinline|debts[0]|,第二个元素为 \lstinline|debts[1]|,...。\\[0.1in]
\item
\red{数组元素的编号从零开始}。\\[0.1in]
\item 每个元素都可以被赋予一个 \lstinline|float| 类型的值。
\begin{lstlisting}[language=c,backgroundcolor=\color{red!10}]
debts[3] = 2.;
debts[7] = 1.2e+10;
\end{lstlisting}
\end{itemize}
\end{frame}

\begin{frame}[fragile]\ft{\secname}
\begin{itemize}
\item
可以像使用相同类型的变量那样使用一个数组元素。例如,你可以把一个值读入一个特定的元素:
\begin{lstlisting}
scanf("%d", &debts[4]); // `为第5个元素读入一个值`
\end{lstlisting} 
\end{itemize}

\end{frame}

\begin{frame}[fragile]\ft{\secname}
\begin{itemize}
\item
易犯的错误:处于执行速度的考虑,C并不检查你是否使用了正确的下标。如
\begin{lstlisting}
debts[20] = 11.0;  // `没有这个数组元素`
debts[31] = 22.22; // `没有这个数组元素`
\end{lstlisting}
但编译器不会报错。当程序运行时,这些语句把数据放在可能由其他数据使用的位置上,因而可能破坏程序的结果甚至是程序崩溃。
\end{itemize}
\end{frame}

\begin{frame}[fragile]\ft{\secname}
\begin{itemize}
\item
数组可以是任意数据类型。
\begin{lstlisting}
int num[10];  // `一个存放10个整数的数组`
char ch[20];  // `一个存放20个字符的数组`
double a[40]; // `一个存放40个double值的数组`
\end{lstlisting}
特别地,字符串就是一个字符数组。
\end{itemize}
\end{frame}

\begin{frame}[fragile]\ft{\secname}
\begin{itemize}
\item
用于标识数组元素的数字称为下标(subscript)、索引(index)或偏移量(offset)。
\\[0.1in]
\item 下标必须是整数,它从0开始。\\[0.1in]
\item 数组中的元素在内存中是顺序存储的。
\end{itemize}
\end{frame}

\begin{frame}[fragile]\ft{在for循环中使用数组}
\begin{li}
读入你5天运动的步数,然后进行处理,最后报告步数的总和、平均值和差点(平均值与标志值之间的差)。
\end{li}
\end{frame}


\begin{frame}[fragile,allowframebreaks]\ft{\secname}
\lstinputlisting[numbers=left]{ch06/code/scores.c}
\end{frame}


\begin{frame}[fragile]\ft{\secname}
  \begin{lstlisting}
Enter steps of 5 days:
9999 9995 11000 12012 11145 95667
The steps read in are as follows:
   9999   9995  11000  12012  11145
Sum of steps = 54151, average = 10830.20
That's a handicap of 830.
\end{lstlisting}
\end{frame}
 
