\section{惯性定理与二次型的规范形}
\begin{frame}
  \begin{dingli}[惯性定理]
    对一个$n$元二次型$\vx^T\MA\vx$,不论做怎样的坐标变换使之化为标准型,其中正平方项的项数$p$与负平方项的项数$q$都是唯一确定的。

    或者说,对一个$n$阶实对称矩阵$\MA$,不论取怎样的可逆矩阵$\MC$,只要使
    $$
    \MC^T\MA\MC=\left[
      \begin{array}{ccccccccc}
        d_1&&&&&&&&\\
           &\ddots&&&&&&&\\
           &&d_p&&&&&&\\
           &&&-d_{p+1}&&&&&\\
           &&&&\ddots&&&&\\
           &&&&&-d_{p+q}&&&\\
           &&&&&&0&&\\
           &&&&&&&\ddots&\\
           &&&&&&&&0\\
      \end{array}
    \right]
    $$
    其中$d_i>0, ~i=1,2,\cd,p+q, p+q\le n$成立,则$p$和$q$是由$\MA$唯一确定的。
  \end{dingli}
\end{frame}

\begin{frame}
  \blue{\bf 证明.}
    因$\rank(\MA)=\rank(\MC^T\MA\MC)=p+q$,故$p+q$由$\MA$的秩唯一确定,因此只需证明$p$由$\MA$唯一确定。 \vspace{.1in}\pause 

    设$p+q=\rank(\MA)=r$,二次型$f=\vx^T\MA\vx$经坐标变换$\vx=\MB\vy$和$\vx=\MC\vz$都可化为标准型:
    \begin{eqnarray}
      f&=&b_1y_1^2+\cd+b_py_p^2-b_{p+1}y_{p+1}^2-\cd-b_{r}y_r^2, \\[.1in]
      f&=&c_1z_1^2+\cd+c_tz_t^2-c_{t+1}z_{t+1}^2-\cd-c_{r}z_r^2.
    \end{eqnarray}
    要证正平方项的项数唯一确定,即证$p=t$。
\end{frame}

\begin{frame}
  用反证法。设$p>t$,则
  $$
  \begin{array}{rrrr}
    f&=\blue{b_1y_1^2+\cd+b_ty_t^2}&\red{+b_{t+1}y_{t+1}^2+\cd+b_py_p^2}&-b_{p+1}y_{p+1}^2-\cd-b_{r}y_r^2\\[.1in]
     &=\blue{c_1z_1^2+\cd+c_tz_t^2}&\red{-c_{t+1}z_{t+1}^2-\cd-c_pz_p^2}&-c_{p+1}z_{p+1}^2-\cd-c_{r}z_r^2
  \end{array}
  $$
  因$\vz = \MC^{-1}\MB\vy := \MD \vy$,可得
  \begin{equation}\label{ls1}
  \left\{
    \begin{array}{c}
      z_1 = d_{11}y_1+d_{12}y_2+\cd+d_{1n}y_n, \\
      \cd\cd\\
      z_t = d_{t1}y_1+d_{t2}y_2+\cd+d_{tn}y_n, \\
      \cd\cd\\
      z_n = d_{n1}y_1+d_{n2}y_2+\cd+d_{nn}y_n.
    \end{array}
  \right.
  \end{equation}
  \pause 

  令$z_1=\cd=z_t=0, y_{p+1}=\cd=y_p=0$,利用上式可得
  \begin{equation}\label{ls2}
  \left\{
    \begin{array}{r}
      d_{11}y_1+d_{12}y_2+\cd+d_{1n}y_n=0, \\
      \cd\cd\\
      d_{t1}y_1+d_{t2}y_2+\cd+d_{tn}y_n=0, \\
      y_{p+1}=0, \\
      \cd\cd\\
      y_n=0.
    \end{array}
  \right.
  \end{equation}
  齐次线性方程组\eqref{ls2}有$n$个未知量,但方程个数为$t+(n-p)=n-(p-t)<n$,故必有非零解。
\end{frame}

\begin{frame}
  由于$y_{p+1}=\cd=y_n=0$,故$y_1,\cd,y_p$不全为零,于是
  $$
  f=b_1y_1^2+\cd+b_py_p^2>0.
  $$
  \pause 

  将\eqref{ls2}的非零解代入\eqref{ls1}得$z_1,\cd,z_t,\cd,z_n$的一组值($z_1=\cd=z_t=0$),于是
  $$
  f=-c_{t+1}z_{t+1}^2-\cd-c_{r}z_r^2\le 0.
  $$
  这导致矛盾,故假设$p>t$不能成立。
  \pause 

  同理可证$p<t$不成立,故$p=t$。这就证明了二次型的标准型中,正平方项的项数与所做的非退化线性变换无关,它是由二次型本身所确定的。 \qed
\end{frame}

\begin{frame}
  \begin{dingyi}
    二次型$\vx^T\MA\vx$的标准型中,
    \begin{itemize}
    \item 正平方项的项数称为二次型(或$\MA$)的\red{正惯性指数};\\[0.1in]
    \item 负平方项的项数称为二次型(或$\MA$)的\red{负惯性指数};\\[0.1in]
    \item 正、负惯性指数的差称为\red{符号差};\\[0.1in]
    \item 矩阵$\MA$的秩也称为二次型$\vx^T\MA\vx$的秩。
    \end{itemize}
  \end{dingyi}

  \vspace{.1in}\pause 

  设$n$阶实对称矩阵的秩为$r$,正惯性指数为$p$,则
  \begin{itemize}
  \item 负惯性指数为$q=r-p$;
  \item 符号差为$p-q=2p-r$;
  \item 与$\MA$合同的对角阵的零对角元个数为$n-r$。
  \end{itemize}
  \end{frame}

\begin{frame}
  \begin{tuilun}
    设$\MA$为$n$阶实对称矩阵,若$\MA$的正、负惯性指数分别为$p$和$q$,则
    \begin{equation}\label{eq1}
    \MA \simeq \diag(\red{\underbrace{1,\cd,1}_{p\mbox{个}}},\blue{\underbrace{-1,\cd,-1}_{q\mbox{个}}},0,\cd,0).
    \end{equation}

    或者说,对于二次型$\vx^T\MA\vx$,存在坐标变换$\vx=\MC\vy$使得
    \begin{equation}\label{eq2}
    \vx^T\MA\vx=y_1^2+\cd+y_p^2-y_{p+1}^2-\cd-y_{p+q}^2.
    \end{equation}
    把\eqref{eq2}右端的二次型称为\red{$\vx^T\MA\vx$的规范形};把\eqref{eq1}中的对角阵称为\red{$\MA$的合同规范形}。
  \end{tuilun}
\end{frame}


\begin{frame}
  \begin{proof}
    $$
    \MC_1^T\MA\MC_1=\diag(d_1,\cd,d_p,-d_{p+1},\cd,-d_{p+q},0,\cd,0)
    $$ \pause 
    取可逆阵
    $$
    \MC_2 = \diag(\frac1{\sqrt{d_1}},\cd,\frac1{\sqrt{d_p}},\frac1{\sqrt{d_{p+1}}},\cd,\frac1{\sqrt{d_{p+q}}},1,\cd,1)
    $$
    则有
    $$
    \MC_2^T(\MC_1^T\MA\MC_1)\MC_2=\diag(1,\cd,1,-1,\cd,-1,0,\cd,0)
    $$
  \end{proof}
\end{frame}

\begin{frame}
  如果两个$n$阶实对称矩阵$\MA,\MB$合同,也称它们对应的二次型$\vx^T\MA\vx$和$\vy^T\MB\vy$合同。
\end{frame}

\begin{frame}
  \begin{jielun}
    \begin{itemize}
    \item 两个实对称矩阵$\MA,\MB$合同的充分必要条件是$\MA,\MB$有相同的正惯性指数和负惯性指数;\\[0.1in]
    \item 全体$n$阶实对称矩阵,按其合同规范形(不考虑$+1,-1,0$的排列次序)分类,共有$\frac{(n+1)(n+2)}2$类。
    \end{itemize}
  \end{jielun}
\end{frame}

