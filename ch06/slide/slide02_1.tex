
\subsection{正交变换法}


\begin{frame}
  
    \begin{block}{回顾}
      对于实对称矩阵$\MA$,存在正交阵$\MQ$,使得
      $$
      \MQ^{-1}\MA\MQ=\Lambdabd,
      $$
      由于$\MQ^{-1}=\MQ^T$,故
      $$
      \MQ^{T}\MA\MQ=\Lambdabd.
      $$
    \end{block}
  
\end{frame}

\begin{frame}
  
    \begin{dingyi}[主轴定理]
      对于任一个$n$元二次型
      $$
      f(x_1,x_2,\cd,x_n)=\vx^T\MA\vx,
      $$
      存在正交变换$\vx=\MQ\vy$($\MQ$为正交阵),使得
      $$
      \vx^T\MA\vx=\vy^T(\MQ^T\MA\MQ)\vy=\lambda_1y_1^2+\lambda_2y_2^2+\cd+\lambda_ny_n^2,
      $$
      其中$\lambda_1,\lambda_2,\cd,\lambda_n$为$\MA$的$n$个特征值,
      $\MQ$的$n$个列向量$\alphabd_1,\alphabd_2,\cd,\alphabd_n$是$\MA$对应于$\lambda_1,\lambda_2,\cd,\lambda_n$的标准正交特征向量。
    \end{dingyi}
  
\end{frame}



\begin{frame}
  
    \begin{li}
      用正交变换法,将二次型
      $$
      f(x_1,x_2,x_3)=2x_1^2+5x_2^2+5x_3^2+4x_1x_2-4x_1x_3-8x_2x_3
      $$
      化为标准型。
    \end{li}
    \pause
    \begin{jie}
    对应方程为
    $$
    \MA=\left(
    \begin{array}{rrr}
      2&2&-2\\
      2&5&-4\\
      -2&-4&5
    \end{array}
    \right)
    $$
    \pause
    其特征多项式为
    $$
    |\lambda\MI-\MA|=(\lambda-1)^2(\lambda-10)
    $$
    得特征值$\lambda_{1,2}=1$和$\lambda_3=10$.
  \end{jie}
\end{frame}


\begin{frame}
  
    $$
    \begin{array}{rl}
      (\lambda_1\MI-\MA)\vx=\M0 & \Rightarrow~~
      \left(
      \begin{array}{rrr}
        -1&-2&2\\
        -2&-4&4\\
        2&4&-4
      \end{array}
      \right)\left(
      \begin{array}{c}
        x_1\\
        x_2\\
        x_3
      \end{array}
      \right)=\M0\\[0.3in] \pause
      & \Rightarrow~~
      \vx_1=(-2,1,0)^T, \quad
      \vx_2=(2,0,1)^T. \\[0.2in]\pause
      (\lambda_3\MI-\MA)\vx=\M0 & \Rightarrow~~
      \left(
      \begin{array}{rrr}
        8&-2&2\\
        -2&5&4\\
        2&4&5
      \end{array}
      \right)\left(
      \begin{array}{c}
        x_1\\
        x_2\\
        x_3
      \end{array}
      \right)=\M0\\[0.3in] \pause
      & \Rightarrow~~
      \vx_3=(1,2,-2)^T.
    \end{array}
    $$

    对$\vx_1,\vx_2$用施密特正交化方法得
    $$
    \xibd_1=\left(-\frac{2\sqrt{5}}5,\frac{2\sqrt{5}}5,0\right)^T,~~~~
    \xibd_2=\left(\frac{2\sqrt{5}}{15},\frac{4\sqrt{5}}{15},\frac{\sqrt{5}}3\right)^T
    $$
    再将$\vx_3$单位化为
    $$
    \xibd_3=\left(\frac13,\frac23,-\frac23\right)^T
    $$
  
\end{frame}


\begin{frame}
  
    取正交矩阵
    $$
    \MQ=(\xibd_1,\xibd_2,\xibd_3)=\left(
    \begin{array}{rrr}
      \ds-\frac{2\sqrt{5}}5&\ds\frac{2\sqrt{5}}{15}&\ds\frac13\\[0.2cm]
      \ds \frac{2\sqrt{5}}5&\ds\frac{4\sqrt{5}}{15}&\ds\frac23\\[0.2cm]
      \ds 0&\ds\frac{\sqrt{5}}3&\ds-\frac23
    \end{array}
    \right)
    $$
    则
    $$
    \MQ^{-1}\MA\MQ=\MQ^{T}\MA\MQ=\diag(1,1,10).
    $$ \pause 
    令$\vx=(x_1,x_2,x_3)^T,\vy=(y_1,y_2,y_3)^T$,做正交变换$\vx=\MQ\vy$,原二次型就化成标准型
    $$
    \vx^T\MA\vx=\vy^T(\MQ^T\MA\MQ)\vy=y_1^2+y_2^2+10y_3^2.
    $$
  
\end{frame}


\begin{frame}
  
    对在自然坐标系$\{\epsilonbd_1,\epsilonbd_2,\epsilonbd_3\}$下的二次曲面
    $$
    2x_1^2+5x_2^2+5x_3^2+4x_1x_2-4x_1x_3-8x_2x_3=1
    $$ \pause
    若将坐标系$\{\epsilonbd_1,\epsilonbd_2,\epsilonbd_3\}$变换为另一直角坐标系
    $$
    \xibd_1=\left(
    \begin{array}{r}
      \ds-\frac{2\sqrt{5}}5\\[0.2cm]
      \ds \frac{2\sqrt{5}}5\\[0.2cm]
      \ds 0
    \end{array}
    \right), \quad
    \xibd_2=\left(
    \begin{array}{r}
      \ds\frac{2\sqrt{5}}{15}\\[0.2cm]
      \ds\frac{4\sqrt{5}}{15}\\[0.2cm]
      \ds\frac{\sqrt{5}}3
    \end{array}
    \right),\quad
    \xibd_3=\left(
    \begin{array}{r}
      \ds\frac13\\[0.2cm]
      \ds\frac23\\[0.2cm]
      \ds-\frac23
    \end{array}
    \right)
    $$\pause
    即
    $$
    (\xibd_1,\xibd_2,\xibd_3)=(\etabd_1,\etabd_2,\etabd_3)\left(
    \begin{array}{rrr}
      \ds-\frac{2\sqrt{5}}5&\ds\frac{2\sqrt{5}}{15}&\ds\frac13\\[0.2cm]
      \ds \frac{2\sqrt{5}}5&\ds\frac{4\sqrt{5}}{15}&\ds\frac23\\[0.2cm]
      \ds 0&\ds\frac{\sqrt{5}}3&\ds-\frac23
    \end{array}
    \right)
    $$\pause
    则在坐标系$\{\xibd_1,\xibd_2,\xibd_3\}$下,曲面方程为
    $$
    y_1^2+y_2^2+10y_3^2=1.
    $$
  
\end{frame}


\begin{frame}
  
    \begin{li}
      将一般二次曲面方程
      $$
      x^2-2y^2+10z^2+28xy-8yz+20zx-26x+32y+28z-38=0
      $$
      化为标准方程。
    \end{li}
    \pause
    \begin{jie}
    二次型部分
    $
    \blue{x^2-2y^2+10z^2+28xy-8yz+20zx}
    $
    对应的矩阵为
    $$
    \left(
    \begin{array}{rrr}
      1&14&10\\
      14&-2&-4\\
      10&-4&10
    \end{array}
    \right)
    $$
    同前例,可得正交阵
    $$
    \MQ=\left(
    \begin{array}{rrr}
      \ds\frac13&\ds\frac23&\ds\frac23\\[0.1in]
      \ds\frac23&\ds\frac13&\ds-\frac23\\[0.1in]
      \ds-\frac23&\ds\frac23&\ds-\frac13
    \end{array}
    \right)
    $$
    使得
    $$
    \MQ^T\MA\MQ=\diag(9,18,-18).
    $$
    \end{jie}
\end{frame}

\begin{frame}
  
    做正交变换$\vx=\MQ\vy,~\vx=(x,y,z)^T,~~\vy=(x^\prime,y^\prime,z^\prime)^T$,可得
    $$
    \vx^T\MA\vx=\vy^T(\MQ^T\MA\MQ)\vy=9{x^\prime}^2+18{y^\prime}^2-18{z^\prime}^2
    $$\pause
    将$\vx=\MQ\vy$代入原方程即得
    $$
    {x^\prime}^2+2{y^\prime}^2-2{z^\prime}^2-\frac23x^\prime+\frac43y^\prime-\frac{16}3z^\prime-\frac{28}9=0
    $$\pause
    配方得
    $$
    \left(x^\prime-\frac13\right)^2+2\left(y^\prime+\frac13\right)^2-2\left(z^\prime+\frac43\right)^2=1
    $$\pause
    再令
    $$
    x^{\prime\prime}=x^\prime-\frac13,~~
    y^{\prime\prime}=y^\prime+\frac13,~~
    z^{\prime\prime}=z^\prime+\frac43,
    $$
    代入上式得标准方程
    $$
    {x^{\prime\prime}}^2+2{y^{\prime\prime}}^2-2{z^{\prime\prime}}^2=1
    $$
  
\end{frame}
