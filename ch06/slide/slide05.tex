\section{其他有定二次型}

\begin{frame}
   \begin{dingyi}
     如果对于任意的非零向量$\vx=(x_1,x_2,\cd,x_n)^T$,恒有二次型
     \begin{itemize}
     \item $\vx^T\MA\vx\ge0$,但至少存在一个$\vx_0\ne \M0$使得$\vx_0^T\MA\vx_0=0$,就称$\vx^T\MA\vx$为\blue{半正定二次型},称$\MA$为\blue{半正定矩阵};\\[.1in]
     \item $\vx^T\MA\vx<0$,称$\vx^T\MA\vx$为\blue{负定二次型},称$\MA$为\blue{负定矩阵};
     \item $\vx^T\MA\vx\le0$,但至少存在一个$\vx_0\ne \M0$使得$\vx_0^T\MA\vx_0=0$,就称$\vx^T\MA\vx$为\blue{半负定二次型},称$\MA$为\blue{半负定矩阵}。
     \end{itemize}     
    \end{dingyi}

    \vspace{.1in}\pause 
    正定、半正定、负定、半负定二次型,统称为\blue{有定二次型};如果二次型不是有定的,就称为\blue{不定二次型}。
\end{frame}

\begin{frame}
  \begin{li}
    对于二次型
    $$
    \vx^T\MA\vx=d_1x_1^2+d_2x_2^2+\cd+d_nx_n^2,
    $$
    \begin{itemize}
      \item 当$d_i<0, ~i=1,2,\cd,n$时,是负定的;\\[.1in]
      \item 当$d_i\ge0, ~i=1,2,\cd,n$但至少有一个为零时,是半正定的;\\[.1in]
      \item 当$d_i\le0, ~i=1,2,\cd,n$但至少有一个为零时,是半负定的。
    \end{itemize}
  \end{li}
\end{frame}

\begin{frame}
  \begin{dingli}
    设$\MA$是$n$阶实对称矩阵,则下列命题等价:
    \begin{enumerate}
    \item $\vx^T\MA\vx$负定;\\[.1in]
    \item $\MA$的负惯性指数为$n$,即$\MA\simeq -\MI$;\\[.1in]
    \item 存在可逆矩阵$\MP$,使得$\MA=-\MP^T\MP$;\\[.1in]
    \item $\MA$的特征值全小于零;\\[.1in]
    \item $\MA$的奇数阶顺序主子式全小于零,偶数阶顺序主子式全大于零。
    \end{enumerate}
  \end{dingli}
\end{frame}

\begin{frame}
  \begin{dingli}
    设$\MA$是$n$阶实对称矩阵,则下列命题等价:
    \begin{enumerate}
    \item $\vx^T\MA\vx$半正定;\\[.1in]
    \item $\MA$的正惯性指数$=\rank(\MA)=r<n$,或$\MA\simeq -\diag(\underbrace{1,1,\cd,1}_{r},0,\cd,0)$;\\[.1in]
    \item 存在非满秩矩阵$\MP(\rank(\MP)<n)$,使得$\MA=\MP^T\MP$;\\[.1in]
    \item $\MA$的特征值全大于等于零,但至少有一个为零;\\[.1in]
    \item $\MA$的各阶顺序主子式$\ge 0$,但至少有一个主子式等于零。
    \end{enumerate}
  \end{dingli}
\end{frame}

\begin{frame}
  \begin{li}
    判断二次型$\ds f=n \sum_{i=1}^n x_i^2 - \left(\sum_{i=1}^n x_i\right)^2$是否为有定二次型。
  \end{li}
  \pause 
  \begin{jie}
    $$
    \begin{aligned}
      f&=(n-1)\sum_{i=1}^n x_i^2 - \sum_{1\le i<j \le n} 2x_ix_j\\
      &=\sum_{1\le i<j \le n}(x_i-x_j)^2\ge 0
    \end{aligned}
    $$
    当$x_1=x_2=\cd=x_n$时,等号成立,故原二次型是半正定的。 \vspace{.1in} \pause 

    也可通过求特征值来判断。二次型矩阵$\MA$的特征值为$\lambda_1=0,\lambda_2=n$($n-1$重),故二次型是半正定的。
  \end{jie}
\end{frame}

