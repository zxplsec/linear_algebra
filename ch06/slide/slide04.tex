\section{正定二次型和正定矩阵}

\begin{frame}  
    \begin{dingyi}
      如果对于任意的非零向量$\vx=(x_1,x_2,\cd,x_n)^T$,恒有
      $$
      \vx^T\MA\vx=\sum_{i=1}^n\sum_{j=1}^na_{ij}x_ix_j>0,
      $$
      就称$\vx^T\MA\vx$为正定二次型,称$\MA$为正定矩阵。
    \end{dingyi}
    \pause\vspace{0.1in}

    
    注:正定矩阵是针对对称矩阵而言的。
    
  
\end{frame}

\begin{frame}
  
    \begin{jielun}
      二次型$f(y_1,y_2,\cd,y_n)=d_1y_1^2+d_2y_2^2+\cd+d_ny_n^2$正定
      $~~~\Longleftrightarrow~~~d_i>0~~(i=1,2,\cd,n)$
    \end{jielun}\pause

    \begin{proof}
    \begin{itemize}
    \item[$\Leftarrow$] 显然 \pause
    \item[$\Rightarrow$] 设$d_i\le 0$,取$y_i=1, y_j=0(j\ne i)$,代入二次型,得
      $$
      f(0,\cd,0,1,0,\cd,0)=d_i\le 0
      $$
      这与二次型$f(y_1,y_2,\cd,y_n)$正定矛盾。
    \end{itemize}
    \end{proof}
\end{frame}


\begin{frame}
  
    \begin{jielun}
      一个二次型$\vx^T\MA\vx$,经过非退化线性变换$\vx=\MC\vy$,化为$\vy^T(\MC^T\MA\MC)\vy$,其正定性保持不变。即当
      $$\vx^T\MA\vx~~~\xLongleftrightarrow[]{\ds \vx=\MC\vy}~~~\vy^T(\MC^T\MA\MC)\vy\quad (\MC\mbox{可逆})$$
      时,等式两端的二次型有相同的正定性。
    \end{jielun}\pause
    \begin{proof}
    $\forall \vy=(y_1,y_2,\cd,y_n)\ne\M0$,由于$\vx=\MC\vy(\MC\mbox{可逆})$,则$\vx\ne \M0$。若$\vx^T\MA\vx$正定,则$\vx^T\MA\vx>0$。
    从而有:$\forall \vy\ne\M0$,
    $$
    \vy^T(\MC^T\MA\MC)\vy=\vx^T\MA\vx>0
    $$
    故$\vy^T(\MC^T\MA\MC)\vy$是正定二次型。\pause 反之亦然。
    \end{proof}
\end{frame}


\begin{frame}
  
    \begin{dingli}
      若$\MA$是$n$阶实对称矩阵,则以下命题等价:
      \begin{itemize}
      \item[(1)]$\vx^T\MA\vx$是正定二次型($\MA$是正定矩阵);
      \item[(2)]$\MA$的正惯性指数为$n$,即$\MA\simeq\MI$;
      \item[(3)]存在可逆矩阵$\MP$使得$\MA=\MP^T\MP$;
      \item[(4)]$\MA$的$n$个特征值$\lambda_1,\lambda_2,\cd,\lambda_n$全大于零。
      \end{itemize}
    \end{dingli}
  
\end{frame}

\begin{frame}
  
    \begin{li}
      $\MA\mbox{正定} ~~\Longrightarrow~~ \MA^{-1}\mbox{正定}$
    \end{li}
  
\end{frame}

\begin{frame}
  
    \begin{li}
      判断二次型
      $$
      f(x_1,x_2,x_3)=x_1^2+2x_2^2+3x_3^2+2x_1x_2-2x_2x_3
      $$
      是否为正定二次型。
    \end{li}
  
\end{frame}

\begin{frame}
  
    \begin{li}
      判断二次型
      $$
      f(x_1,x_2,x_3)=3x_1^2+x_2^2+3x_3^2-4x_1x_2-4x_1x_3+4x_2x_3
      $$
      是否为正定二次型。
    \end{li}
  
\end{frame}

\begin{frame}
  
    \begin{dingli}
      $$
      \MA\mbox{正定}~~\Longrightarrow~~
      a_{ii}>0(i=1,2,\cd,n) \mbox{~~且~~}
      |\MA|>0
      $$
    \end{dingli}
  
\end{frame}

\begin{frame}
  
    \begin{dingli}
      $$\MA\mbox{正定} ~~\Longleftrightarrow~~ \MA\mbox{的$n$个顺序主子式全大于零。}$$
    \end{dingli}
  
\end{frame}
