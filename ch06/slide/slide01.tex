\section{二次型的定义与矩阵表示\quad 合同矩阵}

\begin{frame}
  
    \begin{dingyi}[二次型]
      $n$元变量$x_1,x_2,\cd,x_n$的二次齐次多项式
      $$
      \begin{array}{rcccccccc}
        f(x_1,x_2,\cd,x_n) &=& \\[0.1in]
        a_{11}x_1^2&+&2a_{12}x_1x_2&+&2a_{13}x_1x_3&+&\cd&+&2a_{1n}x_1x_n\\[0.1in]
        &+&a_{22}x_2^2&+&2a_{23}x_2x_3&+&\cd&+&2a_{2n}x_2x_n\\[0.1in]
        &&&&\cd&&\cd\\[0.1in]
        &&&&&&&+&a_{nn}x_n^2
      \end{array}
      $$
     当系数属于数域$F$时,称为数域$F$上的一个\blue{\underline{$n$元二次型}}。
    \end{dingyi}
  
\end{frame}


\begin{frame}
  
    $$
    \begin{array}{l}
      f(x_1,x_2,\cd,x_n) =  \\[0.1in]
      \begin{array}{rcccccccccc}
        &a_{11}x_1^2&+&a_{12}x_1x_2&+&a_{13}x_1x_3&+&\cd&+&a_{1n}x_1x_n\\[0.1in]
        +&a_{21}x_2x_1&+&a_{22}x_2^2&+&a_{23}x_2x_3&+&\cd&+&a_{2n}x_2x_n\\[0.1in]
        &\cd&&\cd&&\cd&&\cd\\[0.1in]
        +&a_{n1}x_nx_1&+&a_{n2}x_nx_2&+&a_{n3}x_2x_3&+&\cd&+&a_{nn}x_n^2
      \end{array} \\[0.5in] \pause
      \ds = \sum_{i=1}^nx_i(a_{i1}x_1+a_{i2}x_2+\cd+a_{in}x_n)
      \ds = \sum_{i=1}^nx_i\sum_{j=1}^n a_{ij}x_{ij}
      \ds = \sum_{i=1}^n\sum_{j=1}^na_{ij}x_ix_j\\[0.2in] \pause
      \ds = (x_1,x_2,\cd,x_n)\left(
      \begin{array}{c}
        a_{11}x_1+a_{12}x_2+\cd+a_{1n}x_n\\
        a_{21}x_1+a_{22}x_2+\cd+a_{2n}x_n\\
        \vd\\
        a_{n1}x_1+a_{n2}x_2+\cd+a_{nn}x_n
      \end{array}
      \right)\\[0.3in]\pause
      \ds = (x_1,x_2,\cd,x_n)\left(
      \begin{array}{cccc}
        a_{11}&a_{12}&\cd&a_{1n}\\
        a_{21}&a_{22}&\cd&a_{2n}\\
        \vd&\vd&&\vd\\
        a_{n1}&a_{n2}&\cd&a_{nn}\\
      \end{array}
      \right)\left(
      \begin{array}{c}
        x_1\\
        x_2\\
        \vd\\
        x_n
      \end{array}\right) \pause = \vx^T\MA\vx
    \end{array}
    $$
  
\end{frame}


\begin{frame}
  
    \begin{itemize}
    \item
      对于任意一个二次型,总可以写成对称形式
      $$
      f(x_1,x_2,\cd,x_n)=\vx^T\MA\vx
      $$
      其中$\MA$为对称矩阵。\\[0.1in]\pause
    \item
      若$\MA,\MB$为对称矩阵,且
      $$
      f(x_1,x_2,\cd,x_n)=\vx^T\MA\vx=\vx^T\MB\vx
      $$
      则必有$\MA=\MB$。\\[0.1in]\pause
    \item 二次型和它的矩阵式相互唯一确定的,因此研究二次型的性质就转化为研究$\MA$所具有的性质。
    \end{itemize}

  
\end{frame}

\begin{frame}
  
    \begin{li}
      设$f(x_1,x_2,x_3,x_4)=2x_1^2+x_1x_2+2x_1x_3+4x_2x_4+x_3^2+5x_4^2$,则它的矩阵为
      $$
      \MA=\left(
      \begin{array}{cccc}
        2&\ds\frac12&1&0\\[0.2cm]
        \ds\frac12&0&0&2\\[0.2cm]
        1&0&1&0\\[0.2cm]
        0&2&0&5
      \end{array}
      \right)
      $$
    \end{li}
  
\end{frame}

\begin{frame}
  
    一个二次型$\vx^T\MA\vx$可看成是$n$维向量$\alphabd$的一个函数,即
    $$
    f(\alphabd)=\vx^T\MA\vx
    $$
    其中$\vx=(x_1,x_2,\cd,x_n)^T$是$\alphabd$在$\mathbb R^n$的一组基下的坐标向量,故二次型$\vx^T\MA\vx$是向量$\alphabd$的$n$个坐标的二次齐次函数。
    因此二次型作为$\alphabd$的函数,其矩阵是与一组基相联系的。
  
\end{frame}

\begin{frame}
  
    设$\alphabd$在两组基$\{\epsilonbd_1,\epsilonbd_2,\cd,\epsilonbd_n\}$和$\{\etabd_1,\etabd_2,\cd,\etabd_n\}$下的坐标向量分别为
    $$
    \vx=(x_1,x_2,\cd,x_n)^T\mbox{~~和~~}\vy=(y_1,y_2,\cd,y_n)^T
    $$
    又
    $$
    (\etabd_1,\etabd_2,\cd,\etabd_n)=(\epsilonbd_1,\epsilonbd_2,\cd,\epsilonbd_n)\MC
    $$
    故
    $$
    \vx=\MC\vy
    $$
    从而
    $$
    f(\alphabd)=\vx^T\MA\vx=\vy^T(\MC^T\MA\MC)\vy
    $$ \pause


    \blue{
      二次型$f(\alphabd)$在两组基$\{\epsilonbd_1,\epsilonbd_2,\cd,\epsilonbd_n\}$和$\{\etabd_1,\etabd_2,\cd,\etabd_n\}$下所对应的矩阵分别为
      $$
      \MA \mbox{~~和~~} \MC^T\MA\MC     
      $$
    }
  
\end{frame}


\begin{frame}
  
    \begin{li}
      设$\alphabd$在自然基$\{\epsilonbd_1,\epsilonbd_2\}$下的坐标$(x_1,x_2)^T$满足方程
      \begin{equation}\label{li2-1}
        5x_1^2+5x_2^2-6x_1x_2=4.
      \end{equation}      
      将$\epsilonbd_1,\epsilonbd_2$逆时针旋转$\pi/4$变为$\etabd_1,\etabd_2$      
    \end{li}
    \pause 
    $$
    (\etabd_1,\etabd_2)=(\epsilonbd_1,\epsilonbd_2)\left(
    \begin{array}{rr}
      \cos \pi/4&-\sin \pi/4\\
      \sin\pi/4&\cos\pi/4
    \end{array}
    \right)
    $$
    则$\alphabd$在基$\{\etabd_1,\etabd_2\}$下的坐标$(y_1,y_2)^T$满足
    $$
    \vx=\left(
    \begin{array}{c}
      x_1\\
      x_2
    \end{array}
    \right)=\left(
    \begin{array}{rr}
      \cos \pi/4&-\sin \pi/4\\
      \sin\pi/4&\cos\pi/4
    \end{array}
    \right)\left(
    \begin{array}{c}
      y_1\\
      y_2
    \end{array}
    \right)=\MC\vy
    $$\pause 
    (\ref{li2-1})的矩阵形式为
    $$
    \vx^T\MA\vx=(x_1,x_2)\left(
    \begin{array}{rr}
      5&-3\\
      -3&5
    \end{array}
    \right)\left(
    \begin{array}{c}
      x_1\\
      x_2
    \end{array}
    \right)=4
    $$

  
\end{frame}


\begin{frame}
  
    $$
    \begin{array}{rl}
      \vx^T\MA\vx& =\vy^T\MC^T\MA\MC\vy\\[0.1in]
      & =(y_1,y_2)\left(
      \begin{array}{rr}
        \frac{\sqrt{2}}2&\frac{\sqrt{2}}2\\[0.2cm]
        -\frac{\sqrt{2}}2&\frac{\sqrt{2}}2
      \end{array}
      \right)\left(
      \begin{array}{rr}
        5&-3\\
        -3&5
      \end{array}
      \right)\left(
      \begin{array}{rr}
        \frac{\sqrt{2}}2&-\frac{\sqrt{2}}2\\[0.2cm]
        \frac{\sqrt{2}}2&\frac{\sqrt{2}}2
      \end{array}
      \right)\left(
      \begin{array}{c}
        y_1\\
        y_2
      \end{array}
      \right)\\[0.2in]
      & =(y_1,y_2)\left(
      \begin{array}{rr}
        2&0\\
        0&8
      \end{array}
      \right)\left(
      \begin{array}{c}
        y_1\\
        y_2
      \end{array}
      \right)  \\[0.2in]
      &= 2y_1^2+8y_2^2=4.      
    \end{array}
    $$ \pause
    此时,方程(\ref{li2-1})化成了在基$\{\etabd_1,\etabd_2\}$的坐标系下的标准方程,其图形是一个椭圆。

  
\end{frame}


\begin{frame}
  
    把一般的二次型$f(x_1,x_2,\cd,x_n)$化为$y_1,y_2,\cd,y_n$的纯平方项之代数和的基本方法是做坐标变换
    $$
    \vx=\MC\vy \mbox{~~~~~($\MC$为可逆矩阵)}
    $$
    使得
    $$
    \vx^T\MA\vx=\vy^T\MC^T\MA\MC\vy=d_1y_1^2+\cd+d_ny_n^2.
    $$
    \pause \vspace{0.1in}

    从矩阵的角度来说,就是对于一个实对称矩阵$\MA$,寻找一个可逆矩阵,使得$\MC^T\MA\MC$称为对角形。
  
\end{frame}


\begin{frame}
  
    \begin{dingyi}[矩阵的合同]
      对于两个矩阵$\MA$和$\MB$,若存在可逆矩阵$\MC$,使得
      $$
      \MC^T\MA\MC=\MB,
      $$
      就称$\MA$合同于$\MB$,记作$\MA\simeq\MB$。
    \end{dingyi}
  
\end{frame}


