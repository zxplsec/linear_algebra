\section{第二章~~矩阵及其运算}

\subsection{往年试题}
\begin{frame}
\begin{footnotesize}
\begin{exampleblock}{2005-2006第一学期}
已知$\A$为$n(n\ge 2)$矩阵,且$\A$非奇异,求$(\A^*)^*$。
\end{exampleblock}
\pause\jiename
由$\A^{-1}=\A^*/|\A|$,可知$\A^*=|\A|\A^{-1}$,从而$|\A^*|=|\A|^{n-1}$。
而
$$
(\A^*)^*=|\A^*|(\A^*)^{-1}=|\A|^{n-1}(|\A|\A^{-1})^{-1}=|\A|^{n-2}\A.
$$
\end{footnotesize}
\end{frame}

\begin{frame}
\begin{footnotesize}
\begin{exampleblock}{2005-2006第一学期, 2009-2010第一学期, 2010-2011第一学期, 2011-2012第一学期}
设$\A=\left(
\begin{array}{rrr}
1&0&1\\
0&2&0\\
1&6&a
\end{array}
\right)$,且$\rr(\A)=2$,$\X$满足$\A\X+\II=\A^2+\X$,求$a$和$\X$。
\end{exampleblock}
\pause\jiename
因$\A \rightarrow \left(
\begin{array}{ccc}
1&0&1\\
0&2&0\\
0&0&a-1
\end{array}
\right)$,
由$\rr(\A)=2$知$a=1$,故$\A=\left(
\begin{array}{rrr}
1&0&1\\
0&2&0\\
1&6&1
\end{array}
\right)$.
\vspace{0.1in}


$$\blue{
\A\X+\II=\A^2+\X ~~\Rightarrow~~
(\A-\II)\X=\A^2-\II=(\A-\II)(\A+\II)
}
$$
因$
\A-\II=\left(
\begin{array}{rrr}
0&0&1\\
0&1&0\\
1&6&0
\end{array}
\right)
$可逆,故
$$
\X=\A+\II=\left(
\begin{array}{rrr}
2&0&1\\
0&3&0\\
1&6&2
\end{array}
\right)
$$
\end{footnotesize}
\end{frame}

\begin{frame}
\begin{footnotesize}
\begin{exampleblock}{2005-2006第一学期}
设$\A$为$n$阶实矩阵,
\begin{itemize}
\item[(1)] 当$n$为奇数且$\A\A^T=\II$及$|\A|=1$时,证明$|\II-\A|=0$;
\item[(2)]  当$m$为任意给定正整数且$(\A+\II)^m=\zero$,证明$\A$可逆。
\end{itemize}
\end{exampleblock}
\pause\proofname
\begin{itemize}
\item[(1)]
由$\A\A^T=\II$知$A^T=\A^{-1}$。
又$\II-\A=(\A^{-1}-\II)\A=(\A^T-\II)\A$,故
$$
|\II-\A|=|\A^T-\II||\A|=|\A-\II|=(-1)^n=|\II-\A|
$$
而$n$为奇数,于是
$
|\II-\A|=-|\II-\A|
$,即$|\II-\A|=0$。\\[0.1in] \pause 
\item[(2)]
由$(\A+\II)^m=\zero$,即$\A^m+C_m^{m-1}\A^{m-1}+\cd++C_m^{1}\A+\II=\zero$可知,
$$
\A(\A^{m-1}+C_m^{m-1}\A^{m-2}+\cd+C_m^{1}\II)=-\II
$$
故
$$
\A^{-1}=-(\A^{m-1}+C_m^{m-1}\A^{m-2}+\cd+C_m^{1}\II).
$$
\end{itemize}
\end{footnotesize}
\end{frame}


\begin{frame}
\begin{footnotesize}
\begin{exampleblock}{2005-2006第二学期}
设$\A=\left(
\begin{array}{rrr}
1&1&-1\\
0&1&1\\
0&0&-1
\end{array}
\right)$,且$\A^2-\A\B=\II$,
\begin{itemize}
\item[(1)] 求$\B$;
\item[(2)]  令$\C=4\A^2-\B^2-2\B\A+2\A\B$,计算$\C^*$。
\end{itemize}
\end{exampleblock}
\pause\jiename
\begin{itemize}
\item[(1)]
由$\A^2-\A\B=\II$知$\A\B=\A^2-\II$。利用如下过程可求得$\B$:
$$\boxed{\blue{
(\A, ~~\A^2-\II) ~~\xlongrightarrow[]{\mbox{初等行变换}} (\II, ~~\B)
}
}
$$
易求得$\A^2-\II=\left(
\begin{array}{rrr}
0&2&1\\
0&0&0\\
0&0&0
\end{array}
\right)$,故
$$\left(
\begin{array}{rrrrrr}
1&1&-1&0&2&1\\
0&1&1&0&0&0\\
0&0&-1&0&0&0
\end{array}
\right) \longrightarrow
\left(
\begin{array}{rrrrrr}
1&0&0&0&2&1\\
0&1&0&0&0&0\\
0&0&1&0&0&0
\end{array}
\right)
$$
\end{itemize}•
\end{footnotesize}
\end{frame}

\begin{frame}
\begin{footnotesize}
\begin{itemize}
\item[(2)]
易求得
$$
\C=(2\A-\B)(2\A+\B)=\left(
\begin{array}{rrr}
4&8&4\\
0&4&0\\
0&0&4
\end{array}
\right) $$
故
$$
\C^{-1}=\frac14\left(
\begin{array}{rrr}
1&-2&-1\\
0&1&0\\
0&0&1
\end{array}
\right), \quad |\C|=64.
$$
故
$$
\C^*=|\C|\C^{-1}=16\left(
\begin{array}{rrr}
1&-2&-1\\
0&1&0\\
0&0&1
\end{array}
\right)
$$
\end{itemize}•
\end{footnotesize}
\end{frame}




\begin{frame}
\begin{footnotesize}
\begin{exampleblock}{2006-2007第一学期}
设三阶方阵$\A=(a_{ij})$,
\begin{itemize}
\item[(1)] 若$\A^T=\A$且$\A^2=\zero$,证明$\A=\zero$;并由反例说明一般情况下$\A^2=\zero$得不出$\A=\zero$;
\item[(2)]  若$\A$可逆,将其第二行的$2$倍加到第三行的矩阵为$\B$,问$\B\A^{-1}-\A\B^{-1}$是否可逆?
\end{itemize}
\end{exampleblock}
\pause\jiename
\begin{itemize}
\item[(1)]
由条件知$\A^T\A=\zero$,而$(\A^T\A)_{ii}=a_{1i}^2+a_{2i}^2+\cd+a_{ni}^2=0$,故$a_{ij}=0$,即$\A=\zero$。
反例:$\A=\left(\begin{array}{cc}
0&1\\0&0
\end{array}\right)\ne \zero$,但$\A^2=\left(\begin{array}{cc}
0&0\\0&0
\end{array}\right)$. \pause
\item[(2)]
由题意知$\B=\PP\A$,其中$\PP=\left(\begin{array}{ccc}
1&&\\0&1&\\0&3&1
\end{array}\right)$. 故
$$
\begin{array}{rl}
\B\A^{-1}-\A\B^{-1}&=\PP\A\A^{-1}-\A\A^{-1}\PP^{-1}=\PP-\PP^{-1} \\[0.1in]
&=\left(\begin{array}{ccc}
1&&\\0&1&\\0&3&1
\end{array}\right)-\left(\begin{array}{ccc}
1&&\\0&1&\\0&-3&1
\end{array}\right)=\left(\begin{array}{ccc}
0&&\\0&0&\\0&6&0
\end{array}\right),
\end{array}
$$ 显然不可逆。
\end{itemize}•
\end{footnotesize}
\end{frame}

\begin{frame}
\begin{footnotesize}
\begin{exampleblock}{2006-2007第二学期}
设$\A=\left(
\begin{array}{rrr}
1&2&1\\
2&1&1\\
1&1&2
\end{array}
\right)$, $\B=\left(
\begin{array}{rrr}
-1&1&0\\
1&3&1\\
-1&0&1
\end{array}
\right)$,
\begin{itemize}
\item[(1)] 求$(\A+\B)^2-(\A^2+2\A\B+\B^2)$;
\item[(2)]  求$\A^{-1}$。
\end{itemize}
\end{exampleblock}
\pause\jiename
\begin{itemize}
\item[(1)]  $(\A+\B)^2-(\A^2+2\A\B+\B^2)=\B\A-\A\B=\left(
\begin{array}{rrr}
1&-1&0\\
8&6&6\\
0&-1&1
\end{array}
\right)-\left(
\begin{array}{rrr}
0&7&3\\
-2&5&2\\
-2&4&3
\end{array}
\right)=\left(
\begin{array}{rrr}
1&-8&-3\\
10&1&4\\
2&-5&-2
\end{array}
\right)$ \pause 
\item[(2)]
$$
\begin{array}{rl}
(\A,~\II)&=\left(
\begin{array}{rrrrrr}
1&2&1&1&0&0\\
2&1&1&0&1&0\\
1&1&2&0&0&1
\end{array}
\right)\rightarrow\left(
\begin{array}{rrrrrr}
1&2&1&1&0&0\\
0&-3&-1&-2&1&0\\
0&-1&1&-1&0&1
\end{array}
\right)\\
&\rightarrow\left(
\begin{array}{rrrrrr}
1&2&1&1&0&0\\
0&1&-1&1&0&-1\\
0&0&-4&1&1&-3\\
\end{array}
\right)\rightarrow\left(
\begin{array}{rrrrrr}
1&0&0&-\frac14&\frac34&-\frac14\\
0&1&0&\frac34&-\frac14&-\frac14\\
0&0&1&-\frac14&-\frac14&\frac34\\
\end{array}
\right)
\end{array}
$$
\end{itemize}
\end{footnotesize}
\end{frame}

\begin{frame}
\begin{footnotesize}
\begin{exampleblock}{2006-2007第二学期}
设$\A=\left(
\begin{array}{rrr}
1&-3&2\\
-2&1&-1\\
1&2&-1
\end{array}
\right)$, $\B=\left(
\begin{array}{rrr}
2&5&4\\
4&-2&2\\
1&4&1
\end{array}
\right)$,
\begin{itemize}
\item[(1)] 求$4\A^2-\B^2-2\B\A+2\A\B$;
\item[(2)]  求$|\A^*|$。
\end{itemize}
\end{exampleblock}
\pause\jiename
\begin{itemize}
\item[(1)] $4\A^2-\B^2-2\B\A+2\A\B=(2\A-\B)(2\A+\B)=\left(
\begin{array}{rrr}
0&0&0\\
-44&-24&-60\\
-5&-25&11
\end{array}
\right)$ \pause 
\item[(2)]  因$|\A^*|=|\A|^{2}$,而
$
|\A|=\left|
\begin{array}{rrr}
1&-3&2\\
0&-5&3\\
0&5&-3
\end{array}
\right|=0
$,
故$|\A^*|=0$。
\end{itemize}
\end{footnotesize}
\end{frame}

\begin{frame}
\begin{footnotesize}
\begin{exampleblock}{2007-2008第一学期}
证明
\begin{itemize}
\item[(1)] 设$\A$为$n$阶方阵,证明:若$|\A|=0$,则$|\A^*|=0$;
\item[(2)]  设$\A,\B$为$n$阶方阵,且满足$\A^2=\A,\B^2=\B,\rr(\A+\B-\II)=n$,证明:$\rr(\A)=\rr(\B)$。
\end{itemize}
\end{exampleblock}
\pause\proofname
\begin{itemize}
\item[(1)] $|\A^*|=|\A|^{n-1}=0$;\pause 
\item[(2)]  由$\rr(\A+\B-\II)=n$知$\A+\B-\II$可逆。故
$$
\rr(\A)=\rr(\A(\A+\B-\II))=\rr(\A\B)
$$
$$
\rr(\B)=\rr((\A+\B-\II)\B)=\rr(\A\B)
$$
于是$\rr(\A)=\rr(\B)$。
\end{itemize}
\end{footnotesize}
\end{frame}


\begin{frame}
\begin{scriptsize}
\begin{exampleblock}{2007-2008第二学期}
设$\A=\left(
\begin{array}{rrr}
2&-1&1\\
1&2&0\\
2&1&2
\end{array}
\right)$, $\B=\left(
\begin{array}{rrr}
0&2&-3\\
2&-1&4\\
0&-1&4
\end{array}
\right)$,$\A\C-\II=\B+\C$,求$\C$。
\end{exampleblock}
\pause\jiename
由题意可知$(\A-\II)\C=\B+\II$,
$$
\begin{array}{rl}
&(\A-\II,\B+\II)=\left(
\begin{array}{rrrrrr}
1&-1&1 &1&2&-3\\
1&1&0 &2&0&4\\
2&1&1&0&-1&5
\end{array}
\right)\rightarrow\left(
\begin{array}{rrrrrr}
1&-1&1 &1&2&-3\\
0&2&-1 &1&-2&7\\
0&3&-1&-2&-5&11
\end{array}
\right)\\[0.2in]
&\rightarrow\left(
\begin{array}{rrrrrr}
1&-1&1 &1&2&-3\\
0&2&-1 &1&-2&7\\
0&0&1&-7&-4&1
\end{array}
\right) \rightarrow\left(
\begin{array}{rrrrrr}
1&-1&0 &8&6&-4\\
0&1&0 &-3&-3&4\\
0&0&1&-7&-4&1
\end{array}
\right)\\[0.2in]
& \rightarrow\left(
\begin{array}{rrrrrr}
1&-1&0 &8&6&-4\\
0&1&0 &-3&-3&4\\
0&0&1&-7&-4&1
\end{array}
\right) \rightarrow\left(
\begin{array}{rrrrrr}
1&0&0 &5&3&0\\
0&1&0 &-3&-3&4\\
0&0&1&-7&-4&1
\end{array}
\right)
\end{array}
$$
\end{scriptsize}
\end{frame}


\begin{frame}
\begin{footnotesize}
\begin{exampleblock}{2008-2009第一学期}
设$\A,\B$为三阶方阵,满足$\A\B+\II=\A^2+\B$,且$\A=\left(
\begin{array}{rrr}
1&0&1\\
0&2&0\\
1&0&1
\end{array}
\right)$, 求$\B$及$\B^*$。
\end{exampleblock}
\pause\jiename
依题意可知
$$
(\A-\II)\B=\A^2-\II=(\A-\II)(\A+\II),
$$
而$\A-\II=\left(
\begin{array}{rrr}
0&0&1\\
0&2&0\\
1&0&0
\end{array}
\right)$非奇异,故$$\B=\A+\II=\left(
\begin{array}{rrr}
2&0&1\\
0&3&0\\
1&0&2
\end{array}
\right)$$
\end{footnotesize}
\end{frame}






\begin{frame}
\begin{footnotesize}
\begin{exampleblock}{2009-2010第一学期}
计算下列各题:
\begin{itemize}
\item[(1)] 已知$\A=\left(
\begin{array}{rrr}
a&a&a\\
b&b&b\\
c&c&c
\end{array}
\right)$,求$\A^{2010}$;
\item[(2)] 设$n(n\ge2)$阶方阵$\A$非奇异,求$(\A^*)^*$。
\end{itemize}
\end{exampleblock}
\pause\jiename
\begin{itemize}
\item[(1)]由$|\A-\lambda\II|
%=\left|
%\begin{array}{rrr}
%a-\lambda&a&a\\
%b&b-\lambda&b\\
%c&c&c-\lambda
%\end{array}
%\right|
%=(a+b+c-\lambda)\left|
%\begin{array}{rrr}
%1&1&1\\
%b&b-\lambda&b\\
%c&c&c-\lambda
%\end{array}
%\right|=(a+b+c-\lambda)\left|
%\begin{array}{rrr}
%1&0&0\\
%b&-\lambda&\\
%c&0&-\lambda
%\end{array}
%\right|
=(a+b+c-\lambda)\lambda^2=0$知,
特征值为$\lambda_{1,2}=0$与$\lambda_3=a+b+c$。
\begin{itemize}
\item
$\lambda_{1,2}=0$,方程为$x+y+z=0$,基础解系为$\boxed{\xx_1=(-1,1,0)^T,\xx_2=(-1,0,1)^T}$。\\[0.1in]
\item
$\lambda_3=a+b+c$,方程为$\left(
\begin{array}{rrr}
-b-c&a&a\\
b&-a-c&b\\
c&c&-a-b
\end{array}
\right)\xx=\zero$,基础解系为$\boxed{\xx_3=(a,b,c)^T}$。
\end{itemize}
\end{itemize}
\end{footnotesize}
\end{frame}

\begin{frame}
\begin{footnotesize}
故
$$
\A(\xx_1,\xx_2,\xx_3)=(\xx_1,\xx_2,\xx_3)\left(
\begin{array}{ccc}
0&&\\
&0&\\
&&a+b+c
\end{array}
\right)
$$
从而
$$
\A^{2010}=\PP\left(
\begin{array}{ccc}
0&&\\
&0&\\
&&a+b+c
\end{array}
\right)^{2010}\PP^{-1}
$$
其中
$$
\PP=\left(
\begin{array}{ccc}
-1&-1&a\\
1&0&b\\
0&1&c
\end{array}
\right), \quad
\PP^{-1}=\frac1{a+b+c}\left(
\begin{array}{ccc}
-b&a+c&-b\\
-c&-c&a+b\\
1&1&1
\end{array}
\right)
$$
故
$$
\A^{2010}= (a+b+c)^{2009}
\left(
\begin{array}{ccc}
a&a&a\\
b&b&b\\
c&c&c
\end{array}
\right)
$$
\end{footnotesize}
\end{frame}

\begin{frame}
\begin{footnotesize}
\begin{exampleblock}{2009-2010第一学期,2011-2012第二学期}
设三阶方阵$\A$满足$\A\X=\A+2\X$,且$\A=\left(
\begin{array}{rrr}
3&0&1\\
1&1&0\\
0&1&4
\end{array}
\right)$,求$\X$。
\end{exampleblock}
\pause\jiename
依题意可知$(\A-2\II)\X=\A$,解此矩阵方程即可求得$\X$。\pause 
$$
\begin{array}{rl}
&(\A-2\II, \A)=\left(
\begin{array}{rrrrrr}
  1&0&1&3&0&1\\
  1&-1&0&1&1&0\\
  0&1&2&0&1&4
\end{array}
\right)\rightarrow\left(
\begin{array}{rrrrrr}
  1&0&1&3&0&1\\
  0&-1&-1&-2&1&-1\\
  0&1&2&0&1&4
\end{array}
\right)\\[0.2in]
&\rightarrow\left(
\begin{array}{rrrrrr}
  1&0&1&3&0&1\\
  0&-1&-1&-2&1&-1\\
  0&0&1&-2&2&3
\end{array}
\right)\rightarrow\left(
\begin{array}{rrrrrr}
  1&0&1&3&0&1\\
  0&1&0&4&-3&-2\\
  0&0&1&-2&2&3
\end{array}
\right)\\[0.2in]
&\rightarrow\left(
\begin{array}{rrrrrr}
  1&0&0&5&-2&-2\\
  0&1&0&4&-3&-2\\
  0&0&1&-2&2&3
\end{array}
\right)
\end{array}
$$
\end{footnotesize}
\end{frame}


\begin{frame}
\begin{scriptsize}
\begin{exampleblock}{2009-2010第二学期}
已知矩阵方程满足$(2\II-\C^{-1}\B)\A^T=\C^{-1}$,求$\A$,其中
$$\B=\left(
\begin{array}{rrrr}
1&2&-3&-2\\
0&1&2&-3\\
0&0&1&2\\
0&0&0&1
\end{array}
\right),\quad
\C=\left(
\begin{array}{rrrr}
1&2&0&1\\
0&1&2&0\\
0&0&1&2\\
0&0&0&1
\end{array}
\right)$$
\end{exampleblock}
\pause\jiename
依题意知$\A^T=(2\II-\C^{-1}\B)^{-1}\C^{-1}=(\C(2\II-\C^{-1}\B))^{-1}=(2\C-\B)^{-1}$
$$
\begin{array}{l}
  (2\C-\B,~\II)=\left(
  \begin{array}{rrrrrrrr}
    1&2&3&4&1&0&0&0\\
    0&1&2&3&0&1&0&0\\
    0&0&1&2&0&0&1&0\\
    0&0&0&1&0&0&0&1
  \end{array}
  \right)\rightarrow\left(
  \begin{array}{rrrrrrrr}
    1&2&3&0&1&0&0&-4\\
    0&1&2&0&0&1&0&-3\\
    0&0&1&0&0&0&1&-2\\
    0&0&0&1&0&0&0&1
  \end{array}
  \right)\\[0.3in]
\rightarrow\left(
  \begin{array}{rrrrrrrr}
    1&2&0&0&1&0&-3&2\\
    0&1&0&0&0&1&-2&1\\
    0&0&1&0&0&0&1&-2\\
    0&0&0&1&0&0&0&1
  \end{array}
  \right)
\rightarrow\left(
  \begin{array}{rrrrrrrr}
    1&0&0&0&1&-2&1&0\\
    0&1&0&0&0&1&-2&1\\
    0&0&1&0&0&0&1&-2\\
    0&0&0&1&0&0&0&1   
  \end{array}
  \right)\\[0.3in]
  \red{\Longrightarrow \A=\left(
  \begin{array}{rrrr}
    1&0&0&0\\
    -2&1&0&0\\
    1&-2&1&0\\
    0&1&-2&1   
  \end{array}
  \right)
}
\end{array}
$$

\end{scriptsize}
\end{frame}

\begin{frame}
\begin{footnotesize}
\begin{exampleblock}{2012-2013第二学期}
已知$\A$为三阶矩阵,$\B=\left(
\begin{array}{rrr}
1&-2&0\\
1&2&0\\
0&0&2
\end{array}
\right)$,且满足$2\A^{-1}\B=\B-4\II$,$\II$为三阶单位矩阵,求矩阵$\A$。
\end{exampleblock}
\pause\jiename
依题意$\A(\B-4\II)=2\B$,可用$\boxed{\left(
  \begin{array}{c}
    \B-4\II\\
    2\B
  \end{array}
\right)\xlongrightarrow[]{\mbox{初等列变换}} \left(
  \begin{array}{c}
    \II\\
    \red{\A}
  \end{array}
\right)}$求$\A$。
$$
\left(
  \begin{array}{c}
    \B-4\II\\
    2\B
  \end{array}
\right)=\left(
\begin{array}{rrr}
-3&-2&0\\
1&-2&0\\
0&0&-2\\
2&-4&0\\
2&4&0\\
0&0&4
\end{array}
\right)\rightarrow
\left(
\begin{array}{rrr}
3  &1&0\\
-1 &1&0\\
0  &0 &1\\
-2 &2 &0\\
-2 &-2&0\\
0  &0 &-2
\end{array}
\right)\rightarrow
\left(
\begin{array}{rrr}
1 &0  &0\\
0 &1 &0\\
0 &0  &1\\
0 &2 &0\\
-1&-1  &0\\
0 &0  &-2
\end{array}
\right)
$$
\end{footnotesize}
\end{frame}


\begin{frame}
\begin{footnotesize}
\begin{exampleblock}{2012-2013第二学期}
设矩阵$\A=\left(
\begin{array}{rrr}
-1&0&0\\
 0&1&0\\
 0&1&1
\end{array}
\right)$,矩阵$\B$满足$\A^*\B\A=2\B\A-9\II$,求$\B$。
\end{exampleblock}
\pause\jiename
易知$|\A|=-1$,即$\A$可逆,由$\A\A^*=|\A|\II=-\II$可得
$$
\begin{array}{l}
\A^*\B\A=2\B\A-9\II ~~\Rightarrow~~
\A\A^*\B\A=\A(2\B\A-9\II) \\[0.1in]
\Rightarrow~~
-\B\A=2\A\B\A-9\A ~~\Rightarrow~~
-\B=2\A\B-9\II   ~~\Rightarrow~~
(2\A+\II)\B=9\II
\end{array}
$$
$$
\begin{array}{l}
(2\A+\II, 9\II)=\left(
\begin{array}{rrrrrr}
  -1   &  0  &   0 &  9&0&0\\
  0    & 3   &  0  &  0&9&0\\
  0    & 2   &  3  &  0&0&9
\end{array}
\right)  \rightarrow
\left(
\begin{array}{rrrrrr}
  1   &  0  &   0 & -9&0&0\\
  0    & 1   &  0  &  0&3&0\\
  0    & 0   &  1  &  0&-2&3
\end{array}
\right)  
\end{array}
$$
\end{footnotesize}
\end{frame}

\begin{frame}
\begin{scriptsize}
\begin{exampleblock}{2012-2013第二学期}
设矩阵$\A=\left(
\begin{array}{rrrr}
1&-1&-1&-1\\
-1&1&-1&-1\\
-1&-1&1&-1\\
-1&-1&-1&1
\end{array}
\right)$,
\begin{itemize}
\item[(1)] 求$\A^n$;
\item[(2)] 设$\A^2+\A\B-\A=\II$,求$|\B|$。
\end{itemize}
\end{exampleblock}
\pause\jiename
\begin{itemize}
\item[(1)] 求矩阵的特征值与特征向量。
  $$
  \begin{array}{l}
    |\A-\lambda\II|=\left|
    \begin{array}{rrrr}
      1-\lambda&-1&-1&-1\\
      -1&1-\lambda&-1&-1\\
      -1&-1&1-\lambda&-1\\
      -1&-1&-1&1-\lambda
    \end{array}
    \right|
    =(-2-\lambda)\left|
    \begin{array}{rrrr}
      1&-1&-1&-1\\
      1&1-\lambda&-1&-1\\
      1&-1&1-\lambda&-1\\
      1&-1&-1&1-\lambda
    \end{array}
    \right|\\[0.3in]=(-2-\lambda)\left|
    \begin{array}{rrrr}
      1&-1&-1&-1\\
      0&2-\lambda&0&0\\
      0&0& 2-\lambda&0\\
      0&0& 0 &2-\lambda
    \end{array}
    \right|   = (\lambda+2)(\lambda-2)^3
  \end{array}
$$
\end{itemize}
\end{scriptsize}
\end{frame}


\begin{frame}
\begin{scriptsize}
  \begin{itemize}
  \item 当$\lambda_{1,2,3}=2$时,$(\A-\lambda\II)\xx=\zero$为
    $$
    x_1+x_2+x_3+x_4=0
    $$
    基础解系为
    $$
    \xx_1=(-1,1,0,0)^T,~~
    \xx_2=(-1,0,1,0)^T,~~
    \xx_3=(-1,0,0,1)^T.
    $$
    故对应于$\lambda_{1,2,3}=2$的特征向量为$k_1\xx_1+k_2\xx_2+k_3\xx_3, (k_1,k_2,k_3\mbox{不全为零})$;
  \item 当$\lambda_{4}=-2$时,$(\A-\lambda\II)\xx=\zero$为
    $$
    \begin{array}{l}
          \left(
    \begin{array}{rrrr}
      3&-1&-1&-1\\
      -1&3&-1&-1\\
      -1&-1&3&-1\\
      -1&-1&-1&3
    \end{array}
    \right) \rightarrow \left(
    \begin{array}{rrrr}
      1&-3&1&1\\
      1&1&-3&1\\
      1&1&1&-3\\
      3&-1&-1&-1
    \end{array}
    \right) \rightarrow \left(
    \begin{array}{rrrr}
      1&-3&1&1\\
      0&4&-4&0\\
      0&4&0&-4\\
      0&8&-4&-4
    \end{array}
    \right) \\[0.2in]
    \rightarrow \left(
    \begin{array}{rrrr}
      1&-3&1&1\\
      0&1&-1&0\\
      0&0&1&-1\\
      0&0&0&0
    \end{array}
    \right)\rightarrow \left(
    \begin{array}{rrrr}
      1&0&0&-1\\
      0&1&0&-1\\
      0&0&1&-1\\
      0&0&0&0
    \end{array}
    \right)
    \end{array}
    $$
    对应方程为
    $$
    \left\{
    \begin{array}{l}
      x_1=x_4\\
      x_2=x_4\\
      x_3=x_4\\
      x_4=x_4
    \end{array}
    \right.
    $$
    基础解系为$$
    \xx_3=(1,1,1,1)^T
    $$
    
  \end{itemize}
\end{scriptsize}
\end{frame}

\begin{frame}
\begin{footnotesize}
  取
  $$
  \PP=(\xx_1,\xx_2,\xx_3,\xx_4)=\left(
  \begin{array}{rrrr}
    -1 & -1& -1&  1 \\
    1  &  0&  0&  1 \\
    0  &  1&  0&  1  \\
    0  &  0&  1&  1 
  \end{array}
  \right), ~~~\PP^{-1}=\frac14 \left(
  \begin{array}{rrrr}
    -1 &   3& -1&  -1 \\
    -1 &  -1&  3&  -1 \\
    -1 &  -1& -1&   3  \\
     1 &   1&  1&   1 
  \end{array}
  \right)
  $$
  则 $\A\PP=\PP\Lambdabd$,即
  $$
  \A=\PP\Lambdabd\PP^{-1},
  $$
  从而$$
  \A^{n}=\PP\Lambdabd^n\PP^{-1}
  =2^n \PP\left(
  \begin{array}{cccc}
    1&&&\\
    &1&&\\
    &&1&\\
    &&&(-1)^n
  \end{array}
  \right)\PP^{-1}
  $$
  当$n$为偶数时,$\A^n=2^n \II$
  当$n$为奇数时,$\A^n=2^n\PP\left(
  \begin{array}{cccc}
    1&&&\\
    &1&&\\
    &&1&\\
    &&&-1
  \end{array}
  \right)\PP^{-1} $
  而
  $
  \A=2\PP\left(
  \begin{array}{cccc}
    1&&&\\
    &1&&\\
    &&1&\\
    &&&-1
  \end{array}
  \right)\PP^{-1} 
  $
  故
  $
  \A^n=2^{n-1}\A
  $。
\end{footnotesize}
\end{frame}

\begin{frame}
\begin{footnotesize}
  \begin{itemize}
  \item[(2)]
    依题意,
    $$
    \A\B=\II-\A+\A^2=\II-\A+4\II=-\A+5\II
    $$
    故
    $$
    |\A||\B|=|-\A+5\II|
    $$
    即
    $$
    -16 |\B| = |\A-5\II| =\left|
    \begin{array}{rrrr}
      -4&-1&-1&-1\\
      -1&-4&-1&-1\\
      -1&-1&-4&-1\\
      -1&-1&-1&-4
    \end{array}
    \right| = 189
    $$
    故
    $$
    |\B|=-\frac{189}{16}.
    $$
  \end{itemize}
\end{footnotesize}
\end{frame}

%% \begin{frame}
%% \begin{footnotesize}

%% \end{footnotesize}
%% \end{frame}

%% \begin{frame}
%% \begin{footnotesize}

%% \end{footnotesize}
%% \end{frame}

%% \begin{frame}
%% \begin{footnotesize}

%% \end{footnotesize}
%% \end{frame}

%% \begin{frame}
%% \begin{footnotesize}

%% \end{footnotesize}
%% \end{frame}

%% \begin{frame}
%% \begin{footnotesize}

%% \end{footnotesize}
%% \end{frame}

%% \begin{frame}
%% \begin{footnotesize}

%% \end{footnotesize}
%% \end{frame}


%% \begin{frame}
%% \begin{footnotesize}

%% \end{footnotesize}
%% \end{frame}

%% \begin{frame}
%% \begin{footnotesize}

%% \end{footnotesize}
%% \end{frame}

%% \begin{frame}
%% \begin{footnotesize}

%% \end{footnotesize}
%% \end{frame}

%% \begin{frame}
%% \begin{footnotesize}

%% \end{footnotesize}
%% \end{frame}

%% \begin{frame}
%% \begin{footnotesize}

%% \end{footnotesize}
%% \end{frame}

%% \begin{frame}
%% \begin{footnotesize}

%% \end{footnotesize}
%% \end{frame}

%% \begin{frame}
%% \begin{footnotesize}

%% \end{footnotesize}
%% \end{frame}

%% \begin{frame}
%% \begin{footnotesize}

%% \end{footnotesize}
%% \end{frame}

%% \begin{frame}
%% \begin{footnotesize}

%% \end{footnotesize}
%% \end{frame}


%% \begin{frame}
%% \begin{footnotesize}

%% \end{footnotesize}
%% \end{frame}

%% \begin{frame}
%% \begin{footnotesize}

%% \end{footnotesize}
%% \end{frame}

%% \begin{frame}
%% \begin{footnotesize}

%% \end{footnotesize}
%% \end{frame}

%% \begin{frame}
%% \begin{footnotesize}

%% \end{footnotesize}
%% \end{frame}

%% \begin{frame}
%% \begin{footnotesize}

%% \end{footnotesize}
%% \end{frame}

%% \begin{frame}
%% \begin{footnotesize}

%% \end{footnotesize}
%% \end{frame}

%% \begin{frame}
%% \begin{footnotesize}

%% \end{footnotesize}
%% \end{frame}


%% \begin{frame}
%% \begin{footnotesize}

%% \end{footnotesize}
%% \end{frame}

%% \begin{frame}
%% \begin{footnotesize}

%% \end{footnotesize}
%% \end{frame}


%% \begin{frame}
%% \begin{footnotesize}

%% \end{footnotesize}
%% \end{frame}

%% \begin{frame}
%% \begin{footnotesize}

%% \end{footnotesize}
%% \end{frame}

%% \begin{frame}
%% \begin{footnotesize}

%% \end{footnotesize}
%% \end{frame}

%% \begin{frame}
%% \begin{footnotesize}

%% \end{footnotesize}
%% \end{frame}

%% \begin{frame}
%% \begin{footnotesize}

%% \end{footnotesize}
%% \end{frame}

%% \begin{frame}
%% \begin{footnotesize}

%% \end{footnotesize}
%% \end{frame}

%% \begin{frame}
%% \begin{footnotesize}

%% \end{footnotesize}
%% \end{frame}

%% \begin{frame}
%% \begin{footnotesize}

%% \end{footnotesize}
%% \end{frame}

%% \begin{frame}
%% \begin{footnotesize}

%% \end{footnotesize}
%% \end{frame}


%% \begin{frame}
%% \begin{footnotesize}

%% \end{footnotesize}
%% \end{frame}

%% \begin{frame}
%% \begin{footnotesize}

%% \end{footnotesize}
%% \end{frame}

%% \begin{frame}
%% \begin{footnotesize}

%% \end{footnotesize}
%% \end{frame}

%% \begin{frame}
%% \begin{footnotesize}

%% \end{footnotesize}
%% \end{frame}

%% \begin{frame}
%% \begin{footnotesize}

%% \end{footnotesize}
%% \end{frame}

%% \begin{frame}
%% \begin{footnotesize}

%% \end{footnotesize}
%% \end{frame}

%% \begin{frame}
%% \begin{footnotesize}

%% \end{footnotesize}
%% \end{frame}
