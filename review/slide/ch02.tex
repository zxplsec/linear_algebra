\section{矩阵及其运算}

\subsection{往年试题}
\begin{frame}

\begin{li}[2005-2006第一学期]
已知$\MA$为$n(n\ge 2)$矩阵,且$\MA$非奇异,求$(\MA^*)^*$。
\end{li}
\pause
\begin{jie}
由$\MA^{-1}=\MA^*/|\MA|$,可知$\MA^*=|\MA|\MA^{-1}$,从而$|\MA^*|=|\MA|^{n-1}$。
而
$$
(\MA^*)^*=|\MA^*|(\MA^*)^{-1}=|\MA|^{n-1}(|\MA|\MA^{-1})^{-1}=|\MA|^{n-2}\MA.
$$
\end{jie}
\end{frame}

\begin{frame}

\begin{li}[2005-2006第一学期, 2009-2010第一学期, 2010-2011第一学期, 2011-2012第一学期]
设$\MA=\left(
\begin{array}{rrr}
1&0&1\\
0&2&0\\
1&6&a
\end{array}
\right)$,且$\rank(\MA)=2$,$\MX$满足$\MA\MX+\MI=\MA^2+\MX$,求$a$和$\MX$。
\end{li}
\pause

\begin{jie}
因$\MA \rightarrow \left(
\begin{array}{ccc}
1&0&1\\
0&2&0\\
0&0&a-1
\end{array}
\right)$,
由$\rank(\MA)=2$知$a=1$,故$\MA=\left(
\begin{array}{rrr}
1&0&1\\
0&2&0\\
1&6&1
\end{array}
\right)$.
\vspace{0.1in}


$$\blue{
\MA\MX+\MI=\MA^2+\MX ~~\Rightarrow~~
(\MA-\MI)\MX=\MA^2-\MI=(\MA-\MI)(\MA+\MI)
}
$$
因$
\MA-\MI=\left(
\begin{array}{rrr}
0&0&1\\
0&1&0\\
1&6&0
\end{array}
\right)
$可逆,故
$$
\MX=\MA+\MI=\left(
\begin{array}{rrr}
2&0&1\\
0&3&0\\
1&6&2
\end{array}
\right)
$$
\end{jie}
\end{frame}

\begin{frame}

\begin{li}[2005-2006第一学期]
设$\MA$为$n$阶实矩阵,
\begin{itemize}
\item[(1)] 当$n$为奇数且$\MA\MA^T=\MI$及$|\MA|=1$时,证明$|\MI-\MA|=0$;
\item[(2)]  当$m$为任意给定正整数且$(\MA+\MI)^m=\M0$,证明$\MA$可逆。
\end{itemize}
\end{li}
\pause
\begin{proof}
\begin{itemize}
\item[(1)]
由$\MA\MA^T=\MI$知$A^T=\MA^{-1}$。
又$\MI-\MA=(\MA^{-1}-\MI)\MA=(\MA^T-\MI)\MA$,故
$$
|\MI-\MA|=|\MA^T-\MI||\MA|=|\MA-\MI|=(-1)^n=|\MI-\MA|
$$
而$n$为奇数,于是
$
|\MI-\MA|=-|\MI-\MA|
$,即$|\MI-\MA|=0$。\\[0.1in] \pause 
\item[(2)]
由$(\MA+\MI)^m=\M0$,即$\MA^m+C_m^{m-1}\MA^{m-1}+\cd++C_m^{1}\MA+\MI=\M0$可知,
$$
\MA(\MA^{m-1}+C_m^{m-1}\MA^{m-2}+\cd+C_m^{1}\MI)=-\MI
$$
故
$$
\MA^{-1}=-(\MA^{m-1}+C_m^{m-1}\MA^{m-2}+\cd+C_m^{1}\MI).
$$
\end{itemize}
\end{proof}
\end{frame}


\begin{frame}

\begin{li}[2005-2006第二学期]
设$\MA=\left(
\begin{array}{rrr}
1&1&-1\\
0&1&1\\
0&0&-1
\end{array}
\right)$,且$\MA^2-\MA\MB=\MI$,
\begin{itemize}
\item[(1)] 求$\MB$;
\item[(2)]  令$\MC=4\MA^2-\MB^2-2\MB\MA+2\MA\MB$,计算$\MC^*$。
\end{itemize}
\end{li}
\pause

\begin{jie}
\begin{itemize}
\item[(1)]
由$\MA^2-\MA\MB=\MI$知$\MA\MB=\MA^2-\MI$。利用如下过程可求得$\MB$:
$$\boxed{\blue{
(\MA, ~~\MA^2-\MI) ~~\xlongrightarrow[]{\mbox{初等行变换}} (\MI, ~~\MB)
}
}
$$
易求得$\MA^2-\MI=\left(
\begin{array}{rrr}
0&2&1\\
0&0&0\\
0&0&0
\end{array}
\right)$,故
$$\left(
\begin{array}{rrrrrr}
1&1&-1&0&2&1\\
0&1&1&0&0&0\\
0&0&-1&0&0&0
\end{array}
\right) \longrightarrow
\left(
\begin{array}{rrrrrr}
1&0&0&0&2&1\\
0&1&0&0&0&0\\
0&0&1&0&0&0
\end{array}
\right)
$$
\end{itemize}
\end{jie}
\end{frame}

\begin{frame}

\begin{itemize}
\item[(2)]
易求得
$$
\MC=(2\MA-\MB)(2\MA+\MB)=\left(
\begin{array}{rrr}
4&8&4\\
0&4&0\\
0&0&4
\end{array}
\right) $$
故
$$
\MC^{-1}=\frac14\left(
\begin{array}{rrr}
1&-2&-1\\
0&1&0\\
0&0&1
\end{array}
\right), \quad |\MC|=64.
$$
故
$$
\MC^*=|\MC|\MC^{-1}=16\left(
\begin{array}{rrr}
1&-2&-1\\
0&1&0\\
0&0&1
\end{array}
\right)
$$
\end{itemize}•

\end{frame}




\begin{frame}

\begin{li}[2006-2007第一学期]
设三阶方阵$\MA=(a_{ij})$,
\begin{itemize}
\item[(1)] 若$\MA^T=\MA$且$\MA^2=\M0$,证明$\MA=\M0$;并由反例说明一般情况下$\MA^2=\M0$得不出$\MA=\M0$;
\item[(2)]  若$\MA$可逆,将其第二行的$2$倍加到第三行的矩阵为$\MB$,问$\MB\MA^{-1}-\MA\MB^{-1}$是否可逆?
\end{itemize}
\end{li}
\pause

\begin{jie}
\begin{itemize}
\item[(1)]
由条件知$\MA^T\MA=\M0$,而$(\MA^T\MA)_{ii}=a_{1i}^2+a_{2i}^2+\cd+a_{ni}^2=0$,故$a_{ij}=0$,即$\MA=\M0$。
反例:$\MA=\left(\begin{array}{cc}
0&1\\0&0
\end{array}\right)\ne \M0$,但$\MA^2=\left(\begin{array}{cc}
0&0\\0&0
\end{array}\right)$. \pause
\item[(2)]
由题意知$\MB=\MP\MA$,其中$\MP=\left(\begin{array}{ccc}
1&&\\0&1&\\0&3&1
\end{array}\right)$. 故
$$
\begin{array}{rl}
\MB\MA^{-1}-\MA\MB^{-1}&=\MP\MA\MA^{-1}-\MA\MA^{-1}\MP^{-1}=\MP-\MP^{-1} \\[0.1in]
&=\left(\begin{array}{ccc}
1&&\\0&1&\\0&3&1
\end{array}\right)-\left(\begin{array}{ccc}
1&&\\0&1&\\0&-3&1
\end{array}\right)=\left(\begin{array}{ccc}
0&&\\0&0&\\0&6&0
\end{array}\right),
\end{array}
$$ 显然不可逆。
\end{itemize}
\end{jie}

\end{frame}

\begin{frame}

\begin{li}[2006-2007第二学期]
设$\MA=\left(
\begin{array}{rrr}
1&2&1\\
2&1&1\\
1&1&2
\end{array}
\right)$, $\MB=\left(
\begin{array}{rrr}
-1&1&0\\
1&3&1\\
-1&0&1
\end{array}
\right)$,
\begin{itemize}
\item[(1)] 求$(\MA+\MB)^2-(\MA^2+2\MA\MB+\MB^2)$;
\item[(2)]  求$\MA^{-1}$。
\end{itemize}
\end{li}
\pause

\begin{jie}
\begin{itemize}
\item[(1)]  $(\MA+\MB)^2-(\MA^2+2\MA\MB+\MB^2)=\MB\MA-\MA\MB=\left(
\begin{array}{rrr}
1&-1&0\\
8&6&6\\
0&-1&1
\end{array}
\right)-\left(
\begin{array}{rrr}
0&7&3\\
-2&5&2\\
-2&4&3
\end{array}
\right)=\left(
\begin{array}{rrr}
1&-8&-3\\
10&1&4\\
2&-5&-2
\end{array}
\right)$ \pause 
\item[(2)]
$$
\begin{array}{rl}
(\MA,~\MI)&=\left(
\begin{array}{rrrrrr}
1&2&1&1&0&0\\
2&1&1&0&1&0\\
1&1&2&0&0&1
\end{array}
\right)\rightarrow\left(
\begin{array}{rrrrrr}
1&2&1&1&0&0\\
0&-3&-1&-2&1&0\\
0&-1&1&-1&0&1
\end{array}
\right)\\
&\rightarrow\left(
\begin{array}{rrrrrr}
1&2&1&1&0&0\\
0&1&-1&1&0&-1\\
0&0&-4&1&1&-3\\
\end{array}
\right)\rightarrow\left(
\begin{array}{rrrrrr}
1&0&0&-\frac14&\frac34&-\frac14\\
0&1&0&\frac34&-\frac14&-\frac14\\
0&0&1&-\frac14&-\frac14&\frac34\\
\end{array}
\right)
\end{array}
$$
\end{itemize}
\end{jie}
\end{frame}

\begin{frame}

\begin{li}[2006-2007第二学期]
设$\MA=\left(
\begin{array}{rrr}
1&-3&2\\
-2&1&-1\\
1&2&-1
\end{array}
\right)$, $\MB=\left(
\begin{array}{rrr}
2&5&4\\
4&-2&2\\
1&4&1
\end{array}
\right)$,
\begin{itemize}
\item[(1)] 求$4\MA^2-\MB^2-2\MB\MA+2\MA\MB$;
\item[(2)]  求$|\MA^*|$。
\end{itemize}
\end{li}
\pause

\begin{jie}
\begin{itemize}
\item[(1)] $4\MA^2-\MB^2-2\MB\MA+2\MA\MB=(2\MA-\MB)(2\MA+\MB)=\left(
\begin{array}{rrr}
0&0&0\\
-44&-24&-60\\
-5&-25&11
\end{array}
\right)$ \pause 
\item[(2)]  因$|\MA^*|=|\MA|^{2}$,而
$
|\MA|=\left|
\begin{array}{rrr}
1&-3&2\\
0&-5&3\\
0&5&-3
\end{array}
\right|=0
$,
故$|\MA^*|=0$。
\end{itemize}
\end{jie}
\end{frame}

\begin{frame}

\begin{li}[2007-2008第一学期]
证明
\begin{itemize}
\item[(1)] 设$\MA$为$n$阶方阵,证明:若$|\MA|=0$,则$|\MA^*|=0$;
\item[(2)]  设$\MA,\MB$为$n$阶方阵,且满足$\MA^2=\MA,\MB^2=\MB,\rank(\MA+\MB-\MI)=n$,证明:$\rank(\MA)=\rank(\MB)$。
\end{itemize}
\end{li}
\pause

\begin{proof}
\begin{itemize}
\item[(1)] $|\MA^*|=|\MA|^{n-1}=0$;\pause 
\item[(2)]  由$\rank(\MA+\MB-\MI)=n$知$\MA+\MB-\MI$可逆。故
$$
\rank(\MA)=\rank(\MA(\MA+\MB-\MI))=\rank(\MA\MB)
$$
$$
\rank(\MB)=\rank((\MA+\MB-\MI)\MB)=\rank(\MA\MB)
$$
于是$\rank(\MA)=\rank(\MB)$.
\end{itemize}
\end{proof}
\end{frame}


\begin{frame}
\begin{li}[2007-2008第二学期]
设$\MA=\left(
\begin{array}{rrr}
2&-1&1\\
1&2&0\\
2&1&2
\end{array}
\right)$, $\MB=\left(
\begin{array}{rrr}
0&2&-3\\
2&-1&4\\
0&-1&4
\end{array}
\right)$,$\MA\MC-\MI=\MB+\MC$,求$\MC$。
\end{li}
\pause

\begin{jie}
由题意可知$(\MA-\MI)\MC=\MB+\MI$,
$$
\begin{array}{rl}
&(\MA-\MI,\MB+\MI)=\left(
\begin{array}{rrrrrr}
1&-1&1 &1&2&-3\\
1&1&0 &2&0&4\\
2&1&1&0&-1&5
\end{array}
\right)\rightarrow\left(
\begin{array}{rrrrrr}
1&-1&1 &1&2&-3\\
0&2&-1 &1&-2&7\\
0&3&-1&-2&-5&11
\end{array}
\right)\\[0.2in]
&\rightarrow\left(
\begin{array}{rrrrrr}
1&-1&1 &1&2&-3\\
0&2&-1 &1&-2&7\\
0&0&1&-7&-4&1
\end{array}
\right) \rightarrow\left(
\begin{array}{rrrrrr}
1&-1&0 &8&6&-4\\
0&1&0 &-3&-3&4\\
0&0&1&-7&-4&1
\end{array}
\right)\\[0.2in]
& \rightarrow\left(
\begin{array}{rrrrrr}
1&-1&0 &8&6&-4\\
0&1&0 &-3&-3&4\\
0&0&1&-7&-4&1
\end{array}
\right) \rightarrow\left(
\begin{array}{rrrrrr}
1&0&0 &5&3&0\\
0&1&0 &-3&-3&4\\
0&0&1&-7&-4&1
\end{array}
\right)
\end{array}
$$
\end{jie}
\end{frame}


\begin{frame}

\begin{li}[2008-2009第一学期]
设$\MA,\MB$为三阶方阵,满足$\MA\MB+\MI=\MA^2+\MB$,且$\MA=\left(
\begin{array}{rrr}
1&0&1\\
0&2&0\\
1&0&1
\end{array}
\right)$, 求$\MB$及$\MB^*$。
\end{li}
\pause

\begin{jie}
依题意可知
$$
(\MA-\MI)\MB=\MA^2-\MI=(\MA-\MI)(\MA+\MI),
$$
而$\MA-\MI=\left(
\begin{array}{rrr}
0&0&1\\
0&2&0\\
1&0&0
\end{array}
\right)$非奇异,故$$\MB=\MA+\MI=\left(
\begin{array}{rrr}
2&0&1\\
0&3&0\\
1&0&2
\end{array}
\right)$$
\end{jie}
\end{frame}






\begin{frame}

\begin{li}[2009-2010第一学期]
计算下列各题:
\begin{itemize}
\item[(1)] 已知$\MA=\left(
\begin{array}{rrr}
a&a&a\\
b&b&b\\
c&c&c
\end{array}
\right)$,求$\MA^{2010}$;
\item[(2)] 设$n(n\ge2)$阶方阵$\MA$非奇异,求$(\MA^*)^*$。
\end{itemize}
\end{li}
\pause

\begin{jie}
\begin{itemize}
\item[(1)]由$|\MA-\lambda\MI|
%=\left|
%\begin{array}{rrr}
%a-\lambda&a&a\\
%b&b-\lambda&b\\
%c&c&c-\lambda
%\end{array}
%\right|
%=(a+b+c-\lambda)\left|
%\begin{array}{rrr}
%1&1&1\\
%b&b-\lambda&b\\
%c&c&c-\lambda
%\end{array}
%\right|=(a+b+c-\lambda)\left|
%\begin{array}{rrr}
%1&0&0\\
%b&-\lambda&\\
%c&0&-\lambda
%\end{array}
%\right|
=(a+b+c-\lambda)\lambda^2=0$知,
特征值为$\lambda_{1,2}=0$与$\lambda_3=a+b+c$。
\begin{itemize}
\item
$\lambda_{1,2}=0$,方程为$x+y+z=0$,基础解系为$\boxed{\vx_1=(-1,1,0)^T,\vx_2=(-1,0,1)^T}$。\\[0.1in]
\item
$\lambda_3=a+b+c$,方程为$\left(
\begin{array}{rrr}
-b-c&a&a\\
b&-a-c&b\\
c&c&-a-b
\end{array}
\right)\vx=\M0$,基础解系为$\boxed{\vx_3=(a,b,c)^T}$。
\end{itemize}
\end{itemize}
\end{jie}
\end{frame}

\begin{frame}

故
$$
\MA(\vx_1,\vx_2,\vx_3)=(\vx_1,\vx_2,\vx_3)\left(
\begin{array}{ccc}
0&&\\
&0&\\
&&a+b+c
\end{array}
\right)
$$
从而
$$
\MA^{2010}=\MP\left(
\begin{array}{ccc}
0&&\\
&0&\\
&&a+b+c
\end{array}
\right)^{2010}\MP^{-1}
$$
其中
$$
\MP=\left(
\begin{array}{ccc}
-1&-1&a\\
1&0&b\\
0&1&c
\end{array}
\right), \quad
\MP^{-1}=\frac1{a+b+c}\left(
\begin{array}{ccc}
-b&a+c&-b\\
-c&-c&a+b\\
1&1&1
\end{array}
\right)
$$
故
$$
\MA^{2010}= (a+b+c)^{2009}
\left(
\begin{array}{ccc}
a&a&a\\
b&b&b\\
c&c&c
\end{array}
\right)
$$

\end{frame}

\begin{frame}

\begin{li}[2009-2010第一学期,2011-2012第二学期]
设三阶方阵$\MA$满足$\MA\MX=\MA+2\MX$,且$\MA=\left(
\begin{array}{rrr}
3&0&1\\
1&1&0\\
0&1&4
\end{array}
\right)$,求$\MX$。
\end{li}
\pause
\begin{jie}
依题意可知$(\MA-2\MI)\MX=\MA$,解此矩阵方程即可求得$\MX$。\pause 
$$
\begin{array}{rl}
&(\MA-2\MI, \MA)=\left(
\begin{array}{rrrrrr}
  1&0&1&3&0&1\\
  1&-1&0&1&1&0\\
  0&1&2&0&1&4
\end{array}
\right)\rightarrow\left(
\begin{array}{rrrrrr}
  1&0&1&3&0&1\\
  0&-1&-1&-2&1&-1\\
  0&1&2&0&1&4
\end{array}
\right)\\[0.2in]
&\rightarrow\left(
\begin{array}{rrrrrr}
  1&0&1&3&0&1\\
  0&-1&-1&-2&1&-1\\
  0&0&1&-2&2&3
\end{array}
\right)\rightarrow\left(
\begin{array}{rrrrrr}
  1&0&1&3&0&1\\
  0&1&0&4&-3&-2\\
  0&0&1&-2&2&3
\end{array}
\right)\\[0.2in]
&\rightarrow\left(
\begin{array}{rrrrrr}
  1&0&0&5&-2&-2\\
  0&1&0&4&-3&-2\\
  0&0&1&-2&2&3
\end{array}
\right)
\end{array}
$$
\end{jie}
\end{frame}


\begin{frame}
\begin{li}[2009-2010第二学期]
已知矩阵方程满足$(2\MI-\MC^{-1}\MB)\MA^T=\MC^{-1}$,求$\MA$,其中
$$\MB=\left(
\begin{array}{rrrr}
1&2&-3&-2\\
0&1&2&-3\\
0&0&1&2\\
0&0&0&1
\end{array}
\right),\quad
\MC=\left(
\begin{array}{rrrr}
1&2&0&1\\
0&1&2&0\\
0&0&1&2\\
0&0&0&1
\end{array}
\right)$$
\end{li}
\pause

\begin{jie}
依题意知$\MA^T=(2\MI-\MC^{-1}\MB)^{-1}\MC^{-1}=(\MC(2\MI-\MC^{-1}\MB))^{-1}=(2\MC-\MB)^{-1}$
$$
\begin{array}{l}
  (2\MC-\MB,~\MI)=\left(
  \begin{array}{rrrrrrrr}
    1&2&3&4&1&0&0&0\\
    0&1&2&3&0&1&0&0\\
    0&0&1&2&0&0&1&0\\
    0&0&0&1&0&0&0&1
  \end{array}
  \right)\rightarrow\left(
  \begin{array}{rrrrrrrr}
    1&2&3&0&1&0&0&-4\\
    0&1&2&0&0&1&0&-3\\
    0&0&1&0&0&0&1&-2\\
    0&0&0&1&0&0&0&1
  \end{array}
  \right)\\[0.3in]
\rightarrow\left(
  \begin{array}{rrrrrrrr}
    1&2&0&0&1&0&-3&2\\
    0&1&0&0&0&1&-2&1\\
    0&0&1&0&0&0&1&-2\\
    0&0&0&1&0&0&0&1
  \end{array}
  \right)
\rightarrow\left(
  \begin{array}{rrrrrrrr}
    1&0&0&0&1&-2&1&0\\
    0&1&0&0&0&1&-2&1\\
    0&0&1&0&0&0&1&-2\\
    0&0&0&1&0&0&0&1   
  \end{array}
  \right)\\[0.3in]
  \red{\Longrightarrow \MA=\left(
  \begin{array}{rrrr}
    1&0&0&0\\
    -2&1&0&0\\
    1&-2&1&0\\
    0&1&-2&1   
  \end{array}
  \right)
}
\end{array}
$$
\end{jie}

\end{frame}

\begin{frame}

\begin{li}[2012-2013第二学期]
已知$\MA$为三阶矩阵,$\MB=\left(
\begin{array}{rrr}
1&-2&0\\
1&2&0\\
0&0&2
\end{array}
\right)$,且满足$2\MA^{-1}\MB=\MB-4\MI$,$\MI$为三阶单位矩阵,求矩阵$\MA$。
\end{li}
\pause

\begin{jie}
依题意$\MA(\MB-4\MI)=2\MB$,可用$\boxed{\left(
  \begin{array}{c}
    \MB-4\MI\\
    2\MB
  \end{array}
\right)\xlongrightarrow[]{\mbox{初等列变换}} \left(
  \begin{array}{c}
    \MI\\
    \red{\MA}
  \end{array}
\right)}$求$\MA$。
$$
\left(
  \begin{array}{c}
    \MB-4\MI\\
    2\MB
  \end{array}
\right)=\left(
\begin{array}{rrr}
-3&-2&0\\
1&-2&0\\
0&0&-2\\
2&-4&0\\
2&4&0\\
0&0&4
\end{array}
\right)\rightarrow
\left(
\begin{array}{rrr}
3  &1&0\\
-1 &1&0\\
0  &0 &1\\
-2 &2 &0\\
-2 &-2&0\\
0  &0 &-2
\end{array}
\right)\rightarrow
\left(
\begin{array}{rrr}
1 &0  &0\\
0 &1 &0\\
0 &0  &1\\
0 &2 &0\\
-1&-1  &0\\
0 &0  &-2
\end{array}
\right)
$$
\end{jie}
\end{frame}


\begin{frame}

\begin{li}[2012-2013第二学期]
设矩阵$\MA=\left(
\begin{array}{rrr}
-1&0&0\\
 0&1&0\\
 0&1&1
\end{array}
\right)$,矩阵$\MB$满足$\MA^*\MB\MA=2\MB\MA-9\MI$,求$\MB$。
\end{li}
\pause

\begin{jie}
易知$|\MA|=-1$,即$\MA$可逆,由$\MA\MA^*=|\MA|\MI=-\MI$可得
$$
\begin{array}{l}
\MA^*\MB\MA=2\MB\MA-9\MI ~~\Rightarrow~~
\MA\MA^*\MB\MA=\MA(2\MB\MA-9\MI) \\[0.1in]
\Rightarrow~~
-\MB\MA=2\MA\MB\MA-9\MA ~~\Rightarrow~~
-\MB=2\MA\MB-9\MI   ~~\Rightarrow~~
(2\MA+\MI)\MB=9\MI
\end{array}
$$
$$
\begin{array}{l}
(2\MA+\MI, 9\MI)=\left(
\begin{array}{rrrrrr}
  -1   &  0  &   0 &  9&0&0\\
  0    & 3   &  0  &  0&9&0\\
  0    & 2   &  3  &  0&0&9
\end{array}
\right)  \rightarrow
\left(
\begin{array}{rrrrrr}
  1   &  0  &   0 & -9&0&0\\
  0    & 1   &  0  &  0&3&0\\
  0    & 0   &  1  &  0&-2&3
\end{array}
\right)  
\end{array}
$$
\end{jie}
\end{frame}

\begin{frame}

\begin{li}[2012-2013第二学期]
设矩阵$\MA=\left(
\begin{array}{rrrr}
1&-1&-1&-1\\
-1&1&-1&-1\\
-1&-1&1&-1\\
-1&-1&-1&1
\end{array}
\right)$,
\begin{itemize}
\item[(1)] 求$\MA^n$;
\item[(2)] 设$\MA^2+\MA\MB-\MA=\MI$,求$|\MB|$。
\end{itemize}
\end{li}
\pause

\begin{jie}
\begin{itemize}
\item[(1)] 求矩阵的特征值与特征向量。
  $$
  \begin{array}{l}
    |\MA-\lambda\MI|=\left|
    \begin{array}{rrrr}
      1-\lambda&-1&-1&-1\\
      -1&1-\lambda&-1&-1\\
      -1&-1&1-\lambda&-1\\
      -1&-1&-1&1-\lambda
    \end{array}
    \right|
    =(-2-\lambda)\left|
    \begin{array}{rrrr}
      1&-1&-1&-1\\
      1&1-\lambda&-1&-1\\
      1&-1&1-\lambda&-1\\
      1&-1&-1&1-\lambda
    \end{array}
    \right|\\[0.3in]=(-2-\lambda)\left|
    \begin{array}{rrrr}
      1&-1&-1&-1\\
      0&2-\lambda&0&0\\
      0&0& 2-\lambda&0\\
      0&0& 0 &2-\lambda
    \end{array}
    \right|   = (\lambda+2)(\lambda-2)^3
  \end{array}
$$
\end{itemize}
\end{jie}

\end{frame}


\begin{frame}

  \begin{itemize}
  \item 当$\lambda_{1,2,3}=2$时,$(\MA-\lambda\MI)\vx=\M0$为
    $$
    x_1+x_2+x_3+x_4=0
    $$
    基础解系为
    $$
    \vx_1=(-1,1,0,0)^T,~~
    \vx_2=(-1,0,1,0)^T,~~
    \vx_3=(-1,0,0,1)^T.
    $$
    故对应于$\lambda_{1,2,3}=2$的特征向量为$k_1\vx_1+k_2\vx_2+k_3\vx_3, (k_1,k_2,k_3\mbox{不全为零})$;
  \item 当$\lambda_{4}=-2$时,$(\MA-\lambda\MI)\vx=\M0$为
    $$
    \begin{array}{l}
          \left(
    \begin{array}{rrrr}
      3&-1&-1&-1\\
      -1&3&-1&-1\\
      -1&-1&3&-1\\
      -1&-1&-1&3
    \end{array}
    \right) \rightarrow \left(
    \begin{array}{rrrr}
      1&-3&1&1\\
      1&1&-3&1\\
      1&1&1&-3\\
      3&-1&-1&-1
    \end{array}
    \right) \rightarrow \left(
    \begin{array}{rrrr}
      1&-3&1&1\\
      0&4&-4&0\\
      0&4&0&-4\\
      0&8&-4&-4
    \end{array}
    \right) \\[0.2in]
    \rightarrow \left(
    \begin{array}{rrrr}
      1&-3&1&1\\
      0&1&-1&0\\
      0&0&1&-1\\
      0&0&0&0
    \end{array}
    \right)\rightarrow \left(
    \begin{array}{rrrr}
      1&0&0&-1\\
      0&1&0&-1\\
      0&0&1&-1\\
      0&0&0&0
    \end{array}
    \right)
    \end{array}
    $$
    对应方程为
    $$
    \left\{
    \begin{array}{l}
      x_1=x_4\\
      x_2=x_4\\
      x_3=x_4\\
      x_4=x_4
    \end{array}
    \right.
    $$
    基础解系为$$
    \vx_3=(1,1,1,1)^T
    $$
    
  \end{itemize}

\end{frame}

\begin{frame}

  取
  $$
  \MP=(\vx_1,\vx_2,\vx_3,\vx_4)=\left(
  \begin{array}{rrrr}
    -1 & -1& -1&  1 \\
    1  &  0&  0&  1 \\
    0  &  1&  0&  1  \\
    0  &  0&  1&  1 
  \end{array}
  \right), ~~~\MP^{-1}=\frac14 \left(
  \begin{array}{rrrr}
    -1 &   3& -1&  -1 \\
    -1 &  -1&  3&  -1 \\
    -1 &  -1& -1&   3  \\
     1 &   1&  1&   1 
  \end{array}
  \right)
  $$
  则 $\MA\MP=\MP\Lambdabd$,即
  $$
  \MA=\MP\Lambdabd\MP^{-1},
  $$
  从而$$
  \MA^{n}=\MP\Lambdabd^n\MP^{-1}
  =2^n \MP\left(
  \begin{array}{cccc}
    1&&&\\
    &1&&\\
    &&1&\\
    &&&(-1)^n
  \end{array}
  \right)\MP^{-1}
  $$
  当$n$为偶数时,$\MA^n=2^n \MI$
  当$n$为奇数时,$\MA^n=2^n\MP\left(
  \begin{array}{cccc}
    1&&&\\
    &1&&\\
    &&1&\\
    &&&-1
  \end{array}
  \right)\MP^{-1} $
  而
  $
  \MA=2\MP\left(
  \begin{array}{cccc}
    1&&&\\
    &1&&\\
    &&1&\\
    &&&-1
  \end{array}
  \right)\MP^{-1} 
  $
  故
  $
  \MA^n=2^{n-1}\MA
  $。

\end{frame}

\begin{frame}

  \begin{itemize}
  \item[(2)]
    依题意,
    $$
    \MA\MB=\MI-\MA+\MA^2=\MI-\MA+4\MI=-\MA+5\MI
    $$
    故
    $$
    |\MA||\MB|=|-\MA+5\MI|
    $$
    即
    $$
    -16 |\MB| = |\MA-5\MI| =\left|
    \begin{array}{rrrr}
      -4&-1&-1&-1\\
      -1&-4&-1&-1\\
      -1&-1&-4&-1\\
      -1&-1&-1&-4
    \end{array}
    \right| = 189
    $$
    故
    $$
    |\MB|=-\frac{189}{16}.
    $$
  \end{itemize}

\end{frame}
