\section{矩阵的秩 \quad 相抵标准形}
\begin{frame}
\begin{dingyi}[行秩 \& 列秩]
  \begin{itemize}
  \item
    把$\MA$的行向量组的秩,称为矩阵$\MA$的\red{行秩},记为$\red{\rank_r(\MA)}$。
  \item
    把$\MA$的列向量组的秩,称为矩阵$\MA$的\red{列秩},记为$\red{\rank_c(\MA)}$。
  \end{itemize}      
\end{dingyi}    
对于$m\times n$阶矩阵$\MA$,
\begin{itemize}
\item $\rank_p(\MA) \le m$;
\item $\rank_c(\MA) \le n$。
\end{itemize}
\end{frame}

\begin{frame}
\begin{dingyi}[阶梯形矩阵]
  若矩阵$\MA$满足
  \begin{itemize}
  \item[(1)] 零行在最下方;
  \item[(2)] 非零行首元的列标号随行标号的增加而严格递增,
  \end{itemize}
  则称$\MA$为\red{阶梯形矩阵}。
\end{dingyi}

\begin{li}
  $$
  \left(
    \begin{array}{rrrr}
      2&0&2&1\\
      0&5&2&-2\\
      0&0&3&2\\
      0&0&0&0
    \end{array}
  \right)
  $$
\end{li}
\end{frame}

\begin{frame}
\begin{dingyi}[行简化阶梯形矩阵]
  若矩阵$\MA$满足
  \begin{itemize}
  \item[(1)] 它是阶梯形矩阵;
  \item[(2)] 非零行首元所在的列除了非零行首元外,其余元素全为零,
  \end{itemize}
  则称$\MA$为\red{行简化阶梯形矩阵}。
\end{dingyi}

\begin{li}
  $$
  \left(
    \begin{array}{rrrr}
      2&0&0&1\\
      0&5&0&-2\\
      0&0&3&2\\
      0&0&0&0
    \end{array}
  \right)
  $$
\end{li}
\end{frame}

\begin{frame}
\begin{dingyi}[行最简阶梯形矩阵]
  若矩阵$\MA$满足
  \begin{itemize}
  \item[(1)] 它是行简化阶梯形矩阵;
  \item[(2)] 非零行首元为$1$,
  \end{itemize}
  则称$\MA$为\red{行最简阶梯形矩阵}。
\end{dingyi}

\begin{li}
  $$
  \left(
    \begin{array}{rrrr}
      1&0&0&1\\
      0&1&0&-2\\
      0&0&1&2\\
      0&0&0&0
    \end{array}
  \right)
  $$
\end{li}
\end{frame}


\begin{frame}
阶梯形矩阵
\begin{figure}
  \centering
  \begin{tikzpicture}
    \matrix (M) [matrix of math nodes]  { 
      \MA = \\
    };
    \matrix(MM) [right=.1in of M, matrix of math nodes,nodes in empty cells,
    column sep=3ex,row sep=1ex,ampersand replacement=\&,left delimiter=(,right delimiter=)] {
      a_{11} \& a_{12} \& a_{13}  \& a_{14} \& a_{15}\\
      0 \& 0 \& a_{23}  \& a_{24} \& a_{25}\\
      0 \& 0 \& 0  \& a_{34} \& a_{35}\\
      0 \& 0 \& 0  \& 0 \& 0\\
    };
    \draw[thick,red,dashed] (MM-2-1.north west)--(MM-2-2.north east)
    --(MM-3-2.north east)--(MM-3-3.north east)
    --(MM-4-3.north east)--(MM-4-5.north east);
  \end{tikzpicture}
\end{figure}
其中$a_{11}\ne0, a_{23}\ne 0, a_{34}\ne 0$。
\red{验证$\rank_r(\MA)=3, \rank_c(\MA)=3$}。
\end{frame}

\begin{frame}
把$\MA$按行和列分块为
$$
\MA = \left(
  \begin{array}{c}
    \alphabd_1\\
    \alphabd_2\\
    \alphabd_3\\
    \alphabd_4
  \end{array}
\right), \quad \MB = (\betabd_1,\betabd_2,\betabd_3,\betabd_4,\betabd_5)
$$
下证$\alphabd_1,\alphabd_2,\alphabd_3$线性无关,$\betabd_1,\betabd_3,\betabd_4$线性无关。
\end{frame}

\begin{frame}
\begin{itemize}
\item[(1)] 设
  $$
  x_1\alphabd_1+x_2\alphabd_2+x_3\alphabd_3=\M0,
  $$
  即
  $$
  \begin{aligned}
    \red{x_1(a_{11},a_{12},a_{13},a_{14},a_{15})+
      x_2(0,0,a_{23},a_{24},a_{25})
      +
      x_3(0,0,0,a_{34},a_{35})=(0,0,0,0,0)}
  \end{aligned}
  $$  
  比较第一个分量
  $$
  x_1a_{11} = 0 \Rightarrow x_1=0.
  $$  \pause 
  $$
  \red{x_2(0,0,a_{23},a_{24},a_{25})+
  x_3(0,0,0,a_{34},a_{35})=(0,0,0,0,0)}
  $$  
  比较第3个分量
  $$
  x_2a_{23} = 0 \Rightarrow x_2=0.
  $$ \pause 
  $$
  \red{x_3(0,0,0,a_{34},a_{35})=(0,0,0,0,0)}
  $$ 
  同理得$x_3=0$。  于是$\alphabd_1,\alphabd_2,\alphabd_3$线性无关。 \pause 
  \vspace{0.1in}

  又$\alphabd_4=\M0$,而零向量可由任何向量线性表示,此时
  $$
  \M0 = 0\alphabd_1+0\alphabd_2+0\alphabd_3.
  $$
  故$\alphabd_1,\alphabd_2,\alphabd_3$是向量组$\alphabd_1,\alphabd_2,\alphabd_3,\alphabd_4$的极大无关组,所以$\rank_r(\MA)=3$.
\end{itemize}
\end{frame}



\begin{frame}
\begin{figure}
  \centering
  \begin{tikzpicture}
    \matrix(MM) [ matrix of math nodes,nodes in empty cells,
    column sep=3ex,row sep=1ex,ampersand replacement=\&,left delimiter=(,right delimiter=)] {
      \red{a_{11}} \& a_{12} \& \red{a_{13}}  \& \red{a_{14}} \& a_{15}\\
      \red{0} \& 0 \& \red{a_{23}}  \& \red{a_{24}} \& a_{25}\\
      \red{0} \& 0 \& \red{0}  \& \red{a_{34}} \& a_{35}\\
      \red{0} \& 0 \& \red{0}  \& \red{0} \& 0\\
    };
    \node[above=5pt  of MM-1-1, blue]  {$\betabd_1$};
    \node[above=5pt  of MM-1-2, blue]  {$\betabd_2$};
    \node[above=5pt  of MM-1-3, blue]  {$\betabd_3$};
    \node[above=5pt  of MM-1-4, blue]  {$\betabd_4$};
    \node[above=5pt  of MM-1-5, blue]  {$\betabd_5$};
  \end{tikzpicture}
\end{figure}
去掉$\betabd_1,\betabd_2,\betabd_3,\betabd_4,\betabd_5$的最后一个分量$0$,得新向量组$\betabd_1^*,\betabd_2^*,\betabd_3^*,\betabd_4^*,\betabd_5^*$。

\pause 
因$$ \left|\begin{array}{ccc}
             \betabd_1^*&\betabd_3^*&\betabd_4^*
           \end{array}\right| =
\left|
  \begin{array}{ccc}
    a_{11}&a_{13}&a_{14}\\
    0&a_{23}&a_{24}\\
    0&0&a_{34}\\
  \end{array}
\right|=a_{11}a_{23}a_{24}\ne 0,
$$
故$\betabd_1^*,\betabd_3^*,\betabd_4^*$线性无关,从而$\betabd_1,\betabd_3,\betabd_4$线性无关\red{(对应位置增减零不影响线性相关性)}。 \pause 

因$\red{\betabd_i^*\in\mathbb R^3}$,故$\red{\betabd_{1}^*,\betabd_{3}^*,\betabd_{4}^*},\blue{\betabd_{j}^*(j=2 ~or~5)}$线性相关,而$\red{\betabd_{1}^*,\betabd_{3}^*,\betabd_{4}^*}$线性无关,故$\blue{\betabd_{j}^*(j=2 ~or~5)}$可由$\red{\betabd_{1}^*,\betabd_{3}^*,\betabd_{4}^*}$线性表示,从而$\blue{\betabd_{j}(j=2 ~or~5)}$可由$\red{\betabd_{1},\betabd_{3},\betabd_{4}}$线性表示。于是$\red{\betabd_1,\betabd_3,\betabd_4}$是极大无关组,从而有$\rank_c(\MA)=3$.
\end{frame}


\begin{frame}
\begin{jielun}
  阶梯形矩阵的行秩等于列秩,其值等于阶梯形矩阵的非零行的行数。
\end{jielun}
\end{frame}

\begin{frame}
\begin{dingli}
  初等行变换不改变矩阵的行秩。
\end{dingli}
\end{frame}

\begin{frame}
\begin{proof}
  只需证明每做一次对换、倍乘和倍加变换,矩阵的行秩不改变。    
  设$\MA$是$m\times n$矩阵,进行一次初等变换所得矩阵为$\MB$。记$\MA$的行向量为
  $\red{A:~\alphabd_1,\alphabd_2,\cd,\alphabd_m.}$ \pause 
  \begin{itemize}
  \item[(1)] 考虑对调变换
    $\red{\MA \xlongrightarrow[]{r_i\leftrightarrow r_j}\MB}$。
    因$\MB$的行向量组
    $$\red{B:~\alphabd_1,\alphabd_2,\cd,\blue{\alphabd_j},\cd,\blue{\alphabd_i},\cd,\alphabd_m}$$
    与$\MA$的行向量组
    $$\red{A:~\alphabd_1,\alphabd_2,\cd,\blue{\alphabd_i},\cd,\blue{\alphabd_j},\cd,\alphabd_m}$$
    一致,故$\rank_r(\MB)=\rank_r(\MA)$。 \pause 

  \item[(2)] 考虑倍乘变换
    $\red{\MA \xlongrightarrow[]{r_i\times c, ~c\ne 0}\MB}$。 
    因$\MB$的行向量组
    $$\red{B:~\alphabd_1,\alphabd_2,\cd,\blue{c\alphabd_i},\cd,\alphabd_m}$$
    与$\MA$的行向量组
    $$\red{A:~\alphabd_1,\alphabd_2,\cd,\blue{\alphabd_i},\cd,\alphabd_m}$$
    等价,故$\rank_r(\MB)=\rank_r(\MA)$。\pause 

  \item[(3)] 考虑倍乘变换
    $
    \red{\MA \xlongrightarrow[]{r_i+ r_j \times c  }\MB}
    $。 
    因$\MB$的行向量组
    $$B:~\red{\alphabd_1,\alphabd_2,\cd,}\blue{\alphabd_i+c\alphabd_j}\red{,\cd,\alphabd_m}$$
    与$\MA$的行向量组
    $$\red{A:~\alphabd_1,\alphabd_2,\cd,\blue{\alphabd_i},\cd,\alphabd_m}$$
    等价,故$\rank_r(\MB)=\rank_r(\MA)$。
  \end{itemize}
\end{proof}
\end{frame}


\begin{frame}
\begin{dingli}
  初等行变换不改变矩阵的列秩。
\end{dingli} \pause 
\begin{proof}
  设
  $$
  \red{\MA = (\alphabd_1,\alphabd_2,\cd,\alphabd_m) \xlongrightarrow[]{\mbox{初等行变换}}
  (\betabd_1,\betabd_2,\cd,\betabd_m) = \MB.}
  $$ 
  在$\MA,\MB$中相同位置任取$s$个列向量:
  $$
  \underbrace{\alphabd_{i_1},\alphabd_{i_2},\cd,\alphabd_{i_s}}_{A^*} \mbox{~~和~~}
  \underbrace{\betabd_{i_1},\betabd_{i_2},\cd,\betabd_{i_s}}_{B^*}.
  $$ \pause 
  设
  \begin{eqnarray}
    x_1\alphabd_{i_1}+x_2\alphabd_{i_2}+\cd+x_s\alphabd_{i_s} =\M0, \label{thm3.3.2-1}\\[0.1in]
    x_1\betabd_{i_1}+x_2\betabd_{i_2}+\cd+x_s\betabd_{i_s} =\M0. \label{thm3.3.2-2}
  \end{eqnarray} \pause 
  注意到方程组(\ref{thm3.3.2-2})是方程组(\ref{thm3.3.2-1})经过高斯消元法得到,故两方程组同解。 即向量组$A^*$和$B^*$有完全相同的线性关系,从而$\MA,\MB$的列秩相等。
\end{proof}

% \end{frame}

% \begin{frame}
\pause 
\blue{上述定理提供了求向量组的秩与极大无关组的一种简便而有效的方法。}
\end{frame}

\begin{frame}
\begin{li}
  设向量组
  $$
  \alphabd_1=\left(
    \begin{array}{r}
      -1\\-1\\0\\0
    \end{array}
  \right),~~ \alphabd_2=\left(
    \begin{array}{r}
      1\\2\\1\\-1
    \end{array}
  \right),~~ \alphabd_3=\left(
    \begin{array}{r}
      0\\1\\1\\-1
    \end{array}
  \right),~~ \alphabd_4=\left(
    \begin{array}{r}
      1\\3\\2\\1
    \end{array}
  \right),~~ \alphabd_5=\left(
    \begin{array}{r}
      2\\6\\4\\-1
    \end{array}
  \right)
  $$
  求向量组的秩及其一个极大无关组,并将其余向量用该极大无关组线性表示。
\end{li}
\end{frame}

\begin{frame}[allowframebreaks]
\begin{jie}
作矩阵$\MA=(\alphabd_1,\alphabd_2,\alphabd_3,\alphabd_4,\alphabd_5)$,由
$$
\begin{array}{rl}
  \MA &= \left(
       \begin{array}{rrrrr}
         -1&1&0&1&2\\
         -1&2&1&3&6\\
         0&1&1&2&4\\
         0&-1&-1&1&-1
       \end{array}
                    \right) \xlongrightarrow[r_2+r_1]{ r_1\times(-1)}
                    \left(
                    \begin{array}{rrrrr}
                      1&-1&0&-1&-2\\
                      0&1&1&2&4\\
                      0&1&1&2&4\\
                      0&-1&-1&1&-1
                    \end{array}
                                 \right)\\[0.4in]
     &\xlongrightarrow[r_4+r_2]{r_3- r_2}
       \left(
       \begin{array}{rrrrr}
         1&-1&0&-1&-2\\
         0&1&1&2&4\\
         0&0&0&0&0\\
         0&0&0&3&3
       \end{array}
                  \right) \xlongrightarrow[r_3\leftrightarrow r_4]{r_4\div 3}
                  \left(
                  \begin{array}{rrrrr}
                    1&-1&0&-1&-2\\
                    0&1&1&2&4\\
                    0&0&0&1&1\\
                    0&0&0&0&0
                  \end{array}
                             \right)
\end{array}
$$

$$
\begin{array}{rl}
  & \xlongrightarrow[r_2 -2 r_3]{r_1+r_3}
    \left(
    \begin{array}{rrrrr}
      1&-1&0&0&-1\\
      0&1&1&0&2\\
      0&0&0&1&1\\
      0&0&0&0&0
    \end{array}
               \right) \xlongrightarrow[]{r_1+r_2}
               \left(
               \begin{array}{rrrrr}
                 1&0&1&0&1\\
                 0&1&1&0&2\\
                 0&0&0&1&1\\
                 0&0&0&0&0
               \end{array}
                          \right) = \MB
\end{array}
$$
将最后一个阶梯矩阵$\MB$记为$(\betabd_1,\betabd_2,\betabd_3,\betabd_4,\betabd_5)$

易知$\betabd_1,\betabd_2,\betabd_4$为$\MB$的列向量组的一个极大无关组,故$\alphabd_1,\alphabd_2,\alphabd_4$也为$\MA$的列向量组的一个极大无关组,故
$$
\rank(\alphabd_1,\alphabd_2,\alphabd_3,\alphabd_4,\alphabd_5)=3,
$$
且
$$
\begin{array}{l}
  \alphabd_3=\alphabd_1+\alphabd_2,\\
  \alphabd_5=\alphabd_1+2\alphabd_2+\alphabd_4,\\
\end{array}
$$
\end{jie}
\end{frame}

\begin{frame}
%由定理3.3.1与定理3.3.2可以推出:
\purple{初等列变换也不改变矩阵的列秩与行秩。}

\begin{dingli}
  初等变换不改变矩阵的行秩与列秩。
\end{dingli}

\end{frame}

\begin{frame}
\begin{dingli}
  矩阵的行秩等于其列秩。
\end{dingli}\pause 
\begin{proof}
对$\MA$做初等行变换得到阶梯矩阵$\MU$,则有
$$
\begin{array}{rl}
  \rank_r(\MA)&=\rank_r(\MU)\\[0.1in]
                 &=\rank_c(\MU)=\rank_c(\MA).
\end{array}
$$
\end{proof}
\end{frame}

\begin{frame}
\begin{dingyi}[矩阵的秩]
  矩阵的行秩或列秩的数值,称为\red{矩阵的秩}。记作
  $$
  \rank(\MA)  \quad \mbox{或} \quad 
  \mbox{rank} (\MA).
  $$
\end{dingyi}

\begin{dingyi}[满秩矩阵]
  对于$n$阶方阵,若
  $$
  \rank(\MA) = n,
  $$
  则称$\MA$为\red{满秩矩阵}。
\end{dingyi}
\end{frame}

\begin{frame}
\begin{dingli}
  对于$n$阶方阵,下列表述等价:
  \begin{itemize}
  \item[(1)] $\MA$为满秩矩阵。
  \item[(2)] $\MA$为可逆矩阵。
  \item[(3)] $\MA$为非奇异矩阵。
  \item[(4)] $\det(\MA)\ne 0$。
  \end{itemize}
\end{dingli} \pause 
\begin{proof}
只需证明前两个表述等价。 
\begin{itemize}
\item [\red{(1)$\Rightarrow$(2)}]    
  设$\rank(\MA)=n$,记$\MA$的行简化阶梯形矩阵为$\MB$,则$\MB$有$n$个非零行, 由行简化阶梯形矩阵的结构知
  $
  \MB=\MI,
  $ 
  即存在可逆矩阵$\MP$使得
  $$
  \MP\MA=\MI,
  $$
  故$\MA^{-1}=\MP$,即$\MA$可逆。 \pause 
\item [\red{(2)$\Rightarrow$(1)}]   
  若$\MA$可逆,记$\MA^{-1}=\MP_0$,则
  $$
  \MP_0\MA=\MI,
  $$ 
  即$\MA$经过初等行变换可以得到$\MI$,故$\rank(\MA)=\rank(\MI)=n$。
\end{itemize}
\end{proof}
\end{frame}

\begin{frame}
\begin{dingyi}[子式与主子式]
  对矩阵$\MA=(a_{ij})_{m\times n}$,任意挑选$k$行($i_1,i_2,\cd,i_k$行)与$k$列($j_1,j_2,\cd,j_k$列),
  其交点上的$k^2$个元素按原顺序排成的$k$阶行列式
  \begin{equation}\label{subdet}
    \left|
      \begin{array}{cccc}
        a_{i_1j_1} & a_{i_1j_2} & \cd & a_{i_1j_k}\\
        a_{i_2j_1} & a_{i_2j_2} & \cd & a_{i_2j_k}\\
        \vd & \vd && \vd\\
        a_{i_kj_1} & a_{i_kj_2} & \cd & a_{i_kj_k}\\
      \end{array}
    \right|
  \end{equation}
  称为$\MA$的\red{$k$阶子行列式},简称$\MA$的\red{$k$阶子式}。 
  \begin{itemize}
  \item 当(\ref{subdet})等于零时,称为\red{$k$阶零子式};
  \item 当(\ref{subdet})不等于零时,称为\red{$k$阶非零子式};
  \item 当(\ref{subdet})的$j_1=i_1,~j_2=i_2,~\cd,~j_k=i_k$,称为$\MA$的\red{$k$阶主子式}。
  \end{itemize}
\end{dingyi}
\end{frame}

\begin{frame}
\begin{zhu}
  若$\MA$存在$r$阶非零子式,而所有$r+1$阶子式(如果有)都等于零,则矩阵$\MA$的非零子式的最高阶数为$r$。
\end{zhu}

\pause 
事实上,由行列式的按行展开可知,若所有$r+1$阶子式都等于零,可得到所有更高阶的子式都等于零。

\end{frame}


\begin{frame}
\begin{dingli}
  $\rank(\MA)=r$的充分必要条件是$\MA$的非零子式的最高阶数为$r$。
\end{dingli} \pause 
\begin{proof}
\begin{itemize}
\item[$(\Rightarrow)$] 设$\rank(\MA)=r$,即$\MA$的行秩为$r$,不妨设$\MA$的前$r$行构成的矩阵$\MA_1$的行秩为$r$,
  其列秩也为$r$;不妨设$\MA_1$的前$r$个列向量线性无关,于是$\MA$的左上角$r$阶子式为非零子式。\vspace{0.05in}  \pause 

  又因为$\MA$的任意$r+1$个行向量线性相关,所以$\MA$的任意$r+1$阶子式都是零子式(\purple{因为其中有一行可由其余$r$行线性表示}),
  因此$\MA$的非零子式的最高阶数为$r$。 \vspace{0.05in}  \pause 

\item[$(\Leftarrow)$] 
  不妨设$\MA$的左上角$r$阶子式$|\MA_r|\ne 0$,于是$\MA_r$可逆,其$r$个行向量线性无关。
  将它们添加分量称为$\MA$的前$r$个行向量,它们也线性无关。\vspace{0.05in}  \pause 

  而$\MA$的任何$r+1$个行向量必线性相关(\purple{否则,$\MA$中存在$r+1$阶非零子式,这与题设矛盾}),故$\MA\mbox{的行秩}=\rank(\MA)=r$.
\end{itemize}
\end{proof}
\end{frame}

\begin{frame}
关于矩阵的秩的基本结论
\begin{itemize}
\item[(1)]  $\red{\rank(\MA)=\rank_r(\MA)=\rank_c(\MA)=\MA\mbox{的非零子式的最高阶数}}$
\item[(2)]  \red{初等变换不改变矩阵的秩}
\end{itemize}
\end{frame}

\begin{frame}
\begin{xingzhi}
  $$
  \red{\max\{\rank(\MA),~\rank(\MB)\}~~\le~~ \rank(\MA,~\MB) ~~\le~~ \rank(\MA) + \rank(\MB).}
  $$
  特别地,当$\MB=\vb$为非零向量时,有
  $$
  \red{\rank(\MA)~~\le~~\rank(\MA,~\vb)~~\le~~\rank(\MA)+1.}
  $$
\end{xingzhi}
% \end{frame}

% \begin{frame}
\pause 
$$
\purple{\rank(\MA,\vb) = \left\{
  \begin{array}{lll}
    \rank(\MA) & \Longleftrightarrow~~ \vb\mbox{可以被}\MA\mbox{的列向量线性表示} & \Longleftrightarrow~~ \MA\vx=\vb\mbox{有解}\\[0.1in]
    \rank(\MA)+1 & \Longleftrightarrow~~ \vb\mbox{不能被}\MA\mbox{的列向量线性表示} & \Longleftrightarrow~~ \MA\vx=\vb\mbox{无解}.
  \end{array}
\right.}
$$
\end{frame}



\begin{frame}
设$$\MA=\left(
  \begin{array}{cc}
    1&0\\
    0&1\\
    0&0
  \end{array}
\right),~~
\vb_1=\left(
    \begin{array}{cc}
      1\\
      2\\
      0
    \end{array}
  \right), ~~
\vb_2=\left(
    \begin{array}{cc}
      0\\
      0\\
      1
    \end{array}
  \right)  
$$
\begin{itemize}
\item[(1)] 因
  $$
  (\MA,~\vb_1) = \left(
    \begin{array}{ccc}
      1&0&\red{1}\\
      0&1&\red{2}\\
      0&0&\red{0}
    \end{array}
  \right) \xlongrightarrow[]{c_3-(c_1+2c_2)}
  \left(
    \begin{array}{ccc}
      1&0&\red{0}\\
      0&1&\red{0}\\
      0&0&\red{0}
    \end{array}
  \right) = (\MA, \M0),
  $$
  故$\rank(\MA,\vb_1)=\rank(\MA,\M0)=\rank(\MA)$,从而$\vb_1$可由$\MA$的列向量线性表示。\\[0.1in]  
\item[(2)] 因
  $$
  (\MA,~\vb_2) = \left(
    \begin{array}{ccc}
      1&0&\red{0}\\
      0&1&\red{0}\\
      0&0&\red{1}
    \end{array}
  \right),
  $$
  故$\rank(\MA,\vb)=\rank(\MA)+1$,
  从而$\vb$不能由$\MA$的列向量线性表示。  
\end{itemize}
\end{frame}

\begin{frame}
\begin{proof}
\begin{itemize}
\item
  因为$\MA$的列均可由$(\MA,\MB)$的列线性表示,故
  $$
  \rank(\MA) \le \rank(\MA,\MB),
  $$  
  同理
  $$
  \rank(\MB) \le \rank(\MA,\MB),
  $$ 
  所以
  $$
  \max\{\rank(\MA),~\rank(\MB)\} \le \rank(\MA,\MB),
  $$ \pause 
\item  
  设$\rank(\MA)=p, ~\rank(\MB)=q$,%将$\MA,\MB$按列分块为
  %% $$
  %% \MA=(\alphabd_1,~\cd,~\alphabd_n), ~~
  %% \MB=(\betabd_1,~\cd,~\betabd_n).
  %% $$  
  $\MA$和$\MB$的列向量组的极大无关组分别为
  $$
  \alphabd_1,~\cd,~\alphabd_p \mbox{~~和~~}
  \betabd_1,~\cd,~\betabd_q.  
  $$  
  显然$(\MA,~\MB)$的列向量组可由向量组$\alphabd_1,~\cd,~\alphabd_p,~
  \betabd_1,~\cd,~\betabd_q$线性表示,故
  $$
  \rank(\MA,~\MB) = \rank_c(\MA,~\MB) \le \rank(\alphabd_1,~\cd,~\alphabd_p,~
  \betabd_1,~\cd,~\betabd_q) \le p+q.
  $$
\end{itemize}
\end{proof}
\end{frame}

\begin{frame}
\begin{zhu}
  \begin{itemize}
  \item 不等式
    $$
    \blue{\min\{\rank(\MA),~\rank(\MB)\} ~~\le~~ \rank(\MA,~\MB)}
    $$
    意味着:在$\MA$的右侧添加新的列,秩会增加(或不变);当$\MB$的列向量能被$\MA$的列向量线性表示时,秩不变。\\[0.1in]  \pause 
  \item 不等式
    $$
    \rank(\MA,~\MB) ~~\le~~ \rank(\MA)+\rank(\MB)
    $$
    意味着:对$(\MA,~\MB)$,有可能$\MA$的列向量与$\MB$的列向量出现线性相关,合并为$(\MA,~\MB)$的秩一般会比$\rank(\MA)+\rank(\MB)$要小。
  \end{itemize}
\end{zhu}
\end{frame}


\begin{frame}
\begin{xingzhi}
  $$
  \red{\rank(\MA+\MB) \le \rank(\MA)+\rank(\MB).}
  $$
\end{xingzhi} \pause
\begin{proof}
设$\rank(\MA)=p, ~\rank(\MB)=q$,
$\MA$和$\MB$的列向量组的极大无关组分别为
$$
\alphabd_1,~\cd,~\alphabd_p \mbox{~~和~~}
\betabd_1,~\cd,~\betabd_q.  
$$  
显然$\MA+\MB$的列向量组可由向量组$\alphabd_1,~\cd,~\alphabd_p,~
\betabd_1,~\cd,~\betabd_q$线性表示,故
$$
\rank(\MA+\MB) = \rank_c(\MA+\MB) \le \rank(\alphabd_1,~\cd,~\alphabd_p,~
\betabd_1,~\cd,~\betabd_q) \le p+q.
$$
\end{proof}
\end{frame}

\begin{frame}
\begin{zhu}
  由
  $$
  \red{\rank(\MA,\MB)\le \rank(\MA)+\rank(\MB), \quad
  \rank(\MA+\MB) \le \rank(\MA)+\rank(\MB)}
  $$
  可知,将矩阵$\MA$和$\MB$合并、相加,只可能使得秩减小(或不变)。
\end{zhu}

\end{frame}

\begin{frame}
\begin{xingzhi}
  $$
  \red{\rank(\MA\MB) \le \min(\rank(\MA),~\rank(\MB)).}
  $$
\end{xingzhi}
\pause 
\begin{proof}
设$\MA,\MB$分别为$m\times n, n\times s$矩阵,将$\MA$按列分块,则
$$
\MA\MB = (\alphabd_1,~\cd,~\alphabd_n) \left(
  \begin{array}{cccc}
    b_{11}&b_{12}&\cd&b_{1s}\\
    b_{21}&b_{22}&\cd&b_{2s}\\
    \vd&\vd&&\vd\\
    b_{n1}&b_{n2}&\cd&b_{ns}
  \end{array}
\right).
$$  
由此可知,$\MA\MB$的列向量组可由$\alphabd_1,~\alphabd_2,~\cd,~\alphabd_n$线性表示,故
$$
\rank(\MA\MB) = \rank_c(\MA\MB) \le \rank_c(\MA)= \rank(\MA).
$$ \pause 

类似地,将$\MB$按行分块,可得$\rank(\MA\MB)\le \rank(\MB).$
\end{proof}\pause 
\red{该性质告诉我们,对一个向量组进行线性组合,可能会使向量组的秩减小。}
\end{frame}

\begin{frame}
\begin{xingzhi}
  设$\MA$为$m\times n$矩阵,$\MP,\MQ$分别为$m$阶、$n$阶可逆矩阵,则
  $$
  \rank(\MA) = \rank(\MP\MA) = \rank(\MA\MQ)  = \rank(\MP\MA\MQ).
  $$
\end{xingzhi}\pause 
\begin{proof}
  以下给出两种证明方法:
\begin{itemize}
\item[法一] 
  可逆矩阵$\MP,~\MQ$可表示为若干个初等矩阵的乘积,而初等变换不改变矩阵的秩,故结论成立。  \pause 
\item[法二]
  因
  $$
  \rank(\MA) = \rank(\MP^{-1}(\MP\MA)) \le \rank(\MP\MA) \le \rank(\MA)
  $$
  故
  $$
  \rank(\MA) = \rank(\MP\MA).
  $$
  
  同理可证其他等式。
\end{itemize}
\end{proof}
\end{frame}

\begin{frame}
\begin{li}
  设$\MA$是$m\times n$矩阵,且$m<n$,证明:$|\MA^T\MA|=0$.
\end{li} \pause 
\begin{jie}
  首先$\rank(\MA)=\rank(\MA^T)\le \min\{m,n\}<n$,另外
  $$
  \red{\rank(\MA^T\MA) \le \min\left\{\rank(\MA^T),~\rank(\MA)\right\}} < n,
  $$
而$\MA^T\MA$为$n$阶矩阵,故$|\MA^T\MA|=0$。
\end{jie}
\end{frame}


\begin{frame}
  \begin{center}
    小结(矩阵的秩)
    $$
    \red{
      \begin{aligned}
        \rank(\MA,\MB) &\le \rank(\MA)+\rank(\MB),\\
        \rank(\MA+\MB) &\le \rank(\MA)+\rank(\MB),\\
        \rank(\MA\MB)  &\le \min(\rank(\MA),~\rank(\MB)).
      \end{aligned}
    }
    $$
  \end{center}
\end{frame}


\begin{frame}
\begin{dingyi}[矩阵的相抵]
  若矩阵$\MA$经过初等变换化为$\MB$(\purple{亦即存在可逆矩阵$\MP$和$\MQ$使得$\MP\MA\MQ=\MB$}),就称$\MA$\red{相抵于}$\MB$,记作$\MA\cong\MB$
\end{dingyi}

\begin{xingzhi}[相抵关系的性质]
  \begin{itemize}
  \item 反身性
    $$
    \MA\cong\MA
    $$
  \item 对称性
    $$
    \MA\cong\MB ~~\Rightarrow~~ \MB\cong\MA
    $$
  \item 传递性
    $$
    \MA\cong\MB,~~\MB\cong\MC ~~\Rightarrow~~ \MA\cong\MC
    $$
  \end{itemize}
\end{xingzhi}
\end{frame}



\begin{frame}
\begin{dingli}
  若$\MA$为$m\times n$矩阵,且$\rank(\MA)=r$,则一定存在可逆的$m$阶矩阵$\MP$和$n$阶矩阵$\MQ$使得
  $$
  \MP\MA\MQ=\left(
    \begin{array}{cc}
      \MI_r&\M0\\
      \M0&\M0
    \end{array}
  \right)_{m\times n} = \MU.
  $$
\end{dingli}\pause 
\begin{proof}
对$\MA$做初等行变换,可将$\MA$化为有$r$个非零行的行最简阶梯形矩阵,即存在初等矩阵$\MP_1,\MP_2,\cd,\MP_s$使得
$$
\MP_s\cd\MP_2\MP_1\MA=\MU_1.
$$

对$\MU_1$做初等列变换可将$\MU_1$化为$\MU$,即存在初等矩阵$\MQ_1,\MQ_2,\cd,\MQ_t$使得
$$
\MU_1\MQ_1\MQ_2\cd\MQ_t=\MU
$$
记
$
\purple{\MP_s\cd\MP_2\MP_1=\MP, ~~\MQ_1\MQ_2\cd\MQ_t=\MQ,}
$
则有
$
\MP\MA\MQ=\MU.
$
\end{proof}
\end{frame}

\begin{frame}
\begin{dingyi}[相抵标准形]
  设$\rank(\MA_{m\times n})=r$,则矩阵
  $$
  \left(
    \begin{array}{cc}
      \MI_r&\M0\\
      \M0&\M0
    \end{array}
  \right)_{m\times n} 
  $$称为$\MA$的\blue{相抵标准形},简称\blue{标准形}。
\end{dingyi}
\begin{itemize}
\item 秩相等的同型矩阵,必有相同的标准形。
\item 两个秩相等的同型矩阵是相抵的。
\end{itemize}

\end{frame}

\begin{frame}
\begin{li}
  设$\MA$为$m\times n$矩阵($m>n$),$\rank(\MA)=n$,证明:存在$n\times m$矩阵$\MB$使得
  $$
  \MB\MA=\MI_n.
  $$
\end{li}\pause 
\begin{proof}
由上述定理可知,存在$m$阶可逆矩阵$\MP$使得
$$
\MP\MA=\left(
  \begin{array}{c}
    \MI_n\\
    \M0_1
  \end{array}
\right)     
~~~\Rightarrow~~~
\MP\MA=\left(
  \begin{array}{c}
    \MI_n\\
    \M0_1
  \end{array}
\right)
$$
其中$\M0_1$为$(m-n)\times n$零矩阵。
取
$$
\MC=(\MI_n~~\M0_2),
$$
其中$\M0_2$为$n\times(m-n)$阶零矩阵, 则
$$
\MC\MP\MA=(\MI_n~~\M0_2)\left(
  \begin{array}{c}
    \MI_n\\
    \M0_1
  \end{array}
\right)=\MI_n.
$$ 
故存在$\MB=\MC\MP$使得
$
\MB\MA=\MI_n.
$
\end{proof}
\end{frame}

\begin{frame}
\begin{li}
  设
  $$
  \begin{aligned}
    \alphabd_1=(1,3,1,2), &~\alphabd_2=(2,5,3,3), \\
    \alphabd_3=(0,1,-1,a),& ~\alphabd_4=(3,10,k,4),
  \end{aligned}
  $$
  试求向量组$\alphabd_1,~\alphabd_2,~\alphabd_3,~\alphabd_4$的秩,并将$\alphabd_4$用$\alphabd_1,~\alphabd_2,~\alphabd_3$线性表示。
\end{li}
\end{frame}

\begin{frame}[allowframebreaks]
\begin{jie}
将4个向量按列排成一个矩阵$\MA$,做初等变换将其化为阶梯形矩阵$\MU$,即
$$
\MA=\left(
  \begin{array}{rrrr}
    1&2&0&3\\
    3&5&1&10\\
    1&3&-1&k\\
    2&3&a&4
  \end{array}
\right) \xlongrightarrow[]{\mbox{初等行变换}}
\left(
  \begin{array}{rrcc}
    1&2&0&3\\
    0&-1&1&1\\
    0&0&a-1&-3\\
    0&0&0&k-2
  \end{array}
\right)
$$
\begin{itemize}
\item[(1)] 当$a=1$或$k=2$时,$\MU$只有3个非零行,故
  $$
  \red{\rank(\alphabd_1,\alphabd_2,\alphabd_3,\alphabd_4)=\rank(\MA)=3.}
  $$ 
\item[(2)]  当$a\ne1$且$k\ne2$时,
  $\red{\rank(\alphabd_1,\alphabd_2,\alphabd_3,\alphabd_4)=\rank(\MA)=4.}$
\item[(3)] 当$k=2$且$a\ne1$时,$\alphabd_4$可由$\alphabd_1,~\alphabd_2,~\alphabd_3$线性表示,
  且
  $$
  \alphabd_4=-\frac{1+5a}{1-a}\alphabd_1+\frac{2+a}{1-a}\alphabd_2+\frac{3}{1-a}\alphabd_3.
  $$
\item[(4)]  当$k\ne2$或$a=1$时,$\alphabd_4$不能由$\alphabd_1,~\alphabd_2,~\alphabd_3$线性表示。
\end{itemize}
\end{jie}
\end{frame}

\begin{frame}
\begin{li}
  设
  $$
  \MA=\left(
    \begin{array}{rrr}
      1&2&1\\
      2&2&-2\\
      -1&t&5\\
      1&0&-3
    \end{array}
  \right)
  $$
  已知$\rank(\MA)=2$,求$t$。
\end{li}\pause 
\begin{jie}
$$
\MA \xlongrightarrow[]{\mbox{初等行变换}} \left(
  \begin{array}{ccr}
    1&2&1\\
    0&-2&-4\\
    0&2+t&6\\
    0&0&0
  \end{array}
\right)=\MB
$$ 
由于$\rank(\MB)=\rank(\MA)=2$,故$\MB$中第2、3行必须成比例,即
$$
\frac{-2}{2+t}=\frac{-4}6,
$$
即得$t=1$。
\end{jie}
\end{frame}






% \begin{li}
%   已知$\rank(\MB)=2$,
%   $$
%   \MA=\left(
%     \begin{array}{ccc}
%       1&2&0\\
%       0&a&1\\
%       1&3&b
%     \end{array}
%   \right).
%   $$
%   问
%   \begin{itemize}
%   \item[(1)] $a,b$满足什么条件时,$\rank(\MA\MB)=2$;
%   \item[(2)] $\MA$与$\MB$满足什么条件时,$\rank(\MA\MB)=1$。
%   \end{itemize}
% \end{li}
% \begin{jie}
% \begin{itemize}
% \item[(1)] 当$\MA$可逆时,$\rank(\MA\MB)=\rank(\MB)=2$。此时,
%   $$
%   |\MA| = ab-1\ne 0.
%   $$
% \item[(2)] 当$ab-1=0$时,$\MA$不可逆,且$\rank(\MA)=2$。
%   $\MA$的列向量组线性相关,故$\MA\x=\M0$有非零解。
% \end{itemize}
% \end{jie}
