
\section{\lstinline|limits.h|的使用}
\begin{frame} \ft{\secname}
limits.h头文件中确定了char、int和long等类型的各种属性。
\end{frame}

\begin{frame}\ft{\secname}
\begin{biancheng} 
编制程序,显示char、int和long等类型的各种属性。
\end{biancheng}
\end{frame}

\begin{frame}[fragile,allowframebreaks]\ft{\secname}
\lstinputlisting{ch03/code/ex01.c}
\end{frame}


\section{\lstinline|float.h|的使用}
\begin{frame}\ft{\secname}
\begin{biancheng} 
编写程序,查看float、double和long double数据的字节大小、表示范围以及精度。
\end{biancheng}
\end{frame}

\begin{frame}[fragile,allowframebreaks]\ft{\secname}
\lstinputlisting{ch03/code/ex02.c}
\end{frame}

\section{\lstinline|sizeof|运算符的使用}
\begin{frame}\ft{\secname}
  \begin{itemize}
  \item \lstinline|sizeof|是运算符,不是函数。 \\[.1in]
  \item [] 
    虽然我们习惯写成\lstinline|sizeof()|的形式,但\lstinline|()|并不是必须的,它只是表示优先级。\\[.1in]
  \item [] 约定\lstinline|sizeof|后面的目标叫\lstinline|sizeof|对象。
  \end{itemize}
\end{frame}

\begin{frame}[fragile]\ft{\secname}
  \begin{itemize}
  \item 若\lstinline|sizeof|对象是表达式,则它求的是该表达式的值的类型大小。如
\begin{lstlisting}
char c = 1;
int i = 2;
cout << sizeof(c + i) << endl;
cout << sizeof(c = c + i) << endl;
\end{lstlisting}  
\item 若\lstinline|sizeof|对象是函数,求的则是函数返回值的类型大小。  
  \end{itemize}
\end{frame}


\begin{frame}[fragile]\ft{\secname}
  \lstinputlisting{ch03/code/sizeof_a_function.c}
\end{frame}

\begin{frame}[fragile]\ft{\secname}
\begin{itemize}
  \item 若\lstinline|sizeof|对象是数组,则求的是数组总大小;\\[.1in]
  \item 若\lstinline|sizeof|对象是指针,则求的是指针本身的大小,而不是所指向的内存空间的大小。\\[.1in]
  \item 当数组名作为实参传入函数时,会自动转化为指针类型。
  \end{itemize}
\end{frame}

\begin{frame}[fragile]\ft{\secname}
  \lstinputlisting{ch03/code/sizeof_array_and_pointer.c}
\end{frame}

\section{上溢与下溢}
\begin{frame}\ft{\secname}
\begin{biancheng} 
编写程序观察系统整型数上溢和下溢的情况。
\end{biancheng}
\end{frame}

\begin{frame}\ft{\secname}
\begin{biancheng} 
编写程序观察系统浮点数上溢和浮点数下溢的情况。
\end{biancheng}
\end{frame}




\section{\lstinline|char|型数据的打印}
\begin{frame}\ft{\secname}
\begin{biancheng} 
编写一个程序,输入一个ASCII码值,然后输出相应的字符。
\end{biancheng}
\end{frame}


\section{\lstinline|float|型数据的打印}
\begin{frame}[fragile]\ft{\secname}
\begin{biancheng} 
编写一个程序,读入一个浮点数,并分别以小数形式和指数形式打印。
输出应如同下面格式:
\begin{lstlisting}
The input is 21.290000 or 2.129000e+001.
\end{lstlisting}
\end{biancheng}
\end{frame}


\begin{frame}[fragile]\ft{\secname}
\begin{biancheng} 
一年约有$1.156\times 10^7$秒。编写一个程序,要求输入您的年龄,然后显示该年龄有多少秒。
\end{biancheng}
\end{frame}


\begin{frame}[fragile]\ft{\secname}
\begin{biancheng} 
编写一个程序,输入圆的半径,然后显示圆的面积和周长。
\end{biancheng}
\end{frame}

