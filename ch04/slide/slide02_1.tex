\subsection{$\MR^n$中向量的内积,欧式空间}

\begin{frame}  
  \begin{dingyi}
    在$\MR^n$中,对于$\alphabd=(a_1,a_2,\cd,a_n)^T$和$\betabd=(b_1,b_2,\cd,b_n)^T$,规定$\alphabd$和$\betabd$的内积为
    $$
    (\alphabd,\betabd)=a_1b_1+a_2b_2+\cd+a_nb_n.
    $$
  \end{dingyi}
  当$\alphabd$和$\betabd$为列向量时,
  $$
  (\alphabd,\betabd)=\alphabd^T\betabd=\betabd^T\alphabd.
  $$
  
\end{frame}

\begin{frame}
  
  \begin{xingzhi}[内积的运算性质]
    对于$\alphabd,\betabd,\gammabd\in\MR^n$和$k\in\MR$,
    \begin{itemize}
    \item[(i)]   $(\alphabd,\betabd)=(\betabd,\alphabd)$
    \item[(ii)]  $(\alphabd+\betabd,\gammabd)=(\alphabd,\gammabd)+(\betabd,\gammabd)$
    \item[(iii)] $(k\alphabd,\betabd)=k(\alphabd,\betabd)$
    \item[(iv)]  $(\alphabd,\alphabd)\ge0$, 等号成立当且仅当$\alphabd=\M0$.
    \end{itemize}
  \end{xingzhi}
  \pause
  \begin{dingyi}[向量长度]
    向量$\alphabd$的长度定义为
    $$
    \|\alphabd\|=\sqrt{(\alphabd,\alphabd)}
    $$
  \end{dingyi}
  
\end{frame}


\begin{frame}
  
  \begin{dingli}[柯西-施瓦茨(Cauchy-Schwarz)不等式]
    $$
    |(\alphabd,\betabd)|\le\|\alphabd\|\|\betabd\|
    $$
  \end{dingli}
  \pause 
  \proofname
  $\forall t \in \MR$,有
  $$
  (\alphabd+t\betabd,\alphabd+t\betabd) \ge 0
  $$
  即
  $$
  (\betabd,\betabd)t^2+2(\alphabd,\betabd)t+(\alphabd,\alphabd)\ge0
  $$
  此为关于$t$的二次函数,由一元二次方程理论可知
  $$
  \Delta = b^2-4ac = 4 (\alphabd,\betabd)^2-4(\alphabd,\alphabd)(\betabd,\betabd)\le 0
  $$
  即
  $$
  (\alphabd,\betabd)^2\le (\alphabd,\alphabd)(\betabd,\betabd)
  $$
  亦即
  $$
  |(\alphabd,\betabd)|\le\|\alphabd\|\|\betabd\|
  $$
  
\end{frame}

\begin{frame}
  
  \begin{dingyi}[向量之间的夹角]
    向量$\alphabd,\betabd$之间的夹角定义为
    $$
    <\alphabd,\betabd>=\arccos\frac{(\alphabd,\betabd)}{\|\alphabd\|\|\betabd\||}
    $$
  \end{dingyi}
  \pause
  \begin{dingli}
    $$\alphabd\perp\betabd ~~\Longleftrightarrow~~
    (\alphabd,\betabd)=0
    $$
  \end{dingli}
  \pause
  注意:零向量与任何向量的内积为零,从而零向量与任何向量正交。
  
\end{frame}



\begin{frame}
  
  \begin{dingli}[三角不等式]
    $$
    \|\alphabd+\betabd\|\le\|\alphabd\|+\|\betabd\|.
    $$
  \end{dingli}
  \pause\proofname
  $$
  \begin{array}{rl}
    (\alphabd+\betabd,\alphabd+\betabd)
    &= (\alphabd,\alphabd)+2(\alphabd,\betabd)+(\betabd,\betabd)\\[0.1in]
    &\le (\alphabd,\alphabd)+2|(\alphabd,\betabd)|+(\betabd,\betabd) \\[0.1in]
    &\le \|\alphabd\|^2+2\|\alphabd\|\|\betabd\|+\|\betabd\|^2 \\[0.1in]
  \end{array}
  $$
  
  \pause
  注意:当$\alphabd\perp\betabd$时,$\|\alphabd+\betabd\|=\|\alphabd\|+\|\betabd\|$。
  
\end{frame}


\begin{frame}
  
  \begin{dingyi}[欧几里得空间]
    定义了内积运算的$n$维实向量空间,称为$n$维欧几里得空间(简称欧氏空间),仍记为$\MR^n$。
  \end{dingyi}
  
\end{frame}
