\subsection{线性变换的运算}
\begin{frame}
\begin{dingyi}
  设$\sigmabd$与$\taubd$是线性空间$V(F)$的两个线性变换,$\lambda\in F$,定义
  $$
  \begin{array}{rl}
    (\sigmabd+\taubd)(\alphabd)&=\sigmabd(\alphabd)+\taubd(\alpha),\\
    (\lambda\sigmabd)(\alphabd)&=\lambda\sigmabd(\alphabd), \\
    (\sigmabd\taubd)(\alphabd)&=\sigmabd(\taubd(\alphabd))
  \end{array}
  $$,
\end{dingyi}
\end{frame}

\begin{frame}
  可以验证上述定义的\blue{$\sigmabd+\taubd,\lambda\sigmabd,\sigmabd\taubd$仍是$V(F)$的线性变换。}\vspace{.1in} \pause 


  以$\sigmabd\taubd$为例,
  $\forall \alphabd_1,\alphabd_2\in V, k_1,k_2\in F$,
  $$
  \begin{array}{rl}
    (\sigmabd\taubd)(k_1\alphabd_1+k_2\alphabd_2)&=\sigmabd(\taubd(k_1\alphabd_1+k_2\alphabd_2))\\[.1in]
                                                 &=\sigmabd(k_1\taubd(\alphabd_1)+k_2\taubd(\alphabd_2))\\[.1in]
                                                 &=k_1\sigmabd(\taubd(\alphabd_1))+k_2\sigmabd(\taubd(\alphabd_2))\\[.1in]
                                                 &=k_1(\sigmabd\taubd)(\alphabd_1)+k_2(\sigmabd\taubd)(\alphabd_2).
  \end{array}
  $$
\end{frame}

\begin{frame}
  \begin{dingli}
    设线性空间$V(F)$的线性变换$\sigmabd$与$\taubd$在$V$的基$\{\alphabd_1,\cd,\alphabd_n\}$下对应的矩阵分别为$\MA$和$\MB$,则\blue{$\sigmabd+\taubd, \lambda\sigmabd$和$\sigmabd\taubd$}在该组基下对应的矩阵分别为\red{$\MA+\MB, \lambda \MA$和$\MA\MB$}。
  \end{dingli}
  \vspace{.1in}\pause 

  \begin{proof}
    由
    $$
    \sigmabd(\alphabd_1,\cd,\alphabd_n)=(\alphabd_1,\cd,\alphabd_n)\MA, ~~
    \taubd(\alphabd_1,\cd,\alphabd_n)=(\alphabd_1,\cd,\alphabd_n)\MB
    $$
    可知
    $$
    \sigmabd(\alphabd_j)=\sum_{j=1}^na_{ij}\alphabd_j, ~~
    \taubd(\alphabd_j)=\sum_{j=1}^nb_{ij}\alphabd_j,
    $$
    于是
    $$
    (\sigmabd+\taubd)(\alphabd_j)=\sigmabd(\alphabd_j)+\taubd(\alphabd_j)
    =\sum_{j=1}^na_{ij}\alphabd_j+\sum_{j=1}^nb_{ij}\alphabd_j
    =\sum_{j=1}^n\blue{(a_{ij}+b_{ij})}\alphabd_j.
    $$
    这表明\blue{$\sigmabd+\taubd$所对应的矩阵是为$\MA+\MB$}。
    $$
    \begin{aligned}
      (\sigmabd \taubd)(\alphabd_j)&=\sigmabd(\taubd(\alphabd_j))
      =\sigmabd(\sum_{j=1}^nb_{ij}\alphabd_j)
      =\sum_{j=1}^nb_{ij}\sigmabd(\alphabd_j)\\
      &=\sum_{j=1}^nb_{ij}\left(\sum_{k=1}^na_{ki}\alphabd_k\right)=\sum_{k=1}^n\blue{\left(\sum_{j=1}^na_{ki}b_{ij}\right)}\alphabd_k
    \end{aligned}
    $$
    这表明\blue{$\sigmabd\taubd$所对应的矩阵是为$\MA\MB$}。
  \end{proof}
\end{frame}

\begin{frame}
  \begin{dingyi}
    如果线性变换$\sigmabd$对应的矩阵$\MA$为可逆矩阵,则称$\sigmabd$是\red{可逆的线性变换}。$\sigmabd$可逆也可定义为:如果存在线性变换$\taubd$使得
    $$
    \sigmabd\taubd=\taubd\sigmabd=\MI
    $$
    则称$\sigmabd$为\red{可逆的线性变换}。
  \end{dingyi}
\end{frame}

% \begin{frame}

% \end{frame}

% \begin{frame}

% \end{frame}


% \begin{frame}

% \end{frame}

% \begin{frame}

% \end{frame}

% \begin{frame}

% \end{frame}


% \begin{frame}

% \end{frame}

% \begin{frame}

% \end{frame}

% \begin{frame}

% \end{frame}


% \begin{frame}

% \end{frame}

% \begin{frame}

% \end{frame}

% \begin{frame}

% \end{frame}
