\section{线性子空间}
\begin{frame}
对于数域$F$上的线性空间$V$(记为$V(F)$),其子集合$W$关于$V(F)$中的两种运算可能是封闭的,也可能是不封闭的。
\end{frame}

\begin{frame}
考虑$\R^3$的两个子集合
$$
\begin{array}{l}
  W_1=\{(x_1,x_2,x_3)~|~x_1-x_2+5x_3=0\},\\
  W_2=\{(x_1,x_2,x_3)~|~x_1-x_2+5x_3=1\}.
\end{array}
$$
显然,
\begin{itemize}
\item $W_1$是过原点的平面$x_1-x_2+5x_3=0$上的全体向量;
\item $W_2$是不过原点的平面$x_1-x_2+5x_3=1$上的全体向量。
\end{itemize}\pause 

容易验证,$W_1$关于向量的加法和数乘是封闭的,而$W_2$对这两种运算不封闭。
\end{frame}

\begin{frame}
  \begin{dingyi}[线性子空间]
    设$V(F)$是一个线性空间,$W$是$V$的一个非空子集合。如果$W$对$V(F)$中定义的\blue{线性运算}也构成数域$F$上的一个线性空间,则称$W$为$V(F)$上的一个\red{线性子空间}(简称\red{子空间})。
  \end{dingyi}
\end{frame}

\begin{frame}
\begin{dingli}
  线性空间$V(F)$的非空子集合$W$为$V$的子空间的充分必要条件是$W$对于$V$的两种运算封闭。
\end{dingli}\pause

\begin{proof}
\begin{itemize}
\item[$\Rightarrow$]显然成立。\\

\item[$\Leftarrow$] 只需验证$W$中的向量满足线性空间定义的$8$条规则。\pause 
  \vspace{.1in} 

  由于$W$是$V$的非空子集,故规则(1),(2),(5),(6),(7),(8)显然成立。\pause 
  \vspace{.1in} 
  
  因此只需证明
  \begin{itemize}
    \item $\M0 \in W$
    \item $\forall \alphabd, ~ -\alphabd \in W$
  \end{itemize} \vspace{.1in} \pause 

  事实上,由于$W$对数乘封闭,即$\forall \lambda \in F, \forall \alphabd \in W$,均有$\lambda\alphabd \in W$。取$\lambda=0, -1$,即有
  $$
  \M0 = 0\alphabd = \M0 \in W, ~~
  -\alphabd = (-1)\alphabd = \in W. 
  $$

\end{itemize}
\end{proof}
\end{frame}

\begin{frame}
  \begin{li}
    在线性空间$V$中,
    \begin{itemize}
    \item 由单个的零向量组成的子集合$\{\M0\}$是$V$的一个子空间,称为\red{零子空间};
    \item $V$本身也是$V$的一个子空间,
    \end{itemize}
    这两个子空间都称为$V$的\red{平凡子空间},而$V$的其他子空间称为\red{非平凡子空间}。
  \end{li}
\end{frame}

\begin{frame}
\begin{li}
  设$\MA\in F^{m\times n}$,则$\MA\vx=\M0$的解集合
  $$
  S = \{\vx ~|~ \MA \vx = \M0\}
  $$
  是$F^n$的一个子空间,称为齐次线性方程组的解空间(也称矩阵$\MA$的零空间,记作$\mathcal N(A)$)。 \vspace{.1in} \pause 

  \blue{注:非齐次线性方程组$\MA\vx=\vb$的解集合不是$F^n$的子空间。}

\end{li}
\end{frame}

\begin{frame}
\begin{li}
  全体$n$阶实数量矩阵、实对角矩阵、实对称矩阵、实上(下)三角矩阵分别组成的集合,都是$\R^{n\times n}$的子空间。
\end{li}
\end{frame}

\begin{frame}
  \begin{li}
    设$\R^3$的子集合
    $$
    \begin{array}{l}
      V_1=\{(x_1,0,0)~|~x_1\in \R\},
      V_2=\{(1,0,x_3)~|~x_3\in \R\},
    \end{array}
    $$
    则$V_1$是$\R^3$的子空间,$V_2$不是$\R^3$的子空间。 \vspace{.1in}
    \pause 

    \blue{注:在$\R^3$中,
      \begin{itemize}
        \item 凡是过原点的平面或直线上的全体向量组成的子集合都是$\R^3$的子空间;
        \item 凡是不过原点的平面或直线上的全体向量组成的子集合都不是$\R^3$的子空间。
      \end{itemize}
    }
  \end{li}
\end{frame}

\begin{frame}
  \begin{dingli}
    设$V$是数域$F$上的线性空间,$S$是$V$的一个非空子集合,则$S$中的一切向量组的所有线性组合组成的集合
    $$
    W = \{k_1\alphabd_1+\cd+k_m\alphabd_m~|~\alphabd_i\in S, ~k_i\in F, ~i=1,\cd,m\}
    $$
    是$V$中包含$S$的最小的子空间。
  \end{dingli}\pause 
  \begin{proof}
    $W$显然包含$S$,设$\alphabd,\betabd\in W$,则存在$\alphabd_1,\cd,\alphabd_m,\betabd_1,\cd,\betabd_n\in S$及$k_1,\cd,k_m,l_1,\cd,l_n\in F$,使得
    $$
    \begin{array}{l}
      \alphabd = k_1\alphabd_1+\cd+k_m\alphabd_m, \quad 
      \betabd = l_1\betabd_1+\cd+l_n\betabd_m
    \end{array}
    $$
    于是
    $$
    \alphabd+\betabd=(k_1\alphabd_1+\cd+k_m\alphabd_m)+(l_1\betabd_1+\cd+l_n\betabd_m)\in W
    $$
    $\forall k\in F$,有
    $$
    k\alphabd=k(k_1\alphabd_1+\cd+k_m\alphabd_m)=kk_1\alphabd_1+\cd+kk_m\alphabd_m\in W,
    $$
    故$W$是$V$的一个子空间。
    \vspace{.1in}\pause
    
    再设$W^*$是$V$中包含$S$的任一子空间,则
    $$
    \forall \alpha = k_1\alphabd_1+\cd+k_m\alphabd_m \in W.
    $$
    由于$\alphabd_1,\cd,\alphabd_m\in S\subset W^*$,故必有$\alphabd\in W^*$,从而有$W\subset W^*$,因此$W$是$V$中包含$S$的最小子空间。
  \end{proof}
\end{frame}

\begin{frame}
以上定理中的$W$称为\blue{由$V$的非空子集$S$生成的子空间}。


特别地,当$S$为有限子集$\{\alphabd_1,\cd,\alphabd_m\}$时,记
$$
W = L(\alphabd_1,\cd,\alphabd_m) \mbox{  或  } W = span\{\alphabd_1,\cd,\alphabd_m\}
$$
为由向量组$\alphabd_1,\cd,\alphabd_m$生成的子空间。\vspace{.1in}\pause 


\begin{li}
\begin{itemize}
	\item $\MA\vx = \M0$的解空间是由它的基础解系生成的子空间;
	\item $\R^3$中任一个过原点的平面上的全体向量所构成的子空间,是由该平面上任意两个线性无关的向量生成的子空间。
\end{itemize}

\end{li}
\end{frame}

\begin{frame}
\begin{dingli}
	设$W_1,W_2$是数域$F$上的线性空间$V$上的两个子空间,且
	$$
	W_1=L(\alphabd_1,\cd,\alphabd_s),~W_2=L(\betabd_1,\cd,\betabd_t),
	$$
	则$W_1=W_2$的充分必要条件是两个向量组$\alphabd_1,\cd,\alphabd_s$和$\betabd_1,\cd,\betabd_t$等价。
\end{dingli}\vspace{.1in}\pause 

\begin{proof}
\begin{itemize}
	\item[$\Rightarrow$] 显然成立。 \\[.1in] \pause 
	\item[$\Leftarrow$]  设$\alphabd = k_1\alphabd_1+\cd+k_s\alphabd_s\in W_1$,由于$\alphabd_i$可由向量组$\betabd_1,\cd,\betabd_t$线性表示,故$\alphabd$也可由向量组$\betabd_1,\cd,\betabd_t$线性表示,即存在$l_1,\cd,l_t\in F$使得
	$$
	\alphabd = l_1\betabd_1+\cd+l_t\betabd_t \in W_2,
	$$
	因此,$W_1\subset W_2$。\vspace{.1in}\pause 
	
	同理可证$W_2\subset W_1$,从而$W_1=W_2$。
\end{itemize}
\end{proof}
\end{frame}

\begin{frame}
\begin{dingyi}
设$W_1,W_2$是线性空间$V$的两个子空间,则$V$的子集合
$$
\begin{array}{ll}
W_1\cap W_2 &= \{\alphabd ~|~ \alphabd \in W_1 \mbox{ 且 }  \alphabd \in W_2\},\\[0.1in]
W_1 +   W_2 &= \{\alphabd_1+\alphabd_2 ~|~ \alphabd_1 \in W_1, ~\alphabd_2\in W_2\}
\end{array}
$$	
分别称为两个子空间的\red{交}与\red{和}。\vspace{.1in}\pause

\red{如果$W_1\cap W_2 = \{\M0\}$,则称$W_1+W_2$为直和,记为\blue{$W_1\oplus W_2$}。}
\end{dingyi}
\end{frame}

\begin{frame}
\begin{dingli}
	线性空间$V(F)$的两个子空间$W_1, W_2$的交与和仍是$V$的子空间。
\end{dingli}

\begin{proof}
只证$W_1+W_2$是$V$的子空间,为此只需证$W_1+W_2$对$V$中的线性运算封闭。\vspace{.1in}\pause

设$\alphabd, \betabd \in W_1+W_2$,即存在$\alphabd_1, \betabd_1 \in W_1; ~ \alphabd_2, \betabd_2 \in W_2$使得
$$
\alphabd = \alphabd_1+\alphabd_2, ~~ 
\betabd = \betabd_1 + \betabd_2,
$$
于是
$$
\alphabd+\betabd=(\alphabd_1+\alphabd_2)+(\betabd_1 + \betabd_2)=(\alphabd_1+\betabd_1)+(\alphabd_2+\betabd_2)\in W_1+W_2,
$$
再设$\lambda\in F$,则
$$
\lambda\alphabd=\lambda(\alphabd_1+\alphabd_2)=\lambda\alphabd_1+\lambda\alphabd_2\in W_1+W_2.
$$
故$W_1+W_2$也是$V$的一个子空间。
\end{proof}

	
\end{frame}

\begin{frame}
\begin{dingyi}
矩阵$\MA$的列(行)向量组生成的子空间,称为矩阵$\MA$的列(行)空间,记为$\mathcal R(A)$($\mathcal R(A^T)$)。
\end{dingyi}\vspace{.1in}\pause

若$\MA \in \R^{m\times n}$,则
\begin{itemize} 
	\item $\MA$的列向量组为
	$$
	\betabd_1,\cd,\betabd_n\in \R^m
	$$
	\item $\MA$的行向量组为
	$$
	\alphabd_1,\cd,\alphabd_m\in \R^n
	$$
\end{itemize}
\vspace{.1in}\pause

于是
\begin{itemize} 
\item $\mathcal R(A)=L(\betabd_1,\cd,\betabd_n)$是$\R^m$的一个子空间;
\item $\mathcal R(A^T)=L(\alphabd_1,\cd,\alphabd_m)$是$\R^n$的一个子空间。
\end{itemize}
\end{frame}

\begin{frame}
$$
\begin{array}{ll}
& \mbox{非齐次线性方程组$\MA\vx = \vb$有解} \\[.1in]
\Leftrightarrow & \mbox{$\vb$是$\MA$的列向量组的线性组合}\\[.1in]
\Leftrightarrow & \mbox{$\vb$属于$\MA$的列空间,即$\vb \in \mathcal R(\MA)$}
\end{array}
$$
\end{frame}

\begin{frame}
\begin{dingyi}
设$\alphabd \in \R^n$,$W$是$\R^n$的一个子空间。如果对于任意的$\gammabd \in W$,均有
$$
(\alphabd, \gammabd) = \M0,
$$
则称$\alphabd$与子空间$W$正交,记作$\alphabd \perp W$。
\end{dingyi}	\vspace{.1in}\pause


\begin{dingyi}
	设$V$和$W$是$\R^n$的两个子空间。如果对于任意的$\alphabd \in V, \betabd \in W$,均有
	$$
	(\alphabd, \betabd) = \M0,
	$$
	则称$V$与$W$正交,记作$V \perp W$。
\end{dingyi}

\end{frame}

\begin{frame}

\begin{li}
对于齐次线性方程组$\MA\vx=\M0$,其每个解向量与系数矩阵$\MA$的每个行向量都正交,故解空间与$\MA$的行空间是正交的,即
$$
\mathcal N(\MA) \perp \mathcal R(A^T).
$$
\end{li}

\end{frame}

\begin{frame}
\begin{dingli}
$\R^n$中与子空间$V$正交的全部向量所构成的集合
$$
W=\{\alphabd ~|~ \alphabd \perp V, ~ \alphabd \in \R^n \}
$$
是$\R^n$的一个子空间。
\end{dingli}\vspace{.1in}\pause

\begin{proof}
因$\M0$与任何子空间正交,故$W$是非空集合。设$\alphabd_1,\alphabd_2\in W$,于是$\forall \gammabd \in W$,都有
$$
(\alphabd_1, \gammabd) = \M0, \quad
(\alphabd_2, \gammabd) = \M0,
$$
从而
$$
(\alphabd_1+\alphabd_2, \gammabd) = \M0, \quad
(k\alphabd_1, \gammabd) = \M0 (k \in \R),
$$
所以$(\alphabd_1+\alphabd_2)\perp V, k\alphabd_1\perp V$,即$\alphabd_1+\alphabd_2\in W, k\alphabd_1\in W$,故$W$是$\R^n$的一个子空间。
\end{proof}

\end{frame}

\begin{frame}
\begin{dingyi}
$\R^n$中与子空间$V$正交的全体向量构成的子空间$W$,称为$V$的\red{正交补},记为$W=V^\perp$。
\end{dingyi}\vspace{.1in}\pause

\begin{li}
$\MA \vx = \M0$的解空间$\mathcal N(\MA)$由与$\MA$的行向量都正交的全部向量构成,故
$$
\mathcal N(A) = \mathcal R(A^T)^\perp. 
$$
这是$\MA \vx = \M0$的解空间的一个基本性质。
\end{li}
\end{frame}