\section{编译C程序的工作原理}
\begin{frame}\ft{\secname}
C是一种高级语言,它需要编译器将其转换为可执行代码,以使得程序能在机器上运行。 

以下介绍在MAC或Linux上使用gcc编译器的几个步骤。

\end{frame}


\begin{frame}[fragile]\ft{\secname}
\begin{itemize}

\item[(1)] 用编辑器(vi,emacs等)创建一个C程序,并将其保存为 \lstinline|add2num.c|。

\begin{lstlisting}
$ emacs add2num.c
\end{lstlisting} 

\pause 
\lstinputlisting[]{ch01/code/add2num.c}

\end{itemize}
\end{frame}

\begin{frame}[fragile]\ft{\secname}
\begin{itemize}
\item[(2)] 编译 
\begin{lstlisting}
$ gcc -wall add2num.c –o add2num
$ ls
add2num    add2num.c
\end{lstlisting}
\begin{itemize}
	\item \lstinline|-Wall| \quad 启动所有编译器的警告信息。建议使用该选项以生成更好的代码 \\[.1in]
	\item \lstinline|-o| \quad 指定可执行文件名。若缺省,则可执行文件名将默认为\lstinline|a.out|
\end{itemize} 
\end{itemize}
\end{frame}

\begin{frame}[fragile]\ft{\secname}
\begin{itemize}
\item[(3)] 运行可执行文件。


\begin{lstlisting}
$ ./add2num
Addition is: 9
\end{lstlisting}

\end{itemize}
\end{frame}



\begin{frame}[fragile]\ft{\secname}
编译器将一个C程序转换为一个可执行文件,需经历了4个阶段:
\begin{itemize}
\item 预处理\\[.1in]
\item 编译\\[.1in]
\item 汇编\\[.1in]
\item 链接
\end{itemize}
\end{frame}


\begin{frame}[fragile]\ft{\secname}
执行以下命令,在当前目录下会生成所有的中间文件以及可执行文件。
\begin{lstlisting}
$ gcc -Wall -save-temps filename.c –o filename 
$ ls 
add2num     add2num.c   add2num.o
add2num.bc  add2num.i   add2num.s
\end{lstlisting}

接下来,让我们一个个地来看这些中间文件中的内容。 
\end{frame}


\begin{frame}[fragile]\ft{\secname}

\begin{itemize}
\item[(1)] \blue{预处理} \quad 其输出保存在文件 \lstinline|add2num.i| 中。\\[.1in]

\begin{itemize}
\item 去掉注释\\[.1in]
\item 宏的展开\\[.1in]
\item 头文件的展开\\[.1in]
\end{itemize}
 
\end{itemize}
\pause 

\begin{lstlisting}
$ less add2num.i
$ ls 
...
int printf(const char * restrict, ...) __attribute__((__format__ (__printf__, 1, 2)));
# 499 "/usr/include/stdio.h" 2 3 4
# 3 "add2num.c" 2

int main(void)
{
  int a = 5, b = 4;
  printf("Addition is: %d\n", (a+b));
  return 0;
}
\end{lstlisting}
\end{frame}


%\begin{frame}[fragile]\ft{\secname}
%\red{分析:}在以上内容中,源文件被附加了很多信息,但在末尾代码仍被保留。{\ttfamily printf()}函数中包含了{\ttfamily a+b}而非{\ttfamily add(a,b)},因为宏已被展开。注释被去除掉。{\ttfamily\#include<stdio.h>}没有了,取而代之的是很多代码。因此,头文件已被展开,并且被包含到了源文件中。
%
%\end{frame}


\begin{frame}[fragile]\ft{\secname}
\begin{itemize}
\item[(2)] \blue{编译} \quad  编译 \lstinline|add2num.i|,生成文件 \lstinline|add2num.s| 。该文件为汇编指令,可用以下命令查看:
\end{itemize} 
\begin{lstlisting}
$ less add2num.s
...
        movl    $0, -4(%rbp)
        movl    $5, -8(%rbp)
        movl    $4, -12(%rbp)
        movl    -8(%rbp), %eax
        addl    -12(%rbp), %eax
        movl    %eax, %esi
        movb    $0, %al
        callq   _printf
        xorl    %esi, %esi
...      
\end{lstlisting}
\end{frame}


\begin{frame}[fragile]\ft{\secname}
\begin{itemize}
\item[(3)] \blue{汇编} \quad 通过汇编器将 \lstinline|add2num.s| 转换成 \lstinline|add2num.o|,该文件包含机器指令。\\[.1in]
\begin{itemize}
	\item  该阶段只会将现有代码转换成机器语言,而诸如 \lstinline|printf()| 的函数调用则不会。
\end{itemize}
\end{itemize}
\pause 
\begin{lstlisting}
$ less add2num.o
<CF><FA><ED><FE>^G^@^@^A^C^@^@^@^A^@^@^@^D^@^@^@^@^B^@^@^@ ^@^@^@^@^@^@^Y^@^@^@<88>^A^@^@^@^@^@^@^@^
...      
\end{lstlisting}

\end{frame}


\begin{frame}[fragile]\ft{\secname}
\begin{itemize}
\item[(4)] \blue{连接} \quad 完成所有函数调用及其定义的连接工作。\\[.1in]

\begin{itemize}
	\item 链接器知道所有这些函数在何处执行 \\[.1in]
	\item 链接器也会做一些额外的工作,以添加一些启动和结束程序所需的额外代码 \\[.1in]
\end{itemize} 
\end{itemize}
\pause 


%在命令行中输入以下命令,可看出从目标文件到可执行文件时文件大小的变化。这是因为链接器为我们的程序添加了额外的代码。
\begin{lstlisting}
$ size add2num.o
__TEXT  __DATA  __OBJC  others    dec         hex
  145     0       0       32      177         b1	
  
$ size add2num
__TEXT  __DATA  __OBJC  others    dec         hex
4096      4096    0  4294971392 4294979584 100003000
\end{lstlisting}

\end{frame}
