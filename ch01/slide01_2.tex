\section{计算机基本软件}

\begin{frame}\ft{\secname}
软件是组成计算机系统的重要部分。

\begin{itemize}
\item 
\red{系统软件}\quad 计算机生产厂商提供的基本软件,如
\begin{AutoMultiColItemize}
\item 操作系统
\item 文字处理程序
\item 计算机语言处理程序
\item 数据库管理程序 
\end{AutoMultiColItemize} 
\item \red{应用软件}\quad 
为满足用户不同领域、不同问题的应用需求而提供的那部分软件,如
\begin{AutoMultiColItemize}
	\item 多媒体软件
	\item 社交软件
	\item 办公软件
	\item 商务软件 
\end{AutoMultiColItemize} 
\end{itemize}
\red{系统软件依赖于机器,而应用软件则更接近用户业务。}
\end{frame}

\begin{frame}\ft{操作系统(OS)}
\begin{itemize}
\item 最基本也是最重要的系统软件。\\[0.1in]
\item \red{功能}\\[0.1in]
\begin{itemize}
	\item 管理计算机系统的各种硬件资源;\\[0.1in]
	\item 解释用户对机器的管理命令,使它转换为机器实际的操作。\\[0.1in]
\end{itemize} 
\item \red{常见的操作系统}\\[0.1in]
\begin{itemize}
	\item DOS \\
	\item WINDOWS \\
	\item UNIX(LINUX) \\
	\item OS X \\
\end{itemize}
\end{itemize}
\end{frame}

\begin{frame}\ft{计算机语言}
计算机语言分为:

\begin{itemize}
\item \red{机器语言}\quad 机器能直接识别的语言,是由“1”和“0”组成的一组代码指令。\\[0.1in]
\item \red{汇编语言}\quad 与机器语言指令一一对应的符号指令和简单语法。\\[0.1in]
\item \red{高级语言}\quad 比较接近日常用语,对机器依赖性低,即适用于各种机器的计算机语言,如
\begin{AutoMultiColItemize}
\item Visual Basic
\item Fortran
\item C/C++
\item Java
\item Python
\item matlab
\end{AutoMultiColItemize}
\end{itemize}
\end{frame}

\begin{frame}\ft{计算机语言}

将高级语言翻译为机器语言,有两种方式: \vspace{0.05in}

\begin{itemize}
\item\red{编译} \\[0.1in]
\begin{itemize}
	\item \red{工作原理}\quad 把高级语言程序作为一个整体进行处理,编译后与子程序库连接,形成一个完整的可执行程序 \\[0.1in]
	\item \red{缺点}\quad 编译、链接比较费时\\[0.1in]
	\item \red{优点}\quad 可执行程序运行速度很快 \\[0.1in]
	\item \red{适用语言}\quad Fortran、C/C++ \\[0.1in]
\end{itemize}

\item\red{解释} \\[0.1in]
\begin{itemize}
	\item \red{工作原理}\quad 对高级语言程序逐句解释执行\\[0.1in]
	\item \red{缺点}\quad 运行效率较低\\[0.1in]
	\item \red{优点}\quad 程序设计的灵活性大\\[0.1in]
	\item \red{适用语言}\quad Basic、Python、Matlab
\end{itemize}
\end{itemize}
\end{frame}