\section{矩阵的转置、对称矩阵}
\begin{frame}
\begin{dingyi}[转置矩阵]
  把一个$m\times n$矩阵
  $$
  \MA = \left(
    \begin{array}{cccc}
      a_{11} & a_{12} & \cd & a_{1n} \\
      a_{21} & a_{22} & \cd & a_{2n} \\
      \vd   & \vd &  & \vd \\
      a_{m1} & a_{m2} & \cd & a_{mn} 
    \end{array}
  \right)
  $$
  的行列互换得到的一个$n\times m$矩阵,称之为$\MA$的\red{转置矩阵},记为$\MA^T$或$\MA^\prime$,即
  $$
  \MA^\prime = \left(
    \begin{array}{cccc}
      a_{11} & a_{21} & \cd & a_{m1} \\
      a_{12} & a_{22} & \cd & a_{m2} \\
      \vd   & \vd &  & \vd \\
      a_{1n} & a_{2n} & \cd & a_{mn} 
    \end{array}
  \right).
  $$  
\end{dingyi}
\end{frame}

\begin{frame}
\begin{dingli}[矩阵转置的运算律]
  \begin{itemize}
  \item[(i)] $(\MA^T)^T=\MA$
  \item[(ii)] $(\MA+\MB)^T=\MA^T+\MB^T$
  \item[(iii)] $(k\MA)^T= k\MA^T$
  \item[(iv)] $(\MA\MB)^T=\MB^T\MA^T$
  \end{itemize}
\end{dingli}
\end{frame}

\begin{frame}
\begin{proof}
  只证(iv)。 设$\MA=(a_{ij})_{m\times n}, \MB=(b_{ij})_{n\times s}, \MA^T=(a_{ij}^T)_{n\times m}, \MB^T=(b_{ij}^T)_{s\times n}$,
  注意到
  $$a_{ij} = a_{ji}^T, b_{ij} = b_{ji}^T,$$  
  有
  $$
  (\MB^T\MA^T)_{ji} = \sum_{k=1}^n b_{jk}^Ta_{ki}^T  = \sum_{k=1}^n a_{ik}b_{kj}  = (\MA\MB)_{ij}  = (\MA\MB)_{ji}^T,
  $$ 
  于是$(\MA\MB)^T=\MB^T\MA^T$.
\end{proof}
\end{frame}

\begin{frame}
\begin{dingyi}[对称矩阵、反对称矩阵]
  设
  $$
  \MA = \left(
    \begin{array}{cccc}
      a_{11} & a_{12} & \cd & a_{1n} \\
      a_{21} & a_{22} & \cd & a_{2n} \\
      \vd   & \vd &  & \vd \\
      a_{n1} & a_{n2} & \cd & a_{nn} 
    \end{array}
  \right)
  $$
  是一个$n$阶矩阵。
  \begin{itemize}
  \item[1]
    如果
    $$
    a_{ij} = a_{ji},
    $$
    则称$\MA$为\red{对称矩阵};
  \item[2]
    如果
    $$
    a_{ij} = -a_{ji},
    $$
    则称$\MA$为\red{反对称矩阵}。
  \end{itemize}      
\end{dingyi}
\end{frame}

\begin{frame}
\begin{zhu}
  关于对称矩阵与反对称矩阵,有如下性质:
  \begin{enumerate}
  \item $\MA$为对称矩阵的充分必要条件是$\MA^T=\MA$; \\[0.15in]
  \item $\MA$为反对称矩阵的充分必要条件是$\MA^T=-\MA$;\\[0.15in]
  \item 反对称矩阵的主对角元全为零。 \\[0.15in]
  \item 奇数阶反对称矩阵的行列式为零。\\[0.15in]
  \item 任何一个方阵都可表示成一个对称矩阵与一个反对称矩阵的和。\\[0.15in]
  \item[]  设$\MA$为一$n$阶方阵,则
    $$
    \MA = \frac{\MA+\MA^T}2 + \frac{\MA-\MA^T}2
    $$
    容易验证$\frac{\MA+\MA^T}2$为对称阵,$\frac{\MA-\MA^T}2$为反对称阵。 \\[0.15in]
  \item 对称矩阵的乘积不一定为对称矩阵。
  \item[]  \red{若$\MA$与$\MB$均为对称矩阵,则$\MA\MB$对称的充分必要条件是$\MA\MB$可交换。}
  \end{enumerate}
\end{zhu}
\end{frame}

\begin{frame}
\begin{li}
  设$\MA$是一个$m\times n$矩阵,则$\MA^T\MA$和$\MA\MA^T$都是对称矩阵。      
\end{li} \pause 

\begin{proof}
  $$
  \begin{array}{l}
    (\MA^T\MA)^T  = \MA^T(\MA^T)^T  = \MA^T \MA, \quad
    (\MA\MA^T)^T  = (\MA^T)^T\MA^T  = \MA \MA^T
  \end{array}
  $$
\end{proof}
\end{frame}

\begin{frame}
\begin{li}
  设$\MA$为$n$阶反对称矩阵,$\MB$为$n$阶对称矩阵,则$\MA\MB+\MB\MA$为$n$阶反对称矩阵。
\end{li}\pause 

\begin{proof}
  $$
  (\MA\MB+\MB\MA)^T =(\MA\MB)^T+(\MB\MA)^T  = \MB^T\MA^T+\MA^T\MB^T 
  = \MB(-\MA) + (-\MA^T)\MB  = - (\MA\MB+\MB\MA).      
  $$  
\end{proof}%

\end{frame}