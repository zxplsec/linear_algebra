\section{矩阵分块}
\begin{frame}

  矩阵
  \begin{figure}[htbp]
    \centering
    \begin{tikzpicture}
      \matrix(A) [matrix of math nodes,nodes in empty cells,ampersand replacement=\&,left delimiter=(,right delimiter=)] {
        a_{11} \& a_{12} \& a_{13}  \& a_{14} \\
        a_{21} \& a_{22} \& a_{23}  \& a_{24} \\
        a_{31} \& a_{32} \& a_{33}  \& a_{34} \\
      };
      \draw[red, dashed, very thick] (A-1-2.north east) -- (A-3-2.south east) (A-2-1.south west) -- (A-2-4.south east);;
    \end{tikzpicture}
  \end{figure}
  可记为
  $$
  \left(
    \begin{array}{cc}
      \MA_{11} &  \MA_{12}\\
      \MA_{21} &  \MA_{22}
    \end{array}
  \right)
  $$
  其中
  $$
  \begin{array}{ll}
    \MA_{11} = 
    \left(
    \begin{array}{cc}
      a_{11} &  a_{12}\\
      a_{21} &  a_{22}
    \end{array} \right),
             &
               \MA_{12} = 
               \left(
               \begin{array}{cc}
                 a_{13} &  a_{14}\\
                 a_{23} &  a_{24}
               \end{array}
                          \right)\\ [0.3cm]
    \MA_{21} = 
    \left(
    \begin{array}{cc}
      a_{31} &  a_{32}
    \end{array}\right) ,
             &
               \MA_{22} = 
               \left(
               \begin{array}{cc}
                 a_{33} &  a_{34}
               \end{array}
                          \right)    
  \end{array}
  $$
\end{frame}

\begin{frame}
  \begin{dingyi}[矩阵的按行分块]
    $$
    \MA = \left(
      \begin{array}{cccc}
        a_{11} & a_{12} & \cd & a_{1n}\\
        a_{21} & a_{22} & \cd & a_{2n}\\
        \vd & \vd & & \vd \\
        a_{m1} & a_{m2} & \cd & a_{mn}
      \end{array}
    \right)
    = \left(
      \begin{array}{c}
        \va_1\\
        \va_2\\
        \vd \\
        \va_m
      \end{array}
    \right)
    $$
    其中
    $$
    \va_i = (a_{i1}, ~a_{i2}, ~\cd, ~a_{in})
    $$
  \end{dingyi}
\end{frame}

\begin{frame}
  \begin{dingyi}[矩阵的按列分块]
    $$
    \MB = \left(
      \begin{array}{cccc}
        b_{11} & b_{12} & \cd & b_{1s}\\
        b_{21} & b_{22} & \cd & b_{2s}\\
        \vd & \vd & & \vd \\
        b_{n1} & b_{n2} & \cd & b_{ns}
      \end{array}
    \right)
    = \left(
      \begin{array}{c}
        \vb_1, ~ \vb_2, ~ \cd, \vb_s
      \end{array}
    \right)
    $$
    其中
    $$
    \vb_j = \left(
      \begin{array}{c}
        b_{1j}\\
        b_{2j}\\
        \vd\\
        b_{nj}
      \end{array}
    \right)
    $$
  \end{dingyi}
\end{frame}

\begin{frame}
  当$n$阶矩阵$\MA$中非零元素都集中在主对角线附近,有时可分块成如下\textcolor{acolor3}{对角块矩阵}
  $$
  \MA = \left(
    \begin{array}{cccc}
      \MA_1 & & &\\
           &\MA_2&&\\
           &&\dd&\\
           &&&\MA_m
    \end{array}
  \right)
  $$
  其中$\MA_i$为$r_i$阶方阵($i=1,2,\cd,m$),且
  $$
  \sum_{i=1}^m r_i = n.
  $$

\end{frame}

\begin{frame}
  如
  \begin{figure}
    \centering
    \begin{tikzpicture}       
      [column 1/.style={anchor=base east},
      column 2/.style={anchor=base east},
      column 3/.style={anchor=base east},
      column 4/.style={anchor=base east},
      column 5/.style={anchor=base east},
      column 6/.style={anchor=base east}]
      \matrix(A) [matrix of math nodes,nodes in empty cells,ampersand replacement=\&,left delimiter=(,right delimiter=)] {
        0 \& -1 \& 0 \& 0 \& 0 \& 0\\
        1 \&  2 \& 0 \& 0 \& 0 \& 0\\
        0 \&  0 \& 1 \&-1 \& 0 \& 0\\
        0 \&  0 \&-1 \& 1 \& 2 \& 0\\
        0 \&  0 \& 0 \& 2 \&-2 \& 0\\
        0 \&  0 \& 0 \& 0 \& 0 \& 3\\
      };
      \draw[red, dashed, very thick] 
      (A-1-2.north east) -- (A-6-2.south east)
      (A-1-5.north east) -- (A-6-5.south east) 
      (A-2-1.south west) -- (A-2-6.south east)
      (A-5-1.south west) -- (A-5-6.south east);

    \end{tikzpicture}
  \end{figure}

\end{frame}

\begin{frame}



  \begin{dingyi}[分块矩阵的加法]
    设$\MA, \MB$为同型矩阵,采用相同的分块法,有
    $$
    \MA = \left(
      \begin{array}{ccc}
        \MA_{11} & \cd & \MA_{1r} \\
        \vd   &     & \vd   \\
        \MA_{s1} & \cd & \MA_{sr}
      \end{array}
    \right), \ \ 
    \MB = \left(
      \begin{array}{ccc}
        \MB_{11} & \cd & \MB_{1r} \\
        \vd   &     & \vd   \\
        \MB_{s1} & \cd & \MB_{sr}
      \end{array}
    \right),
    $$
    其中$\MA_{ij}$与$\MB_{ij}$为同型矩阵,则
    $$
    A = \left(
      \begin{array}{ccc}
        \MA_{11} + \MB_{11}  & \cd & \MA_{1r} + \MB_{1r} \\
        \vd   &     & \vd   \\
        \MA_{s1} + \MB_{s1}  & \cd & \MA_{sr} + \MB_{sr}
      \end{array}
    \right).
    $$
  \end{dingyi}
\end{frame}

\begin{frame}

  \begin{dingyi}[分块矩阵的数乘]
    $$
    \lambda \MA = \left(
      \begin{array}{ccc}
        \lambda \MA_{11} & \cd & \lambda \MA_{1r} \\
        \vd   &     & \vd   \\
        \lambda \MA_{s1} & \cd & \lambda \MA_{sr}
      \end{array}
    \right)
    $$    
  \end{dingyi}
\end{frame}

\begin{frame}

  \begin{dingyi}[分块矩阵的乘法]
    设$\MA$为$m\times n$矩阵, $\MB$为$n \times s$矩阵,
    $$
    \MA = \left(
      \begin{array}{ccc}
        \MA_{11} & \cd & \MA_{1s} \\
        \vd   &     & \vd   \\
        \MA_{r1} & \cd & \MA_{rs}
      \end{array}
    \right), \ \ 
    \MB = \left(
      \begin{array}{ccc}
        \MB_{11} & \cd & \MB_{1t} \\
        \vd   &     & \vd   \\
        \MB_{s1} & \cd & \MB_{st}
      \end{array}
    \right),
    $$
    其中\textcolor{acolor3}{$\MA_{i1}, \MA_{i2}, \cd, A_{is}$的列数分别等于$\MB_{1j}, \MB_{2j}, \cd, \MB_{sj}$的行数},则
    $$
    \MA \MB = \left(
      \begin{array}{ccc}
        \MC_{11}   & \cd & \MC_{1t}  \\
        \vd   &     & \vd   \\
        \MC_{r1}   & \cd & \MC_{rt}
      \end{array}
    \right),
    $$
    其中
    $$
    \MC_{ij} = \sum_{k=1}^s \MA_{ik} \MB_{kj}.
    $$
  \end{dingyi}
\end{frame}

\begin{frame}


  \begin{li} 
    用分块矩阵的乘法计算$\MA\MB$,其中
    $$
    \MA = \left(
      \begin{array}{rrrrr}
        1&0&0&0&0\\
        0&1&0&0&0\\
        -1&2&1&0&0\\
        1&1&0&1&0\\
        -2&0&0&0&1
      \end{array}
    \right), \quad
    \MB = \left(
      \begin{array}{rrrrr}
        3&2&0&1&0\\
        1&3&0&0&1\\
        -1&0&0&0&0\\
        0&-1&0&0&0\\
        0&0&-1&0&0
      \end{array}
    \right)
    $$
  \end{li}
\end{frame}

\begin{frame}
  \begin{center}
    \begin{tikzpicture}       
      [column 1/.style={anchor=base east},
      column 2/.style={anchor=base east},
      column 3/.style={anchor=base east},
      column 4/.style={anchor=base east},
      column 5/.style={anchor=base east}]
      \matrix(A1) [matrix of math nodes]{
        \MA = \\
      };
      \matrix(A2) [right=.1in of A1,matrix of math nodes,nodes in empty cells,inner sep=0.2cm,ampersand replacement=\&,left delimiter=(,right delimiter=)] {
        1 \& 0 \& 0 \& 0 \& 0 \\
        0 \& 1 \& 0 \& 0 \& 0 \\
        -1 \& 2 \& 1 \& 0 \& 0 \\
        1 \& 1 \& 0 \& 1 \& 0 \\
        -2 \& 0 \& 0 \& 0 \& 1 \\
      };
      \draw[red, dashed, very thick] 
      (A2-1-2.north east) -- (A2-5-2.south east)
      (A2-2-1.south west) -- (A2-2-5.south east);
      \matrix(A3) [right=.1in of A2,matrix of math nodes]{
        = \\
      };
      \matrix(A4) [right=.1in of A3,matrix of math nodes,nodes in empty cells,inner sep=0.2cm,ampersand replacement=\&,left delimiter=(,right delimiter=)] {
        \MI_2 \& \M0_{2\times 3} \\
        \MA_1 \& \MI_3\\
      };
    \end{tikzpicture}
  \end{center}


  \begin{center}
    \begin{tikzpicture}       
      [column 1/.style={anchor=base east},
      column 2/.style={anchor=base east},
      column 3/.style={anchor=base east},
      column 4/.style={anchor=base east},
      column 5/.style={anchor=base east}]
      \matrix(A1) [matrix of math nodes]{
        \MB = \\
      };
      \matrix(A2) [right=.1in of A1,matrix of math nodes,nodes in empty cells,inner sep=0.2cm,ampersand replacement=\&,left delimiter=(,right delimiter=)] {
        3\&2\&0\&1\&0\\
        1\&3\&0\&0\&1\\
        -1\&0\&0\&0\&0\\
        0\&-1\&0\&0\&0\\
        0\&0\&-1\&0\&0\\
      };
      \draw[red, dashed, very thick] 
      (A2-1-3.north east) -- (A2-5-3.south east)
      (A2-2-1.south west) -- (A2-2-5.south east);
      \matrix(A3) [right=.1in of A2,matrix of math nodes]{
        = \\
      };
      \matrix(A4) [right=.1in of A3,matrix of math nodes,nodes in empty cells,inner sep=0.2cm,ampersand replacement=\&,left delimiter=(,right delimiter=)] {
        \MB_1 \& \MI_2 \\
        -\MI_3 \& \M0_{3\times 2}\\
      };
    \end{tikzpicture}
  \end{center}
\end{frame}

\begin{frame}
  则
  $$
  \MA\MB = \left(
    \begin{array}{cc}
      \MI_2 & \M0\\
      \MA_1 & \MI_3
    \end{array}
  \right)\left(
    \begin{array}{cc}
      \MB_1 & \MI_2\\
      -\MI_3 & \M0
    \end{array}
  \right) = \left(
    \begin{array}{cc}
      \MB_1 & \MI_2\\
      \MA_1\MB_1-\MI_3 & \MA_1
    \end{array}
  \right)
  $$
  其中
  $$
  \MA_1\MB_1-\MI_3 = \left(
    \begin{array}{rr}
      -1&2\\
      1&1\\
      -2&0
    \end{array}
  \right)\left(
    \begin{array}{rrr}
      3&2&0\\
      1&3&0
    \end{array}
  \right)-\left(
    \begin{array}{ccc}
      1&0&0\\
      0&1&0\\
      0&0&1
    \end{array}
  \right)=\left(
    \begin{array}{rrr}
      -2&4&0\\
      4&4&0\\
      -6&-4&-1
    \end{array}
  \right)
  $$
\end{frame}

\begin{frame}

  \begin{li}
    设$\MA$为$m\times n$矩阵,$\MB$为$n\times s$矩阵,$\MB$按列分块成$1\times s$分块矩阵,
    将$\MA$看成$1\times 1$分块矩阵,则
    $$
    \MA\MB=\MA(\vb_1,\vb_2,\cd,\vb_s)=(\MA\vb_1,\MA\vb_2,\cd,\MA\vb_s)      
    $$
    若已知$\MA\MB=0$,则显然
    $$
    \MA\vb_j=0, \quad j=1,2,\cd,s.
    $$
    因此,$\MB$的每一列$\vb_j$都是线性方程组$\MA\vx=0$的解。
  \end{li}    
\end{frame}

\begin{frame}
  \begin{li}
    设$\MA^T\MA=\M0$,证明$\MA=\M0$.
  \end{li}
  \pause
  \begin{proof}
    设$\MA=(a_{ij})_{m\times n}$,把$\MA$用列向量表示为$\MA=(\va_1, ~\va_2,~\cd,~\va_n)$,则
    $$
    \MA^T\MA = \left(
      \begin{array}{c}
        \va_1^T\\
        \va_2^T\\
        \cd \\
        \va_n^T
      \end{array}
    \right) (\va_1, ~\va_2,~\cd,~\va_n) = \left(
      \begin{array}{cccc}
        \va_1^T\va_1 & \va_1^T\va_2 & \cd & \va_1^T\va_n\\
        \va_2^T\va_1 & \va_2^T\va_2 & \cd & \va_2^T\va_n\\
        \vd & \vd & & \vd \\
        \va_n^T\va_1 & \va_n^T\va_2 & \cd & \va_n^T\va_n
      \end{array}
    \right)
    $$
    \pause
    因为$\MA^T\MA=\M0$,故
    $$
    \va_i^T \va_j = 0, \quad i,j=1,2,\cd,n.
    $$
    \pause
    特别地,有
    $$
    \va_j^T \va_j = 0, \quad j=1,2,\cd,n,
    $$
    即
    $$
    a_{1j}^2+a_{2j}^2+\cd+a_{mj}^2=0  ~\Rightarrow~ a_{1j}=a_{2j}=\cd=a_{mj}=0 ~\Rightarrow~ \MA = \M0.
    $$
  \end{proof}
\end{frame}

\begin{frame}

  \begin{li}
    若$n$阶矩阵$\MC,\MD$可以分块成同型对角块矩阵,即
    $$
    \MC = \left(
      \begin{array}{cccc}
        \MC_1&&&\\
            &\MC_2&&\\
            &&\cd&\\
            &&&\MC_m
      \end{array}
    \right),\quad
    \MD = \left(
      \begin{array}{cccc}
        \MD_1&&&\\
            &\MD_2&&\\
            &&\cd&\\
            &&&\MD_m
      \end{array}
    \right)
    $$
    其中$\MC_i$和$\MD_i$为同阶方阵($i=1,2,\cd,m$),则
    $$
    \MC\MD = \left(
      \begin{array}{cccc}
        \MC_1\MD_1&&&\\
                &\MC_2\MD_2&&\\
                &&\cd&\\
                &&&\MC_m\MD_m
      \end{array}
    \right)
    $$
  \end{li}

\end{frame}

\begin{frame}





  \begin{li}
    证明:若方阵$\MA$为可逆的上三角阵,则$\MA^{-1}$也为上三角阵。
  \end{li}
\end{frame}

\begin{frame}
  \begin{proof}
    对阶数$n$用数学归纳法。\pause
    \begin{itemize}
    \item[1] 当$n=1$时,$(a)^{-1}=(\frac1a)$,结论成立。 \pause
    \item[2] 假设命题对$n-1$阶可逆上三角矩阵成立,考虑$n$阶情况,设
      \begin{center}
        \begin{tikzpicture} [column 1/.style={anchor=base east}]
          \matrix (M) [matrix of math nodes]  { 
            \MA = \\
          };
          \matrix(MM) [right=.1in of M, matrix of math nodes,nodes in empty cells,
          column sep=3ex,row sep=1ex,ampersand replacement=\&,left delimiter=(,right delimiter=)] {
            a_{11} \& a_{12} \& a_{13}  \& a_{14} \\
            0    \& a_{22} \& a_{23}  \& a_{24} \\
            \vd  \& \vd   \& \dd  \& \vd \\
            0    \& 0     \& \cd  \& a_{nn} \\
          };

          \draw[red, dashed, very thick]
          (MM-1-1.north east) -- (MM-4-1.south east)
          (MM-1-1.south west) -- (MM-1-4.south east);

          \matrix (MMM) [right=.1in of MM,matrix of math nodes]  { 
            = \\
          };
          \matrix(MMMM) [right=.1in of MMM, matrix of math nodes,nodes in empty cells,
          column sep=3ex,row sep=1ex,ampersand replacement=\&,left delimiter=(,right delimiter=)] {
            a_{11} \& \alphabd\\
            \M0 \& \MA_1\\
          };
        \end{tikzpicture}        
      \end{center}
      其中$\MA_1$为$n-1$阶可逆上三角阵。
    \end{itemize}
  \end{proof}
\end{frame}

\begin{frame}
  \begin{proof}[续]
    设$\MA$的逆阵为
    $$
    \begin{aligned}
      \MB = \left(
        \begin{array}{cc}
          b_{11} & \betabd\\
          \gammabd & \MB_1 
        \end{array}    
      \right),  
    \end{aligned}
    $$
    其中
    $$
    \begin{aligned}
      \betabd = \left(
        \begin{array}{c}
          b_{12}\\
          \vd \\
          b_{1n}
        \end{array}
      \right)^T,  
      \quad \gammabd = \left(
        \begin{array}{c}
          b_{21}\\
          \vd \\
          b_{n1}
        \end{array}
      \right), \quad
      \MB_1 = \left(
        \begin{array}{ccc}
          b_{22} & \cd & b_{2n}\\
          \vd   & \dd & \vd \\
          b_{n2} & \cd & b_{nn}
        \end{array}
      \right),
    \end{aligned}
    $$\pause
    则
    $$
    \begin{aligned}
      \MA\MB &= \left(
        \begin{array}{cc}
          a_{11} & \alphabd \\
          \M0 & \MA_1
        \end{array}
      \right)\left(
        \begin{array}{cc}
          b_{11} & \betabd \\
          \gammabd & \MB_1
        \end{array}
      \right) \\
      & = \left(
        \begin{array}{cc}
          a_{11}b_{11}+\alphabd\gammabd & a_{11}\betabd+\alphabd \MB_1\\
          \MA_1\gammabd & \MA_1\MB_1
        \end{array}
      \right)  \textcolor{acolor3}{
        = \left(
          \begin{array}{cc}
            1 & \M0 \\
            \M0 & \MI_{n-1}
          \end{array}
        \right)
      }
    \end{aligned}
    $$
  \end{proof}
\end{frame}

\begin{frame}
  \begin{proof}[续]
    于是
    $$
    \begin{array}{l}
      \MA_1 \gammabd = \M0 ~ \Rightarrow ~ \gammabd=\M0, \\[0.2cm]
      \MA_1\MB_1 = \MI_1 ~\Rightarrow~ \MB_1=\MA_1^{-1}.
    \end{array}
    $$\pause
    由归纳假设,$\MB_1$为$n-1$阶上三角矩阵,因此
    $$
    \MA^{-1} = \MB = \left(
      \begin{array}{cc}
        b_{11} & \betabd\\
        \M0 & \MB_1 
      \end{array}    
    \right)
    $$
    为上三角矩阵。
  \end{proof}
\end{frame}

\begin{frame}



  \begin{dingyi}[分块矩阵的转置]
    分块矩阵$\MA=(\MA_{kl})_{s\times t}$的转置矩阵为
    $$
    \MA^T = (\MB_{lk})_{t\times s},
    $$
    其中$\MB_{lk}=\MA_{kl}$.
  \end{dingyi}
  \pause
  \begin{li}
    $$
    \MA = \left(
      \begin{array}{ccc}
        \MA_{11} & \MA_{12} & \MA_{13}\\
        \MA_{21} & \MA_{22} & \MA_{23}
      \end{array}
    \right) ~\Rightarrow~
    \MA = \left(
      \begin{array}{cc}
        \MA_{11}^T & \MA_{21}^T \\[0.2cm]
        \MA_{12}^T & \MA_{22}^T \\[0.2cm]
        \MA_{13}^T & \MA_{23}^T
      \end{array}
    \right)
    $$
    \pause
    $$
    \MB \xlongequal[]{\mbox{按行分块}} \left(
      \begin{array}{c}
        \vb_1\\
        \vb_2\\
        \vd\\
        \vb_m
      \end{array}
    \right) ~\Rightarrow~
    \MB^T = \left(
      \begin{array}{cccc}
        \vb_1^T & \vb_2^T & \cd & \vb_m^T
      \end{array}
    \right)
    $$
  \end{li}



\end{frame}

\begin{frame}


  \begin{dingyi}[可逆分块矩阵的逆矩阵]
    对角块矩阵(准对角矩阵)
    $$
    \MA = \left(
      \begin{array}{cccc}
        \MA_1&&&\\
            &\MA_2&&\\
            &&\dd&\\
            &&&\MA_m
      \end{array}
    \right)
    $$
    的行列式为$|\MA|=|\MA_1||\MA_2|\cd|\MA_m|$,因此,$\MA$可逆的充分必要条件为
    $$
    |\MA_i|\ne 0, \quad i=1,2,\cd, m.
    $$

    其逆矩阵为
    $$
    \MA^{-1} = \left(
      \begin{array}{cccc}
        \MA_1^{-1}&&&\\
                 &\MA_2^{-1}&&\\
                 &&\dd&\\
                 &&&\MA_m^{-1}
      \end{array}
    \right)
    $$
  \end{dingyi}
\end{frame}

\begin{frame}
  分块矩阵的作用:
  \begin{itemize}
  \item   用分块矩阵求逆矩阵,可将高阶矩阵的求逆转化为低阶矩阵的求逆。
  \item   一个$2\times 2$的分块矩阵求逆,可以根据逆矩阵的定义,用解矩阵方程的方法解得。
  \end{itemize}
\end{frame}

\begin{frame}
  \begin{li}
    设$\MA=\left(
      \begin{array}{cc}
        \MB&\M0\\
        \MC&\MD
      \end{array}
    \right)$,其中$\MB,\MD$皆为可逆矩阵,证明$\MA$可逆并求$\MA^{-1}$.
  \end{li}
\end{frame}

\begin{frame}
  \begin{jie}
    因$|\MA|=|\MB||\MD|\ne 0$,故$\MA$可逆。\pause 设$\MA^{-1}=\left(
      \begin{array}{cc}
        \MX&\MY\\
        \MZ&\MT
      \end{array}
    \right)$,则
    $$
    \left(
      \begin{array}{cc}
        \MB&\M0\\
        \MC&\MD
      \end{array}
    \right) \left(
      \begin{array}{cc}
        \MX&\MY\\
        \MZ&\MT
      \end{array}
    \right)=\left(
      \begin{array}{cc}
        \MB\MX&\MB\MY\\
        \MC\MX+\MD\MZ&\MC\MY+\MD\MT
      \end{array}
    \right) = \left(
      \begin{array}{cc}
        \MI & \M0\\
        \M0 & \MI
      \end{array}
    \right)
    $$
    \pause
    由此可知
    $$
    \begin{array}{ll}
      \MB\MX = \MI   & \Rightarrow ~ \MX = \MB^{-1}\\[0.2cm]
      \MB\MY = \M0 & \Rightarrow ~ \MY = \M0\\[0.2cm]
      \MC\MX+\MD\MZ = \M0 & \Rightarrow ~ \MZ = -\MD^{-1}\MC\MB^{-1}\\[0.2cm]
      \MC\MY+\MD\MT = \MI & \Rightarrow ~ \MT = \MD^{-1}
    \end{array}
    $$
    \pause
    故
    $$
    \MA^{-1} = \left(
      \begin{array}{cc}
        \MB^{-1} & \M0\\
        -\MD^{-1}\MC\MB^{-1} & \MD^{-1}
      \end{array}
    \right).
    $$
  \end{jie}
\end{frame}

\begin{frame}




  \begin{dingyi}[分块矩阵的初等变换与分块初等矩阵]
    对于分块矩阵
    $$
    \MA = \left(
      \begin{array}{cc}
        \MA_{11} & \MA_{12}\\
        \MA_{21} & \MA_{22}
      \end{array}
    \right)
    $$
    同样可以定义它的3类初等行变换与列变换,并相应地定义3类分块矩阵:
    \begin{itemize}
    \item[(i)] 分块倍乘矩阵($\MC_1,\MC_2$为可逆阵)
      $$
      \left(
        \begin{array}{cc}
          \MC_1 & \M0\\
          \M0 & \MI_n
        \end{array}
      \right) ~~\mbox{或}~~
      \left(
        \begin{array}{cc}
          \MI_m & \M0\\
          \M0 & \MC_2
        \end{array}
      \right)
      $$
    \item[(ii)] 分块倍加矩阵
      $$
      \left(
        \begin{array}{cc}
          \MI_m & \M0\\
          \MC_3 & \MI_n
        \end{array}
      \right) ~~\mbox{或}~~
      \left(
        \begin{array}{cc}
          \MI_m & \MC_4\\
          \M0 & \MI_n
        \end{array}
      \right)
      $$
    \item[(iii)] 分块对换矩阵
      $$
      \left(
        \begin{array}{cc}
          \M0 & \MI_n\\
          \MI_m & \M0
        \end{array}
      \right)
      $$
    \end{itemize}
  \end{dingyi}
\end{frame}

\begin{frame}

  \begin{li}
    设$n$阶矩阵$\MA$分块表示为
    $$
    \MA = \left(
      \begin{array}{cc}
        \MA_{11} & \MA_{12}\\
        \MA_{21} & \MA_{22}
      \end{array}
    \right)
    $$
    其中$\MA_{11},\MA_{22}$为方阵,且$\MA$与$\MA_{11}$可逆。证明:$\MA_{22}-\MA_{21}\MA_{11}^{-1}\MA_{12}$可逆,并求$\MA^{-1}$。
  \end{li}
\end{frame}

\begin{frame}
  \begin{jie}
    构造分块倍加矩阵
    $$
    \MP_1 = \left(
      \begin{array}{cc}
        \MI_1 & \M0\\
        -\MA_{21}\MA_{11}^{-1} & \MI_2
      \end{array}
    \right)
    $$
    则$$
    \MP_1\MA = \left(
      \begin{array}{cc}
        \MA_{11} & \MA_{12} \\
        \M0 & \MA_{22}-\MA_{21}\MA_{11}^{-1}\MA_{12}
      \end{array}
    \right)
    $$
    两边同时取行列式可知
    $$
    |\MA| = |\MP_1\MA| = |\MA_{11}|\cdot |\MA_{22}-\MA_{21}\MA_{11}^{-1}\MA_{12}|
    $$
    故$\MA_{22}-\MA_{21}\MA_{11}^{-1}\MA_{12}$可逆。
  \end{jie}
\end{frame}

\begin{frame}
  \begin{jie}[续]
    $$
    \MP_1\MA = \left(
      \begin{array}{cc}
        \MA_{11} & \MA_{12} \\
        \M0 & \MA_{22}-\MA_{21}\MA_{11}^{-1}\MA_{12}
      \end{array}
    \right)\xlongequal[]{\textcolor{acolor3}{\ds \MQ=\MA_{22}-\MA_{21}\MA_{11}^{-1}\MA_{12}}}
    \left(
      \begin{array}{cc}
        \MA_{11} & \MA_{12} \\
        \M0 & \MQ
      \end{array}
    \right)
    $$ \pause
    构造分块倍加矩阵
    $$
    \MP_2 = \left(
      \begin{array}{cc}
        \MI_1 & -\MA_{12}\MQ^{-1}\\
        \M0 & \MI_2
      \end{array}
    \right)
    $$ \pause
    则
    $$
    \MP_2\MP_1\MA = \left(
      \begin{array}{cc}
        \MA_{11} & \M0\\
        \M0 & \MQ
      \end{array}
    \right)
    $$ \pause
    于是
    $$
    \begin{array}{rl}
      \MA^{-1} & = \left(
                \begin{array}{cc}
                  \MA_{11}^{-1} & \M0\\
                  \M0 & \MQ^{-1}
                \end{array}
                          \right)\left(
                          \begin{array}{cc}
                            \MI_1 & -\MA_{12}\MQ^{-1}\\
                            \M0 & \MI_2
                          \end{array}
                                    \right)\left(
                                    \begin{array}{cc}
                                      \MI_1 & \M0\\[0.2cm]
                                      -\MA_{21}\MA_{11}^{-1} & \MI_2
                                    \end{array}
                                                             \right) \\[0.3in]
              & = \left(
                \begin{array}{cc}
                  \MA_{11}^{-1} & \M0\\[0.2cm]
                  \M0 & \MQ^{-1}
                \end{array}
                          \right)\left(
                          \begin{array}{cc}
                            \MI_1+ \MA_{12}\MQ^{-1}\MA_{21}\MA_{11}^{-1}& -\MA_{12}\MQ^{-1}\\[0.2cm]
                            -\MA_{21}\MA_{11}^{-1} & \MI_2
                          \end{array}
                                                   \right)\\[0.3in]
              & = \left(
                \begin{array}{cc}
                  \MA_{11}^{-1}+ \MA_{11}^{-1}\MA_{12}\MQ^{-1}\MA_{21}\MA_{11}^{-1}& -\MA_{11}^{-1}\MA_{12}\MQ^{-1}\\[0.2cm]
                  -\MQ^{-1}\MA_{21}\MA_{11}^{-1} & \MQ^{-1}
                \end{array}
                                                 \right)
    \end{array}
    $$
  \end{jie}
\end{frame}

\begin{frame}

  \begin{li}
    设$\MQ=\left(
      \begin{array}{cc}
        \MA&\MB\\
        \MC&\MD
      \end{array}
    \right)$,且$\MA$可逆,证明:
    $$
    |\MQ| = |\MA| \cdot |\MD-\MC\MA^{-1}\MB|
    $$
  \end{li}\pause
  \begin{proof}
    构造分块倍加矩阵
    $$
    \MP_1 = \left(
      \begin{array}{cc}
        \MI_1 & \M0\\
        -\MC\MA^{-1} & \MI_2
      \end{array}
    \right)
    $$ \pause
    则
    $$
    \MP_1 \MQ = \left(
      \begin{array}{cc}
        \MA & \MB\\
        \M0 & \MD-\MC\MA^{-1}\MB
      \end{array}
    \right)
    $$
    \pause
    两边同时取行列式得
    $$
    |\MQ| = |\MP_1\MQ| = |\MA|\cdot |\MD-\MC\MA^{-1}\MB|.
    $$
  \end{proof}
\end{frame}

\begin{frame}
  \begin{li}
    设$\MA$与$\MB$均为$n$阶分块矩阵,证明
    $$
    \left|
      \begin{array}{cc}
        \MA&\MB\\
        \MB&\MA
      \end{array}
    \right| = |\MA+\MB|~|\MA-\MB|
    $$
  \end{li}
\end{frame}

\begin{frame}
  \begin{proof}

    将分块矩阵$
    \MP = 
    \left(
      \begin{array}{cc}
        \MA&\MB\\
        \MB&\MA
      \end{array}
    \right)$的第一行加到第二行,得\pause
    $$
    \left(
      \begin{array}{cc}
        \MI & \M0\\
        \MI & \MI
      \end{array}
    \right) \left(
      \begin{array}{cc}
        \MA&\MB\\
        \MB&\MA
      \end{array}
    \right) = \left(
      \begin{array}{cc}
        \MA&\MB\\
        \MA+\MB&\MA+\MB
      \end{array}
    \right)
    $$\pause
    再将第一列减去第二列,得\pause
    $$
    \left(
      \begin{array}{cc}
        \MA&\MB\\
        \MA+\MB&\MA+\MB
      \end{array}
    \right) \left(
      \begin{array}{cc}
        \MI&\M0\\
        -\MI&\MI
      \end{array}
    \right) = \left(
      \begin{array}{cc}
        \MA-\MB & \MB\\
        \M0 & \MA+\MB
      \end{array}
    \right)
    $$\pause
    总之有
    $$
    \left(
      \begin{array}{cc}
        \MI & \M0\\
        \MI & \MI
      \end{array}
    \right) \left(
      \begin{array}{cc}
        \MA&\MB\\
        \MB&\MA
      \end{array}
    \right) 
    \left(
      \begin{array}{cc}
        \MI&\M0\\
        -\MI&\MI
      \end{array}
    \right) = \left(
      \begin{array}{cc}
        \MA-\MB & \MB\\
        \M0 & \MA+\MB
      \end{array}
    \right)
    $$
    两边同时取行列式即得结论。
  \end{proof}
\end{frame}
