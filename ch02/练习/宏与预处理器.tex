\documentclass[10pt,a4paper%,twoside,openright,titlepage,fleqn,%
%headinclude,footinclude,BCOR5mm,%
%numbers=noenddot,cleardoublepage=empty,%
tablecaptionabove]{article}

\usepackage{geometry}
\usepackage{enumerate}
\usepackage{ifthen}
\geometry{left=2.5cm,right=2.5cm,top=2.5cm,bottom=2.5cm}

\usepackage{amsmath,amssymb,amsthm}

%% -----------------设置数学公式字体-------------------------
%% Font style 1
%% \newcommand\ibinom[2]{\genfrac\lbrace\rbrace{0pt}{}{#1}{#2}}
%% \usepackage{bm}

%% Font style 2
%% \newcommand\ibinom[2]{\genfrac\lbrace\rbrace{0pt}{}{#1}{#2}} 
%% \usepackage[boldsans]{ccfonts} 
%% \usepackage{bm} 

%% Font style 3
\newcommand\ibinom[2]{\genfrac\lbrace\rbrace{0pt}{}{#1}{#2}}
\usepackage[euler-digits]{eulervm}
\usepackage{bm}

%% Font style 4
%% \usepackage{fourier}
%% \newcommand\ibinom[2]{\genfrac\lbrace\rbrace{0pt}{}{#1}{#2}}
%% \usepackage{bm}

%% Font style 5
%% \newcommand\ibinom[2]{\genfrac\lbrace\rbrace{0pt}{}{#1}{#2}}
%% \usepackage{mathptmx}
%% \usepackage{bm} 


%% %% Font style 6
%% \newcommand\ibinom[2]{\genfrac\lbrace\rbrace{0pt}{}{#1}{#2}}
%% \usepackage{txfonts}
%% \usepackage{bm}
%% -----------------------------------------------------------

\usepackage{titlesec} %设置标题
\usepackage{titletoc}

\usepackage{extarrows}
\usepackage{verbatim,color,xcolor}
\usepackage{pgf}
\usepackage{tikz}
\usetikzlibrary{calc}
\usetikzlibrary{arrows,snakes,backgrounds,shapes,patterns}
\usetikzlibrary{matrix,fit,positioning,decorations.pathmorphing}
%% \usepackage{classicthesis}
\usepackage{CJK}
\usepackage{mathdots}

\usepackage{listings}
\lstset{
  keywordstyle=\color{blue!70},
  frame=single,
  basicstyle=\ttfamily\small,
  commentstyle=\small\color{red},
  breakindent=0pt,
  rulesepcolor=\color{red!20!green!20!blue!20},
  rulecolor=\color{black},
  tabsize=4,
  numbersep=5pt,
  breaklines=true,
  %% backgroundcolor=\color{red!10},
  showspaces=false,
  showtabs=false,
  extendedchars=false,
  escapeinside=``,
  frame=no,
}


\newcommand{\blue}{\textcolor{blue}}
\newcommand{\red}{\textcolor{red}}
\newcommand{\purple}{\textcolor{electricpurple}}
\newcommand{\ds}{\displaystyle}
\newcommand{\cd}{\cdots}
\newcommand{\dd}{\ddots}
\newcommand{\vd}{\vdots}
\newcommand{\id}{\iddots}

\newcommand{\R}{\mathbb R}

\def\tf{\ttfamily}

\newlength{\la}
\newlength{\lb}
\newlength{\lc}
\newlength{\ld}
\newlength{\lhalf}
\newlength{\lquarter}
\newlength{\lmax}
\newcommand{\xx}[4]{\\[.5pt]%
\settowidth{\la}{A.~#1~~}
\settowidth{\lb}{B.~#2~~}
\settowidth{\lc}{C.~#3~~}
\settowidth{\ld}{D.~#4~~}
%%
\ifthenelse{\lengthtest{\la>\lb}}
{\setlength{\lmax}{\la}}
{\setlength{\lmax}{\lb}}
\ifthenelse{\lengthtest{\lmax<\lc}}
{\setlength{\lmax}{\lc}}
{}
\ifthenelse{\lengthtest{\lmax<\ld}}
{\setlength{\lmax}{\ld}}
{}
%%
\setlength{\lhalf}{0.5\linewidth}
\setlength{\lquarter}{0.25\linewidth}
%%
\ifthenelse{\lengthtest{\lmax>\lhalf}}
{\noindent{}A.~#1 \\ B.~#2 \\ C.~#3 \\ D.~#4 }
{
\ifthenelse{\lengthtest{\lmax>\lquarter}}
{\noindent
\makebox[\lhalf][l]{A.~#1~~}%
\makebox[\lhalf][l]{B.~#2~~}\\
\makebox[\lhalf][l]{C.~#3~~}%
\makebox[\lhalf][l]{D.~#4~~}
}%
{\noindent
\makebox[\lquarter][l]{A.~#1~~}%
\makebox[\lquarter][l]{B.~#2~~}%
\makebox[\lquarter][l]{C.~#3~~}%
\makebox[\lquarter][l]{D.~#4~~}
}
}
}



\begin{document}

\begin{CJK}{UTF8}{gkai}


 

\newtheorem{li}{例}
\newtheorem{jielun}{结论}
\newtheorem{dingli}{定理}
\newtheorem{mingti}{{命题}} 
\newtheorem{yinli}{{引理}} 
\newtheorem{tuilun}{{推论}}
\newtheorem{dingyi}{{定义}} 
\newtheorem{example}{{例}}
\newtheorem*{example*}{{例}}
\newtheorem*{jie}{{解}}
\newtheorem*{zhengming}{{证明}}
\newtheorem{zhu}{{注}}
\newtheorem*{zhu*}{{注}}
\newtheorem{xingzhi}{{性质}}
\newtheorem{wenti}{{问题}}
\newtheorem{rem}{{Remark}}
\newtheorem{lem}{{Lemma}}
\pagenumbering{roman}
\pagestyle{plain}

\pagenumbering{arabic}



\title{宏与预处理器}
%\author{张晓平}
%\date{}                                           % Activate to display a given date or no date
\maketitle

\begin{enumerate}
\item 以下程序打印了\line(1,0){20}次{\bf Hello world}。
\begin{lstlisting}[language=c,frame=single]
#include <stdio.h>
#define PRINT(i, limit) do \
                        { \
                            if (i++ < limit) \
                            { \
                                printf("Hello world\n"); \
                                continue; \
                            } \
                        }while(1)
 
int main(void)
{
    PRINT(0, 3);
    return 0;
}
\end{lstlisting} ~
\xx
{1}
{3}
{4}
{编译时错误}

\item 以下程序的输出是\line(1,0){20}。
\begin{lstlisting}[language=c,frame=single]
#include <stdio.h>
#if X == 3
    #define Y 3
#else
    #define Y 5
#endif
 
int main(void)
{
    printf("%d", Y);
    return 0;
}
\end{lstlisting} ~
\xx
{3}
{5}
{3或5,依赖于X的值}
{编译时错误}

\item 以下程序的输出是\line(1,0){20}。
\begin{lstlisting}[language=c,frame=single]
#include <stdio.h>
#define macro(n, a, i, m) m##a##i##n
#define MAIN macro(n, a, i, m)
 
int MAIN(void)
{
    printf("Hello World");
    return 0;
}
\end{lstlisting} ~
\xx
{编译时错误}
{Hello World}
{MAIN}
{main}

\item 以下程序的输出是\line(1,0){20}。
\begin{lstlisting}[language=c,frame=single]
#include <stdio.h>
#define X 3
#if !X
    printf("Hello");
#else
    printf("World");
  
#endif
int main(void)
{
        return 0;
}
\end{lstlisting} ~
\xx
{Hello}
{World}
{编译时错误}
{运行时错误}

\item 以下程序的输出是\line(1,0){20}。
\begin{lstlisting}[language=c,frame=single]
#include <stdio.h>
#define ISEQUAL(X, Y) X == Y
int main()
{
    #if ISEQUAL(X, 0)
        printf("Hello");
    #else
        printf("World");
    #endif
    return 0;
}
\end{lstlisting} ~
\xx
{Hello}
{World}
{Hello或World中的一个}
{编译时错误}

\item 以下程序的输出是\line(1,0){20}。
\begin{lstlisting}[language=c,frame=single]
#include <stdio.h>
#define square(x) x*x
int main(void)
{
  int x;
  x = 36/square(6);
  printf("%d", x);
  return 0;
}
\end{lstlisting} ~
\xx
{1}
{36}
{0}
{编译时错误}

\item 以下程序的输出是\line(1,0){20}。
\begin{lstlisting}[language=c,frame=single]
#include <stdio.h>
#define scanf  "%s Hello World "
int main()
{
   printf(scanf, scanf);
   return 0;
}
\end{lstlisting} ~
\xx
{\%s Hello World}
{Hello World}
{\%s Hello World Hello World}
{编译时错误}

\item 以下程序的输出是\line(1,0){20}。
\begin{lstlisting}[language=c,frame=single]
#include <stdio.h>
#define a 10
int main()
{
  printf("%d ", a);
 
  #define a 50
 
  printf("%d ", a);
  return 0;
}
\end{lstlisting} ~
\xx
{10 50}
{50 50}
{10 10}
{编译时错误}

\item 以下程序的输出是\line(1,0){20}。
\begin{lstlisting}[language=c,frame=single]
#include<stdio.h> 
#define f(g,g2) g##g2 
int main() 
{ 
   int var12 = 100; 
   printf("%d", f(var,12)); 
   return 0; 
}
\end{lstlisting} ~
\xx
{100}
{0}
{1}
{编译时错误}

\item 一个C程序预处理后所产生的文件是\line(1,0){20}。
\xx
{.p}
{.i}
{.o}
{.m}

\item 以下程序的输出是\line(1,0){20}。
\begin{lstlisting}[language=c,frame=single]
#include <stdio.h>
#define MAX 1000
int main(void)
{
   int MAX = 100;
   printf("%d ", MAX);
   return 0;
}
\end{lstlisting} ~
\xx
{1000}
{100}
{垃圾值}
{编译时错误}

\item 以下程序的输出是\line(1,0){20}。
\begin{lstlisting}[language=c,frame=single]
#include<stdio.h>
#define max abc
#define abc 100
 
int main()
{
    printf("maximum is %d", max);
    return 0;
}
\end{lstlisting} ~
\xx
{maximum is 100}
{abcimum is 100}
{100imum is 100}
{abcimum is abc}

\item 以下程序的输出是\line(1,0){20}。
\begin{lstlisting}[language=c,frame=single]
#include <stdio.h>
#define get(s) #s
 
int main()
{
    char str[] = get(HelloWorld);
    printf("%s", str);
    return 0;
}\end{lstlisting} ~
\xx
{编译时错误}
{\#HelloWorld}
{HelloWorld}
{HHelloWorld}

\item 假设某人用如下方式写增量宏:
\begin{lstlisting}[language=c,frame=single]
#define INC1(a) ((a)+1) 
#define INC2 (a) ((a)+1) 
#define INC3( a ) (( a ) + 1) 
#define INC4 ( a ) (( a ) + 1)
\end{lstlisting} 
关于这些宏,正确的选项是\line(1,0){20}。
\xx
{只有INC1是正确的}
{所有都是正确的}
{只有INC1和INC3是正确的}
{只有INC1和INC2是正确的}

\item 以下程序不能编译,是因为在宏名与圆括号之间有空格。
\begin{lstlisting}[language=c,frame=single]
#include "stdio.h"
 
#define MYINC   (  a  )  (  ( a )  +  1 )
 
int main()
{
 
 printf("GeeksQuiz!");
 
 return 0;
}
\end{lstlisting} 
这样的说法正确吗?\line(1,0){20}。
\xx
{正确}
{错误}{}{}

%\item 以下程序的输出是\line(1,0){20}。
%\begin{lstlisting}[language=c,frame=single]
%#include<stdio.h>
%#define A -B
%#define B -C
%#define C 5
% 
%int main(void)
%{
%  printf("The value of A is %d\n", A); 
%  return 0;
%}
%\end{lstlisting} ~
%\xx
%{The value of A is 4}
%{The value of A is 5}
%{编译时错误}
%{运行时错误}
\end{enumerate}

\end{CJK}
\end{document}