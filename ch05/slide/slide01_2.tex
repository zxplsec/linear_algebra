\subsection{特征值与特征值的性质}

\begin{frame}  
  \begin{dingli}
    若$\vx_1$和$\vx_2$都是$\MA$的对应于特征值$\lambda_0$的特征向量,则$k_1\vx_1+k_2\vx_2$也是$\MA$的对应于特征值$\lambda_0$的特征向量(其中$k_1,k_2$为任意常数,但$k_1\vx_1+k_2\vx_2\ne 0$)。
  \end{dingli}  \vspace{.1in}\pause 
  \begin{proof}
    由于$\vx_1,\vx_2$是齐次线性方程组
    $$
    (\lambda_0\MI-\MA)\vx=\M0
    $$
    的解,因此$k_1\vx_1+k_2\vx_2$也是上式的解,故当$k_1\vx_1+k_2\vx_2\ne 0$时,它是$\MA$属于$\lambda_0$的特征向量。
  \end{proof}
\end{frame}

\begin{frame}
  在
  $$
  (\lambda\MI-\MA)\vx=\M0
  $$
  的解空间中,除零向量外的全体解向量就是$\MA$的属于特征值$\lambda$的全体特征向量。因此,$(\lambda\MI-\MA)\vx=\M0$的解空间也称为矩阵$\MA$关于特征值$\lambda$的\red{特征子空间},记作$\red{V_\lambda}$。 \vspace{.1in} \pause 

  $n$阶矩阵$\MA$的特征子空间是$n$维向量空间的子空间,其维数为
  $$
  \dim V_\lambda = n - \rank(\lambda\MI-\MA).
  $$
\end{frame}

\begin{frame}
  
  \begin{dingli}
    设$n$阶矩阵$\MA=(a_{ij})$的$n$个特征值为$\lambda_1,\lambda_2,\cd,\lambda_n$,则
    \begin{itemize}
    \item[(1)] \red{$\ds \sum_{i=1}^n\lambda_i=\tr(\MA)$}
    \item[(2)] \red{$\ds \prod_{i=1}^n\lambda_i=\det(\MA)$}     
    \end{itemize}
    其中
    $\ds
    \tr(A)=\sum_{i=1}^n a_{ii}
    $
    为$\MA$的迹。
  \end{dingli}
  \vspace{.1in}\pause 

  \begin{itemize}
  \item 当$\det(\MA)\ne 0$,即$\MA$为可逆矩阵时,其特征值全为非零数;
  \item 奇异矩阵$\MA$至少有一个零特征值。      
  \end{itemize}  
\end{frame}

\begin{frame}
  
  \begin{dingli}
    一个特征向量不能属于不同的特征值。
  \end{dingli}
  \pause 

  \begin{proof}
  若$\vx$是$\MA$的属于特征值$\lambda_1,\lambda_2(\lambda_1\ne\lambda_2)$的特征向量,
  即有
  $$
  \MA\vx=\lambda_1\vx, ~~ \MA\vx=\lambda_2\vx
  ~~\Rightarrow~~ (\lambda_1-\lambda_2)\vx=\M0
  ~~\Rightarrow~~ \vx=\M0
  $$
  这与$\vx\ne\M0$矛盾。
\end{proof}
\end{frame}

\begin{frame}
  
  \begin{xingzhi}
    \begin{table}
      \caption{特征值与特征向量}
      
      \begin{tabular}{|c|c|c|}\hline
        &特征值&特征向量\\\hline
        \red{$\MA$}&\red{$\lambda$}&\red{$\vx$}\\ \hline 
        \hline  
        $k\MA$&$k\lambda$&$\vx$\\\hline
        $\MA^m$&$\lambda^m$&$\vx$\\\hline
        $\MA^{-1}$&$\lambda^{-1}$&$\vx$\\\hline
      \end{tabular}
    \end{table}
  \end{xingzhi}
  
\end{frame}

\begin{frame}
  
  \begin{xingzhi}
    矩阵$\MA$与$\MA^T$的特征值相同。
  \end{xingzhi} \vspace{.1in}\pause 

  \begin{proof}
    因$(\lambda\MI-\MA)^T=\lambda\MI-\MA^T$,故
    $$
    \det(\lambda\MI-\MA)=\det(\lambda\MI-\MA^T)
    $$
    因此,$\MA$与$\MA^T$有完全相同的特征值。
  \end{proof}
  
\end{frame}

\begin{frame}
  
  \begin{dingli}
    设$\MA=(a_{ij})$是$n$阶矩阵,若
    \begin{itemize}
    \item[(1)] $\ds \sum_{j=1}^n|a_{ij}|<1, ~~(i=1,2,\cd,n)$
    \item[(2)] $\ds \sum_{i=1}^n|a_{ij}|<1, ~~(j=1,2,\cd,n)$
    \end{itemize}
    有一个成立,则$\MA$的所有特征值的模都小于$1$。
  \end{dingli}\vspace{.1in}\pause 

  \begin{proof}
    设$\lambda$为$\MA$的任一特征值,$\vx$为$\lambda$对应的特征向量,由$\MA\vx=\lambda\vx$可得
    $$
    \sum_{j=1}^n a_{ij} x_j = \lambda x_i, \quad i=1,2,\cd,n.
    $$
    记$\ds x_k=\max_{1\le j\le n} |x_j|$,则有
    $$
    |\lambda|=\left|\frac{\lambda x_k}{x_k}\right|=\left|\frac{\sum_{j=1}^n a_{kj} x_j}{x_k}\right|
    \le \sum_{j=1}^n |a_{kj}| \left|\frac{x_j}{x_k}\right|\le \sum_{j=1}^n |a_{kj}|.
    $$
    由此可知,若条件(1)成立,则$|\lambda|<1$;再由$\lambda$的任意性即得$|\lambda_k|<1,~~k=1,2,\cd,n$。 

    \pause\vspace{.1in}
 
    \blue{同理,若条件(2)成立,则$\MA^T$的所有特征值,亦即$\MA$的所有特征值的模小于$1$。}
  \end{proof}
  
\end{frame}

\begin{frame}
  
  \begin{li}
    设$\MA=\left(
      \begin{array}{rrr}
        1&-1&1\\
        2&-2&2\\
        -1&1&-1
      \end{array}
    \right)$
    \begin{itemize}
    \item[(i)]求$\MA$的特征值与特征向量
    \item[(ii)] 求可逆矩阵$\MP$,使得$\MP^{-1}\MA\MP$为对角阵。 
    \end{itemize}
  \end{li}

  \begin{jie}
    \begin{itemize}
    \item[(i)]
      由
      $$
      \begin{aligned}
        |\MA-\lambda\MI|
        &=\left|
          \begin{array}{rrr}
            1-\lambda&-1&1\\
            2&-2-\lambda&2\\
            -1&1&-1-\lambda
          \end{array}
        \right|=-\lambda^2(\lambda+2)
        \end{aligned}
        $$
        知$\MA$的特征值为$\lambda_1=\lambda_2=0$和$\lambda_3=-2$。
    \end{itemize}
  \end{jie}
  
\end{frame}

