\subsection{实对称矩阵的对角化}

\begin{frame}
  \begin{dingli}
    对于任何一个$n$阶实对称矩阵$\MA$,存在$n$阶正交矩阵$\MT$,使得
    $$
    \MT^{-1}\MA\MT=\diag(\lambda_1,\lambda_2,\cd,\lambda_n).
    $$
  \end{dingli}\pause 
  \begin{proof}
    用数学归纳法证明。

    \begin{itemize}
      \item 当$n=1$时,结论显然成立。
    \end{itemize}
  \end{proof}
\end{frame}

\begin{frame}
\begin{proof}
    \begin{itemize}
      \item 假设定理对任何一个$n-1$阶实对称矩阵$\MB$成立,即存在$n-1$阶正交矩阵$\MQ$,使得$\MA^{-1}\MB\MQ=\Lambdabd_1$。
        下证对$n$阶实对称矩阵$\MA$也成立。 \vspace{.1in}\pause 

        设$\MA\vx_1=\lambda_1\vx_1$,其中$\vx_1$是长度为$1$的特征向量。现将$\vx_1$扩充为$\R^n$的一组标准正交基
        $$
        \vx_1,\vx_2,\cd,\vx_n
        $$
        其中$\vx_2,\cd,\vx_n$不一定$\MA$的特征向量,于是就有
        $$
        \begin{aligned}
          \MA(\vx_1,\vx_2,\cd,\vx_n)&=(\MA\vx_1,\MA\vx_2,\cd,\MA\vx_n)\\
          &=(\vx_1,\vx_2,\cd,\vx_n)\left[
            \begin{array}{cccc}
              \lambda_1&b_{12}&\cd&b_{1n}\\
                       &b_{22}&\cd&b_{2n}\\
                       &\vd   &   &\vd\\
                       &b_{n2}&\cd&b_{nn}
            \end{array}
          \right]
        \end{aligned}
        $$
        \pause 
        记
        $$
        \MP=(\vx_1,\vx_2,\cd,\vx_n)
        $$
        则上式可写为
        $$
        \MP^{-1}\MA\MP=\left[
          \begin{array}{cc}
            \lambda_1&\vb\\
                     &\MB
          \end{array}
        \right]
        $$
        
    \end{itemize}
  \end{proof}
\end{frame}

\begin{frame}
\begin{proof}
  由于$\MP^{-1}=\MP^T, (\MP^{-1}\MA\MP)^T=\MP^T\MA(\MP^{-1})^T=\MP^{-1}\MA\MP$,因此
  $$
  \left[
    \begin{array}{cc}
      \lambda_1&\\
      \vb^T&\MB^T
    \end{array}
  \right]=
  \left[
    \begin{array}{cc}
      \lambda_1&\vb\\
               &\MB
    \end{array}
  \right]
  $$
  \pause 
  因此$\vb^T=\vb, \MB^T=\MB$,从而
  $$
  \MP^{-1}\MA\MP=\left[
    \begin{array}{cc}
      \lambda_1& \\
               &\MB
    \end{array}
  \right]
  $$
  由归纳假设,构造一个正交阵
  $$
  \MS=\left[
    \begin{array}{cc}
      1&\\
       &\MQ
    \end{array}
  \right]
  $$
  便有
  $$
  \MS^{-1}(\MP^{-1}\MA\MP)\MS=
  \left[
    \begin{array}{cc}
      1&\\
       &\MQ^{-1}
    \end{array}
  \right]
  \left[
    \begin{array}{cc}
      \lambda_1& \\
               &\MB
    \end{array}
  \right]
  \left[
    \begin{array}{cc}
      1&\\
       &\MQ
    \end{array}
  \right]=
  \left[
    \begin{array}{cc}
      \lambda_1& \\
               &\MQ^{-1}\MB\MQ
    \end{array}
  \right]=\left[
    \begin{array}{cc}
      \lambda_1& \\
               &\Lambda_1
    \end{array}
  \right]
  $$
  \pause 
  取$\MT=\MP\MS, \MT^{-1}=\MS^{-1}\MP^{-1}$,则
  $$
  \MT^{-1}\MA\MT=\diag(\lambda_1,\lambda_2,\cd,\lambda_n).
  $$
\end{proof}
\end{frame}

\begin{frame}
  给定一个$n$阶实对称矩阵$\MA$,如何求正交矩阵$\MA$,使得$\MT^{-1}\MA\MT=\Lambdabd$呢?
  \vspace{.1in}\pause

  \begin{enumerate}
  \item 有特征多项式$|\MA-\lambda\MI|=\Pi_{i=1}^m(\lambda-\lambda_i)^{r_i}=0$求出全部互异的特征值$\lambda_1,\cd,\lambda_m$。\\[.1in]
  \item 因$\MA$可对角化,故$r_i$重特征值$\lambda_i$对应$r_i$个线性无关的特征向量$\vx_{i_1},\cd,\vx_{i_r}$。\\[.1in]
  \item 利用施密特正交化过程,得到$r_i$个相互正交的单位向量$\vy_{i_1},\cd,\vy_{i_r}$。\\[.1in]
  \item 因不同特征值对应的特征向量正交,故得到$\{\vy_{i_1},\cd,\vy_{i_{r_i}} ~|~ i=1,\cd,m \}$为$n$个相互正交的单位特征向量,将其按列排成$n$阶矩阵,即得正交矩阵$\MT$.
  \end{enumerate}
\end{frame}

\begin{frame}
  \begin{li}
    设$$
    \MA=\left[
      \begin{array}{rrr}
        1&-2&2\\
        -2&-2&4\\
        2&4&-2
      \end{array}
    \right]
    $$
    求正交阵$\MT$,使得$\MT^{-1}\MA\MT$为对角阵。
  \end{li}
  \pause 
  \begin{jie}
    由
    $$
    |\MA-\lambda\MI|=\left|
      \begin{array}{rrr}
        1-\lambda&-2&2\\
        -2&-2-\lambda&4\\
        2&4&-2-\lambda
      \end{array}
    \right|=-(\lambda-2)^2(\lambda+7)
    $$得
    $\lambda_{1,2}=2$和$\lambda_3=-7$。 \pause 
  \end{jie}
\end{frame}

\begin{frame}
  \begin{jie}
    当$\lambda_{1,2}=2$时,由
    $$
    (\MA-\lambda\MI)\vx=\left[
      \begin{array}{rrr}
        -1&-2&2\\
        -2&-4&4\\
        2&4&-4
      \end{array}
    \right]
    \left[
      \begin{array}{c}
        x_1\\
        x_2\\
        x_3
      \end{array}
    \right]=
    \left[
      \begin{array}{c}
        0\\
        0\\
        0
      \end{array}
    \right]
    $$
    得线性无关的特征向量$\vx_1=(2,-1,0)^T$和$\vx_2=(2,0,1)^T$。 \pause 
    
    用施密特正交化过程:先正交化
    $$
    \begin{aligned}
      \betabd_1&=\vx_1,\\
      \betabd_2&=\vx_2-\frac{(\vx_2,\betabd_1)}{(\betabd_1,\betabd_1)}\betabd_1=\frac15\left[
      \begin{array}{c}
        2\\
        4\\
        5
      \end{array}
    \right],\\
    \end{aligned}
    $$
    再单位化得
    $$
    \vy_1=\left[
      \begin{array}{c}
        \frac{2\sqrt5}5\\[.1in]
        -\frac{\sqrt 5}5\\[.1in]
        0
      \end{array}
    \right],
    \vy_2=\left[
      \begin{array}{c}
        \frac{2\sqrt5}{15}\\[.1in]
        \frac{4\sqrt5}{15}\\[.1in]
        \frac{\sqrt5}3
      \end{array}
    \right]
    $$
  \end{jie}
\end{frame}

\begin{frame}
  \begin{jie}
    当$\lambda_3=-7$时,由
    $$
    (\MA-\lambda\MI)\vx=\left[
      \begin{array}{rrr}
        8&-2&2\\
        -2&5&4\\
        2&4&5
      \end{array}
    \right]\left[
      \begin{array}{c}
        x_1\\
        x_2\\
        x_3
      \end{array}
    \right]=
    \left[
      \begin{array}{c}
        0\\
        0\\
        0
      \end{array}
    \right]
    $$
    得特征向量$\vx_3=(1,2,-2)^T$,单位化得$\vy_3=(\frac13,\frac23,-\frac23)^T$。
     \end{jie}
\end{frame}

\begin{frame}
  \begin{jie}
    取正交阵
    $$
    \MT=(\vy_1,\vy_2,\vy_3)=\left[
      \begin{array}{rrr}
        \frac{2\sqrt5}5 &\frac{2\sqrt5}{15}&\frac13\\
        -\frac{\sqrt 5}5&\frac{4\sqrt5}{15}&\frac23\\
        0&\frac{\sqrt5}3&-\frac23
      \end{array}
    \right]
    $$
    则$\MT^{-1}\MA\MT=\diag(2,2,-7)$。
  \end{jie}
\end{frame}

\begin{frame}
  \begin{li}
    设实对称矩阵$\MA$和$\MB$是相似矩阵,证明:存在正交矩阵$\MT$使得$\MT^{-1}\MA\MT=\MB$.
  \end{li}
\end{frame}

\begin{frame}
  \begin{li}
    设$n$阶实对称矩阵$\MA,\MB$有完全相同的$n$个特征值,证明:村早正交阵$\MT$和$n$阶矩阵$\MQ$使得$\MA=\MQ\MT$和$\MB=\MT\MQ$同时成立。
  \end{li}
\end{frame}

\begin{frame}
  \begin{li}
    设$\MA$和$\MB$都是$n$实对称矩阵,若存在正交矩阵$\MT$使得$\MT^{-1}\MA\MT, \MT^{-1}\MB\MT$都是对角阵,则$\MA\MB$是实对称阵。
  \end{li}
\end{frame}